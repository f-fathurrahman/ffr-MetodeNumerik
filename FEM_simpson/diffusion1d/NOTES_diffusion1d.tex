\RequirePackage{luatex85} % tuftebook not yet compatible with recent luatex

\documentclass[a4paper,english]{tufte-handout}

\usepackage{fontspec}
\setmonofont{JuliaMono-Regular}

\usepackage{hyperref}
\usepackage{url}
\usepackage{xcolor}

\usepackage{amsmath}
\usepackage{amssymb}

\usepackage{graphicx}
\usepackage{float}

\usepackage{minted}
\usepackage{enumitem}

\newminted{julia}{breaklines,fontsize=\footnotesize}
\newminted{python}{breaklines,fontsize=\footnotesize}

\newminted{bash}{breaklines,fontsize=\footnotesize}
\newminted{text}{breaklines,fontsize=\footnotesize}

\newcommand{\txtinline}[1]{\mintinline[breaklines,fontsize=\footnotesize]{text}{#1}}
\newcommand{\jlinline}[1]{\mintinline[breaklines,fontsize=\footnotesize]{julia}{#1}}
\newcommand{\pyinline}[1]{\mintinline[breaklines,fontsize=\footnotesize]{python}{#1}}

\newmintedfile[juliafile]{julia}{breaklines,fontsize=\footnotesize}
\newmintedfile[pythonfile]{python}{breaklines,fontsize=\footnotesize}
\newmintedfile[fortranfile]{fortran}{breaklines,fontsize=\footnotesize}
% f-o-otnotesize

\usepackage{mdframed}
\usepackage{setspace}
\onehalfspacing

\usepackage{mhchem}
\usepackage{appendix}

\newcommand{\highlighteq}[1]{\colorbox{blue!25}{$\displaystyle#1$}}
\newcommand{\highlight}[1]{\colorbox{red!25}{#1}}

\newcounter{soal}%[section]
\newenvironment{soal}[1][]{\refstepcounter{soal}\par\medskip
   \noindent {\color{blue}\textbf{Soal~\thesoal. #1}} \sffamily}{\medskip}


\definecolor{mintedbg}{rgb}{0.95,0.95,0.95}
\BeforeBeginEnvironment{minted}{
  \begin{mdframed}[ %backgroundcolor=mintedbg,%
    topline=false,bottomline=false,%
    leftline=false,rightline=false]
}
\AfterEndEnvironment{minted}{\end{mdframed}}


%\BeforeBeginEnvironment{soal}{
%  \begin{mdframed}[%
%    topline=true,bottomline=true,%
%    leftline=true,rightline=true]
%}
%\AfterEndEnvironment{soal}{\end{mdframed}}


\setcounter{secnumdepth}{1}

% Noindent for paragraph
% https://tex.stackexchange.com/questions/77999/remove-indent-of-paragraph-and-add-line-skip-with-tufte-latex

\makeatletter
% Paragraph indentation and separation for normal text
\renewcommand{\@tufte@reset@par}{%
  \setlength{\RaggedRightParindent}{0pt}%
  \setlength{\JustifyingParindent}{0pt}%
  \setlength{\parindent}{0pt}%
  \setlength{\parskip}{1pt}%
}
\@tufte@reset@par

% Paragraph indentation and separation for marginal text
\renewcommand{\@tufte@margin@par}{%
  \setlength{\RaggedRightParindent}{0pt}%
  \setlength{\JustifyingParindent}{0pt}%
  \setlength{\parindent}{0pt}%
  \setlength{\parskip}{1pt}%
}
\makeatother

% You can copy those commands to the preamble of your document and fill in the values that you prefer (e.g., 0pt for the indents and \baselineskip for the \parskip).

\begin{document}

\title{Pengenalan Metode Elemen Hingga: Difusi Transien 1d}
\author{Fadjar Fathurrahman}
\date{}
\maketitle

Difusi Transien 1d

Persamaan yang ingin diselesasikan adalah:
\begin{equation*}
\frac{\partial T}{\partial t} = \kappa \frac{\partial^2 T}{\partial x^2} + H
\end{equation*}

\begin{figure}[h]
{\centering
\includegraphics[width=\textwidth]{images_priv/Simpson_Fig_2_1.pdf}
\par}
\end{figure}

Syarat awal:
\begin{equation*}
T(x,t=0) = 0 \,\, \forall x \in [0,L_{x}]
\end{equation*}

Syarat batas:
\begin{equation*}
T(x=0,t) = 0 \,\, \text{dan} \,\, T(x=L_{x},t) = 0
\end{equation*}


Ekspansi solusi dengan fungsi basis (fungsi bentuk, \textit{shape functions}):
\begin{equation*}
T \approx \sum_{i=1}^{n} N_{i}(x) T_{i}
\end{equation*}

Elemen linear
\begin{align*}
N_{1}(x) = 1 - \frac{x}{L} \\
N_{2}(x) = \frac{x}{L}
\end{align*}

Discretized equations
\begin{equation*}
\mathbf{L} \mathbf{T}^{n+1} = \mathbf{R} \mathbf{T}^{n} + \mathbf{F}
\end{equation*}

\begin{equation*}
\mathbf{L} = \frac{\mathbf{M}}{\Delta t} + \mathbf{K}
\end{equation*}

Mass matrix
\begin{equation*}
\mathbf{M} = \begin{bmatrix}
\dfrac{\Delta x}{3} & \dfrac{\Delta x}{6} \\
\dfrac{\Delta x}{6} & \dfrac{\Delta x}{3}
\end{bmatrix}
\end{equation*}

\begin{equation*}
\mathbf{R} = \frac{\mathbf{M}}{\Delta t}
\end{equation*}


Stiffness matrix
\begin{equation*}
\mathbf{K} = \kappa\begin{bmatrix}
\dfrac{1}{\Delta x} & -\dfrac{1}{\Delta x} \\[0.4cm]
-\dfrac{1}{\Delta x} & \dfrac{1}{\Delta x}
\end{bmatrix}
\end{equation*}



Load vector
\begin{equation*}
\mathbf{F} = H\begin{bmatrix}
\dfrac{\Delta x}{2} \\
\dfrac{\Delta x}{2}
\end{bmatrix}
\end{equation*}

$$
\mathbf{T} = \begin{bmatrix}
T_1 \\ T_2
\end{bmatrix}
$$

Sistem global (menggunakan matriks dan vektor global):
$$
\tilde{\mathbf{L}} \tilde{\mathbf{T}}^{n+1} = \tilde{\mathbf{R}}\ 
\tilde{\mathbf{T}}^{n} + \tilde{\mathbf{F}} = \tilde{\mathbf{b}}
$$

%\bibliographystyle{unsrt}
%\bibliography{BIBLIO}

\end{document}

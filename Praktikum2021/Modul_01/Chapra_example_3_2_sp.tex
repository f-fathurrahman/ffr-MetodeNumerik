\section{Chapra Contoh 3.2, \textit{single precision}}
Secara default, perhitungan dengan \textit{floating number} pada Python (dan NumPy)
dilakukan dengan menggunakan \textit{double precision}. Pada bagian ini, kita akan
mengulangi Chapra Contoh 3.2 dengan menggunakan \textit{single precision}.
Pada C dan C++, tipe yang relevan adalah \txtinline{float}
untuk \textit{single precision} dan \txtinline{double} untuk \textit{double
precision}.
Pada Fortran kita dapat menggunakan \txtinline{REAL(4)} untuk \textit{single precision}
dan \txtinline{REAL(8)} untuk \textit{double precision}.

Karena Python merupakan bahasa pemrograman dinamik yang \textit{type-loose} kita tidak dapat
dengan mudah memberikan spesifikasi pada variabel yang kita gunakan. Meskipun demikian, 
kita dapat menggunakan \textit{single precision} pada Python melalui
\txtinline{np.float32}
\footnote{\txtinline{np} digunakan sebagai
singkatan dari modul \txtinline{numpy}},
meskipun program yang dihasilkan kurang elegan. Selain itu, kita juga harus mengecek
apakah hasil akhir yang diberikan tetap berupa \textit{single precision} (tidak terjadi
\textit{type promotion} ke \textit{double precision}).

\begin{pythoncode}
from math import factorial
import numpy as np
  
def approx_exp(x, N):
  assert(N >= 0)
  if N == 0:
      return 1
  s = np.float32(0.0)
  for i in range(N+1):
      s = s + np.float32(x**i)/np.float32(factorial(i))
  return s
  
x = np.float32(0.5)
true_val = np.exp(x) # from np module
  
n_digit = 3
# Equation 3.7
ε_s_percent = np.float32(0.5)*np.float32(10**(2-n_digit))
  
prev_approx = np.float32(0.0)
for N in range(50):
  approx_val = approx_exp(x, N)
  ε_t_percent = abs(approx_val - true_val)/true_val * 100
  if N > 0:
    ε_a_percent = abs(approx_val - prev_approx)/approx_val * 100
  else:
    ε_a_percent = float('nan')
  prev_approx = approx_val
  print("%3d %18.10f %10.5f%% %10.5f%%" % (N+1, approx_val, ε_t_percent, ε_a_percent))
  if ε_a_percent < ε_s_percent:
    print("Converged within %d significant digits" % n_digit)
    break
  
print("true_val   is %18.10f" % true_val)
print("approx_val is %18.10f" % approx_val)
  
# Make sure that float32 is used
print()
print("type(true_val)   = ", type(true_val))
print("type(approx_val) = ", type(approx_val))
\end{pythoncode}

Berikut ini adalah keluaran dari program.
\begin{textcode}
  1       1.0000000000   39.34693%        nan%
  2       1.5000000000    9.02040%   33.33333%
  3       1.6250000000    1.43876%    7.69231%
  4       1.6458333731    0.17516%    1.26583%
  5       1.6484375000    0.01721%    0.15798%
  6       1.6486979723    0.00141%    0.01580%
Converged within 3 significant digits
true_val   is       1.6487212181
approx_val is       1.6486979723

type(true_val)   =  <class 'numpy.float32'>
type(approx_val) =  <class 'numpy.float32'>
\end{textcode}

\begin{soal}
Ulangi perhitungan pada Chapra Contoh 3.2 dengan menggunakan \textit{single precision}
dengan jumlah digit signifikan yang berbeda, misalnya 5, 8, dan 10 digit signifikan
(berdasarkan kriteria Scarborough). Bandingkan hasil yang Anda dapatkan jika
\textit{double precision}. Apa yang dapat Anda simpulkan?
\end{soal}
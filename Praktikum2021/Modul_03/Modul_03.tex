\documentclass[a4paper,11pt,bahasa]{article} % screen setting
\usepackage[a4paper]{geometry}

%\documentclass[b5paper,11pt,bahasa]{article} % screen setting
%\usepackage[b5paper]{geometry}

\geometry{verbose,tmargin=1.5cm,bmargin=1.5cm,lmargin=1.5cm,rmargin=1.5cm}

\setlength{\parskip}{\smallskipamount}
\setlength{\parindent}{0pt}

%\usepackage{cmbright}
%\renewcommand{\familydefault}{\sfdefault}

\usepackage[libertine]{newtxmath}
\usepackage[no-math]{fontspec}

\setmainfont{Linux Libertine O}

\setmonofont{JuliaMono-Regular}


\usepackage{hyperref}
\usepackage{url}
\usepackage{xcolor}

\usepackage{amsmath}
\usepackage{amssymb}

\usepackage{graphicx}
\usepackage{float}

\usepackage{minted}

\newminted{julia}{breaklines,fontsize=\scriptsize}
\newminted{python}{breaklines,fontsize=\scriptsize}

\newminted{bash}{breaklines,fontsize=\scriptsize}
\newminted{text}{breaklines,fontsize=\scriptsize}

\newcommand{\txtinline}[1]{\mintinline[breaklines,fontsize=\scriptsize]{text}{#1}}
\newcommand{\jlinline}[1]{\mintinline[breaklines,fontsize=\scriptsize]{julia}{#1}}
\newcommand{\pyinline}[1]{\mintinline[breaklines,fontsize=\scriptsize]{python}{#1}}

\newmintedfile[juliafile]{julia}{breaklines,fontsize=\scriptsize}
\newmintedfile[pythonfile]{python}{breaklines,fontsize=\scriptsize}
% f-o-otnotesize

\usepackage{mdframed}
\usepackage{setspace}
\onehalfspacing

\usepackage{mhchem}
\usepackage{appendix}

\newcommand{\highlighteq}[1]{\colorbox{blue!25}{$\displaystyle#1$}}
\newcommand{\highlight}[1]{\colorbox{red!25}{#1}}

\newcounter{soal}%[section]
\newenvironment{soal}[1][]{\refstepcounter{soal}\par\medskip
   \noindent \textbf{Soal~\thesoal. #1} \sffamily}{\medskip}


\definecolor{mintedbg}{rgb}{0.95,0.95,0.95}
\BeforeBeginEnvironment{minted}{
    \begin{mdframed}[backgroundcolor=mintedbg,%
        topline=false,bottomline=false,%
        leftline=false,rightline=false]
}
\AfterEndEnvironment{minted}{\end{mdframed}}


\BeforeBeginEnvironment{soal}{
    \begin{mdframed}[%
        topline=true,bottomline=true,%
        leftline=true,rightline=true]
}
\AfterEndEnvironment{soal}{\end{mdframed}}

% -------------------------
\begin{document}

\title{%
{\small TF2202 Komputasi Rekayasa}\\
Sistem Persamaan Linear
}
\author{Tim Praktikum Komputasi Rekayasa 2021\\
Teknik Fisika\\
Institut Teknologi Bandung}
\date{}
\maketitle


\section{Eliminasi Gauss}

Verbose Gaussian elimination.
Gambar 9.6 pada Chapra.
Lakukan modifikasi jika diperlukan.
Ingat bahwa untuk Python indeks dimulai dari 0.

\section{Dekomposisi LU}

Verbose dekomposisi LU.

Gambar 10.2 pada Chapra.

Uji untuk mencari invers dari matriks berikut.

\section{Menggunakan Pustaka Python}

Pada bagian ini, kita akan menggunakan beberapa pustaka SciPy yang dapat digunakan
untuk mencari akar persamaan nonlinear. Beberapa fungsi tersebut dapat ditemukan
pada modul \txtinline{scipy.optimize}.
\begin{itemize}
\item \txtinline{scipy.optimize.root_scalar}: untuk mencari akar persamaan nonlinear
yang terdiri dari satu variabel.
\item \txtinline{scipy.optimize.fsolve}: untuk mencari akar dari sistem persamaan
nonlinear (lebih dari satu variabel)
\end{itemize}

Pada contoh berikut, kita akan mencari akar dari $f(x) = x - \cos(x)$.
Untuk metode terbuka, kita menggunakan tebakan akar awal $x_0 = 0$
sedangkan untuk 
\begin{pythoncode}
from scipy import optimize
import math
    
def f(x):
    return x - math.cos(x)
    
def fprime(x):
    return 1 + math.sin(x)
    
print("\nUsing Newton-Raphson method")
sol = optimize.root_scalar(f, x0=0.0, fprime=fprime, method='newton')
print(sol)
    
print("\nUsing secant method")
sol = optimize.root_scalar(f, x0=0.0, x1=1.0, method='secant')
print(sol)
    
# Bracketing methods
for method in ["brentq", "brenth", "ridder", "bisect"]:
    print("\nUsing %s method" % (method))
    sol = optimize.root_scalar(f, bracket=[0.0, 1.0], method=method)
    print(sol)

print("\nUsing bisect directly")
xroot, sol = optimize.bisect(f, a=0.0, b=1.0, full_output=True)
print(sol)    
\end{pythoncode}
Modul \txtinline{scipy.optimize} juga menyediakan beberapa fungsi lain seperti
\txtinline{optimize.bisect} yang dapat digunakan untuk mencari akar persamaan
nonlinear, yang berbeda hanyalah \txtinline{interface} atau \txtinline{function signature}-nya.

Pada contoh berikut ini, kita akan mencari akar dari sistem persamaan nonlinear
pada Chapra Contoh 6.12 dengan menggunakan \txtinline{fsolve}:
\begin{pythoncode}
from scipy.optimize import fsolve

def f(x_):
    # x = x_[0] and y = x_[1] 
    x = x_[0]
    y = x_[1]
    u = x**2 + x*y - 10
    v = y + 3*x*y**2 - 57
    return [u, v]
    
root = fsolve(f, [1.0, 3.5]) # using initial guess as in the book
print(root)
\end{pythoncode}

Numpy menyediakan modul khusus untuk merepresentasikan polinomial, yaitu
\txtinline{numpy.polynomial}. Modul ini menyediakan banyak fungsi untuk
melakukan berbagai operasi terkait polinomial.
Pada contoh berikut ini, kita akan mencari akar-akar (real dan kompleks)
dari polinomial:
\begin{equation*}
f(x) = x^5 - 3.5x^4 + 2.75x^3 + 2.125x^2 - 3.875x + 1.25
\end{equation*}

Kode Python:
\begin{pythoncode}
from numpy.polynomial import Polynomial

p = Polynomial([1.25, -3.875, 2.125, 2.75, -3.5, 1.0])
print(p.roots())
\end{pythoncode}

Silakan membaca dokumentasi berikut untuk informasi lebih lanjut.
\begin{itemize}
\item {\scriptsize\url{https://docs.scipy.org/doc/scipy/reference/optimize.html\#root-finding}}
\item {\scriptsize\url{https://docs.scipy.org/doc/scipy/reference/generated/scipy.optimize.fsolve.html}}
\item {\scriptsize\url{https://numpy.org/doc/stable/reference/routines.polynomials.package.html}}
\end{itemize}

\begin{soal}
Gunakan Numpy untuk menentukan akar-akar dari polinomial berikut:
\begin{itemize}
\item $f(x) = x^3 - x^2 + 2x - 2$
\item $f(x) = 2x^4 + 6x^2 + 8$
\item $f(x) = x^4 - 2x^3 + 6x^2 - 2x + 5$
\item $f(x) = -2 + 6.2x - 4x^2 + 0.7x^3$
\item $f(x) = 9.34 - 21.97x + 16.3x^2 - 3.704x^3$
\item $f(x) = x^4 - 2x^3 + 6x^2 - 2x + 5$
\end{itemize}
\end{soal}


\end{document}

\documentclass[a4paper,11pt,bahasa]{article} % screen setting
\usepackage[a4paper]{geometry}

%\documentclass[b5paper,11pt,bahasa]{article} % screen setting
%\usepackage[b5paper]{geometry}

\geometry{verbose,tmargin=1.5cm,bmargin=1.5cm,lmargin=1.5cm,rmargin=1.5cm}

\setlength{\parskip}{\smallskipamount}
\setlength{\parindent}{0pt}

%\usepackage{cmbright}
%\renewcommand{\familydefault}{\sfdefault}

\usepackage[libertine]{newtxmath}
\usepackage[no-math]{fontspec}

\setmainfont{Linux Libertine O}

\setmonofont{JuliaMono-Regular}


\usepackage{hyperref}
\usepackage{url}
\usepackage{xcolor}

\usepackage{amsmath}
\usepackage{amssymb}

\usepackage{graphicx}
\usepackage{float}

\usepackage{minted}
\usepackage{enumitem}

\newminted{julia}{breaklines,fontsize=\scriptsize}
\newminted{python}{breaklines,fontsize=\scriptsize}

\newminted{bash}{breaklines,fontsize=\scriptsize}
\newminted{text}{breaklines,fontsize=\scriptsize}

\newcommand{\txtinline}[1]{\mintinline[breaklines,fontsize=\scriptsize]{text}{#1}}
\newcommand{\jlinline}[1]{\mintinline[breaklines,fontsize=\scriptsize]{julia}{#1}}
\newcommand{\pyinline}[1]{\mintinline[breaklines,fontsize=\scriptsize]{python}{#1}}

\newmintedfile[juliafile]{julia}{breaklines,fontsize=\scriptsize}
\newmintedfile[pythonfile]{python}{breaklines,fontsize=\scriptsize}
% f-o-otnotesize

\usepackage{mdframed}
\usepackage{setspace}
\onehalfspacing

\usepackage{mhchem}
\usepackage{appendix}

\newcommand{\highlighteq}[1]{\colorbox{blue!25}{$\displaystyle#1$}}
\newcommand{\highlight}[1]{\colorbox{red!25}{#1}}

\newcounter{soal}%[section]
\newenvironment{soal}[1][]{\refstepcounter{soal}\par\medskip
   \noindent \textbf{Soal~\thesoal. #1} \sffamily}{\medskip}


\definecolor{mintedbg}{rgb}{0.95,0.95,0.95}
\BeforeBeginEnvironment{minted}{
    \begin{mdframed}[backgroundcolor=mintedbg,%
        topline=false,bottomline=false,%
        leftline=false,rightline=false]
}
\AfterEndEnvironment{minted}{\end{mdframed}}


\BeforeBeginEnvironment{soal}{
    \begin{mdframed}[%
        topline=true,bottomline=true,%
        leftline=true,rightline=true]
}
\AfterEndEnvironment{soal}{\end{mdframed}}

% -------------------------
\begin{document}

\title{%
{\small TF2202 Komputasi Rekayasa}\\
Integral dan Diferensiasi Numerik
}
\author{Tim Praktikum Komputasi Rekayasa 2021\\
Teknik Fisika\\
Institut Teknologi Bandung}
\date{}
\maketitle

\begin{soal}[Chapra Latihan 21.1]
Untuk integral-integral berikut ini
\begin{align}
& \int_{0}^{\pi/2} (6 + 3\cos(x))\ \mathrm{d}x \\
& \int_{0}^{3} (1 - e^{-2x})\ \mathrm{d}x \\
& \int_{-2}^{4} (1 - x - 4x^3 + 2x^5)\ \mathrm{d}x \\
& \int_{1}^{2} (x - 2/x)^2\ \mathrm{d}x \\
& \int_{-3}^{5} (4x - 3)^3 \mathrm{d}x \\
& \int_{0}^{3} x^2 e^x \mathrm{d}x \\
& \int_{0}^{1} 14^{2x}\ \mathrm{d}x
\end{align}
dengan menggunakan metode:
\begin{enumerate}[label=(\alph*)]
\item analitik
\item aturan trapesium
\item aturan 1/3 Simpson
\item aturan 3/8 Simpson
\item aturan Boole
\end{enumerate}
Variasikan parameter numerik yang terkait, seperti jumlah titik yang dievaluasi,
untuk setiap metode yang digunakan.
Bandingkan hasil numerik yang diberikan dengan hasil analitik.
Anda dapat menggunakan SymPy untuk menghitung integral secara analitik.
\end{soal}

Contoh penggunaan SymPy untuk perhitungan integral.
\begin{pythoncode}
import sympy
x = sympy.symbols("x")
func_symb = 6 + 3*sympy.cos(x)
resExact = sympy.N(sympy.integrate(func_symb, (x, 0, sympy.pi/2)))
# fungsi sympy.N digunakan untuk memaksa hasil dalam bentuk numerik.
\end{pythoncode}


\begin{soal}[Chapra 21.10 dan 21.11]
Hitung integral dari data berikut ini dengan menggunakan aturan trapesium dan aturan
Simpson.

Data Chapra 21.10

{\centering
\begin{tabular}{|c|cccccc|}
\hline
$x$    & 0 & 0.1 & 0.2 & 0.3 & 0.4 & 0.5 \\
$f(x)$ & 1 & 8   & 4   & 3.5 & 5   & 1 \\
\hline
\end{tabular}
\par}

Data Chapra 21.11

{\centering
\begin{tabular}{|c|ccccccc|}
\hline
$x$    & -2 & 0 & 2 & 4 & 6 & 8 & 10 \\
$f(x)$ & 35 & 5 & -10 & 2 & 5 & 3 & 20 \\
\hline
\end{tabular}
\par}
\end{soal}

\begin{soal}[Chapra 21.13]
Diketahui data berikut ini dihasilkan dari fungsi $f(x)=2e^{-1.5x}$

{\centering
\begin{tabular}{|c|ccccccc|}
\hline
$x$    & 0 & 0.05   & 0.15 & 0.25 & 0.35 & 0.475 & 0.6 \\
$f(x)$ & 2 & 1.8555 & 1.5970 & 1.3746 & 1.1831 & 0.9808 & 0.8131 \\
\hline
\end{tabular}
\par}

Hitung integral dari data ini dari $a=0$ dan $b=0.6$ dengan menggunakan
aturan trapesium dan aturan Simpson. Bandingkan hasil yang Anda peroleh
dengan melakukan integrasi $f(x)$ secara analitik.
\end{soal}


\end{document}

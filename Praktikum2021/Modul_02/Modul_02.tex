\documentclass[a4paper,11pt,bahasa]{article} % screen setting
\usepackage[a4paper]{geometry}

%\documentclass[b5paper,11pt,bahasa]{article} % screen setting
%\usepackage[b5paper]{geometry}

\geometry{verbose,tmargin=1.5cm,bmargin=1.5cm,lmargin=1.5cm,rmargin=1.5cm}

\setlength{\parskip}{\smallskipamount}
\setlength{\parindent}{0pt}

%\usepackage{cmbright}
%\renewcommand{\familydefault}{\sfdefault}

\usepackage[libertine]{newtxmath}
\usepackage[no-math]{fontspec}

\setmainfont{Linux Libertine O}

\setmonofont{JuliaMono-Regular}


\usepackage{hyperref}
\usepackage{url}
\usepackage{xcolor}

\usepackage{amsmath}
\usepackage{amssymb}

\usepackage{graphicx}
\usepackage{float}

\usepackage{minted}

\newminted{julia}{breaklines,fontsize=\scriptsize}
\newminted{python}{breaklines,fontsize=\scriptsize}

\newminted{bash}{breaklines,fontsize=\scriptsize}
\newminted{text}{breaklines,fontsize=\scriptsize}

\newcommand{\txtinline}[1]{\mintinline[breaklines,fontsize=\scriptsize]{text}{#1}}
\newcommand{\jlinline}[1]{\mintinline[breaklines,fontsize=\scriptsize]{julia}{#1}}
\newcommand{\pyinline}[1]{\mintinline[breaklines,fontsize=\scriptsize]{python}{#1}}

\newmintedfile[juliafile]{julia}{breaklines,fontsize=\scriptsize}
\newmintedfile[pythonfile]{python}{breaklines,fontsize=\scriptsize}
% f-o-otnotesize

\usepackage{mdframed}
\usepackage{setspace}
\onehalfspacing

\usepackage{appendix}

\newcommand{\highlighteq}[1]{\colorbox{blue!25}{$\displaystyle#1$}}
\newcommand{\highlight}[1]{\colorbox{red!25}{#1}}

\newcounter{soal}%[section]
\newenvironment{soal}[1][]{\refstepcounter{soal}\par\medskip
   \noindent \textbf{Soal~\thesoal. #1} \sffamily}{\medskip}


\definecolor{mintedbg}{rgb}{0.95,0.95,0.95}
\BeforeBeginEnvironment{minted}{
    \begin{mdframed}[backgroundcolor=mintedbg,%
        topline=false,bottomline=false,%
        leftline=false,rightline=false]
}
\AfterEndEnvironment{minted}{\end{mdframed}}


\BeforeBeginEnvironment{soal}{
    \begin{mdframed}[%
        topline=true,bottomline=true,%
        leftline=true,rightline=true]
}
\AfterEndEnvironment{soal}{\end{mdframed}}

% -------------------------
\begin{document}

\title{%
{\small TF2202 Komputasi Rekayasa}\\
Aproksimasi, Kesalahan, dan Deret Taylor
}
\author{Tim Praktikum Komputasi Rekayasa 2021\\
Teknik Fisika\\
Institut Teknologi Bandung}
\date{}
\maketitle

\section{Chapra Contoh 5.3}

Kita ingin mencari nilai dari $c$ dari persamaan
\begin{equation*}
v(t) = \frac{gm}{c}(1 - e^{-(c/m)t})
\end{equation*}
sehingga untuk $v(t=10) = 40$, dengan $m=9.81$ dan $m=68.1$.
Nilai $c$ dapat dicari sebagai akar dari persamaan:
\begin{equation*}
f(c) \equiv \frac{gm}{c}(1 - e^{-(c/m)t}) - v(t)
\end{equation*}

Kita akan menggunakan metode \textit{bisection} untuk mengaproksimasi akar atau solusi dari
persamaan $f(c) = 0$. Diberikan dua nilai $x_{l}$ (lower) dan $x_{u}$ (upper), di mana
$x_{u} > x_{l}$ dan $f(c=x_{l})f(c=x_{u}) < 0$, metode \textit{bisection} memberikan
aproksimasi akar $x_{r}$ sebagai berikut:
\begin{equation}
x_{r} = \frac{x_{l} + x_{u}}{2}
\end{equation}

Kode Python berikut ini mengilustrasikan penggunaan 
\begin{pythoncode}
import numpy as np

m = 68.1 # mass, kg
v = 40.0 # velocity, m/s
t = 10.0 # time, s
g = 9.81
    
def f(c):
    return ... # lengkapi
    
x_true = 14.8011 # from the text
    
# Initial guess
xl = 12.0
xu = 16.0
    
# First iteration
print("\n1st iteration: ")
print("xl = %f, xu = %f" % (xl, xu))
print("f(xl) = %f, f(xu) = %f" % (f(xl), f(xu)))
xr = # ... lengkapi
print("xr = ", xr)
ε_t = abs(xr - x_true)/x_true*100 # error in percent
print("ε_t = %.1f %%" % ε_t)

# Determine new xr should replace xu or xl (make new interval)
if f(xl)*f(xr) < 0:
    xu = xr
else:
    xl = xr
    
# Second iteration
print("\n2nd iteration: ")
print("xl = %f, xu = %f" % (xl, xu))
print("f(xl) = %f, f(xu) = %f" % (f(xl), f(xu)))
xr = (xl + xu)/2
print("xr = ", xr)
ε_t = abs(xr - x_true)/x_true*100 # error in percent
print("ε_t = %.1f %%" % ε_t)

if f(xl)*f(xr) < 0:
    xu = xr
else:
    xl = xr
    
# Third iteration
# ... teruskan jika diperlukan

# Jika xr merupakan akar dari f, maka f(xr) harus mendekati 0
# Tampilkan hasil dari f(xr) di sini
\end{pythoncode}

\begin{soal}
Lengkapi kode untuk ilustrasi penggunakan bisection tersebut. Lakukan iterasi sampai
suatu kriteria tertentu yang Anda tentukan. Anda boleh menggunakan loop
untuk menghindari pengulangan kode.
\end{soal}

\section{Chapra Contoh 5.5}
Dengan menggunakan metode \textit{regula falsi} aproksimasi akar diberikan oleh:
\begin{equation}
x_{r} = x_{u} - \frac{f(x_u)(x_{l} - x_{u})}{f(x_l) - f(x_u)}
\end{equation}

\begin{soal}
Lakukan modifikasi pada kode yang diberikan di soal sebelumnya sehingga dapat
memberikan ilustrasi penggunaan metode \textit{regula falsi}. Bandingkan hasil yang
Anda dapatkan dengan metode \textit{bisection}.
\end{soal}

\section{Chapra Contoh 5.6}

Kita ingin menghitung akar dari persamaan:
\begin{equation}
f(x) = x^{10} - 1
\label{eq:chapra-example-5.6}
\end{equation}
dengan menggunakan metode \textit{bisection} dan \textit{regula falsi}.

\begin{soal}
Aplikasikan metode \textit{bisection} dan \textit{regula falsi} untuk mendapatkan
aproksimasi akar dari persamaan \eqref{eq:chapra-example-5.6}
\end{soal}

\section{Chapra Contoh 6.1}

Pada soal ini, kita ingin mencari akar dari $f(x) = e^{-x} - x$ dengan
menggunakan metode iterasi \textit{fixed-point}. Metode ini juga dikenal
dengan nama iterasi satu titik atau substitusi berurutan. Metode ini
bekerja dengan cara mengubah persamaan awal $f(x) = 0$ menjadi
$x = g(x)$. Untuk contoh $f(x)$ yang diberikan kita memiliki:
\begin{equation*}
x = e^{-x}
\end{equation*}
Dimulai dari tebakan awal $x_{0} = 0$, kita dapat melakukan iterasi
\textit{fixed-point} dengan menggunakan kode Python berikut.

\begin{pythoncode}
# Simple fixed-point iteration
    
import numpy as np # math module also can be used
    
def g(x):
    return .... # lengkapi
    
# Initial guess
x = 0.0
x_true = 0.56714329
    
print()
print("Initial point: x = ", x)
print()
for i in range(1,11):
    xnew = .... # lengkapi
    ε_a = np.abs( (xnew - x)/xnew )*100 # in percent
    ε_t = np.abs( (x_true - xnew)/x_true )*100
    print("%3d %10.6f %10.2f%% %10.2f%%" % (i, xnew, ε_a, ε_t))
    x = xnew    
\end{pythoncode}

\section{Soal tambahan}

Blah

\end{document}

\documentclass[a4paper,11pt,bahasa]{article} % screen setting

\usepackage[a4paper]{geometry}
\geometry{verbose,tmargin=1.5cm,bmargin=1.5cm,lmargin=1.5cm,rmargin=1.5cm}

\setlength{\parskip}{\smallskipamount}
\setlength{\parindent}{0pt}

%\usepackage{cmbright}
%\renewcommand{\familydefault}{\sfdefault}

%\usepackage{fontspec}
\usepackage[libertine]{newtxmath}
\usepackage[no-math]{fontspec}
\setmainfont{Linux Libertine O}
\setmonofont{DejaVu Sans Mono}
%\setmonofont{JuliaMono-Regular}


\usepackage{hyperref}
\usepackage{url}
\usepackage{xcolor}

\usepackage{amsmath}
\usepackage{amssymb}

\usepackage{graphicx}
\usepackage{float}

\usepackage{minted}

\newminted{julia}{breaklines,fontsize=\footnotesize}
\newminted{python}{breaklines,fontsize=\footnotesize}

\newminted{bash}{breaklines,fontsize=\footnotesize}
\newminted{text}{breaklines,fontsize=\footnotesize}

\newcommand{\txtinline}[1]{\mintinline[breaklines,fontsize=\footnotesize]{text}{#1}}
\newcommand{\jlinline}[1]{\mintinline[breaklines,fontsize=\footnotesize]{julia}{#1}}
\newcommand{\pyinline}[1]{\mintinline[breaklines,fontsize=\footnotesize]{python}{#1}}

\newmintedfile[juliafile]{julia}{breaklines,fontsize=\footnotesize}
\newmintedfile[pythonfile]{python}{breaklines,fontsize=\footnotesize}

\definecolor{mintedbg}{rgb}{0.90,0.90,0.90}
\usepackage{mdframed}
\BeforeBeginEnvironment{minted}{
    \begin{mdframed}[backgroundcolor=mintedbg,%
        topline=false,bottomline=false,%
        leftline=false,rightline=false]
}
\AfterEndEnvironment{minted}{\end{mdframed}}

\usepackage{mhchem}
\usepackage{setspace}

\onehalfspacing

\usepackage{appendix}


\newcommand{\highlighteq}[1]{\colorbox{blue!25}{$\displaystyle#1$}}
\newcommand{\highlight}[1]{\colorbox{red!25}{#1}}


\newcounter{soal}[section]
\newenvironment{soal}[1][]{\refstepcounter{soal}\par\medskip
   \noindent \textbf{Soal~\thesoal. #1} \rmfamily}{\medskip}

\begin{document}

\title{%
{\small TF2202 Komputasi Rekayasa}\\
Akar Persamaan Nonlinear
}
\author{Tim Praktikum Komputasi Rekayasa 2021\\
Teknik Fisika\\
Institut Teknologi Bandung}
\date{}
\maketitle


\begin{soal}
Tentukan akar positif terkecil dari
\begin{equation*}
f(x) = 7\sin(x)e^{-x} - 1
\end{equation*}
\end{soal}


\begin{soal}[Fraksi mol air]
Dalam proses kimia, uap air ($\mathrm{H}_{\mathrm{2}}\mathrm{O}$) dipanaskan sampai
suhu yang cukup tinggi sehingga sebagian besar dari air akan terdisosiasi membentuk oksigen
($\mathrm{O}_{2}$) dan hidrogen ($\mathrm{H}_{2}$) menurut persamaan reaksi
\begin{equation*}
\mathrm{H}_2\mathrm{O} \rightarrow \mathrm{H}_{2}\,+\,\mathrm{O}_{2}
\end{equation*}
Jika diasumsikan bahwa ini adalah satu-satunya reaksi yang terlibat, fraksi
mol $\ce{H2O}$ yang berdisosiasi, dilambangkan dengan $x$, dapat dihitung dari persamaan
\begin{equation*}
K = \frac{x}{1-x}\sqrt{\frac{2p_t}{2+x}}
\end{equation*}
dimana $K$ = kesetimbangan konstan reaksi dan $p_t$: tekanan total campuran
dalam satuan atm.
Jika diketahui bahwa $p_t = 3.5$ atm dan $K = 0.4$, tentukan nilai $x$ yang memenuhi
persamaan diatas.
\end{soal}


\begin{soal}[Rangkaian RLC]
Impedansi dari rangkaian paralel RLC dinyatakan oleh persamaan
\begin{equation*}
\frac{1}{Z} = \sqrt{\frac{1}{R^2} + \left(
\omega C - \frac{1}{\omega L}
\right)^2
}
\end{equation*}
Cari frekuensi angular $\omega$ untuk $Z=75$ ohm,
$R=225$ ohm,
$C = 0.6 \times 10^{-6}$ F
dan $L = 0.5 H$.
\end{soal}


\begin{soal}[Balok elastis]
Gambar menunjukkan balok yang dikenai beban terdistribusi yang meningkat secara linier.
Persamaan untuk kurva elastis diberikan oleh
\begin{equation*}
y = \frac{w_0}{120EIL}
\left( -x^5 + 2L^2 x^3 - L^4 x \right)
\end{equation*}
Tentukan defleksi maksimum dengan menggunakan nilai parameter dalam perhitungan berikut:
$L = 450\, \mathrm{cm}$,
$E = 50000 \, \mathrm{kN}/\mathrm{cm}^2$,
$I = 30000 \, \mathrm{cm}^4$,
dan $w0 = 2.5$ kN/cm.
Hint: akar berada di sekitar 200, gunakan $y'$ sebagai fungsi untuk mencari akar persamaan)
\end{soal}


\begin{soal}
Kecepatan kearah atas sebuah roket dapat dihitung menggunakan persamaan berikut
\begin{equation*}
v = u \mathrm{ln}\frac{m_{0}}{m_{0} - qt} - gt
\end{equation*}
dimana $v$ = kecepatan kearah atas roket,
$u$ = kecepatan keluar bahan bakar relatif terhadap
roket, $m_0 $ = masa awal roket pada t = 0, $q$ = laju konsumsi bahan bakar,
dan $g$ = percepatan gravitasi. Jika $g = 9.81 \, \mathrm{m/s}^2$,
$u = 2200$ m/s, $m_0 = 160000$ kg,
dan $q = 2680$ kg/s, hitunglah waktu
dimana kecepatan mencapai $ v = 1000 $ m/s.
\end{soal}


\begin{soal}[Persamaan Redlich-Kwong]
Persamaan keadan Redlich-Kwong dinyatakan sebagai:
\begin{equation*}
P = \frac{R_u T}{v - b} - \frac{a}{v(v + b)\sqrt{T}}
\end{equation*}
di mana $ R_u $ = konstanta gas universal = 0.518 kJ/(kg K),
$T$ = Temperature absolute ($K$),
$P$ = Tekanan absolute (kPa),
and $v$ = volume spesifik gas (m3/kg).
Parameter a dan b dihitung dengan persamaan:
\begin{align*}
a & = -0.427 \frac{R^{2}_{u} T_{c}^{2.5}}{P_{c}} \\
b & = 0.0866 R_{u} \frac{T_c}{P_c}
\end{align*}
di mana $P_c$ dan $T_c$
adalah tekanan dan temperatur kritis.
Tentukan jumlah methana ( $P_c$ = 4600 kPa and $T_c$ = 191 K)
yang dapat disimpan dalam 3 $m^{3}$ tangki pada temperatur
−40$^{\circ}\,\mathrm{C}$ dengan tekanan 65 MPa.
\end{soal}

\begin{soal}
Gaya $F$ yang bekerja antara partikel bermuatan $q$ dengan
piringan bulat dengan jari-jari $R$ dan
rapat muatan $Q$ diberikan oleh persamaan:
\begin{equation*}
F = \frac{Qq}{2\epsilon 0}\left(
1 - \frac{z}{\sqrt{z^2 + R^2}}
\right)
\end{equation*}
di mana $\epsilon_{0} = 0.885 \times 10^{-12}$
$\mathrm{C}^{2}/(\mathrm{Nm}^{2})$
adalah konstanta permitivitas dan $z$ adalah jarak
partikel terhadap piringan.
Tentukan jarak z jika
$F = 0.3 \, \mathrm{N}$
$Q = 9.4 \times 10^{-6} \, \mathrm{C/m}^2$
$q = 2.4 \times 10^{-5} \, C$
dan $R=0.1$ m.
\end{soal}


\begin{soal}
Berdasar prinsip Archimedes, gaya apung yang bekerja pada
sebuah benda yang sebagian
terbenam dalam fluida sama dengan berat badan yang
dipindahkan oleh bagian dari objek yang
terendam. Sebuah pelampung bola dengan massa $ m_f $ = $ 70 $ kg
dan diameter 0.90 m ditempatkan di laut
(densitas air laut 1030 $\mathrm{kg/m}^3$).
Ketinggian $h$ dari bagian pelampung yang berada di atas
air dapat ditentukan dengan memecahkan persamaan yang menyamakan massa pelampung
dengan massa air yang dipindahkan oleh bagian dari pelampung yang terendam
$\rho V_{\mathrm{celup}} = m_{f}$
di mana, untuk bola dengan jari-jari $r$,
volume yang tercelup pada kedalaman d diberikan oleh:
$V_{\mathrm{celup}} = \frac{1}{3}\pi d^2 (3r - d)$
Carilah $h$ yang memenuhi kondisi tersebut.
\end{soal}


\begin{soal}
Sebuah filter bandpass melewatkan sinyal dengan frekuensi yang berada dalam kisaran tertentu.
Dalam filter ini rasio besaran tegangan diberikan oleh
\begin{equation*}
G = \frac{\left|V_0\right|}{\left|V_1\right|} =
\frac{\omega RC}{\sqrt{(1-\omega^2 LC)^2 + (\omega RC)^2}}
\end{equation*}
dimana $\omega$
adalah frekuensi sinyal input. Diberikan $L = 11$ mH,
$C = 8 \, \mu\mathrm{F}$
dan $R = 1000$ ohm, tentukan rentang frekuensi yang sesuai
agar $G \geq 0.87$.
\end{soal}


\begin{soal}
Daya output dari sel surya bervariasi terhadap tegangan output yang dikeluarkan. 
Tegangan $V_{m}$ di mana daya output maksimum diberikan oleh persamaan:
\begin{equation*}
e^{\frac{qV_m}{k_{B}T}} \left( 1 + \frac{qV_m}{k_{B}T} \right) = 
e^{\frac{qV_{oc}}{k_{B}T}}
\end{equation*}
di mana $ V_{oc} $ adalah tegangan rangkaian terbuka,
$T$ adalah suhu dalam Kelvin, $q=1.6022\times10^{-19}$ C
adalah muatan elektron, dan
$k_{B} = 1.3806\times10^{-23}$ J/K adalah konstanta Boltzmann.
Untuk $V_{oc} = 0.5$ V dan suhu kamar (T = 297 K),
tentukan tegangan $V_{m}$ di mana output daya dari sel surya adalah maksimum.
\end{soal}


\begin{soal}
Sebuah model sederhana dari sistem suspensi mobil terdiri dari massa $m$,
pegas dengan kekakuan $k$, dan peredam dengan
koefisien redaman $c$.
Sebuah jalan bergelombang dapat dimodelkan oleh gerakan
naik-turun sinusoidal roda $y = Y \sin(\omega t)$.
Dari solusi persamaan gerak untuk model ini, kondisi tunak gerak naik-turun mobil
(massa) diberikan oleh $x = X\sin(\omega-\varphi)$.
Rasio antara amplitudo $X$ dengan amplitudo $Y$ diberikan oleh:
\begin{equation*}
\frac{X}{Y} = \sqrt{
\frac{\omega^2 c + k^2}{(k - m\omega^2) + (\omega c)^2} }
\end{equation*}
Dengan nilai numerik berikut: $m$ = 2500 kg, $k$ = 300 kN/m,
dan $c = 36\times10^{3}$ Ns/m, tentukan frekuensi $\omega$ untuk
$X/Y = 0.4$.
\end{soal}

\begin{soal}[Fungsi Bessel sferis]
Fungsi Bessel bola (spherical Bessel) $j_{n}(x)$ dapat dituliskan sebagai
berikut:
\begin{equation*}
j_{n}(x) = (-x)^{n}
\left(
\frac{1}{x} \frac{\mathrm{d}}{\mathrm{d}x}
\right)^n
\frac{\sin(x)}{x}
\end{equation*}
Buatlah plot untuk $j_{2}(x)$ dan carilah semua akar-akarnya pada interval (0,20).
Gunakan salah satu metode untuk mencari akar persamaan nonlinear.
\end{soal}


\end{document}

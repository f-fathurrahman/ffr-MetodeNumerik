% ----------------------------------
\subsection*{Menggunakan deret Taylor orde-tinggi}

Misalnya, menggunakan orde dua:
\begin{equation*}
y_{i+1} = y_{i} + f(x_i, y_i)h + \frac{f'(x_i, y_i)}{2!}h^2
\end{equation*}
dengan kesalahan pemotongan lokal:
\begin{equation*}
E_{a} = \frac{f''(x_i, y_i)}{6} h^3
\end{equation*}
Dengan menggunakan aturan rantai untuk turunan:
\begin{equation*}
f'(x_i, y_i) = \frac{\partial f(x,y)}{\partial x} +
\frac{\partial f(x,y)}{\partial y}\frac{\mathrm{d}y}{\mathrm{d}x}
\end{equation*}


\subsection*{Metode Runge-Kutta}

\begin{equation*}
y_{i+1} = y_{i} + \phi(x_i, y_i, h) h
\end{equation*}

$\phi(x_i, y_i, h)$: fungsi kenaikan (\textit{increment})

\begin{equation*}
\phi = a_1 k_1 + a_2 k_2 + \cdots + a_n k_n
\end{equation*}

di mana $a$ adalah konstanta dan $k$ adalah:
\begin{align*}
k_1 & = f(x_i, y_i) \\
k_2 & = f(x_i + p_1 h, y_i + q_{11} k_1 h ) \\
k_3 & = f(x_i + p_2 h, y_i + q_{21} k_1 h + q_{22} k_2 h) \\
\cdots & \cdots \\
k_n & = f(x_i + p_{n-1}h, y_i + q_{n-1,1} k_1 h + q_{n-1,2} k_2 h + \cdots + q_{n-1,n-1} k_{n-1} h)
\end{align*}
di mana $p$ dan $q$ adalah konstanta.
Perhatikan bahwa $k$ memiliki hubungan rekurensi (perulangan).


\subsection{Runge-Kutta orde 2}

\begin{equation*}
y_{i+1} = y_i + (a_1 k_1 + a_2 k_2) h
\end{equation*}
dengan
\begin{align*}
k_1 = f(x_i, y_i) \\
k_2 = f(x_i + p_1 h, y_i + q_{11} k_1 h)
\end{align*}
Nilai-nilai dari $a_1$, $a_2$, $p_1$, dan $q_{11}$ dapat diperoleh dari persamaan
berikut.
\begin{align*}
a_1 + a_2  & = 1 \\
a_2 p_1    & = \frac{1}{2} \\
a_2 q_{11} & = \frac{1}{2}
\end{align*}
Karena ada tiga persamaan dan empat variabel yang tidak diketahui, kita harus mengasumsikan satu
nilai dari variabel tersebut untuk mendapatkan tiga variabel yang lain.
Misalkan nilai $a_2$ telah dipilih, maka variabel-variabel yang lain
dapat ditentukan sebagai berikut.
\begin{align*}
a_1 = 1 - a_2
p_1 = q_{11} = \frac{1}{2a_2}
\end{align*}
Dengan kata lain, ada tak hingga versi dari metode Runge-Kutta orde-2.

Ada tiga versi yang populer:

Metode Heun dengan korektor tunggal, $a_2 = 1/2$
$a_1 = 1/2$ dan $p_1 = q_{11} = 1$. Dengan parameter tersebut, diperoleh skema
sebagai berikut:
\begin{equation*}
y_{i+1} = y_{i} + \left(
\frac{1}{2}k_1 + \frac{1}{2}k_2
\right)h
\end{equation*}
dengan
\begin{align*}
k_1 = f(x_i, y_i) \\
k_2 = f(x_i + h, y_i + k_1 h)
\end{align*}
Perhatikan bahwa $k_1$ adalah kemiringan pada awal interval dan $k_2$ adalah
kemiringan pada akhir interval. Oleh karena itu, metode Runge-Kutta orde-2 ini
tidak lain adalah metode Heun tanpa iterasi.


Metode titik tengah, $a_2 = 1$, $a_1 = 0$, $p_1 = q_{11} = 1/2$, diperoleh
skema sebagai berikut.
\begin{equation*}
y_{i+1} = y_i + k_2 h
\end{equation*}
dengan
\begin{align*}
k_1 = f(x_i, y_i) \\
k_2 = f\left( x_i + \frac{1}{2}h, y_i + \frac{1}{2} k_1 h \right)
\end{align*}
yang merupakan metode titik tengah.

Metode Ralston, dikembangkan oleh Ralston (1962) dan Ralston dan Rabinowitz (1978),
yang memilih parameter-parameter sehingga batas minimum untuk kesalahan pemotongan,
dengan parameter $a_2 = 2/3$, $a_1 = 1/3$, dan $p_1 = q_{11} = 3/4$, yang
memberikan skema sebagai berikut:
\begin{equation*}
y_{i+1} = y_i + \left( \frac{1}{3}k_1 + \frac{2}{3}k_2 \right) h
\end{equation*}
dengan
\begin{align*}
k_1 = f(x_i, y_i) \\
k_2 = f\left( x_i + \frac{3}{4}h, y_i + \frac{3}{4}k_1 h \right)
\end{align*}



\subsection*{Persamaan diferensial orde dua}

Contoh:
\begin{equation*}
m \frac{\mathrm{d}^2 x}{\mathrm{d}t^2} + c \frac{\mathrm{d}x}{\mathrm{d}t} + kx = 0
\end{equation*}
$c$: koefisien redaman, $k$ konstant pegas.

Definisikan $y_{1}(t) = x(t)$, $y_{2}(t) = x'(t)$, sehingga:
$y'_{1}(t) = x'(t) = y_{2}(t)$, dan
$y'_{2}(t) = x''(t)$:
\begin{align*}
m \frac{\mathrm{d}^2 x}{\mathrm{d}t^2} + c \frac{\mathrm{d}x}{\mathrm{d}t} + kx & = 0 \\
m y'_{2}(t) + c y_{2}(t) + k y_{1}(t) & = 0 \\
y'_{2}(t) & = -\frac{c y_{2}(t) + k y_{1}(t)}{m}
\end{align*}

\begin{align*}
y'_{1}(t) & = y_{2}(t) \\
y'_{2}(t) & = -\frac{c y_{2}(t) + k y_{1}(t)}{m}
\end{align*}
%m \frac{\mathrm{d}^2 x}{\mathrm{d}t^2} + c \frac{\mathrm{d}x}{\mathrm{d}t} + kx = 0


\subsection*{Metode multilangkah}

Pada metode satu langkah, kita menggunakan informasi sebelumnya pada satu titik $x_i$ untuk
memprediksi nilai dari variabel dependen $y_{i+1}$ pada titik depan $x_{i+1}$.

Pada metode multilangkah (\textit{multistep}), kita menggunakan informasi sebelumnya
yang diperoleh pada beberapa titik sebelumnya.

\subsection*{Metode Heun \textit{non-self-starting}}

Ingat bahwa metode Heun menggunakan metode Euler sebagai prediktor:
\begin{equation*}
y_{i+1}^{0} = y_{i} + f(x_i, y_i) h
\end{equation*}
dan aturan trapesium sebagai korektor:
\begin{equation*}
y_{i+1} = y_{i} + \frac{f(x_i,y_i) + f(x_{i+1},y^{0}_{i+1})}{2} h 
\end{equation*}
Prediktor dan korektor memiliki kesalahan pemotongan lokal $\mathcal{O(h^2)}$
dan $\mathcal{O}(h^3)$. Hal ini menyarankan bahwa prediktor merupakan
hubungan yang lemah pada metode ini karena memiliki kesalahan yang terbesar.
Kelemahan ini signifikan karena efisiensi dari korektor iteratif bergantung
pada akurasi dari prediksi awal.
Salah satu cara untuk memperbaiki metode Heun adalah dengan mengembangkan prediktor
yang memiliki kesalahan pemotongan lokal $\mathcal{O}(h^3)$.
Hal ini dapat dicapai dengan menggunakan metode Euler dan kemiringan pada $y_i$ dan
informasi tambahan dari titik sebelumnya $y_{i-1}$ sebagai berikut:
\begin{equation*}
y^{0}_{i+1} = y_{i-1} + f(x_i, y_i)2h
\end{equation*}
Perhatikan bahwa persamaan diatas memiliki kesalahan lokal $\mathcal{O}(h^3)$
dengan menggunakan ukuran langkah yang lebih besar $2h$. Selain itu, persamaan
ini tidak self-starting karena melibatkan nilai sebelumnya, $y_{i-1}$.
Informasi ini tidak tersedia pada permasalahan nilai awal. Oleh karena itu
skema yang dihasilkan disebut sebagai metode Heun non-self-starting.

\begin{align*}
y^{0}_{i+1} = y^{m}_{i-1} + f(x_i,y^{m}_i)2h \\
y^{j}_{i+1} = y^{m}_{i} + \frac{f(x_i,y^{m}_i) + f(x_{i+1},y^{j-1}_{i+1})}{2} h 
\end{align*}

$m$: indeks iterasi

Perhatikan bahwa $y^{m}_{i}$ dan $y^{m}_{i-1}$


\subsection*{Formula Adams-Bashforth}

Ekspansi Taylor:
\begin{equation*}
y_{i+1} = y_{i} + f_{i} h + \frac{f'_{i}}{2!} h^2 + \frac{f''_{i}}{3!} h^3 + \cdots
\end{equation*}
yang juga dapat dituliskan sebagai:
\begin{equation*}
y_{i+1} = y_{i} + h\left( f_i + \frac{h}{2}f'_{i} + \frac{h^2}{3!} + \cdots \right)
\end{equation*}

Backward difference:
\begin{equation*}
f'_{i} = \frac{f_i - f_{i-1}}{h} + \frac{f''_{i}}{2!} + \mathcal{O}(h^2)
\end{equation*}

Substitusi:
\begin{equation*}
y_{i+1} = y_{i} + h \left(
f_{i} + \frac{h}{2} \left[ \frac{f_i - f_{i-1}}{h} + \frac{f''_{i}}{2} + \mathcal{O}(h^2) \right]
+ \frac{h^2}{6} f''_{i} + \cdots \right)
\end{equation*}
atau:
\begin{equation*}
y_{i+1} = y_{i} + h \left( \frac{3}{2} f_{i} - \frac{1}{2} f_{i-1} \right) +
\frac{5}{12} h^3 f''_{i} + \mathcal{O}(h^4)
\end{equation*}
(2nd order open Adams formula)
or Adams-Bashforth formula.

Bentuk umum formula Adams-Bashforth:
\begin{equation*}
y_{i+1} = y_{i} + h \sum_{k=0}^{n-1} \beta_{k} f_{i-k} + \mathcal{O}(h^{n+1})
\end{equation*}



\subsection*{Formula Adams-Moulton}

Deret Taylor mundur disekitar $x_{i+1}$:
\begin{equation*}
y_{i} = y_{i+1} - f_{i+1}h + \frac{f'_{i+1}}{2!} h^2 - \frac{f''_{i+1}}{3!} h^3 + \cdots
\end{equation*}

Cari $y_{i+1}$:
\begin{equation*}
y_{i+1} = y_i + h \left( f_{i+1} - \frac{h}{2}f'_{i+1} + \frac{h^2}{6} f''_{i+1} + \cdots \right)
\end{equation*}

Aproksimasi turunan pertama:
\begin{equation*}
f'_{i+1} = \frac{f_{i+1} - f_{i}}{h} + \frac{f''_{i+1}}{2}h + \mathcal{O}(h^2)
\end{equation*}
Substitusi:
\begin{equation*}
y_{i+1} = y_{i} + h \left( \frac{1}{2}f_{i+1} + \frac{1}{2}f_{i} \right) - 
\frac{1}{12} h^3 f''_{i+1} - \mathcal{O}(h^4)
\end{equation*}

Formula Adams tertutup orde-2 atau
Formula Adams-Moulton orde-2

Bentuk umum:
\begin{equation*}
y_{i+1} = y_{i} + h \sum_{k=0}^{n-1} \beta_{k} f_{i+1-k} + \mathcal{O}(h^{n+1})
\end{equation*}


\subsection*{Metode Milne}

Menggunakan formula Newton-Cotes terbuka tiga-titik sebagai prediktor:
\begin{equation*}
y^{0}_{i+1} = y^{m}_{i-3} + \frac{4h}{3}\left(
2f^{m}_{i} - f^{m}_{i-1} + 2f^{m}_{i-2}
\right)
\end{equation*}
dan Newton-Cotes tertutup tiga-titik sebagai korektor:
\begin{equation*}
y^{j}_{i+1} = y^{m}_{i-1} + \frac{h}{3} \left(
f^{m}_{i-1} + 4f^{m}_{i} + f^{j-1}_{i+1}
\right)
\end{equation*}
$j$ adalah indeks iterasi.

\subsection*{Metode Adams orde-4}

Menggunakan formula Adams-Bashforth orde-4 sebagai prediktor:
\begin{equation*}
y^{0}_{i+1} = y^{m}_{i} + h \left(
\frac{55}{24} f^{m}_{i} - \frac{59}{24}f^{m}_{i-1} + \frac{37}{24}f^{m}_{i-2}
- \frac{9}{24} f^{m}_{i-3}
\right)
\end{equation*}
dan formula Adams-Moulton orde-4 sebagai korektor:
\begin{equation*}
y^{j}_{i+1} = y^{m}_{i} + h \left( 
\frac{9}{24} f^{j-1}_{i+1} + \frac{19}{24} f^{m}_{i} - \frac{5}{24} f^{m}_{i-1}
+ \frac{1}{24} f^{m}_{i-2}
\right)
\end{equation*}



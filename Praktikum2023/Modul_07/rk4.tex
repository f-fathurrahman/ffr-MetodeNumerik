\subsection{Metode Runge-Kutta Orde-3}

Skema:
\begin{equation*}
y_{i+1} = y_i + \frac{1}{6}(k_1 + 4k_2 + k_3)h
\end{equation*}
dengan
\begin{align*}
k_1 & = f(x_i, y_i) \\
k_2 & = f(x_i + \frac{1}{2}h, y_i + \frac{1}{2} k_1 h ) \\
k_3 & = f(x_i + h, y_i - k_1 h + 2 k_2 h)
\end{align*}
Perhatikan bahwa, jika fungsi turunan merupakan fungsi dari $x$ saja, maka metode ini menjadi
aturan 1/3 Simpson.


\begin{pythoncode}
def ode_rk3_1step(dfunc, xi, yi, h):
    k1 = dfunc(xi, yi)
    k2 = dfunc(xi + 0.5*h, yi + 0.5*k1*h)
    k3 = dfunc(xi + h, yi - k1*h + 2*k2*h)
    yip1 = yi + (k1 + 4*k2 + k3)*h/6
    return yip1
\end{pythoncode}

\subsection{Metode Runge-Kutta Orde 4}

Skema:
\begin{equation*}
y_{i+1} = y_{i} + \frac{1}{6}(k_1 + 2k_2 + 2k_3 + k_4) h
\end{equation*}
dengan
\begin{align*}
k_1 & = f(x_i, y_i) \\
k_2 & = f\left( x_i + \frac{1}{2}h, y_i + \frac{1}{2} k_1 h \right) \\
k_3 & = f\left( x_i + \frac{1}{2}h, y_i + \frac{1}{2} k_2 h \right) \\
k_4 & = f(x_i + h, y_i + k_3 h)
\end{align*}

\begin{pythoncode}
def ode_rk4_1step(dfunc, xi, yi, h):
    k1 = dfunc(xi, yi)
    k2 = dfunc(xi + 0.5*h, yi + 0.5*k1*h)
    k3 = dfunc(xi + 0.5*h, yi + 0.5*k2*h)
    k4 = dfunc(xi + h, yi + k3*h)
    yip1 = yi + (k1 + 2*k2 + 2*k3 + k4)*h/6
    return yip1
\end{pythoncode}


\subsection{Metode Runge-Kutta Orde 5}

Skema (Butcher 1964):
\begin{equation*}
y_{i+1} = y_i + \frac{1}{90} ( 7 k_1 + 32 k_3 + 12 k_4 + 32 k_5 + 7 k_6) h
\end{equation*}
dengan:
\begin{align*}
k_1 & = f(x_i, y_i) \\
k_2 & = f\left( x_i + \frac{1}{4}h, y_i + \frac{1}{4} k_1 h \right) \\
k_3 & = f\left( x_i + \frac{1}{4}h, y_i + \frac{1}{8} k_1 h + \frac{1}{8} k_2 h \right) \\
k_4 & = f\left( x_i + \frac{1}{2}h, y_i - \frac{1}{2} k_2 h + k_3 h \right) \\
k_5 & = f\left( x_i + \frac{3}{4}h, y_i + \frac{3}{16} k_1 h + \frac{9}{16} k_4 h \right) \\
k_6 & = f\left( x_i + h, y_i - \frac{3}{7} k_1 h + \frac{2}{7} k_2 h +
\frac{12}{7}k_3 h - \frac{12}{7}k_4 h + \frac{8}{7}k_5 h \right)
\end{align*}

\begin{pythoncode}
def ode_rk5_1step(dfunc, xi, yi, h):
    k1 = dfunc(xi, yi)
    k2 = dfunc(xi + h/4, yi + k1*h/4)
    k3 = dfunc(xi + h/4, yi + k1*h/8 + k2*h/8)
    k4 = dfunc(xi + h/2, yi - k2*h/2 + k3*h)
    k5 = dfunc(xi + 3*h/4, yi + 3*k1*h/16 + 9*k4*h/16)
    k6 = dfunc(xi + h, yi - 3*k1*h/7 + 2*k2*h/7 + 12*k3*h/7 - 12*k4*h/7 + 8*k5*h/7)
    yip1 = yi + (7*k1 + 32*k3 + 12*k4 + 32*k5 + 7*k6)*h/90
    return yip1
\end{pythoncode}


\begin{soal}
Gunakan metode \textit{midpoint}, Ralston, Runge-Kutta orde-3, -4, dan -5 untuk
menyelesaikan persamaan diferensial yang sama pada Chapra Contoh 25.5.
\end{soal}






\section{Kuadratur Gauss-Legendre}

\begin{soal}[Chapra Contoh 22.3 dan 22.4]
Aplikasikan kuadratur Gauss-Legendre dengan dua dan tiga titik untuk
menghitung integral dari fungsi pada Chapra Contoh 21.1
\end{soal}

Anda dapat melengkapi kode berikut.
\begin{pythoncode}
# .... Definisikan atau import fungsi-fungsi yang diperlukan

a = 0.0
b = 0.8

a_0 = (b + a)/2
a_1 = (b - a)/2

# mapping from x_d -> x
# NOTE: using a_0 and a_1 as global variables
def mapping_func(x_d):
    return a_0 + a_1*x_d
    
dx_dx_d = (b - a)/2 # Eq. (22.30)

NGaussPoints = 2
GAUSS2_c = [1.0, 1.0]
GAUSS2_x = [-1.0/sqrt(3), 1.0/sqrt(3)]
    
I_exact = 1.640533
    
I = 0.0
for i in range(NGaussPoints):
    x_d = GAUSS2_x[i]
    c = GAUSS2_c[i]
    I = .... # LENGKAPI
    
ε_t = (I_exact - I)/I_exact * 100
print("I = %.6f, ε_t = %.1f%%" % (I, ε_t))
\end{pythoncode}

Untuk tiga titik, Anda dapat mengganti daftar titik-titik Gauss sebagai berikut.
\begin{pythoncode}
# Can be found for example at:
# https://en.wikipedia.org/wiki/Gaussian_quadrature#Gauss%E2%80%93Legendre_quadrature
NGaussPoints = 3
GAUSS3_c = [5/9, 8/9, 5/9]
GAUSS3_x = [-sqrt(3/5), 0, sqrt(3/5)]
\end{pythoncode}

Untuk mendapatkan jumlah titik Gauss-Legendre yang lebih lengkap, kita dapat
menggunakan fungsi yang tersedia pada Numpy seperti pada contoh berikut.

\begin{soal}[Chapra Contoh 22.5]
Hitung integral berikut ini dengan menggunakan kuadratur Gauss-Legendre
\begin{equation*}
d = \frac{gm}{c} \int_{0}^{10} (1 - e^{-(c/m)t}) \, \mathrm{d}t
\end{equation*}
Variasikan jumlah titik yang digunakan dan
bandingkan hasil yang diperoleh dengan hasil analitik.
\end{soal}

Anda dapat melengkapi kode berikut.
\begin{pythoncode}
import numpy as np
from math import exp
    
def my_func(t):
    g = 9.8
    m = 68.1
    c = 12.5
    return g*m/c * (1 - exp(-(c/m)*t) )
    
def calc_exact_sympy():
    # Calculate "Exact" result using SymPy
    import sympy
    g = 9.8
    m = 68.1
    c = 12.5
    t = sympy.symbols("t")
    f = g*m/c * (1 - sympy.exp(-(c/m)*t) )
    d_sympy = sympy.integrate( f, (t,0,10))
    return d_sympy
    
I_exact = calc_exact_sympy()
    
a = 0.0
b = 10.0

a_0 = (b + a)/2
a_1 = (b - a)/2

def mapping_func(x_d):
    return a_0 + a_1*x_d

dx_dx_d = (b - a)/2
    
for NGaussPoints in range(2,7):
    GAUSS_x, GAUSS_c = np.polynomial.legendre.leggauss(NGaussPoints)
    I = 0.0
    for i in range(NGaussPoints):
        x_d = GAUSS_x[i]
        c = GAUSS_c[i]
        I = .... # LENGKAPI
    Δ = abs(I_exact-I)
    print("NGaussPoints = %d, I = %15.10f, err = %10.5e" % (NGaussPoints,I,Δ))
\end{pythoncode}
\section{Integrasi Romberg}

\begin{soal}[Chapra Contoh 22.1 dan 22.2]
Tinjau kembali integral dari fungsi pada Chapra Contoh 21.1. Gunakan aturan trapesium
dengan jumlah segmen 1, 2, dan 4. Gunakan ekstrapolasi Richardson untuk
mendapatkan estimasi integral yang lebih akurat.
\end{soal}

Anda dapat melengkapi kode berikut.
\begin{pythoncode}
# Lengkapi definisi fungsi-fungsi yang diperlukan
# ....

a = 0.0
b = 0.8
I_exact = 1.640533

print("Using trapezoidal rule")

I_1 = integ_trapz_multiple(my_func, a, b, 1) # satu segmen
ε_t = (I_exact - I_1)/I_exact * 100
print("I_1  = %.6f ε_t = %2.1f%%" % (I_1, ε_t))

I_2 = integ_trapz_multiple(my_func, a, b, 2) # dua segmen
ε_t = (I_exact - I_2)/I_exact * 100
print("I_2  = %.6f ε_t = %2.1f%%" % (I_2, ε_t))

I_3 = integ_trapz_multiple(my_func, a, b, 3) # 3 segmen
ε_t = (I_exact - I_3)/I_exact * 100
print("I_3  = %.6f ε_t = %2.1f%%" % (I_3, ε_t))

I_4 = integ_trapz_multiple(my_func, a, b, 4) # 4 segmen
ε_t = (I_exact - I_4)/I_exact * 100
print("I_4  = %.6f ε_t = %2.1f%%" % (I_4, ε_t))

print()
print("Using Richardson's extrapolation:")

I_12 = .... # LENGKAPI: kombinasi I_2 dan I_1
ε_t = (I_exact - I_12)/I_exact * 100
print("I_12 = %.6f ε_t = %2.1f%%" % (I_12, ε_t))

I_24 = .... # LENGKAPI: kombinasi I_4 dan I_2
ε_t = (I_exact - I_24)/I_exact * 100
print("I_24 = %.6f ε_t = %2.1f%%" % (I_24, ε_t))

print()
print("Using Richardson's extrapolation (2nd iter):")

I_124 = .... # LENGKAPI: kombinasi I_24 dan I_12
ε_t = (I_exact - I_124)/I_exact * 100
print("I_124 = %.6f ε_t = %2.1f%%" % (I_124, ε_t))
\end{pythoncode}

\begin{soal}
Implementasikan algoritma pada Gambar 22.4 untuk integrasi Romberg.
\end{soal}

Anda dapat melengkapi kode berikut:
\begin{pythoncode}
def integ_romberg(f, a, b, es=1e-10, MAXIT=10):
    I = np.zeros( (MAXIT+2,MAXIT+2) )
    n = 1
    # We start from I[1,1], to follow the book's notation
    I[1,1] = integ_trapz_multiple(f, a, b, n)
    iterConv = 0
    for i in range(1,MAXIT+1):
        n = 2**i
        I[i+1,1] = integ_trapz_multiple(f, a, b, n)
        #
        for k in range(2,i+2):
            j = 2 + i - k
            I[j,k] = ....# LENGKAPI
        #
        ea = abs( (I[1,i+1] - I[2,i])/I[1,i+1] )*100 # in percent
        if ea <= es:
            iterConv = i
            # For debugging, please comment if not needed
            print("Converged)
            break
        # Here we set iterConv to i to make sure that the last value
        # will be returned in case of no convergence.
        iterConv = i
    # to make sure that we are use variable that is defined outside the loop
    # we use iterConv instead of i
    return I[1,iterConv+1]
\end{pythoncode}

Contoh pengujian (membandingkan dengan hasil dari SymPy):
\begin{pythoncode}
from math import cos, pi

import sympy
x = sympy.symbols("x")
func_symb = 6 + 3*sympy.cos(x)
resExact = sympy.N(sympy.integrate(func_symb, (x, 0, sympy.pi/2)))

# Definisikan atau import fungsi-fungsi yang diperlukan
# ...
    
def my_func(x):
    return 6 + 3*cos(x)
    
a = 0.0
b = pi/2
resN = integ_romberg(my_func, a, b)
print("resN = %18.12f" % resN)
print("res  = %18.12f" % resExact)
print("err  = %18.12e" % abs(resExact-resN))    
\end{pythoncode}

Contoh hasil keluaran:
\begin{textcode}
Converged
iterConv =  5
resN =    12.424777960769
res  =    12.424777960769
err  = 0.000000000000e+00    
\end{textcode}


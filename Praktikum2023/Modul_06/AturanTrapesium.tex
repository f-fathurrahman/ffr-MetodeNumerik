\section{Aturan trapesium}

\begin{soal}[Chapra Contoh 21.1]
Gunakan aturan trapesium untuk aproksimasi integral berikut.
\begin{equation*}
f(x) = 0.2 + 25x - 200x^2 + 675x^3 - 900x^4 + 400x^5
\end{equation*}
dari $a = 0$ sampai $b = 0.8$. Bandingkan hasilnya dengan hasil analitik,
yaitu 1.640533. Hitung juga estimasi error dari aproksimasi yang digunakan.
\end{soal}

Anda dapat melengkapi kode berikut ini untuk Chapra Contoh 21.1
\begin{pythoncode}
def my_func(x):
    return 0.2 + 25*x - 200*x**2 + 675*x**3 - 900*x**4 + 400*x**5

a = 0.0
b = 0.8
f_a = my_func(a)
f_b = my_func(b)

I_exact = 1.640533 # from the book
# Alternatively, you can use SymPy to do the integration

# Use trapezoid rule here
I = .... # Lengkapi

E_t = I_exact - I
ε_t = E_t/I_exact * 100 # in percent
print("Integral result = %.6f" % I)
print("True integral   = %.6f" % I_exact)
print("True error      = %.6f" % E_t)
print("ε_t             = %.1f%%" % ε_t)

import sympy
x = sympy.symbols("x")
f = 0.2 + 25*x - 200*x**2 + 675*x**3 - 900*x**4 + 400*x**5
d2f = f.diff(x,2) # calculate 2nd derivative
avg_d2f_xi = sympy.integrate( d2f, (x,a,b) )/(b - a)
E_a = -1/12*avg_d2f_xi*(b - a)**3 # Persamaan 21.6
print("Approx error    = %.6f" % E_a)
\end{pythoncode}

Aturan trapesium (dan aturan integrasi Newton-Cotes lainnya) dapat digunakan
untuk lebih dari satu interval.

\begin{soal}
Gunakan aturan trapesium multi-interval dengan jumlah segmen 2 untuk menghitung
aproksimasi integral pada Soal sebelumnya (Chapra Contoh 21.1).
Coba juga ganti jumlah segmen yang digunakan dan bandingkan hasilnya
dengan hasil analitik.
\end{soal}

Anda dapat melengkapi kode berikut ini. Lihat juga persaman yang digunakan pada
buku Chapra.
\begin{pythoncode}
def my_func(x):
    return 0.2 + 25*x - 200*x**2 + 675*x**3 - 900*x**4 + 400*x**5

# N is number of segments
# Npoints is N + 1
# f is function
def integ_trapz_multiple( f, a, b, N ):
    x0 = a
    xN = b
    h = (b - a)/N
    ss = 0.0
    for i in range(1,N):
        xi = x0 + i*h
        ss = ss + f(xi)
    I = .... # lengkapi
    return I

Nsegments = 2
a = 0.0
b = 0.8
I_exact = 1.640533 # from the book

I = integ_trapz_multiple( my_func, a, b, Nsegments )

print("Nsegments = ", Nsegments)
print("Integral result = %.6f" % I)
E_t = I_exact - I
print("True integral   = %.6f" % I_exact)
print("True error      = %.6f" % E_t)
ε_t = E_t/I_exact * 100
print("ε_t             = %.1f%%" % ε_t)

import sympy
x = sympy.symbols("x")
f = 0.2 + 25*x - 200*x**2 + 675*x**3 - 900*x**4 + 400*x**5
d2f = f.diff(x,2)
#sympy.pprint(d2f)
avg_d2f_xi = sympy.integrate( d2f, (x,a,b) )/(b - a)
#print("avg_d2f_xi = ", avg_d2f_xi)
E_a = -1/12*avg_d2f_xi*(b - a)**3/Nsegments**2
print("Approx error    = %.6f" % E_a)
\end{pythoncode}

\begin{soal}[Chapra Contoh 21.3]
Hitung integral berikut ini dengan menggunakan aturan trapesium:
\begin{equation*}
d = \frac{gm}{c} \int_{0}^{10} (1 - e^{-(c/m)t}) \, \mathrm{d}t
\end{equation*}
Bandingkan hasil yang diperoleh dengan hasil analitik. Lihat buku Chapra untuk nilai-nilai
numerik yang diperlukan atau lihat pada kode Python di bawah.
\end{soal}

Anda dapat melengkapi kode berikut ini.
\begin{pythoncode}
from math import exp

def my_func(t):
    g = 9.8
    m = 68.1
    c = 12.5
    return g*m/c * (1 - exp(-(c/m)*t) )
    
# ... definisi integ_trapz_multiple
# ... LENGKAPI

t = 10.0
a = 0.0
b = t
    
d_exact = 289.43515 # dari buku
    
print("------------------------------------------------------")
print("   N          h          d          E_t         ε_t")
print("------------------------------------------------------")
for Nsegments in [10, 20, 50, 100, 200, 500, 1000, 2000, 5000, 5000, 10000]:
    h = (b - a)/Nsegments
    d = integ_trapz_multiple( my_func, a, b, Nsegments )
    E_t = d_exact - d
    ε_t = E_t/d_exact * 100
    print("%5d  %10.4f  %10.4f   %10.4e %10.2e%%" % (Nsegments, h, d, E_t, ε_t))


# Calculate "Exact" result using SymPy
import sympy
g = 9.8
m = 68.1
c = 12.5
t = sympy.symbols("t")
f = g*m/c * (1 - sympy.exp(-(c/m)*t) )
d_sympy = sympy.integrate( f, (t,0,10))
    
print()
print("SymPy result:")
print("d_sympy = ", d_sympy)    
\end{pythoncode}

Apakah hasil yang Anda peroleh sama dengan yang diberikan pada Chapra?
Jika ada perbedaan, apakah yang menyebabkan perbedaan tersebut?
\subsection{Aproksimasi dengan deret Taylor}

\textbf{Chapra Contoh 3.2}

Hubungan antara nilai eksak atau benar dan aproksimasi dapat dinyatakan
sebagai:
\begin{equation*}
\text{nilai benar} = \text{aproksimasi} + \text{error atau galat}
\end{equation*}
Dari persamaan di atas, diberikan nilai benar dan aproksimasi, kita dapat menghitung
kesalahan sebenarnya, dilambangkan dengan $E_t$ (subskrip $t$: \textit{true}):
\begin{equation}
E_t = \text{nilai benar} - \text{aproksimasi}
\end{equation}
Terkadang, kita juga dapat menggunakan nilai relatif, dengan cara membandingkannya
dengan nilai benar:
\begin{equation}
\epsilon_{t} = \frac{\text{error sebenarnya}}{\text{nilai sebenarnya}}\times 100\%
\end{equation}
yang dapat dinyatakan dalam persentase.

Dalam banyak kasus, kita tidak memiliki nilai sebenarnya sehingga $\epsilon_t$
tidak dapat dihitung. Sebagai alternatif, kita dapat menggunakan error relatif:
\begin{equation}
\epsilon_{a} = \frac{\text{error aproksimasi}}{\text{nilai aproksimasi}}
\end{equation}
Ada beberapa cara untuk mengestimasi error aproksimasi. Dalam kasus proses iteratif,
kita dapat menggunakan persamaan berikut.
\begin{equation}
\epsilon_{a} = \frac{\text{aproksimasi sekarang} - \text{aproksimasi sebelumnya}}%
{\text{aproksimasi sekarang}} \times 100\%
\end{equation}
Dalam konteks perhitungan iteratif, biasanya perhitungan dilakukan sampai:
\begin{equation}
|\epsilon_{a}| < \epsilon_{s}
\end{equation}
di mana $\epsilon_{s}$ adalah suatu nilai toleransi yang sudah ditentukan.

Scarborough memberikan suatu kriteria yang menghubungkan antara jumlah digit signifikan
dengan dan nilai aproksimasi. Menurut kriteria ini, jika menggunakan:
\begin{equation}
\epsilon_{s} = (0.5 \times 10^{2-n})\%
\end{equation}
maka hasil yang diperoleh akan benar untuk setidaknya $n$ digit signifikan.

Fungsi eksponensial dapat dihitung dengan menggunakan deret sebagai berikut:
\begin{equation}
e^{x} = 1 + x + \frac{x^2}{2} + \frac{x^3}{3!} + \cdots + \frac{x^n}{n!}
\label{eq:exp-deret}
\end{equation}
Kita ingin menggunakan Persamaan \eqref{eq:exp-deret} untuk menghitung estimasi
dari $e^{0.5}$ untuk setidaknya tiga digit signifikan.
Dengan menggunakan kriteria dari Scarborough, nilai $n=3$ diperoleh:
\begin{equation*}
\epsilon_{s} = (0.5 \times 10^{2-3}) \% = 0.05 \%
\end{equation*}
Kita akan menambahkan suku-suku pada Persamaan \eqref{eq:exp-deret} sampai $\epsilon_{a}$
lebih kecil dari $\epsilon_{s}$.

Program Python berikut ini dapat digunakan untuk melakukan perhitungan yang ada pada buku teks.
\begin{pythoncode}
from math import factorial, exp

def approx_exp(x, N):
    assert(N >= 0)
    if N == 0:
        return 1
    s = 0.0
    for i in range(N+1):
        s = s + x**i/factorial(i)
    return s
  
x = 0.5
true_val = exp(x) # from math module
  
n_digit = 3
# Equation 3.7
ε_s_percent = 0.5*10**(2-n_digit)
  
prev_approx = 0.0
for N in range(50):
    approx_val = approx_exp(x, N)
    ε_t_percent = abs(approx_val - true_val)/true_val * 100
    if N > 0:
        ε_a_percent = abs(approx_val - prev_approx)/approx_val * 100
    else:
        ε_a_percent = float('nan')
    prev_approx = approx_val
    print("%3d %18.10f %10.5f%% %10.5f%%" % (N+1, approx_val, ε_t_percent, ε_a_percent))
    if ε_a_percent < ε_s_percent:
        print("Converged within %d significant digits" % n_digit)
        break

print("true_val   is %18.10f" % true_val)
print("approx_val is %18.10f" % approx_val)
\end{pythoncode}

Catatan: Pada program di atas, \pyinline{for}-loop digunakan dengan jumlah iterasi yang relatif
besar. Anda dapat menggunakan \pyinline{while}-loop sebagai gantinya.

Contoh output:
\begin{textcode}
  1       1.0000000000   39.34693%        nan%
  2       1.5000000000    9.02040%   33.33333%
  3       1.6250000000    1.43877%    7.69231%
  4       1.6458333333    0.17516%    1.26582%
  5       1.6484375000    0.01721%    0.15798%
  6       1.6486979167    0.00142%    0.01580%
Converged within 3 significant digits
true_val   is       1.6487212707
approx_val is       1.6486979167
\end{textcode}


\begin{soal}
Ulangi perhitungan ini untuk jumlah digit signifikan yang berbeda, misalnya 5, 8, dan 10
digit signifikan. Silakan melakukan modifikasi terhadap program yang diberikan.
\end{soal}
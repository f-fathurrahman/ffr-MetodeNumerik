\section{Interpolasi Spline Kubik (OPSIONAL)}

Pada spline kubik, polinomial orde-3 diberikan untuk tiap interval di antara dua titik simpul.
Polinomial orde-3 ini dapat dituliskan sebagai:
$$
f_i(x) = a_i x^3 + b_i x^2 + c_i x + d_i
$$
Untuk, $n+1$ titik data, dengan $i=0,1,2,\ldots,n$, terdapat $n$ interval sehingga ada $4n$ konstanta yang harus dievaluasi. Terdapat $4n$ kondisi yang diperlukan untuk mendapatkan konstanta-konstanta tersebut. Kondisi ini dapat dituliskan sebagai berikut.
\begin{itemize}
\item Nilai fungsi harus sama pada titik simpul dalam (interior): ($2n-2$ kondisi)
\item Fungsi pertama dan terakhir harus melalui titik-titik ujung: ($2$ kondisi)
\item Turunan pertama pada titik simpul dalam harus sama: ($n-1$ kondisi)
\item Turunan kedua pada titik simpul dalam harus sama: ($n-1$ kondisi)
\item Turunan kedua pada titik-titik ujung bernilai 0: ($2$ kondisi)
\end{itemize}
Kondisi 5 mensyaratkan bahwa fungsi menjadi garis lurus pada titik-titik ujung. Spesifikasi ini mendefinisikan spline kubik natural atau alami.

Persamaan polinomial kubik untuk tiap interval (silakan Box 18.3 untuk penurunan rumus):
\begin{multline}
f_{i}(x) =
\frac{f''_{i}(x_{i-1})}{6(x_{i} - x_{i-1})}(x_{i} - x)^3 +
\frac{f''_{i}(x_{i})}{6(x_{i} - x_{i-1})}(x - x_{i-1})^3
 + \left[
\frac{f(x_{i-1})}{x_{i} - x_{i-1}} - \frac{f''(x_{i-1}) (x_i - x_{i-1})}{6}
\right] (x_i - x) \\
 + \left[
\frac{f(x_{i})}{x_{i} - x_{i-1}} - \frac{f''(x_{i}) (x_i - x_{i-1})}{6}
\right] (x - x_{i-1})
\label{eq:cubicspline}
\end{multline}
Persamaan ini hanya memiliki dua variabel yang tidak diketahui, yaitu turunan kedua
pada tiap ujung-ujung interval. Variabel-variabel tersebut dapat dihitung dengan menyelesaikan
persamaan berikut:
\begin{multline}
(x_i - x_{i-1}) f''(x_{i-1}) + 2(x_{i+1} - x_{i-1}) f''(x_{i}) + (x_{i+1} - x_{i})f''(x_{i+1}) = \\
\frac{6}{x_{i+1} - x_{i}} \left[ f(x_{i+1}) - f(x_i) \right] +
\frac{6}{x_{i} - x_{i+1}} \left[ f(x_{i+1}) - f(x_i) \right]
\label{eq:tridiagsys}
\end{multline}

\begin{soal}
Implementasikan fungsi atau subroutin dalam Python untuk implementasi algoritma
pada Gambar 18.18 untuk implementasi spline kubik (natural).
Uji hasil yang Anda dapatkan dengan menggunakan data-data yang diberikan pada contoh
18.10.
\end{soal}

Anda dapat melengkapi kode berikut ini sebagai referensi.
Untuk mengevaluasi polinomial spline pada setiap interval diperlukan data $f''(x_{i})$, sesuai
Persamaan \eqref{eq:cubicspline}.
Untuk mengevaluasi $f''(x_{i})$ kita perlu menyelesaikan sistem persamaan linear pada
Persamaan \eqref{eq:tridiagsys}. Sistem persamaan linear ini memiliki struktur tridiagonal, dan
dapat diselesaikan dengan menggunakan algoritma khusus. Pada kode yang akan diberikan algoritma
ini sudah diimplementasikan sebagai fungsi \pyinline{decomp_trid} dan \pyinline{subs_trid}.
Koefisien-koefisien yang diperlukan untuk membangun matriks tridiagonal diimplementasikan
dalam fungsi \pyinline{gen_trid_matrix}. Lihat kode contoh penggunaan (Chapra Contoh 18.10)
di bawah.

Berikut ini adalah fungsi untuk interpolasi kubik spline natural. Fungsi ini memerlukan
input sebagai berikut:
\begin{itemize}
\item Data atau fungsi diskrit yang akan diinterpolasi: array \pyinline{x} dan \pyinline{y}
\item Turunan kedua fungsi: array \pyinline{d2x} yang dihitung sebagai solusi dari sistem
persamaan linear (Persamaan \eqref{eq:tridiagsys}).
\item Titik di mana nilai fungsi interpolasi akan dicari
\end{itemize}
Selain nilai fungsi interpolasi pada titik yang dicari, algoritma pada Gambar 18.18 (di buku Chapra)
juga akan menghitung turunan pertama dan kedua dari fungsi spline hasil interpolasi.

\begin{pythoncode}
import numpy as np

def interp_nat_cubic_spline( x, y, d2x, xu ):    
    assert len(x) == len(y)
    N = len(x) - 1
    success = False
    i = 1
    #
    while True:
        is_in_interval = (x[i-1] <= xu <= x[i])    
        if is_in_interval:
            c1 = d2x[i-1]/6.0/( x[i] - x[i-1] )
            c2 = ....
            c3 = ....
            c4 = ....
            #
            t1 = c1*( x[i] - xu )**3
            t2 = ....
            t3 = ....
            t4 = ....
            yu = t1 + t2 + t3 + t4
            # Turunan pertama
            t1 = -3.0*c1*( x[i] - xu )**2
            t2 = 3.0*c2*( xu - x[i-1] )**2
            t3 = -c3
            t4 = c4
            dy = t1 + t2 + t3 + t4
            # Turunan kedua
            t1 = 6.0*c1*( x[i] - xu )
            t2 = 6.0*c2*( xu - x[i-1] )
            d2y = t1 + t2
            #
            success = True
            #
        else:
            #
            i = i + 1
        
        if i == (N + 1) or success:
            break # break out of the loop
    
    if not success:
        raise RuntimeError("xu is outside range of spline")
        
    return yu, dy, d2y
\end{pythoncode}

Fungsi berikut ini akan membuat sistem persamaan linear yang akan diselesaikan: matriks tridiagonal
(array \pyinline{e,f,g})
dan vektor pada ruas kanan (array \pyinline{r}).
\begin{pythoncode}
def gen_trid_matrix(x, y):
    assert len(x) == len(y)
    N = len(x) - 1
    #
    e = np.zeros(N-1)
    f = np.zeros(N-1)
    g = np.zeros(N-1)
    r = np.zeros(N-1)
    #
    f[0] = 2.0*( x[2] - x[0] )
    g[0] = x[2] - x[1]
    r[0] = 6.0/( x[2] - x[1] ) * ( y[2] - y[1] )
    r[0] = r[0] + 6.0/( x[1] - x[0] ) * ( y[0] - y[1] )
    #
    for i in range(2,N-1):
        e[i-1] = ....
        f[i-1] = ....
        g[i-1] = ....
        r[i-1] = ....
        r[i-1] = ....
    #    
    e[N-2] = x[N-1] - x[N-2]
    f[N-2] = 2.0*( x[N] - x[N-2] )
    r[N-2] = 6.0/( x[N] - x[N-1] ) * ( y[N] - y[N-1] )
    r[N-2] = r[N-2] + 6.0/( x[N-1] - x[N-2] ) * ( y[N-2] - y[N-1] )
    
    return e, f, g, r    
\end{pythoncode}


Berikut ini adalah fungsi-fungsi untuk menyelesaikan sistem persamaan linear dengan
struktur matriks tridiagonal:
\begin{pythoncode}
def decomp_trid(e, f, g):
    N = len(e)
    assert N == len(f)
    assert N == len(g)
    #
    for k in range(1,N):
        e[k] = e[k]/f[k-1]
        f[k] = f[k] - e[k]*g[k-1]
    return
    
# should be called after calling decomp_trid
def subs_trid(e, f, g, r):
    N = len(e)
    assert N == len(f)
    assert N == len(g)
    # Forward subs
    for k in range(1,N):
        r[k] = r[k] - e[k]*r[k-1]
    # back subs
    x = np.zeros(N)
    x[N-1] = r[N-1]/f[N-1]
    for k in range(N-2,-1,-1):
        x[k] = ( r[k] - g[k]*x[k+1] ) / f[k]
    return x    
\end{pythoncode}


Contoh kode untuk pengujian (Chapra Contoh 18.10):
\begin{pythoncode}
import numpy as np
# import or define the required functions here
    
x = np.array([3.0, 4.5, 7.0, 9.0])
y = np.array([2.5, 1.0, 2.5, 0.5])

e, f, g, r = gen_trid_matrix(x, y)
print("e = ", e) # subdiagonal
print("f = ", f) # main diagonal
print("g = ", g) # subdiagonal
print("r = ", r) # rhs vector

Npts_known = len(x)
d2x = np.zeros(len(x))
# Natural spline conditions
d2x[0] = 0.0
d2x[Npts_known-1] = 0.0 # last element

# Solve tridiagonal system (decomposition and back substitution)
decomp_trid(e, f, g)
d2x[1:Npts_known-1] = subs_trid( e, f, g, r )
print("d2x = ", d2x) # compare with Chapra's

xu = 5
# We do not use the derivatives, we use underscores to represent them
yu, _, _ = interp_nat_cubic_spline( x, y, d2x, xu )
print("yu = ", yu) # compare with Chapra's

# Make a plot
NptsPlot = 50
xgrid = np.linspace(3.0, 9.0, NptsPlot)
ygrid = np.zeros(NptsPlot)
for i in range(NptsPlot):
    ygrid[i], _, _ = interp_nat_cubic_spline( x, y, d2x, xgrid[i] )

import matplotlib.pyplot as plt
plt.scatter(x, y, label="data")
plt.plot(xgrid, ygrid, label="interpolated")
plt.grid(True)
plt.legend()
plt.tight_layout()
\end{pythoncode}

Contoh hasil plot:

{\centering
\includegraphics[width=0.7\textwidth]{../../chapra_7th/ch18/IMG_chapra_example_18_10.pdf}
\par}

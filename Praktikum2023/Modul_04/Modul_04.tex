\documentclass[a4paper,11pt,bahasa]{extarticle} % screen setting
\usepackage[a4paper]{geometry}

%\documentclass[b5paper,11pt,bahasa]{article} % screen setting
%\usepackage[b5paper]{geometry}

%\geometry{verbose,tmargin=1.5cm,bmargin=1.5cm,lmargin=1.5cm,rmargin=1.5cm}

\geometry{verbose,tmargin=2.0cm,bmargin=2.0cm,lmargin=2.0cm,rmargin=2.0cm}

\setlength{\parskip}{\smallskipamount}
\setlength{\parindent}{0pt}

%\usepackage{cmbright}
%\renewcommand{\familydefault}{\sfdefault}

\usepackage{amsmath}
\usepackage{amssymb}

\usepackage[libertine]{newtxmath}

\usepackage[no-math]{fontspec}
\setmainfont{Linux Libertine O}

%\usepackage{fontspec}
%\usepackage{lmodern}

\setmonofont{JuliaMono-Regular}


\usepackage{hyperref}
\usepackage{url}
\usepackage{xcolor}
\usepackage{enumitem}
\usepackage{mhchem}
\usepackage{graphicx}
\usepackage{float}

\usepackage{minted}

\newminted{julia}{breaklines,fontsize=\footnotesize}
\newminted{python}{breaklines,fontsize=\footnotesize}

\newminted{bash}{breaklines,fontsize=\footnotesize}
\newminted{text}{breaklines,fontsize=\footnotesize}

\newcommand{\txtinline}[1]{\mintinline[breaklines,fontsize=\footnotesize]{text}{#1}}
\newcommand{\jlinline}[1]{\mintinline[breaklines,fontsize=\footnotesize]{julia}{#1}}
\newcommand{\pyinline}[1]{\mintinline[breaklines,fontsize=\footnotesize]{python}{#1}}

\newmintedfile[juliafile]{julia}{breaklines,fontsize=\footnotesize}
\newmintedfile[pythonfile]{python}{breaklines,fontsize=\footnotesize}
\newmintedfile[fortranfile]{fortran}{breaklines,fontsize=\footnotesize}

\usepackage{mdframed}
\usepackage{setspace}
\onehalfspacing

\usepackage{babel}
\usepackage{appendix}

\newcommand{\highlighteq}[1]{\colorbox{blue!25}{$\displaystyle#1$}}
\newcommand{\highlight}[1]{\colorbox{red!25}{#1}}

\newcounter{soal}%[section]
\newenvironment{soal}[1][]{\refstepcounter{soal}\par\medskip
   \noindent \textbf{Soal~\thesoal. #1} \sffamily}{\medskip}


\definecolor{mintedbg}{rgb}{0.95,0.95,0.95}
\BeforeBeginEnvironment{minted}{
    \begin{mdframed}[%
        topline=false,bottomline=false,%
        leftline=false,rightline=false]
}
\AfterEndEnvironment{minted}{\end{mdframed}}


\BeforeBeginEnvironment{soal}{
    \begin{mdframed}[%
        topline=true,bottomline=false,%
        leftline=true,rightline=false]
}
\AfterEndEnvironment{soal}{\end{mdframed}}


% -------------------------
\begin{document}

\title{%
{\small TF2202 Komputasi Rekayasa}\\
Optimisasi
}
\author{Tim Praktikum Komputasi Rekayasa 2023\\
Teknik Fisika\\
Institut Teknologi Bandung}
\date{}
\maketitle

\textbf{Catatan}

Optimisasi dapat merujuk pada pencarian nilai minimum atau nilai maksimum
dari sebuah fungsi objektif.
Pada modul ini, kita akan memilih untuk mengimplementasikan
pencarian nilai minimum (bentuk standard).
Algoritma atau subrutin yang sudah diimplementasikan
untuk minimisasi suatu fungsi objektif dapat digunakan untuk mencari nilai maksimum
dengan cara mengubah tanda dari fungsi objektif, misalnya dari positif menjadi negatif,
atau sebaliknya.


\section{Optimisasi Variabel Tunggal}
% subsections
%\section{Metode bagi-dua}

Selain digunakan untuk mencari akar persamaan nonlinear, metode bagi-dua
juga dapat digunakan untuk mencari nilai optimum dari suatu fungsi.

\section{Metode rasio emas (\textit{golden section})}

Berikut ini adalah implementasi dari metode rasio emas,
diadaptasi dari Chapra 7th (Gambar 13.5) untuk mencari
\textbf{nilai minimum} dari suatu fungsi.

\begin{pythoncode}
import numpy as np

# This will search for minimum value
# If you want to use this for searching maximum value please
# define f = -f_actual, where f_actual is the function you want
# to maximize.
# The sign keyword argument is not implemented here.

def optim_golden_ratio(f, xlow, xhigh, NiterMax=100, TOL=1e-10, verbose=True):

    SMALL = np.finfo(np.float64).resolution

    ϕ = .... # lengkapi
    xl = xlow
    xu = xhigh
    iiter = 1
    d = ϕ*(xu - xl)
    x1 = xl + d
    x2 = xu - d
    f1 = f(x1)
    f2 = f(x2)

    if verbose:
        print("Iteration: ", 0)
        print()
        print("xl = %18.10f xu = %18.10f" % (xl, xu))
        print("-"*47)
        print("x2 = %18.10f x1 = %18.10f" % (x2, x1))
        print("f2 = %18.10f f1 = %18.10f" % (f2, f1))

    if f1 < f2:
        xopt = x1
        fxopt = f1
        if verbose:
            print("-"*47)
            print("f1 is the current optimum value")
            print("(xopt,fxopt) = (%18.10f,%18.10f)" % (xopt, fxopt))
    else:
        xopt = x2
        fxopt = f2
        if verbose:
            print("-"*47)
            print("f2 is the current optimum value")
            print("(xopt,fxopt) = (%18.10f,%18.10f)" % (xopt, fxopt))

    for iiter in range(1,NiterMax+1):

        d = ϕ*d
        xint = xu - xl

        if f1 < f2:
            xl = x2
            x2 = x1
            x1 = .... # lengkapi
            f2 = f1
            f1 = f(x1)
            if verbose:
                print("")
                print("For next iteration: ")
                print("- Replacing xl with x2")
                print("- New point: x1 = %18.10f" % x1)
                print("- xu is not changed")
                print("- Replacing x2 with x1")
        else:
            xu = x1
            x1 = x2
            x2 = .... # lengkapi
            f1 = f2
            f2 = f(x2)
            if verbose:
                print("")
                print("For next iteration: ")
                print("- xl is not changed")
                print("- x1 is replaced by x2")
                print("- New point: x2 = %18.10f" % x2)
                print("- xu is replaced by x1")

        if verbose:
            print("")
            print("Iteration: ", iiter)
            print("xl = %18.10f xu = %18.10f" % (xl, xu))
            print("-"*47)
            print("x2 = %18.10f x1 = %18.10f" % (x2, x1))
            print("f2 = %18.10f f1 = %18.10f" % (f2, f1))

        #
        if f1 < f2:
            xopt = x1
            fxopt = f1
            if verbose:
                print("-"*47)
                print("f1 is the current optimum value")
                print("(xopt,fxopt) = (%18.10f,%18.10f)" % (xopt, fxopt))
        else:
            xopt = x2
            fxopt = f2
            if verbose:
                print("-"*47)
                print("f2 is the current optimum value")
                print("(xopt,fxopt) = (%18.10f,%18.10f)" % (xopt, fxopt))

        if abs(xopt) > SMALL:
            ea = (1 - ϕ)*abs(xint/xopt)
        else:
            # The above might fail if xopt is very close to zero
            # We set xint as the convergence criteria
            ea = xint

        if verbose:
            print("Interval length = %18.10e" % xint)
            print("ea              = %18.10e" % ea)

        if ea <= TOL:
            if verbose:
                print("Converged")
            break

    return xopt, fxopt
\end{pythoncode}

Contoh penggunaan pada kasus Chapra Contoh 13.1, di mana kita ingin mencari
nilai maksimum dari fungsi:
\begin{equation*}
f(x) = 2\sin(x) - \frac{x}{10}
\end{equation*}
pada selang $x \in [0,4]$.
Pada program berikut, selain mendefinisikan fungsi \pyinline{my_func},
kita juga mendefinisikan fungsi \pyinline{m_my_func} yang mengembalikan
$-f(x)$ karena pada fungsi \pyinline{optim_golden_ratio} akan
mencari nilai minumum. Ingat bahwa nilai maksimum dari $f(x)$ sama
dengan nilai minimum dari $-f(x)$.

\begin{pythoncode}
import numpy as np

# definisikan atau import modul yang berisi optim_golden_ratio

def my_func(x):
    return 2*np.sin(x) - x**2/10

def m_my_func(x):
    return -my_func(x)

xopt, fx = optim_golden_ratio(m_my_func, 0.0, 4.0, TOL=1e-10, verbose=True)

print("\nOptimization result")
print("xopt    = %18.10f" % xopt)
print("f(xopt) = %18.10f" % my_func(xopt))
# gunakan fungsi asal, yaitu my_func
\end{pythoncode}


Contoh keluaran:
\begin{textcode}
Iteration:  0

xl =       0.0000000000 xu =       4.0000000000
-----------------------------------------------
x2 =       1.5278640450 x1 =       2.4721359550
f2 =      -1.7647202483 f1 =      -0.6299744700
-----------------------------------------------
f2 is the current optimum value
(xopt,fxopt) = (      1.5278640450,     -1.7647202483)

For next iteration: 
- xl is not changed
- x1 is replaced by x2
- New point: x2 =       0.9442719100
- xu is replaced by x1

Iteration:  1
xl =       0.0000000000 xu =       2.4721359550
-----------------------------------------------
x2 =       0.9442719100 x1 =       1.5278640450
f2 =      -1.5309755469 f1 =      -1.7647202483
-----------------------------------------------
f1 is the current optimum value
(xopt,fxopt) = (      1.5278640450,     -1.7647202483)
Interval length =   4.0000000000e+00
ea              =   1.0000000000e+00

For next iteration: 
- Replacing xl with x2
- New point: x1 =       1.8885438200
- xu is not changed
- Replacing x2 with x1

Iteration:  2
xl =       0.9442719100 xu =       2.4721359550
-----------------------------------------------
x2 =       1.5278640450 x1 =       1.8885438200
f2 =      -1.7647202483 f1 =      -1.5432233694
-----------------------------------------------
f2 is the current optimum value
(xopt,fxopt) = (      1.5278640450,     -1.7647202483)
Interval length =   2.4721359550e+00
ea              =   6.1803398875e-01

For next iteration: 
- xl is not changed
- x1 is replaced by x2
- New point: x2 =       1.3049516850
- xu is replaced by x1

...... # dan seterusnya

Iteration:  49
xl =       1.4275517651 xu =       1.4275517653
-----------------------------------------------
x2 =       1.4275517652 x1 =       1.4275517652
f2 =      -1.7757256531 f1 =      -1.7757256531
-----------------------------------------------
f2 is the current optimum value
(xopt,fxopt) = (      1.4275517652,     -1.7757256531)
Interval length =   3.7209457737e-10
ea              =   9.9560299663e-11
Converged

Optimization result
xopt    =       1.4275517652
f(xopt) =       1.7757256531
\end{textcode}




\subsection{Interpolasi parabola}

Polinomial orde dua atau parabola seringkali dapat digunakan sebagai pendekatan
dari $f(x)$ di sekitar minimum, misalnya untuk mengaproksimasi fungsi potensial Morse
antara dua atom.
Pada metode interpolasi parabola, dipilih tiga titik $x_0, x_1, x_2$
yang kemudian diinterpolasi menjadi
fungsi polinomial orde dua. Nilai optimum dari $f(x)$ kemudian diaproksimasi sebagai
optimum dari parabola yang dihasilkan. Nilai optimum diperoleh dari
\begin{equation}
x_3 = \frac{f(x_0)(x_{1}^2 - x_{2}^2) + f(x_1)(x_{2}^2 - x_{0}^2) + f(x_2)(x_{0}^2 - x_{1}^2)}%
{2f(x_0)(x_{1} - x_{2}) + 2f(x_1)(x_{2} - x_{0}) + 2f(x_2)(x_{0} - x_{1})}
\end{equation}
Untuk iterasi selanjutnya dapat digunakan strategi yang sama dengan pemilihan titik pada
metode rasio emas. Alternatif lain yang akan kita gunakan di sini adalah
penggunaan titik secara sekuensial (lebih mudah untuk diimplementasikan):
$x_0 \leftarrow x_1$, $x_1 \leftarrow x_2$, dan $x_2 \leftarrow x_3$.

\begin{soal}
Program Python berikut ini mengimplementasikan metode interpolasi parabola untuk
mencari maksimum dari $f(x) = 2\sin(x) - \frac{x^2}{10}$. Lengkapi kode yang tersebut.
\end{soal}

Catatan: kode berikut ini belum dibuat dalam bentuk fungsi seperti pada kasus
\pyinline{optim_golden_ratio}, Anda dapat mengubahnya jika diperlukan.
Kriteria konvergensi yang digunakan adalah ketika nilai optimum sudah tidak berubah
berdasarkan suatu nilai tertentu.

\begin{pythoncode}
import numpy as np

def my_func(x):
    return 2*np.sin(x) - x**2/10

# f0 = f(x0), f1 = f(x1), f2 = f(x2)
def calc_parabolic_x3(x0, f0, x1, f1, x2, f2):
    num = ....  # lengkapi
    denum = .... # lengkapi
    return num/denum

# Initial guess
x0 = 0.0; f0 = my_func(x0)
x1 = 1.0; f1 = my_func(x1)
x2 = 4.0; f2 = my_func(x2)

x3 = calc_parabolic_x3(x0, f0, x1, f1, x2, f2)
f3 = my_func(x3)
print("x3 = %18.10f f3 = %18.10f" % (x3, f3))

TOL = 1e-10
NiterMax = 100

for iiter in range(1,NiterMax+1):
    xopt_old = x3
    fopt_old = f3

    # Sequentially choose the next points
    x0 = x1; f0 = f1
    x1 = x2; f1 = f2
    x2 = x3; f2 = f3

    x3 = calc_parabolic_x3(x0, f0, x1, f1, x2, f2)
    f3 = my_func(x3)
    print("x3 = %18.10f f3 = %18.10f" % (x3, f3))

    if abs(fopt_old - f3) < TOL:
        print("Converged")
        break
\end{pythoncode}

Contoh keluaran:
\begin{textcode}
x3 =       1.5055348740 f3 =       1.7690789285
x3 =       1.4902527509 f3 =       1.7714309125
x3 =       1.3908075360 f3 =       1.7742568388
x3 =       1.4275400017 f3 =       1.7757256530
x3 =       1.4275037854 f3 =       1.7757256506
x3 =       1.4275518296 f3 =       1.7757256531
x3 =       1.4275526174 f3 =       1.7757256531
Converged
\end{textcode}
\subsection{Metode Newton}

Metode Newton untuk pencarian akar dapat digunakan pada masalah optimisasi.
Pada kasus ini metode Newton digunakan untuk mencari akar dari turunan
fungsi. Skema iterasi yang digunakan adalah
\begin{equation}
x_{i+1} = x_{i} - \frac{f'(x_i)}{f''(x_i)}
\end{equation}

Perhatikan bahwa metode Newton memerlukan informasi turunan pertama dan kedua
dari fungsi yang ingin dicari nilai optimumnya.

\begin{soal}
Lengkapi kode berikut untuk mengimplementasikan metode Newton
pada fungsi $f(x) = 2\sin(x) - \dfrac{x^2}{10}$.
\end{soal}

\begin{pythoncode}
import numpy as np

def my_func(x):
    return 2*np.sin(x) - x**2/10

def d1_my_func(x): # turunan pertama
    return .... # lengkapi

def d2_my_func(x): # turunan kedua
    return .... # lengkapi

SMALL = np.finfo(np.float64).resolution # atau nilai yang cukup kecil
NiterMax = 100

# Initial guess
x0 = 2.5
fopt_old = np.nan

for iiter in range(1,NiterMax+1):
    f0 = my_func(x0)
    df0 = d1_my_func(x0)
    d2f0 = d2_my_func(x0)
    if abs(df0) > SMALL:
        x1 = x0 - df0/d2f0
        f1 = my_func(x1)
        print("%18.10f %18.10f %18.10e" % (x1, f1, abs(f1 - fopt_old)))
    else:
        print("Converged")
        break
    x0 = x1
    fopt_old = f1
\end{pythoncode}

Contoh keluaran:
\begin{textcode}
      0.9950815513       1.5785880072                nan
      1.4690107528       1.7738493793   1.9526137207e-01
      1.4276423210       1.7757256442   1.8762648994e-03
      1.4275517793       1.7757256531   8.9337626186e-09
      1.4275517788       1.7757256531   0.0000000000e+00
Converged
\end{textcode}
\subsection{Metode Brent}

\begin{soal}
Berikut ini adalah implementasi dari pseudocode yang diberikan pada
Gambar 13.7 pada Chapra.
Lengkapi kode Python yang diberikan dan uji pada fungsi
$f(x) = 2\sin(x) - \dfrac{x^2}{10}$ (untuk mencari nilai maksimum).
\end{soal}

\begin{pythoncode}
import numpy as np

def optim_brent(f, xl, xu, TOL=1e-10, NiterMax=100, verbose=True):
    
    SMALL = np.finfo(np.float64).resolution

    ϕ = .... # lengkapi
    ρ = 2.0 - ϕ
    u = .... # lengkapi
    v = u; w = u; x = u
    fu = f(u)
    fv = fu; fw = fu; fx = fu
    xm = .... # lengkapi
    d = 0.0
    e = 0.0

    iiter = 0
    while True:
        iiter = iiter + 1
        if verbose:
            print("\nBegin iter = ", iiter)
            print("x      = %18.10f" % x)
            print("xm     = %18.10f" % xm)
            print("fu     = %18.10f" % fu)
            print("x - xm = %18.10e" % abs(x - xm))

        if abs(x - xm) <= TOL:
            if verbose:
                print("Converged")
            break

        if iiter >= NiterMax:
            if verbose:
                print("WARNING: Maximum iterations reached")
            break

        para = abs(e) > TOL
        
        if para:
            # Try using parabolic interp
            r = (x - w)*(fx - fv)
            q = (x - v)*(fx - fw)
            p = (x - v)*q - (x - w)*r
            s = 2*(q - r)
            if s > 0:
                p = -q
            s = abs(s)
            #
            cond1 = abs(p) < abs(0.5*s*e)
            cond2 = p > s*(xl - x)
            cond3 = p < s*(xu - x)
            para = cond1 and cond2 and cond3
            # Parabolic interpolation step
            if para:
                if verbose:
                    print("Parabolic interpolation is used")
                e = d
                d = p/s

        if not para:
            if verbose:
                print("Using golden section")
            if x >= xm:
                e = xl - x
            else:
                e = xu - x
            d = ρ*e

        u = x + d
        fu = f(u)

        if fu <= fx:
            if u >= x:
                xl = x
            else:
                xu = x
            v = w; fv = fw
            w = x; fw = fx
            x = u; fx = fu
        else:
            #
            if u < x:
                xl = u
            else:
                xu = u
            #
            if (fu <= fw) or (abs(w - x) <= SMALL):
                v = w; fv = fw
                w = u; fw = fu
            elif (fu <= fv) or (abs(v - x) <= SMALL) or (abs(v - w) <= SMALL):
                v = u; fv = fu
        #
        xm = 0.5*(xl + xu)

    return xu, fu
\end{pythoncode}

Contoh pemanggilan fungsi:
\begin{pythoncode}
import numpy as np

# definisi atau import optim_brent di sini

def my_func(x):
    return 2*np.sin(x) - x**2/10

def m_my_func(x):
    return -my_func(x)

xopt, fxopt = optim_brent(m_my_func, 0.0, 4.0, TOL=1e-10)
print()
print("xopt    = %18.10f" % xopt)
print("f(xopt) = %18.10f" % my_func(xopt))
\end{pythoncode}

Contoh hasil keluaran (\pyinline{verbose=True}):
\begin{textcode}
Begin iter =  1
x      =       1.5278640450
xm     =       2.0000000000
fu     =      -1.7647202483
x - xm =   4.7213595500e-01
Using golden section

Begin iter =  2
x      =       1.5278640450
xm     =       1.2360679775
fu     =      -0.6299744700
x - xm =   2.9179606750e-01
Using golden section

.... # dan seterusnya

Begin iter =  30
x      =       1.4275517790
xm     =       1.4275517790
fu     =      -1.7757256531
x - xm =   5.9506621852e-11
Converged

xopt    =       1.4275517793
f(xopt) =       1.7757256531
\end{textcode}


\section{Optimisasi Variabel Banyak}
Pada bagian ini kita akan mengeksplorasi beberapa metode yang dapat
digunakan untuk optimisasi banyak variabel (multivariabel). Untuk
memudahkan visualisasi, kita akan lebih fokus pada masalah dua variabel
(atau dua dimensi), sehingga fungsi objektif lebih mudah untuk divisualisasi.


% New section
\section{Pemrograman Linear}

Pada bagian ini, kita akan menggunakan pustaka SciPy untuk menyelesaikan permasalahan
pemrograman linear, di mana fungsi objektif dan kendala yang terlibat berbentuk
persamaan dan pertidaksamaan linear. Fungsi yang akan digunakan adalah \pyinline{linprog}
yang terdefinisi pada modul \pyinline{scipy.optimize}.
Deklarasi API (\textit{application programming interface}) dari fungsi ini adalah sebagai berikut.
%
\begin{pythoncode}
scipy.optimize.linprog(
    c, A_ub=None, b_ub=None, A_eq=None, b_eq=None, bounds=None,
    method='highs', callback=None, options=None, x0=None, integrality=None)
\end{pythoncode}
%
Fungsi ini menyelesaikan permasalahan
pemrograman linear yang memiliki bentuk standard sebagai berikut.
\begin{align*}
& \min_{x} \mathbf{c}^{\mathrm{T}} \mathbf{x} \\
\text{dengan kendala:}\,\, & \mathbf{A}_{\mathrm{ub}} \leq \mathbf{b}_{\mathrm{ub}} \\
& \mathbf{A}_{\mathrm{eq}} = \mathbf{b}_{\mathrm{eq}} \\
& \mathbf{l} \leq \mathbf{x} \leq \mathbf{u}
\end{align*}
dengan:
\begin{itemize}
\item $\mathbf{x}$: (output, array 1d) variabel yang akan diubah-ubah
atau dicari nilainya, atau variabel keputusan (\textit{decision variables}).
\item $\mathbf{c}$: (argumen \pyinline{c}, array 1d) koefisien fungsi objektif (linear).
\item $\mathbf{A}_{\mathrm{ub}}$: (argumen \pyinline{A_ub}, array 2d) matriks kendala batas atas
(\pyinline{ub}: \textit{upper bound})
\item $\mathbf{b}_{\mathrm{ub}}$: (argumen \pyinline{b_ub}, array 1d) vektor kendala batas atas
\item $\mathbf{A}_{\mathrm{eq}}$: (argumen \pyinline{A_eq}, array 2d) matriks kendala persamaan
(\pyinline{eq}: \textit{equality})
\item $\mathbf{b}_{\mathrm{eq}}$: (argumen \pyinline{b_eq}, array 1d) vektor kendala persamaan
\item $\mathbf{l}$ dan $\mathbf{u}$: (argumen \pyinline{l} dan \pyinline{b},
array 1d) kendala rentang nilai untuk
variabel keputusan.
\end{itemize}

Untuk selengkapnya silakan membaca dokumentasi terkait
\footnote{\url{https://docs.scipy.org/doc/scipy/reference/generated/scipy.optimize.linprog.html}}.

Sebagai contoh, kita ingin mencari nilai minimum dari
\begin{equation*}
f(x_0, x_1) = -x_0 + 4x_1
\end{equation*}
dan $x_0$ serta $x_1$ yang membuat $f(x_0, x_1)$ minimum dengan kendala sebagai
berikut:
\begin{align*}
-3x_0 + x_1 & \leq 6 \\
-x_0 - 2x_1 & \geq -4 \\
x_1 \geq -3
\end{align*}
Bentuk di atas belum sesuai dengan bentuk standard yang diasumsikan oleh
\pyinline{linprog}. Oleh karena itu kita perlu mengubah terlebih dahulu bentuk kendala
standard. Kendala kedua dikalikan dulu dengan -1 sehingga kendala menjadi
sebagai berikut.
\begin{align*}
-3x_0 + x_1 & \leq 6 \\
x_0 + 2x_1 & \leq 4 \\
x_1 \geq -3
\end{align*}
Pertidaksamaan terakhir memberikan batas untuk bawah untuk $x_1$.
Kita tidak diberikan bentuk kendala persamaan sehingga \pyinline{A_eq=None} dan
\pyinline{b_eq=None}, yang sudah merupakan nilai \textit{default} dari \pyinline{linprog}.

Kode berikut ini akan mencari solusi dari permasalahan di atas.
\begin{pythoncode}
from scipy.optimize import linprog
c = [-1, 4] # perhatikan bentuk fungsi objektif
# Matriks pertidaksamaan A_ub,
# perhatikan kendala pertidaksamaan untuk variabel keputusan dalam bentuk standard.
A = [
    [-3, 1],
    [1, 2]
]
# Vektor pertidaksamaan A_ub,
b = [6, 4]
x0_bounds = (None, None) # tidak ada batasan untuk x0
x1_bounds = (-3, None) # batas untuk x1, hanya batas bawah yang diberikan

res = linprog(c, A_ub=A, b_ub=b, bounds=[x0_bounds, x1_bounds])

print("Min value: ", res.fun)
print("Decision variables: ", res.x)
\end{pythoncode}

Contoh keluaran:
\begin{textcode}
Min value:  -22.0
Decision variables:  [10. -3.]
\end{textcode}


\begin{soal}
Selesaikan masalah pemrograman linear yang diberikan pada Chapra Contoh 15.2
dengan menggunakan \pyinline{linprog}.
Pada contoh ini kita diminta untuk mencari \textbf{nilai maksimum} dari:
\begin{equation*}
Z = 150x_1 + 175x_2
\end{equation*}
dengan kendala
\begin{align*}
7x_1 + 11x_2 & \leq 77 \\
10x_1 + 8x_2 & \leq 80
\end{align*}
serta $0 \leq x_1 \leq 9$, $0 \leq x_1 \leq 9$ dan bandingkan hasil yang Anda
peroleh dengan metode grafis dan simplex seperti pada buku.
\end{soal}

Anda dapat melengkapi kode berikut.
\begin{pythoncode}
from scipy.optimize import linprog

c = [-150, -175]
# menggunakan tanda berlawan (negatif) karena kita ingin mencari nilai maksimum
# sedangkan bentuk standard linprog hanya menerima masalah minimisasi.

# Bentuk kendala pertidaksamaan sudah dalam bentuk standard yang diasumsikan
# oleh linprog.
A = [
    [7, 11],
    [10, 8]
]
b = [..., ...] # lengkapi

x0_bounds = (0, 9)
x1_bounds = (0, 6)

res = linprog(c, A_ub=A, b_ub=b, bounds=[x0_bounds, x1_bounds])

# We search for max, the result is negative the minimum we found from linprog
print("Max value: ", -res.fun)
print("Decision variables: ", res.x)
\end{pythoncode}





\end{document}


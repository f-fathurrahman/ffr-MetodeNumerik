\input{../PREAMBLE}

% -------------------------
\begin{document}

\title{%
{\small TF2202 Komputasi Rekayasa}\\
Optimisasi
}
\author{Tim Praktikum Komputasi Rekayasa 2023\\
Teknik Fisika\\
Institut Teknologi Bandung}
\date{}
\maketitle

\textbf{Catatan}

Optimisasi dapat merujuk pada pencarian nilai minimum atau nilai maksimum
dari sebuah fungsi objektif.
Pada modul ini, kita akan memilih untuk mengimplementasikan
pencarian nilai minimum (bentuk standard).
Algoritma atau subrutin yang sudah diimplementasikan
untuk minimisasi suatu fungsi objektif dapat digunakan untuk mencari nilai maksimum
dengan cara mengubah tanda dari fungsi objektif, misalnya dari positif menjadi negatif,
atau sebaliknya.


\section{Optimisasi Variabel Tunggal}
% subsections
%\section{Metode bagi-dua}

Selain digunakan untuk mencari akar persamaan nonlinear, metode bagi-dua
juga dapat digunakan untuk mencari nilai optimum dari suatu fungsi.

\section{Metode rasio emas (\textit{golden section})}

Berikut ini adalah implementasi dari metode rasio emas,
diadaptasi dari Chapra 7th (Gambar 13.5) untuk mencari
\textbf{nilai minimum} dari suatu fungsi.

\begin{pythoncode}
import numpy as np

# This will search for minimum value
# If you want to use this for searching maximum value please
# define f = -f_actual, where f_actual is the function you want
# to maximize.
# The sign keyword argument is not implemented here.

def optim_golden_ratio(f, xlow, xhigh, NiterMax=100, TOL=1e-10, verbose=True):

    SMALL = np.finfo(np.float64).resolution

    ϕ = .... # lengkapi
    xl = xlow
    xu = xhigh
    iiter = 1
    d = ϕ*(xu - xl)
    x1 = xl + d
    x2 = xu - d
    f1 = f(x1)
    f2 = f(x2)

    if verbose:
        print("Iteration: ", 0)
        print()
        print("xl = %18.10f xu = %18.10f" % (xl, xu))
        print("-"*47)
        print("x2 = %18.10f x1 = %18.10f" % (x2, x1))
        print("f2 = %18.10f f1 = %18.10f" % (f2, f1))

    if f1 < f2:
        xopt = x1
        fxopt = f1
        if verbose:
            print("-"*47)
            print("f1 is the current optimum value")
            print("(xopt,fxopt) = (%18.10f,%18.10f)" % (xopt, fxopt))
    else:
        xopt = x2
        fxopt = f2
        if verbose:
            print("-"*47)
            print("f2 is the current optimum value")
            print("(xopt,fxopt) = (%18.10f,%18.10f)" % (xopt, fxopt))

    for iiter in range(1,NiterMax+1):

        d = ϕ*d
        xint = xu - xl

        if f1 < f2:
            xl = x2
            x2 = x1
            x1 = .... # lengkapi
            f2 = f1
            f1 = f(x1)
            if verbose:
                print("")
                print("For next iteration: ")
                print("- Replacing xl with x2")
                print("- New point: x1 = %18.10f" % x1)
                print("- xu is not changed")
                print("- Replacing x2 with x1")
        else:
            xu = x1
            x1 = x2
            x2 = .... # lengkapi
            f1 = f2
            f2 = f(x2)
            if verbose:
                print("")
                print("For next iteration: ")
                print("- xl is not changed")
                print("- x1 is replaced by x2")
                print("- New point: x2 = %18.10f" % x2)
                print("- xu is replaced by x1")

        if verbose:
            print("")
            print("Iteration: ", iiter)
            print("xl = %18.10f xu = %18.10f" % (xl, xu))
            print("-"*47)
            print("x2 = %18.10f x1 = %18.10f" % (x2, x1))
            print("f2 = %18.10f f1 = %18.10f" % (f2, f1))

        #
        if f1 < f2:
            xopt = x1
            fxopt = f1
            if verbose:
                print("-"*47)
                print("f1 is the current optimum value")
                print("(xopt,fxopt) = (%18.10f,%18.10f)" % (xopt, fxopt))
        else:
            xopt = x2
            fxopt = f2
            if verbose:
                print("-"*47)
                print("f2 is the current optimum value")
                print("(xopt,fxopt) = (%18.10f,%18.10f)" % (xopt, fxopt))

        if abs(xopt) > SMALL:
            ea = (1 - ϕ)*abs(xint/xopt)
        else:
            # The above might fail if xopt is very close to zero
            # We set xint as the convergence criteria
            ea = xint

        if verbose:
            print("Interval length = %18.10e" % xint)
            print("ea              = %18.10e" % ea)

        if ea <= TOL:
            if verbose:
                print("Converged")
            break

    return xopt, fxopt
\end{pythoncode}

Contoh penggunaan pada kasus Chapra Contoh 13.1, di mana kita ingin mencari
nilai maksimum dari fungsi:
\begin{equation*}
f(x) = 2\sin(x) - \frac{x}{10}
\end{equation*}
pada selang $x \in [0,4]$.
Pada program berikut, selain mendefinisikan fungsi \pyinline{my_func},
kita juga mendefinisikan fungsi \pyinline{m_my_func} yang mengembalikan
$-f(x)$ karena pada fungsi \pyinline{optim_golden_ratio} akan
mencari nilai minumum. Ingat bahwa nilai maksimum dari $f(x)$ sama
dengan nilai minimum dari $-f(x)$.

\begin{pythoncode}
import numpy as np

# definisikan atau import modul yang berisi optim_golden_ratio

def my_func(x):
    return 2*np.sin(x) - x**2/10

def m_my_func(x):
    return -my_func(x)

xopt, fx = optim_golden_ratio(m_my_func, 0.0, 4.0, TOL=1e-10, verbose=True)

print("\nOptimization result")
print("xopt    = %18.10f" % xopt)
print("f(xopt) = %18.10f" % my_func(xopt))
# gunakan fungsi asal, yaitu my_func
\end{pythoncode}


Contoh keluaran:
\begin{textcode}
Iteration:  0

xl =       0.0000000000 xu =       4.0000000000
-----------------------------------------------
x2 =       1.5278640450 x1 =       2.4721359550
f2 =      -1.7647202483 f1 =      -0.6299744700
-----------------------------------------------
f2 is the current optimum value
(xopt,fxopt) = (      1.5278640450,     -1.7647202483)

For next iteration: 
- xl is not changed
- x1 is replaced by x2
- New point: x2 =       0.9442719100
- xu is replaced by x1

Iteration:  1
xl =       0.0000000000 xu =       2.4721359550
-----------------------------------------------
x2 =       0.9442719100 x1 =       1.5278640450
f2 =      -1.5309755469 f1 =      -1.7647202483
-----------------------------------------------
f1 is the current optimum value
(xopt,fxopt) = (      1.5278640450,     -1.7647202483)
Interval length =   4.0000000000e+00
ea              =   1.0000000000e+00

For next iteration: 
- Replacing xl with x2
- New point: x1 =       1.8885438200
- xu is not changed
- Replacing x2 with x1

Iteration:  2
xl =       0.9442719100 xu =       2.4721359550
-----------------------------------------------
x2 =       1.5278640450 x1 =       1.8885438200
f2 =      -1.7647202483 f1 =      -1.5432233694
-----------------------------------------------
f2 is the current optimum value
(xopt,fxopt) = (      1.5278640450,     -1.7647202483)
Interval length =   2.4721359550e+00
ea              =   6.1803398875e-01

For next iteration: 
- xl is not changed
- x1 is replaced by x2
- New point: x2 =       1.3049516850
- xu is replaced by x1

...... # dan seterusnya

Iteration:  49
xl =       1.4275517651 xu =       1.4275517653
-----------------------------------------------
x2 =       1.4275517652 x1 =       1.4275517652
f2 =      -1.7757256531 f1 =      -1.7757256531
-----------------------------------------------
f2 is the current optimum value
(xopt,fxopt) = (      1.4275517652,     -1.7757256531)
Interval length =   3.7209457737e-10
ea              =   9.9560299663e-11
Converged

Optimization result
xopt    =       1.4275517652
f(xopt) =       1.7757256531
\end{textcode}




\subsection{Interpolasi parabola}

Polinomial orde dua atau parabola seringkali dapat digunakan sebagai pendekatan
dari $f(x)$ di sekitar minimum, misalnya untuk mengaproksimasi fungsi potensial Morse
antara dua atom.
Pada metode interpolasi parabola, dipilih tiga titik $x_0, x_1, x_2$
yang kemudian diinterpolasi menjadi
fungsi polinomial orde dua. Nilai optimum dari $f(x)$ kemudian diaproksimasi sebagai
optimum dari parabola yang dihasilkan. Nilai optimum diperoleh dari
\begin{equation}
x_3 = \frac{f(x_0)(x_{1}^2 - x_{2}^2) + f(x_1)(x_{2}^2 - x_{0}^2) + f(x_2)(x_{0}^2 - x_{1}^2)}%
{2f(x_0)(x_{1} - x_{2}) + 2f(x_1)(x_{2} - x_{0}) + 2f(x_2)(x_{0} - x_{1})}
\end{equation}
Untuk iterasi selanjutnya dapat digunakan strategi yang sama dengan pemilihan titik pada
metode rasio emas. Alternatif lain yang akan kita gunakan di sini adalah
penggunaan titik secara sekuensial (lebih mudah untuk diimplementasikan):
$x_0 \leftarrow x_1$, $x_1 \leftarrow x_2$, dan $x_2 \leftarrow x_3$.

\begin{soal}
Program Python berikut ini mengimplementasikan metode interpolasi parabola untuk
mencari maksimum dari $f(x) = 2\sin(x) - \frac{x^2}{10}$. Lengkapi kode yang tersebut.
\end{soal}

Catatan: kode berikut ini belum dibuat dalam bentuk fungsi seperti pada kasus
\pyinline{optim_golden_ratio}, Anda dapat mengubahnya jika diperlukan.
Kriteria konvergensi yang digunakan adalah ketika nilai optimum sudah tidak berubah
berdasarkan suatu nilai tertentu.

\begin{pythoncode}
import numpy as np

def my_func(x):
    return 2*np.sin(x) - x**2/10

# f0 = f(x0), f1 = f(x1), f2 = f(x2)
def calc_parabolic_x3(x0, f0, x1, f1, x2, f2):
    num = ....  # lengkapi
    denum = .... # lengkapi
    return num/denum

# Initial guess
x0 = 0.0; f0 = my_func(x0)
x1 = 1.0; f1 = my_func(x1)
x2 = 4.0; f2 = my_func(x2)

x3 = calc_parabolic_x3(x0, f0, x1, f1, x2, f2)
f3 = my_func(x3)
print("x3 = %18.10f f3 = %18.10f" % (x3, f3))

TOL = 1e-10
NiterMax = 100

for iiter in range(1,NiterMax+1):
    xopt_old = x3
    fopt_old = f3

    # Sequentially choose the next points
    x0 = x1; f0 = f1
    x1 = x2; f1 = f2
    x2 = x3; f2 = f3

    x3 = calc_parabolic_x3(x0, f0, x1, f1, x2, f2)
    f3 = my_func(x3)
    print("x3 = %18.10f f3 = %18.10f" % (x3, f3))

    if abs(fopt_old - f3) < TOL:
        print("Converged")
        break
\end{pythoncode}

Contoh keluaran:
\begin{textcode}
x3 =       1.5055348740 f3 =       1.7690789285
x3 =       1.4902527509 f3 =       1.7714309125
x3 =       1.3908075360 f3 =       1.7742568388
x3 =       1.4275400017 f3 =       1.7757256530
x3 =       1.4275037854 f3 =       1.7757256506
x3 =       1.4275518296 f3 =       1.7757256531
x3 =       1.4275526174 f3 =       1.7757256531
Converged
\end{textcode}
\subsection{Metode Newton}

Metode Newton untuk pencarian akar dapat digunakan pada masalah optimisasi.
Pada kasus ini metode Newton digunakan untuk mencari akar dari turunan
fungsi. Skema iterasi yang digunakan adalah
\begin{equation}
x_{i+1} = x_{i} - \frac{f'(x_i)}{f''(x_i)}
\end{equation}

Perhatikan bahwa metode Newton memerlukan informasi turunan pertama dan kedua
dari fungsi yang ingin dicari nilai optimumnya.

\begin{soal}
Lengkapi kode berikut untuk mengimplementasikan metode Newton
pada fungsi $f(x) = 2\sin(x) - \dfrac{x^2}{10}$.
\end{soal}

\begin{pythoncode}
import numpy as np

def my_func(x):
    return 2*np.sin(x) - x**2/10

def d1_my_func(x): # turunan pertama
    return .... # lengkapi

def d2_my_func(x): # turunan kedua
    return .... # lengkapi

SMALL = np.finfo(np.float64).resolution # atau nilai yang cukup kecil
NiterMax = 100

# Initial guess
x0 = 2.5
fopt_old = np.nan

for iiter in range(1,NiterMax+1):
    f0 = my_func(x0)
    df0 = d1_my_func(x0)
    d2f0 = d2_my_func(x0)
    if abs(df0) > SMALL:
        x1 = x0 - df0/d2f0
        f1 = my_func(x1)
        print("%18.10f %18.10f %18.10e" % (x1, f1, abs(f1 - fopt_old)))
    else:
        print("Converged")
        break
    x0 = x1
    fopt_old = f1
\end{pythoncode}

Contoh keluaran:
\begin{textcode}
      0.9950815513       1.5785880072                nan
      1.4690107528       1.7738493793   1.9526137207e-01
      1.4276423210       1.7757256442   1.8762648994e-03
      1.4275517793       1.7757256531   8.9337626186e-09
      1.4275517788       1.7757256531   0.0000000000e+00
Converged
\end{textcode}
\subsection{Metode Brent}

\begin{soal}
Berikut ini adalah implementasi dari pseudocode yang diberikan pada
Gambar 13.7 pada Chapra.
Lengkapi kode Python yang diberikan dan uji pada fungsi
$f(x) = 2\sin(x) - \dfrac{x^2}{10}$ (untuk mencari nilai maksimum).
\end{soal}

\begin{pythoncode}
import numpy as np

def optim_brent(f, xl, xu, TOL=1e-10, NiterMax=100, verbose=True):
    
    SMALL = np.finfo(np.float64).resolution

    ϕ = .... # lengkapi
    ρ = 2.0 - ϕ
    u = .... # lengkapi
    v = u; w = u; x = u
    fu = f(u)
    fv = fu; fw = fu; fx = fu
    xm = .... # lengkapi
    d = 0.0
    e = 0.0

    iiter = 0
    while True:
        iiter = iiter + 1
        if verbose:
            print("\nBegin iter = ", iiter)
            print("x      = %18.10f" % x)
            print("xm     = %18.10f" % xm)
            print("fu     = %18.10f" % fu)
            print("x - xm = %18.10e" % abs(x - xm))

        if abs(x - xm) <= TOL:
            if verbose:
                print("Converged")
            break

        if iiter >= NiterMax:
            if verbose:
                print("WARNING: Maximum iterations reached")
            break

        para = abs(e) > TOL
        
        if para:
            # Try using parabolic interp
            r = (x - w)*(fx - fv)
            q = (x - v)*(fx - fw)
            p = (x - v)*q - (x - w)*r
            s = 2*(q - r)
            if s > 0:
                p = -q
            s = abs(s)
            #
            cond1 = abs(p) < abs(0.5*s*e)
            cond2 = p > s*(xl - x)
            cond3 = p < s*(xu - x)
            para = cond1 and cond2 and cond3
            # Parabolic interpolation step
            if para:
                if verbose:
                    print("Parabolic interpolation is used")
                e = d
                d = p/s

        if not para:
            if verbose:
                print("Using golden section")
            if x >= xm:
                e = xl - x
            else:
                e = xu - x
            d = ρ*e

        u = x + d
        fu = f(u)

        if fu <= fx:
            if u >= x:
                xl = x
            else:
                xu = x
            v = w; fv = fw
            w = x; fw = fx
            x = u; fx = fu
        else:
            #
            if u < x:
                xl = u
            else:
                xu = u
            #
            if (fu <= fw) or (abs(w - x) <= SMALL):
                v = w; fv = fw
                w = u; fw = fu
            elif (fu <= fv) or (abs(v - x) <= SMALL) or (abs(v - w) <= SMALL):
                v = u; fv = fu
        #
        xm = 0.5*(xl + xu)

    return xu, fu
\end{pythoncode}

Contoh pemanggilan fungsi:
\begin{pythoncode}
import numpy as np

# definisi atau import optim_brent di sini

def my_func(x):
    return 2*np.sin(x) - x**2/10

def m_my_func(x):
    return -my_func(x)

xopt, fxopt = optim_brent(m_my_func, 0.0, 4.0, TOL=1e-10)
print()
print("xopt    = %18.10f" % xopt)
print("f(xopt) = %18.10f" % my_func(xopt))
\end{pythoncode}

Contoh hasil keluaran (\pyinline{verbose=True}):
\begin{textcode}
Begin iter =  1
x      =       1.5278640450
xm     =       2.0000000000
fu     =      -1.7647202483
x - xm =   4.7213595500e-01
Using golden section

Begin iter =  2
x      =       1.5278640450
xm     =       1.2360679775
fu     =      -0.6299744700
x - xm =   2.9179606750e-01
Using golden section

.... # dan seterusnya

Begin iter =  30
x      =       1.4275517790
xm     =       1.4275517790
fu     =      -1.7757256531
x - xm =   5.9506621852e-11
Converged

xopt    =       1.4275517793
f(xopt) =       1.7757256531
\end{textcode}


\section{Optimisasi Variabel Banyak}
Pada bagian ini kita akan mengeksplorasi beberapa metode yang dapat
digunakan untuk optimisasi banyak variabel (multivariabel). Untuk
memudahkan visualisasi, kita akan lebih fokus pada masalah dua variabel
(atau dua dimensi), sehingga fungsi objektif lebih mudah untuk divisualisasi.

\subsection{Pencarian Acak (\textit{random search})}

\begin{soal}[Chapra Contoh 14.1]
Lengkapi kode program berikut ini yang mengimplementasikan metode
pencarian acak untuk mencari nilai maksimum dari:
\begin{equation*}
f(x,y) = y - x - 2x^2 - 2xy - y^2
\end{equation*}
Coba variasikan nilai tebakan awal dan jumlah sampel yang digunakan.
\end{soal}

\begin{pythoncode}
import numpy as np

np.random.seed(13306023) # optional, for reproducibility
# silakan coba ganti dengan seed yang lain, misalnya dengan NIM Anda.

def my_func(x, y):
    return ... # LENGKAPI

xmin, xmax = -2.0, 2.0
ymin, ymax = 1.0, 3.0

Nsample = 10000

# First sample
x = xmin + (xmax - xmin)*np.random.rand()
y = ymin + (ymax - ymin)*np.random.rand()

fopt = my_func(x, y)
xopt, yopt = x, y

for i in range(1, Nsample+1):
    x = .... # LENGKAPI
    y = .... # LENGKAPI
    f = my_func(x, y)
    if f > fopt: # we search for max, if you search for min, use <
        fopt = f
        xopt, yopt = x, y
    if i % 1000 == 0:
        print("Current no. sample = ", i)
        print("Current max: fopt = ", fopt, " at xopt = ", xopt, " yopt = ", yopt)
\end{pythoncode}
\subsection{Metode Penurunan Paling Tajam (\textit{steepest descent})}

Metode \textit{steepest descent} (SD) adalah salah satu metode berbasis gradien
yang paling sederhana dan banyak digunakan dalam optimisasi multivariabel.
Pada metode ini, selain fungsi objektif, kita juga memerlukan gradien dari
fungsi objektif. Gradien fungsi objektif ini sendiri merupakan vektor yang
elemen-elemennya merupakan turunan parsial dari fungsi objektif terhadap
variabel independennya.

\begin{soal}[Chapra Contoh 14.2]
Berikut ini adalah contoh perhitungan gradien dari:
\begin{equation*}
f(x,y) = xy^2
\end{equation*}
pada titik (2,2).
Lengkapi kode Python berikut ini. Variasikan nilai \pyinline{Δ} untuk
ukuran langkah, misalnya \pyinline{1e-2},
\pyinline{1e-1}, \pyinline{1e0}, \pyinline{1e1}, \pyinline{1e2},
dan \pyinline{1e3}.
\end{soal}


\begin{pythoncode}
import numpy as np
import matplotlib.pyplot as plt

def my_func(x,y):
    return x*y**2

def grad_my_func(x,y):
    dfdx = .... # LENGKAPI
    dfdy = .... # LENGKAPI
    return dfdx, dfdy


x = np.linspace(0.0, 4.0, 100)
y = np.linspace(0.0, 4.0, 100)
X, Y = np.meshgrid(x, y)
FXY = my_func(X, Y)

# Calc the gradient at (2,2)
x0, y0 = 2.0, 2.0
gx, gy = grad_my_func(x0, y0)

fig, ax = plt.subplots()
my_contour = ax.contour(X, Y, FXY, levels=np.linspace(8.0, 40.0, 5), colors="black")
ax.quiver(x0, y0, gx, gy, color="blue")
ax.set_aspect("equal", "box")
ax.clabel(my_contour, inline=True, fontsize=10)

plt.show()

Δ = 1.0 # Try to vary this value

# Now suppose that we want to search for MAXIMUM value
# We want to search for new point: (x0 + Δ*gx, y0 + Δ*gy)
# or we move in the direction of STEEPEST ASCENT
# which gives larger value than current value

f1 = my_func(x0, y0) # Value of the function at (x0,y0)
f2 = my_func(x0 + Δ*gx, y0 + Δ*gy) # Value of the function at new point
print("\nTrying to find maximum")
print("Old value: %18.10f" % f1)
print("New value: %18.10f" % f2)
if f2 > f1:
    print("Good: Function value is increasing")
else:
    print("Bad: Function value is decreasing")
    print("Step length is too large") # or Δ is too large

# Now suppose that we want to search for MINIMUM value
# We want to search for new point: (x0 - Δ*gx, y0 - Δ*gy)
# or we move in the direction of STEPEEST DESCENT
# which gives smaller value than current value

f1 = my_func(x0, y0) # Value of the function at (x0,y0)
f2 = my_func(x0 - Δ*gx, y0 - Δ*gy) # Value of the function at new point
print("\nTrying to find minimum")
print("Old value: %18.10f" % f1)
print("New value: %18.10f" % f2)
if f2 < f1:
    print("Good: Function value is decreasing")
else:
    print("Bad: Function value is increasing")
    print("Step length is too large") # or Δ is too large
\end{pythoncode}


Berikut ini adalah contoh plot yang dihasilkan.

{\centering
\includegraphics[scale=0.5]{../../chapra_7th/ch14/IMG_chapra_example_14_2.pdf}
\par}

\begin{soal}
Ulangi yang dilakukan pada soal sebelumnya untuk fungsi berikut ini.
\begin{equation*}
f(x,y) = 2xy + 2x - x^2 - 2y^2
\end{equation*}
Anda perlu mengganti definisi fungsi dan gradien.
\end{soal}

Dari soal sebelumnya kita akan mengimplementasikan metode \textit{steepest descent}
untuk mencari minimum sebuah fungsi.

\begin{soal}[Chapra Latihan 14.4]
Lengkapi program Python berikut. Program ini
mencari nilai maksimum dari
\begin{equation*}
f(x,y) = 2xy + 2x - x^2 - 2y^2
\end{equation*}
dengan menggunakan metode \textit{steepest descent} dengan ukuran langkah tetap.
Perhatikan bahwa program ini dapat diubah menjadi subrutin atau fungsi, namun
kita lebih memilih untuk tidak melakukannya supaya lebih mudah untuk dilakukan
visualisasi langkah-langkah yang ditempuh.
Coba lakukan variasi ukuran langkah, variabel \pyinline{α},
dan jumlah iterasi maksimum.
\end{soal}

\begin{pythoncode}
import numpy as np
import matplotlib.pyplot as plt

def my_func(X): # input as vector X
    x, y = X[0], X[1]
    return .... # LENGKAPI: gunakan x dan y 

def my_func_plot(X, Y): # for plotting purpose
    return 2*X*Y + 2*X - X**2 - 2*Y**2

def grad_my_func(X): # input as vector X
    x, y = X[0], X[1]
    dfdx = .... # LENGKAPI: gunakan x dan y
    dfdy = .... # LENGKAPI: gunakan x dan y
    return np.array([dfdx, dfdy]) # return as numpy array

# Definisikan negatif dari fungsi dan gradien yang ingin
# dicari nilai maksimumnya. Hal ini diperlukan karena
# kita mengimplementasikan steepest descent yang akan mencari
# nilai MINIMUM dari suatu fungsi
def m_my_func(X):
    return -my_func(X)

def grad_m_my_func(X):
    return -grad_my_func(X)

xgrid = np.linspace(-2.0, 4.5, 100)
ygrid = np.linspace(-1.0, 3.0, 100)
Xgrid, Ygrid = np.meshgrid(xgrid, ygrid)

fig, ax = plt.subplots()
ax.contour(Xgrid, Ygrid, my_func_plot(Xgrid, Ygrid), levels=10)

x0 = np.array([-1.0, 1.0]) # Initial point
NiterMax = 40
α = 0.1 # ukuran langkah
func = m_my_func # fungsi yang akan dimininumkan
grad_func = grad_m_my_func  # gradien
x = np.copy(x0)

ax.plot(x[0], x[1], marker="o", color="black")
ax.set_aspect("equal")
plt.savefig("IMG_optim_SD_" + str(0) + ".png", dpi=150)

for iiter in range(1,NiterMax+1):

    print("\nIteration: ", iiter)
    print("Current point: ", x)

    f = func(x)
    g = grad_func(x)
    d = -g # step direction, we search for minimum

    ax.quiver(x[0], x[1], d[0], d[1], color="blue") # also plot the direction

    norm_g = np.sqrt(np.dot(g,g))
    print("f      = %18.10f" % f)
    print("norm g = %18.10e" % norm_g)
    if norm_g < 1e-10:
        print("Converged")
        break

    # Update x
    xprev = np.copy(x)
    x = x + α*d

    # draw a line from xprev to x
    ax.plot([xprev[0], x[0]], [xprev[1], x[1]], marker="o", color="black")
    plt.savefig("IMG_optim_SD_" + str(iiter) + ".png", dpi=150)
\end{pythoncode}

Contoh hasil visualisasi yang diperoleh \pyinline{NiterMax = 40}
dan \pyinline{α = 0.1}.

{\centering
\includegraphics[scale=0.75]{../../chapra_7th/ch14/IMG_debug_optim_SD.pdf}
\par}

Hasil akhir yang diperoleh:
\begin{pythoncode}
....
Iteration:  40
Current point:  [1.90213631 0.93951691]
f      =      -1.9949444864
norm g =   8.7887071781e-02
\end{pythoncode}
sudah cukup dekat dengan hasil analitik namun belum cukup konvergen.

Jika kita menggunakan \pyinline{α = 0.01} dan \pyinline{NiterMax = 40}
berikut hasil visualisasi yang diperoleh.

{\centering
\includegraphics[scale=0.75]{../../chapra_7th/ch14/IMG_debug_optim_SD_alpha_001.pdf}
\par}
dan hasil akhir yang diperoleh adalah
\begin{textcode}
....
Iteration:  40
Current point:  [0.28853671 0.1698791 ]
f      =      -0.5341348867
norm g =   1.7656591408e+00
\end{textcode}
yang masih cukup jauh dari nilai analitik.


Jika kita menggunakan ukuran langkah yang digunakan terlalu besar,
misalnya \pyinline{α = 0.4} dan \pyinline{NiterMax = 5} (dibatasi agar tidak
terlalu besar)
berikut hasil visualisasi yang diperoleh.

{\centering
\includegraphics[scale=0.75]{../../chapra_7th/ch14/IMG_debug_optim_SD_alpha_04.pdf}
\par}

dan hasil akhir yang diperoleh adalah
\begin{textcode}
....
Iteration:  5
Current point:  [0.3056 2.6128]
f      =      11.5386956800
norm g =   1.1856470274e+01
\end{textcode}
yang menunjukkan bahwa iterasi terlihat akan divergen.

Dari soal ini kita dapat melihat bahwa ukuran langkah yang diambil pada metode berbasis gradien,
misalnya \textit{steepest descent} akan mempengarui konvergensi dari algoritma.
Jika terlalu kecil konvergensi akan lambat diperoleh, sedangkan jika terlalu besar ada kemungkinan
algoritma akan divergen.

Dalam literatur, penentuan ukuran langkah pada metode berbasis gradien
dikenal sebagai \textit{line minimization}. Pada buku Chapra, pengambilan ukuran langkah dilakukan dengan cara
proyeksi ke fungsi 1-d dimensi atau suatu \emph{garis} (karena itu dinamakan \textit{line minimization}),
kemudian menentukan nilai langkah
dari optimum fungsi 1-d ini secara analitik, yang dalam hal ini dapat dibantu dengan perhitungan simbolik,
misalnya dengan menggunakan SymPy.
Hal ini dapat dilakukan untuk fungsi-fungsi dengan variabel
yang relatif sedikit dan/atau cukup sederhana untuk dimanipulasi secara simbolik.
Akan tetapi manipulasi ini tidak praktis untuk dilakukan pada optimisasi numerik sehingga
banyak dikembangkan banyak metode untuk \textit{line minimization} ini dalam literatur.
Pada modul ini kita akan menggunakan metode yang memerlukan evaluasi gradien pada suatu titik uji.
Berikut ini adalah implementasi yang siap untuk kita gunakan, yang akan kita namai
\pyinline{linmin_grad}:
\begin{pythoncode}
def linmin_grad(grad_func, x, g, d, αt=1e-5):
    xt = x + αt*d
    gt = grad_func(xt)
    denum = np.dot(g - gt, d)
    if denum != 0.0:
        α = abs( αt * np.dot(g, d)/denum )
    else:
        α = 0.0
    return α
\end{pythoncode}

Program \textit{steepest descent} yang kita gunakan sebelumnya dapat dimodifikasi menjadi sebagai
berikut.
\begin{pythoncode}
# ... sama seperti sebelumnya

x0 = np.array([-1.0, 1.0]) # Initial point
NiterMax = 40
func = m_my_func
grad_func = grad_m_my_func
x = np.copy(x0)
# Tidak ada definisi α

ax.plot(x[0], x[1], marker="o", color="black")
ax.set_aspect("equal")
plt.savefig("IMG_optim_SD_linmin_" + str(0) + ".png", dpi=150)
# ubah nama untuk plot yang dihasilkan

for iiter in range(1,NiterMax+1):

    # ... sama seperti sebelumnya

    # Update x
    xprev = np.copy(x)
    α = linmin_grad(grad_func, x, g, d) # hitung ukuran langkah
    x = x + α*d 
    # draw a line from xprev to x
    ax.plot([xprev[0], x[0]], [xprev[1], x[1]], marker="o", color="black")
    plt.savefig("IMG_optim_SD_linmin_" + str(iiter) + ".png", dpi=150)
\end{pythoncode}

\begin{soal}
Modifikasi kode yang Anda buat sehingga menggunakan \pyinline{linmin_grad}.
\end{soal}

Kita memperoleh nilai minimum (konvergen) yang diperlukan dalam iterasi ke-32.
\begin{textcode}
Iteration:  32
Current point:  [2. 1.]
f      =      -2.0000000000
norm g =   7.1152888650e-11
Converged
\end{textcode}
dengan visualisasi langkah pada gambar berikut.

{\centering
\includegraphics[scale=0.75]{../../chapra_7th/ch14/IMG_debug_optim_SD_linmin.pdf}
\par}

Pada gambar terlihat bahwa ukuran langkah yang diambil cukup bagus pada bagian awal,
tetapi metode ini masih mengalami kesulitan untuk konvergen ketika sudah cukup dekat
dengan minimum. Hal ini disebabkan karena metode \textit{steepest descent} hanya
menggunakan informasi yang ada pada langkah saat sekarang dan mengabaikan
informasi pada langkah-langkah sebelumnya. Selain itu, langkah yang diambil oleh metode
ini akan tegak lurus terhadap langkah sebelumnya, hal ini dapat menyebabkan masalah
konvergensi. Meskipun demikian metode \textit{steepest descent} dengan penentuan
langkah adaptif (menggunakan \pyinline{linmin_grad}) sudah cukup bagus digunakan
untuk minimisasi fungsi.

\subsection{Metode Gradien Konjugat (\textit{Conjugate Gradient}) (OPSIONAL)}

Metode gradien konjugat merupakan modifikasi dari metode \textit{steepest descent} yang
menambahkan informasi langkah sebelumnya. Ada banyak sekali varian dari metode ini, kita
akan menggunakan salah satu yang paling sederhana.

Berikut ini adalah implementasi dari metode gradien konjugat.
\begin{pythoncode}
# ... sama dengan sebelumnya

# Initial point
x0 = np.array([-1.0, 1.0])
NiterMax = 40
func = m_my_func
grad_func = grad_m_my_func
x = np.copy(x0)
d_prev = np.zeros(np.size(x0))
g_prev = np.zeros(np.size(x0))

ax.plot(x[0], x[1], marker="o", color="black")
ax.set_aspect("equal")
plt.savefig("IMG_optim_CG_linmin_" + str(0) + ".png", dpi=150)

for iiter in range(1,NiterMax+1):

    print("\nIteration: ", iiter)
    print("Current point: ", x)

    f = func(x)
    g = grad_func(x)

    if iiter > 1:
        # Choose one of these
        β = np.dot(g, g)/np.dot(g_prev, g_prev) # Fletcher-Reeves
        #β = np.dot(g-g_prev,g) / np.dot(g_prev,g_prev) # Polak-Ribiere
    else:
        β = 0.0

    if β < 0:
        β = 0.0

    print("β = ", β)
    d = -g + β*d_prev # step direction, we search for minimum

    ax.quiver(x[0], x[1], d[0], d[1], color="blue") # also plot the direction

    norm_g = np.sqrt(np.dot(g,g))
    print("f      = %18.10f" % f)
    print("norm g = %18.10e" % norm_g)
    if norm_g < 1e-10:
        print("Converged")
        break

    # Update x
    xprev = np.copy(x)
    α = linmin_grad(grad_func, x, g, d)
    x = x + α*d 
    # draw a line from xprev to x
    ax.plot([xprev[0], x[0]], [xprev[1], x[1]], marker="o", color="black")
    plt.savefig("IMG_optim_CG_linmin_" + str(iiter) + ".png", dpi=150)

    d_prev = np.copy(d)
    g_prev = np.copy(g)
\end{pythoncode}

Perhatikan bahwa metode gradien konjugat memerlukan variabel tambahan berupa
gradien dan/atau arah pencarian sebelumnya: variabel \pyinline{g_prev} dan
\pyinline{d_prev} pada program. Metode gradien konjugat akan tereduksi menjadi
metode \textit{steepest descent} ketika variabel \pyinline{β=0}.

Metode gradien konjugat memberikan konvergensi yang jauh lebih cepat daripada
metode \textit{steepest descent}.
\begin{textcode}
Iteration:  3
Current point:  [2. 1.]
β =  4.688332967786487e-22
f      =      -2.0000000000
norm g =   3.6745610550e-11
Converged
\end{textcode}

Hasil visualisasi proses optimisasi:

{\centering
\includegraphics[scale=0.75]{../../chapra_7th/ch14/IMG_debug_optim_CG_FR_linmin.pdf}
\par}



% New section
\section{Pemrograman Linear}

Pada bagian ini, kita akan menggunakan pustaka SciPy untuk menyelesaikan permasalahan
pemrograman linear, di mana fungsi objektif dan kendala yang terlibat berbentuk
persamaan dan pertidaksamaan linear. Fungsi yang akan digunakan adalah \pyinline{linprog}
yang terdefinisi pada modul \pyinline{scipy.optimize}. Fungsi ini menyelesaikan permasalahan
pemrograman linear yang memiliki bentuk standard sebagai berikut.
\begin{align*}
& \min_{x} \mathbf{c}^{\mathrm{T}} \mathbf{x} \\
\text{dengan kendala:}\,\, & \mathbf{A}_{\mathrm{ub}} \leq \mathbf{b}_{\mathrm{ub}} \\
& \mathbf{A}_{\mathrm{eq}} = \mathbf{b}_{\mathrm{eq}} \\
& \mathbf{l} \leq \mathbf{x} \leq \mathbf{u}
\end{align*}
dengan:
\begin{itemize}
\item $\mathbf{x}$: (array 1d) variabel yang akan diubah-ubah
atau dicari nilainya, atau variabel keputusan (\textit{decision variables}).
\item $\mathbf{c}$: (array 1d)
\item $\mathbf{A}_{\mathrm{ub}}$: (array 2d) matriks kendala batas atas
(\pyinline{ub}: \textit{upper bound})
\item $\mathbf{x}_{\mathrm{ub}}$: (array 1d) vektor kendala batas atas
\item $\mathbf{A}_{\mathrm{eq}}$: (array 2d) matriks kendala persamaan
(\pyinline{eq}: \textit{equality})
\item $\mathbf{x}_{\mathrm{ub}}$: (array 1d) vektor kendala persamaan
\item $\mathbf{l}$ dan $\mathbf{b}$: (array 1d) kendala rentang nilai untuk
variabel keputusan.
\end{itemize}



\section{Soal tambahan}
\begin{soal}[Chapra Latihan 13.6]
Gunakan metode:
\begin{itemize}
\item rasio emas (titik awal $x_l = -2$ dan $x_u = 4$)
\item interpolasi parabola (titik awal $x_0 = 1.75$, $x_1 = 2$, dan $x_2 = 2.5$)
\item Newton (dengan tebakan awal $x_0 = 3$)
\end{itemize}
untuk mencari nilai maksimum dari
\begin{equation*}
f(x) = 4x - 1.8x^2 + 1.2x^3 - 0.3x^4
\end{equation*}
\end{soal}

\begin{soal}[Chapra Latihan 13.9]
Cari minimum dari:
\begin{equation*}
f(x) = 2x + \frac{3}{x}
\end{equation*}
dengan menggunakan metode rasio emas, interpolasi parabolik, dan Newton.
Silakan pilih titik-titik awal yang diperlukan dengan cara membuat plot
dari $f(x)$ terlebih dahulu.
\end{soal}


\begin{soal}[Chapra Latihan 14.6]
Cari nilai minimum dari
\begin{equation*}
f(x,y) = (x - 3)^2 + (y - 2)^2
\end{equation*}
dengan menggunakan metode \textit{steepest descent}. Buat plot kontur
dari $f(x,y)$ dan tunjukkan titik yang dilalui ketika proses optimisasi.
Gunakan tiga titik awal yang berbeda pada saat optimisasi.
\end{soal}


\begin{soal}[Chapra Latihan 14.7]
Cari nilai maksimum dari
\begin{equation*}
f(x,y) = 4x + 2y + x^2 - 2x^4 + 2xy - 3y^2
\end{equation*}
dengan menggunakan metode \textit{steepest descent}.
Buat plot kontur
dari $f(x,y)$ dan tunjukkan titik yang dilalui ketika proses optimisasi.
Gunakan tiga titik awal yang berbeda pada saat optimisasi.
\end{soal}
\begin{soal}[Chapra Latihan 15.3]
Selesaikan permasalahan pemrograman linear berikut.
Cari nilai maksimum dari:
\begin{equation*}
f(x,y) = 1.75x + 1.25y
\end{equation*}
dengan kendala:
\begin{align*}
1.2x + 2.25y \leq 14 \\
x + 1.1y \leq 8 \\
2.5x + y \leq 9
\end{align*}
serta batas $x \geq 0$ dan $y \geq 0$.
\end{soal}


\begin{soal}[Chapra Latihan 15.4]
Selesaikan permasalahan pemrograman linear berikut.
Cari nilai maksimum dari:
\begin{equation*}
f(x,y) = 6x + 8y
\end{equation*}
dengan kendala:
\begin{align*}
5x + 2y \leq 40 \\
6x + 6y \leq 60 \\
2x + 4y \leq 32
\end{align*}
serta batas $x \geq 0$ dan $y \geq 0$.
\end{soal}


\end{document}


\documentclass[a4paper,11pt,bahasa]{extarticle} % screen setting
\usepackage[a4paper]{geometry}

%\documentclass[b5paper,11pt,bahasa]{article} % screen setting
%\usepackage[b5paper]{geometry}

%\geometry{verbose,tmargin=1.5cm,bmargin=1.5cm,lmargin=1.5cm,rmargin=1.5cm}

\geometry{verbose,tmargin=2.0cm,bmargin=2.0cm,lmargin=2.0cm,rmargin=2.0cm}

\setlength{\parskip}{\smallskipamount}
\setlength{\parindent}{0pt}

%\usepackage{cmbright}
%\renewcommand{\familydefault}{\sfdefault}

\usepackage{amsmath}
\usepackage{amssymb}

\usepackage[libertine]{newtxmath}

\usepackage[no-math]{fontspec}
\setmainfont{Linux Libertine O}

%\usepackage{fontspec}
%\usepackage{lmodern}

\setmonofont{JuliaMono-Regular}


\usepackage{hyperref}
\usepackage{url}
\usepackage{xcolor}
\usepackage{enumitem}
\usepackage{mhchem}
\usepackage{graphicx}
\usepackage{float}

\usepackage{minted}

\newminted{julia}{breaklines,fontsize=\footnotesize}
\newminted{python}{breaklines,fontsize=\footnotesize}

\newminted{bash}{breaklines,fontsize=\footnotesize}
\newminted{text}{breaklines,fontsize=\footnotesize}

\newcommand{\txtinline}[1]{\mintinline[breaklines,fontsize=\footnotesize]{text}{#1}}
\newcommand{\jlinline}[1]{\mintinline[breaklines,fontsize=\footnotesize]{julia}{#1}}
\newcommand{\pyinline}[1]{\mintinline[breaklines,fontsize=\footnotesize]{python}{#1}}

\newmintedfile[juliafile]{julia}{breaklines,fontsize=\footnotesize}
\newmintedfile[pythonfile]{python}{breaklines,fontsize=\footnotesize}
\newmintedfile[fortranfile]{fortran}{breaklines,fontsize=\footnotesize}

\usepackage{mdframed}
\usepackage{setspace}
\onehalfspacing

\usepackage{babel}
\usepackage{appendix}

\newcommand{\highlighteq}[1]{\colorbox{blue!25}{$\displaystyle#1$}}
\newcommand{\highlight}[1]{\colorbox{red!25}{#1}}

\newcounter{soal}%[section]
\newenvironment{soal}[1][]{\refstepcounter{soal}\par\medskip
   \noindent \textbf{Soal~\thesoal. #1} \sffamily}{\medskip}


\definecolor{mintedbg}{rgb}{0.95,0.95,0.95}
\BeforeBeginEnvironment{minted}{
    \begin{mdframed}[%
        topline=false,bottomline=false,%
        leftline=false,rightline=false]
}
\AfterEndEnvironment{minted}{\end{mdframed}}


\BeforeBeginEnvironment{soal}{
    \begin{mdframed}[%
        topline=true,bottomline=false,%
        leftline=true,rightline=false]
}
\AfterEndEnvironment{soal}{\end{mdframed}}


% -------------------------
\begin{document}

\title{%
{\small TF2202 Komputasi Rekayasa}\\
Interpolasi dan Pencocokan Kurva
}
\author{Tim Praktikum Komputasi Rekayasa 2022\\
Teknik Fisika\\
Institut Teknologi Bandung}
\date{}
\maketitle


\section{Polinomial Interpolasi Newton}

\begin{soal}
Implementasikan fungsi atau subroutin dalam Python untuk implementasi algoritma
pada Gambar 18.7 untuk implementasi polinomial Newton.
Uji hasil yang Anda dapatkan dengan menggunakan data-data yang diberikan pada contoh
18.2 (polinomial kuadrat) dan 18.3 (polinomial kubik) pada Chapra.
Lengkapi jawaban Anda dengan membuat plot seperti pada Gambar 18.4 dan 18.6 pada Chapra.
\end{soal}

\begin{soal}
Gunakan data pada soal Chapra 18.5 untuk mengevaluasi nilai $f(x) = \mathrm{ln}(x)$
pada $x = 2$ dengan menggunakan polinomial kubik. Coba variasikan titik-titik
yang digunakan (\textit{base points}) dan perhatikan nilai estimasi
kesalahan yang diberikan oleh
fungsi/subrutin yang sudah Anda buat pada soal sebelumnya.
\end{soal}

\begin{pythoncode}
import numpy as np
def newton_interp(x, y, xi):
    assert(len(x) == len(y))
    N = len(x) - 1  # length of array is (N + 1)
    yint = np.zeros(N+1)
    ea = np.zeros(N)
    # finite divided difference table
    fdd = np.zeros((N+1,N+1))
    for i in range(0,N+1):
        fdd[i,0] = y[i]
    for j in range(1,N+1):
        for i in range(0,N-j+1):
            fdd[i,j] = ( fdd[i+1,j-1] - fdd[i,j-1] ) / ( x[i+j] - x[i] )
    xterm = 1.0
    yint[0] = fdd[0,0]  
    for order in range(1,N+1):
        xterm = xterm * ( xi - x[order-1] )
        yint2 = yint[order-1] + fdd[0,order]*xterm
        ea[order-1] = yint2 - yint[order-1]
        yint[order] = yint2 
      
    return yint, ea  
\end{pythoncode}
\section{Polinomial Interpolasi Lagrange}
\begin{soal}
Implementasikan fungsi atau subroutin dalam Python untuk implementasi algoritma
pada Gambar 18.11 untuk implementasi polinomial Lagrange.
Uji hasil yang Anda dapatkan dengan menggunakan data-data yang diberikan pada contoh
18.2 (polinomial kuadrat) dan 18.3 (polinomial kubik) pada Chapra.
\end{soal}

\begin{soal}
Kerjakan Contoh 18.7 pada Chapra dengan menggunakan fungsi/subrutin interpolasi Lagrange
yang sudah Anda buat pada soal sebelumnya sehingga Anda dapat mereproduksi Gambar 18.12
pada Chapra. Pada Contoh 18.17 Anda diminta untuk mengestimasi kecepatan penerjun
pada $t=10$ s, yang berada di antara dua titik data terakhir.
Untuk polinom orde-1, gunakan dua titik data terakhir, untuk orde-2 gunakan tiga titik
data terakhir, dan seterusnya sampai orde-4.
\end{soal}

\begin{pythoncode}
def lagrange_interp(x, y, xx):
    assert(len(x) == len(y))
    N = len(x) - 1  # length of array is (N + 1)
    ss = 0.0
    for i in range(0,N+1):
        pp = ....
        for j in range(0,N+1):
            if i != j:
                ....
        ss = ....
    return ss
\end{pythoncode}
\section{Interpolasi Spline Kubik}

Pada spline kubik, polinomial orde-3 diberikan untuk tiap interval di antara dua titik simpul. Polinomial orde-3 ini dapat dituliskan sebagai:
$$
f_i(x) = a_i x^3 + b_i x^2 + c_i x + d_i
$$
Untuk, $n+1$ titik data, dengan $i=0,1,2,\ldots,n$, terdapat $n$ interval sehingga ada $4n$ konstanta yang harus dievaluasi. Terdapat $4n$ kondisi yang diperlukan untuk mendapatkan konstanta-konstanta tersebut. Kondisi ini dapat dituliskan sebagai berikut.
\begin{itemize}
\item Nilai fungsi harus sama pada titik simpul dalam (interior): ($2n-2$ kondisi)
\item Fungsi pertama dan terakhir harus melalui titik-titik ujung: ($2$ kondisi)
\item Turunan pertama pada titik simpul dalam harus sama: ($n-1$ kondisi)
\item Turunan kedua pada titik simpul dalam harus sama: ($n-1$ kondisi)
\item Turunan kedua pada titik-titik ujung bernilai 0: ($2$ kondisi)
\end{itemize}
Kondisi 5 mensyaratkan bahwa fungsi menjadi garis lurus pada titik-titik ujung. Spesifikasi ini mendefinisikan spline kubik natural atau alami.

Karena tiap pasang titik simpul pada interval dihubungkan oleh suatu polinomial kubik,
maka turunan kedua dari fungsi tersebut adalah fungsi linear (garis lurus).
Turunan kedua dapat direpresentasikan dengan suatu polinomial Lagrange orde-1:
$$
f''_{i}(x) = f''_i(x_{i-1})\frac{x - x_{i}}{x_{i-1} - x_{i}} +
f''_i(x_{i})\frac{x - x_{i-1}}{x_{i} - x_{i-1}}
$$
di mana $f''_{i}(x)$ adalah nilai dari turunan kedua pada sembarang titik $x$ pada interval ke-$i$.

...


Persamaan polinomial kubik untuk tiap interval:
\begin{align*}
f_{i}(x) & =
\frac{f''_{i}(x_{i-1})}{6(x_{i} - x_{i-1})}(x_{i} - x)^3 +
\frac{f''_{i}(x_{i})}{6(x_{i} - x_{i-1})}(x - x_{i-1})^3 \\
& + \left[
\frac{f(x_{i-1})}{x_{i} - x_{i-1}} - \frac{f''(x_{i-1}) (x_i - x_{i-1})}{6}
\right] (x_i - x) \\
& + \left[
\frac{f(x_{i})}{x_{i} - x_{i-1}} - \frac{f''(x_{i}) (x_i - x_{i-1})}{6}
\right] (x - x_{i-1})
\end{align*}
Persamaan ini hanya memiliki dua variabel yang tidak diketahui, yaitu turunan kedua
pada tiap ujung-ujung interval. Variabel-variabel tersebut dapat dihitung dengan menyelesaikan
persamaan berikut:
\begin{multline*}
(x_i - x_{i-1}) f''(x_{i-1}) + 2(x_{i+1} - x_{i-1}) f''(x_{i}) + (x_{i+1} - x_{i})f''(x_{i+1}) = \\
\frac{6}{x_{i+1} - x_{i}} \left[ f(x_{i+1}) - f(x_i) \right] +
\frac{6}{x_{i} - x_{i+1}} \left[ f(x_{i+1}) - f(x_i) \right]
\end{multline*}

\begin{soal}
Implementasikan fungsi atau subroutin dalam Python untuk implementasi algoritma
pada Gambar 18.18 untuk implementasi spline kubik (natural).
Uji hasil yang Anda dapatkan dengan menggunakan data-data yang diberikan pada contoh
18.10.
\end{soal}

\begin{pythoncode}
import numpy as np

def interp_nat_cubic_spline( x, y, d2x, xu ):    
    assert len(x) == len(y)
    N = len(x) - 1
    success = False
    i = 1
    
    while True:
        is_in_interval = (x[i-1] <= xu <= x[i])    
        if is_in_interval:
            c1 = d2x[i-1]/6.0/( x[i] - x[i-1] )
            c2 = ....
            c3 = ....
            c4 = ....
            #
            t1 = c1*( x[i] - xu )**3
            t2 = ....
            t3 = ....
            t4 = ....
            yu = t1 + t2 + t3 + t4
            #
            # Turunan pertama
            #
            t1 = -3.0*c1*( x[i] - xu )**2
            t2 = 3.0*c2*( xu - x[i-1] )**2
            t3 = -c3
            t4 = c4
            dy = t1 + t2 + t3 + t4
            #
            # Turunan kedua
            #
            t1 = 6.0*c1*( x[i] - xu )
            t2 = 6.0*c2*( xu - x[i-1] )
            d2y = t1 + t2
            #
            success = True
            #
        else:
            #
            i = i + 1
        
        if i == (N + 1) or success:
            break # break out of the loop
    
    if not success:
        raise RuntimeError("xu is outside range of spline")
        
    return yu, dy, d2y

def decomp_trid(e, f, g):
    N = len(e)
    assert N == len(f)
    assert N == len(g)
    #
    for k in range(1,N):
        e[k] = e[k]/f[k-1]
        f[k] = f[k] - e[k]*g[k-1]
    return
    
# should be called after calling decomp_trid
def subs_trid(e, f, g, r):
    N = len(e)
    assert N == len(f)
    assert N == len(g)
    # Forward subs
    for k in range(1,N):
        r[k] = r[k] - e[k]*r[k-1]
    # back subs
    x = np.zeros(N)
    x[N-1] = r[N-1]/f[N-1]
    for k in range(N-2,-1,-1):
        x[k] = ( r[k] - g[k]*x[k+1] ) / f[k]
    return x    
    
def gen_trid_matrix(x, y):
    assert len(x) == len(y)
    N = len(x) - 1
    #
    e = np.zeros(N-1)
    f = np.zeros(N-1)
    g = np.zeros(N-1)
    r = np.zeros(N-1)
    #
    f[0] = 2.0*( x[2] - x[0] )
    g[0] = x[2] - x[1]
    r[0] = 6.0/( x[2] - x[1] ) * ( y[2] - y[1] )
    r[0] = r[0] + 6.0/( x[1] - x[0] ) * ( y[0] - y[1] )
    #
    for i in range(2,N-1):
        e[i-1] = x[i] - x[i-1]
        f[i-1] = 2.0*( x[i+1] - x[i-1] )
        g[i-1] = x[i+1] - x[i]
        r[i-1] = 6.0/( x[i+1] - x[i] ) * ( y[i+1] - y[i] )
        r[i-1] = r[i-1] + 6.0/( x[i] - x[i-1] ) * ( y[i-1] - y[i] )
    #    
    e[N-2] = x[N-1] - x[N-2]
    f[N-2] = 2.0*( x[N] - x[N-2] )
    r[N-2] = 6.0/( x[N] - x[N-1] ) * ( y[N] - y[N-1] )
    r[N-2] = r[N-2] + 6.0/( x[N-1] - x[N-2] ) * ( y[N-2] - y[N-1] )
    
    return e, f, g, r    
\end{pythoncode}


\section{Regresi Linear}
Diberikan pasangan data: $(x_1, y_1), (x_2, y_2), \ldots, (x_n, y_n)$, kita ingin
mencari persamaan kurva garis lurus yang paling cocok:
\begin{equation}
y = a_0 + a_1 x + e
\end{equation}
di mana $a_0$ dan $a_1$ adalah parameter kurva dan $e$ adalah error atau residual.
Error dapat dituliskan sebagai:
\begin{equation}
e = y - a_0 - a_1 x
\end{equation}
Jumlah kuadrat dari error untuk seluruh data yang diberikan adalah:
\begin{equation}
S_r = \sum_{i=1}^{n} e_{i}^2 = \sum_{i=1}^{n} \left( y_{i} - a_0 - a_1 x_{i} \right)^2
\end{equation}
$a_0$ dan $a_1$ yang meminimumkan $S_r$ dapat dicari dengan syarat:
\begin{align}
\frac{\partial S_r}{\partial a_0} & = -2 \sum \left( y_i - a_0 - a_1 x_i \right) = 0 \\
\frac{\partial S_r}{\partial a_1} & = -2 \sum x_{i} \left( y_i - a_0 - a_1 x_i \right) = 0
\end{align}



Standard error of the estimate:
\begin{equation}
s_{y/x} = \sqrt{\frac{S_r}{n - 2}}
\end{equation}


\begin{soal}
Implementasikan fungsi atau subroutin dalam Python untuk implementasi algoritma
pada Gambar 17.6 untuk implementasi regresi linear. Uji dengan menggunakan data
pada Contoh 17.1.
\end{soal}

\section{Regresi Linear (Notasi Matriks-Vektor)}

Tinjau model linear berikut:
\begin{equation}
y = f(x; w_{0}, w_{1}) = w_{0} + w_{1}x
\end{equation}
di mana $w_{1}$ (atau kemiringan atau \textit{slope}) dan $w_{0}$
(titik potong sumbu $y$ atau \textit{intercept}) adalah parameter model.
Misalnya, diberikan suatu nilai atau input $x_{n}$, kita dapat menghitung output dari model
sebagai:
\begin{equation}
y_{n} = f(x_{n}; w_{0}, w_{1}) = w_{0} + w_{1}x_{n}
\label{eq:linmodel1}
\end{equation}
Definisikan:
\begin{equation}
\mathbf{x}_{n} = \begin{bmatrix}
1 \\
x_{n}
\end{bmatrix}
,\,\,\,%
\mathbf{w} = \begin{bmatrix}
w_{0} \\
w_{1}
\end{bmatrix}
\end{equation}
Dengan menggunakan notasi ini, model linear pada Persamaan \eqref{eq:linmodel1} dapat ditulis
menjadi:
\begin{equation}
y_{n} = f(x_{n}; w_0, w_1) = \mathbf{w}^{\mathsf{T}} \mathbf{x}_{n}
\label{eq:linmodel2}
\end{equation}
Dalam bentuk matriks-vektor, dapat dituliskan sebagai berikut:
\begin{equation}
\mathbf{y} = \mathbf{X}\mathbf{w}
\end{equation}
dengan $\mathbf{X}$ adalah matriks input:
\begin{equation}
\mathbf{X} = \begin{bmatrix}
\mathbf{x}^{\mathsf{T}}_{1} \\
\mathbf{x}^{\mathsf{T}}_{2} \\
\vdots \\
\mathbf{x}^{\mathsf{T}}_{N}
\end{bmatrix} =
\begin{bmatrix}
1 & x_{1} \\
1 & x_{2} \\
\vdots & \vdots \\
1 & x_{N} \\
\end{bmatrix}
\end{equation}
dan $\mathbf{y}$ adalah vektor kolom:
\begin{equation}
\mathbf{y} =
\begin{bmatrix}
y_{1} \\
y_{2} \\
\vdots \\
y_{N} \\
\end{bmatrix}
\end{equation}

Diberikan himpunan pasangan data
$\left\{(x_{1},y_{1}), (x_{2}, y_{2}), \ldots, (x_{n}, y_{n})\right\}$
kita ingin
mencari parameter $\mathbf{w}$ yang meminimumkan rata-rata kesalahan kuadrat
$\mathcal{L}$ yang didefinisikan sebagai:
\begin{equation}
\mathcal{L} \equiv \frac{1}{N} \sum_{n=1}^{N} \left( y_{n} - \mathbf{w}^{\mathsf{T}}
\mathbf{x}_{n} \right)^2
\end{equation}
Dapat ditunjukkan bahwa parameter yang membuat $\mathcal{L}$ menjadi mininum adalah
\begin{equation}
\mathbf{w} = \left(\mathbf{X}^{\mathsf{T}}\mathbf{X} \right)^{-1} \mathbf{X}^{\mathsf{T}} \mathbf{y}
\label{eq:w_vektor}
\end{equation}

\begin{soal}
Buat fungsi/subrutin dalam Python untuk membuat matriks $\mathbf{X}$ dengan argumen
input $\mathbf{x}$ dan $\mathbf{y}$ dan menghitung $\mathbf{w}$ berdasarkan
persamaan \eqref{eq:w_vektor}.
Uji fungsi yang sudah Anda buat dengan menggunakan data pada Contoh 17.1 dan bandingkan
parameter \textit{slope} dan \textit{intercept} yang Anda dapatkan pada soal ini dan
soal sebelumnya. Anda dapat menggunakan \pyinline{np.linalg.inv} untuk menghitung
invers matriks atau \pyinline{np.linalg.solve} jika Anda mengubah permasalahan
ini menjadi sistem persamaan linear.
\end{soal}

Kita juga dapat menggunakan Persamaan \eqref{eq:w_vektor} untuk model linear
\footnote{Model linear didefinisikan sebagai model yang parameternya linear
(berpangkat satu)} polinom lebih tinggi dari satu. Misalkan, pada model
polinom kuadrat:
\begin{equation}
y = w_{0} + w_{1} x + w_{2} x^2
\end{equation}
matriks $\mathbf{X}$ menjadi:
\begin{equation}
\mathbf{X} = \begin{bmatrix}
1 & x_{1} & x_{1}^2 \\
1 & x_{2} & x_{2}^2 \\
\vdots & \vdots & \vdots \\
1 & x_{N} & x_{N}^2
\end{bmatrix}
\end{equation}
dan vektor $\mathbf{w}$ menjadi:
\begin{equation}
\mathbf{w} = 
\begin{bmatrix}
w_{0} \\
w_{1} \\
w_{2}
\end{bmatrix}
\end{equation}

Formula yang sama juga dapat digunakan untuk model multilinear dengan
dua variabel independen $x^{(1)}$ dan $x^{(2)}$ (tanda $^{(1)}$ dan $^{(2)}$ bukan
menyatakan pangkat, namun variabel yang berbeda)
\footnote{Silakan gunakan variabel lain seperti $t$, $x$, $y$, $z$}:
\begin{equation}
y = w_{0} + w_{1} x^{(1)} + w_{2} x^{(2)}
\end{equation}
di mana sekarang matrix $\mathbf{X}$ menjadi:
\begin{equation}
\mathbf{X} = \begin{bmatrix}
1 & x^{(1)}_{1} & x^{(2)}_{1} \\
1 & x^{(1)}_{2} & x^{(2)}_{2} \\
\vdots & \vdots & \vdots \\
1 & x^{(1)}_{N} & x^{(2)}_{N}
\end{bmatrix}
\end{equation}

\begin{soal}
Kembangkan fungsi/subrutin Python yang sudah Anda buat sehingga dapat menerima argumen
opsional \pyinline{m}, di mana \pyinline{m >= 1} adalah orde polinomial yang ingin digunakan.
Uji dengan menggunakan data pada Contoh 17.5 di buku Chapra.
\end{soal}

\begin{soal}
Modifikasi fungsi/subrutin Python sudah Anda buat untuk regresi linear sehingga
dapat digunakan untuk regresi multilinear dan uji fungsi yang Anda buat
dengan menggunakan data pada Contoh 17.6 di buku Chapra.
\end{soal}



\section{Soal Tambahan}

\begin{soal}[Chapra Latihan 18.5]

Diberikan data berikut:

{\centering
\begin{tabular}{|c|cccccc|}
\hline
$x$    & 1.6 & 2 & 2.5 & 3.2 & 4 & 4.5 \\
$f(x)$ & 2   & 8 & 14  & 15  & 8 & 2 \\
\hline
\end{tabular}
\par}

\begin{enumerate}[label=(\alph*)]
\item Hitung $f(2.8)$ dengan menggunakan polinomial interpolasi Newton dengan orde 1 sampai 3.
\item Gunakan Pers. (18.18) pada Chapra untuk mengestimasi kesalahan untuk setiap prediksi.
\end{enumerate}
\end{soal}

\begin{soal}[Chapra Latihan 18.11]
Gunakan interpolasi invers dengan menggunakan polinomial interpolasi kubik dan
metode bagi dua (\textit{bisection}) untuk menentukan nilai $x$ yang memenuhi
$f(x) = 0.23$ untuk data dalam tabel berikut.

{\centering
\begin{tabular}{|c|cccccc|}
\hline
x & 2   &   3   & 4    & 5 & 6 & 7 \\
y & 0.5 & 0.333 & 0.25 & 0.2 & 0.1667 & 1.1429 \\
\hline
\end{tabular}
\par}
\end{soal}

\begin{soal}{Chapra Latihan 18.26}
Fungsi Runge dapat dinyatakan sebagai berikut:
\begin{equation*}
f(x) = \frac{1}{1 + 25x^2}
\end{equation*}
\begin{enumerate}[label=(\alph*)]
\item Buat plot dari fungsi tersebut interval dari $x=-1$ sampai $x=1$.
\item Buat polinomial interpolasi Lagrange orde 4 dengan menggunakan nilai
fungsi yang disampel secara seragam: $x = -1, -0.5, 0, 0.5, 1$. Buat juga
plot dari polinomial tersebut. Gunakan untuk menghitung nilai $f(0.8)$.
\item Ulangi bagian sebelumnya dengan menggunakan polinom orde 5 sampai 10.
\end{enumerate}
\end{soal}

\begin{soal}[Chapra Latihan 18.27]
Fungsi \textit{humps} dapat ditulis sebagai berikut.
\begin{equation*}
f(x) = \frac{1}{(x - 0.3)^2 + 0.01} + \frac{1}{(x - 0.9)^2 + 0.04} - 6
\end{equation*}
Hitung nilai fungsi ini pada titik-titik dalam interval x = 0 sampai x = 1
dengan jarak antar titik 0.1. Gunakan interpolasi spline kubik pada data
yang Anda hasilkan dan buat plot dari fungsi \textit{humps} dengan
hasil interpolan spline yang Anda dapatkan.
\end{soal}


\begin{soal}[Chapra Latihan 17.4]
Gunakan regresi kuadrat terkecil untuk mencocokkan garis lurus
ke data berikut.

{\centering
\begin{tabular}{|c|ccccccccccc|}
\hline
$x$ & 6 & 7 & 11 & 15 & 17 & 21 & 23 & 29 & 29 & 37 & 39 \\
$x$ & 29 & 21 & 29 & 14 & 21 & 15 & 7 & 7 & 13 & 0 & 3 \\
\hline
\end{tabular}
\par}

Plot data dan garis lurus (persamaan linear) yang Anda dapatkan
dalam satu plot. Hitung juga koefisien korelasi dan determinasi.
\end{soal}




\begin{soal}[Chapra Latihan 17.6]
Gunakan regresi kuadrat terkecil untuk mencocokkan polinomial
orde-1 dan orde-2

{\centering
\begin{tabular}{|c|ccccccccc|}
\hline
$x$ & 1 & 2 & 3 & 4 & 5 & 6 & 7 & 8 & 9 \\
$x$ & 1 & 1.5 & 2 & 3 & 4 & 5 & 8 & 10 & 13 \\
\hline
\end{tabular}
\par}
\end{soal}

\begin{soal}[Chapra Latihan 17.8]

Lakukan pencocokan model pangkat:
\begin{equation*}
y = a x^{b}
\end{equation*}
pada data berikut.

{\centering
\begin{tabular}{|c|cccc cccc cc|}
\hline
$x$ & 2.5 & 3.5 & 5   & 6   & 7.5 & 10  & 12.5 & 15  & 17.5 & 20 \\
$y$ & 13  & 11  & 8.5 & 8.2 & 7   & 6.2 & 5.2  & 4.8 & 4.6  & 4.3 \\
\hline
\end{tabular}
\par}

Gunakan persamaan yang diperoleh utuk memprediksi keluaran model jika
$x = 9$.
Jelaskan transformasi apa yang harus Anda buat untuk mengubah permasalahan
ini menjadi permasalahan regresi linear.
Plot data dan kurva yang dihasilkan pada koordinat $x-y$ dan koordinat di mana
model menjadi linear (Lihat Gambar 17.9).
\end{soal}

\begin{soal}[Chapra Latihan 17.9]

Lakukan pencocokan model eksponensial:
\begin{equation*}
y = \alpha_{1} e^{\beta_{1}x}
\end{equation*}
pada data berikut.

{\centering
\begin{tabular}{|c|ccccccccc|}
\hline
$x$ & 0.4 & 0.8 & 1.2 & 1.6 & 2 & 2.3 \\
$y$ & 800 & 975 & 1500 & 1950 & 2900 & 3600 \\
\hline
\end{tabular}
\par}
Jelaskan transformasi apa yang harus Anda buat untuk mengubah permasalahan
ini menjadi permasalahan regresi linear.
Plot data dan kurva yang dihasilkan pada koordinat $x-y$ dan koordinat di mana
model menjadi linear (Lihat Gambar 17.9).
\end{soal}

\begin{soal}
Titik simpul Chebyshev jenis pertama didefinisikan pada selang $[-1,1]$ dan
dapat dituliskan sebagai berikut:
\begin{equation}
t_{k} = -\cos\left( \frac{k\pi}{n} \right), \,\,\,\, k = 0, \ldots, n
\end{equation}
Gunakan titik simpul Chebyshev sebagai ganti dari titik-titik $x$ yang memiliki jarak seragam
untuk melakukan interpolasi dengan polinom orde 10 dan orde 20 untuk fungsi Runge
pada interval $[-1,1]$.
Bandingkan hasilnya jika Anda menggunakan titik-titik $x$ yang memiliki jarak
seragam.
\end{soal}

\end{document}


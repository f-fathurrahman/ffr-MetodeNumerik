\documentclass[a4paper,11pt,bahasa]{extarticle} % screen setting
\usepackage[a4paper]{geometry}

%\documentclass[b5paper,11pt,bahasa]{article} % screen setting
%\usepackage[b5paper]{geometry}

%\geometry{verbose,tmargin=1.5cm,bmargin=1.5cm,lmargin=1.5cm,rmargin=1.5cm}

\geometry{verbose,tmargin=2.0cm,bmargin=2.0cm,lmargin=2.0cm,rmargin=2.0cm}

\setlength{\parskip}{\smallskipamount}
\setlength{\parindent}{0pt}

%\usepackage{cmbright}
%\renewcommand{\familydefault}{\sfdefault}

\usepackage{amsmath}
\usepackage{amssymb}

\usepackage[libertine]{newtxmath}

\usepackage[no-math]{fontspec}
\setmainfont{Linux Libertine O}

%\usepackage{fontspec}
%\usepackage{lmodern}

\setmonofont{JuliaMono-Regular}


\usepackage{hyperref}
\usepackage{url}
\usepackage{xcolor}
\usepackage{enumitem}
\usepackage{mhchem}
\usepackage{graphicx}
\usepackage{float}

\usepackage{minted}

\newminted{julia}{breaklines,fontsize=\footnotesize}
\newminted{python}{breaklines,fontsize=\footnotesize}

\newminted{bash}{breaklines,fontsize=\footnotesize}
\newminted{text}{breaklines,fontsize=\footnotesize}

\newcommand{\txtinline}[1]{\mintinline[breaklines,fontsize=\footnotesize]{text}{#1}}
\newcommand{\jlinline}[1]{\mintinline[breaklines,fontsize=\footnotesize]{julia}{#1}}
\newcommand{\pyinline}[1]{\mintinline[breaklines,fontsize=\footnotesize]{python}{#1}}

\newmintedfile[juliafile]{julia}{breaklines,fontsize=\footnotesize}
\newmintedfile[pythonfile]{python}{breaklines,fontsize=\footnotesize}
\newmintedfile[fortranfile]{fortran}{breaklines,fontsize=\footnotesize}

\usepackage{mdframed}
\usepackage{setspace}
\onehalfspacing

\usepackage{babel}
\usepackage{appendix}

\newcommand{\highlighteq}[1]{\colorbox{blue!25}{$\displaystyle#1$}}
\newcommand{\highlight}[1]{\colorbox{red!25}{#1}}

\newcounter{soal}%[section]
\newenvironment{soal}[1][]{\refstepcounter{soal}\par\medskip
   \noindent \textbf{Soal~\thesoal. #1} \sffamily}{\medskip}


\definecolor{mintedbg}{rgb}{0.95,0.95,0.95}
\BeforeBeginEnvironment{minted}{
    \begin{mdframed}[%
        topline=false,bottomline=false,%
        leftline=false,rightline=false]
}
\AfterEndEnvironment{minted}{\end{mdframed}}


\BeforeBeginEnvironment{soal}{
    \begin{mdframed}[%
        topline=true,bottomline=false,%
        leftline=true,rightline=false]
}
\AfterEndEnvironment{soal}{\end{mdframed}}


\begin{document}

\title{%
{\small TF2202 Komputasi Rekayasa}\\
Akar Persamaan Nonlinear
}
\author{Tim Praktikum Komputasi Rekayasa 2023\\
Teknik Fisika\\
Institut Teknologi Bandung}
\date{}
\maketitle

\section{Chapra Contoh 5.3}

Kita ingin mencari nilai dari $c$ dari persamaan
\begin{equation*}
v(t) = \frac{gm}{c}(1 - e^{-(c/m)t})
\end{equation*}
sehingga untuk $v(t=10) = 40$, dengan $m=9.81$ dan $m=68.1$.
Nilai $c$ dapat dicari sebagai akar dari persamaan:
\begin{equation*}
f(c) \equiv \frac{gm}{c}(1 - e^{-(c/m)t}) - v(t)
\end{equation*}

Kita akan menggunakan metode \textit{bisection} untuk mengaproksimasi akar atau solusi dari
persamaan $f(c) = 0$. Diberikan dua nilai $x_{l}$ (lower) dan $x_{u}$ (upper), di mana
$x_{u} > x_{l}$ dan $f(c=x_{l})f(c=x_{u}) < 0$, metode \textit{bisection} memberikan
aproksimasi akar $x_{r}$ sebagai berikut:
\begin{equation}
x_{r} = \frac{x_{l} + x_{u}}{2}
\end{equation}

Kode Python berikut ini mengilustrasikan penggunaan 
\begin{pythoncode}
import numpy as np

m = 68.1 # mass, kg
v = 40.0 # velocity, m/s
t = 10.0 # time, s
g = 9.81
    
def f(c):
    return ... # lengkapi
    
x_true = 14.8011 # from the text
    
# Initial guess
xl = 12.0
xu = 16.0
    
# First iteration
print("\n1st iteration: ")
print("xl = %f, xu = %f" % (xl, xu))
print("f(xl) = %f, f(xu) = %f" % (f(xl), f(xu)))
xr = # ... lengkapi
print("xr = ", xr)
ε_t = abs(xr - x_true)/x_true*100 # error in percent
print("ε_t = %.1f %%" % ε_t)

# Determine new xr should replace xu or xl (make new interval)
if f(xl)*f(xr) < 0:
    xu = xr
else:
    xl = xr
    
# Second iteration
print("\n2nd iteration: ")
print("xl = %f, xu = %f" % (xl, xu))
print("f(xl) = %f, f(xu) = %f" % (f(xl), f(xu)))
xr = (xl + xu)/2
print("xr = ", xr)
ε_t = abs(xr - x_true)/x_true*100 # error in percent
print("ε_t = %.1f %%" % ε_t)

if f(xl)*f(xr) < 0:
    xu = xr
else:
    xl = xr
    
# Third iteration
# ... teruskan jika diperlukan

# Jika xr merupakan akar dari f, maka f(xr) harus mendekati 0
# Tampilkan hasil dari f(xr) di sini
\end{pythoncode}

\begin{soal}
Lengkapi kode untuk ilustrasi penggunakan bisection tersebut. Lakukan iterasi sampai
suatu kriteria tertentu yang Anda tentukan. Anda boleh menggunakan loop
untuk menghindari pengulangan kode.
\end{soal}

\section{Chapra Contoh 5.5}
Dengan menggunakan metode \textit{regula falsi} aproksimasi akar diberikan oleh:
\begin{equation}
x_{r} = x_{u} - \frac{f(x_u)(x_{l} - x_{u})}{f(x_l) - f(x_u)}
\end{equation}

\begin{soal}
Lakukan modifikasi pada kode yang diberikan di soal sebelumnya sehingga dapat
memberikan ilustrasi penggunaan metode \textit{regula falsi}. Bandingkan hasil yang
Anda dapatkan dengan metode \textit{bisection}.
\end{soal}

\section{Metode \textit{regula falsi}}

\textbf{Chapra Contoh 5.5}
Dengan menggunakan metode \textit{regula falsi} aproksimasi akar diberikan oleh:
\begin{equation}
x_{r} = x_{u} - \frac{f(x_u)(x_{l} - x_{u})}{f(x_l) - f(x_u)}
\end{equation}

\begin{soal}
Lakukan modifikasi pada kode yang diberikan di soal sebelumnya sehingga dapat
memberikan ilustrasi penggunaan metode \textit{regula falsi}. Bandingkan hasil yang
Anda dapatkan dengan metode \textit{bisection}.
\end{soal}
\section{Chapra Contoh 5.6}

Kita ingin menghitung akar dari persamaan:
\begin{equation}
f(x) = x^{10} - 1
\label{eq:chapra-example-5.6}
\end{equation}
dengan menggunakan metode \textit{bisection} dan \textit{regula falsi}.

\begin{soal}
Aplikasikan metode \textit{bisection} dan \textit{regula falsi} untuk mendapatkan
aproksimasi akar dari persamaan \eqref{eq:chapra-example-5.6}
\end{soal}

\section{Chapra Contoh 6.1}

Pada soal ini, kita ingin mencari akar dari $f(x) = e^{-x} - x$ dengan
menggunakan metode iterasi \textit{fixed-point}. Metode ini juga dikenal
dengan nama iterasi satu titik atau substitusi berurutan. Metode ini
bekerja dengan cara mengubah persamaan awal $f(x) = 0$ menjadi
$x = g(x)$. Untuk contoh $f(x)$ yang diberikan kita memiliki:
\begin{equation*}
x = e^{-x}
\end{equation*}
Dimulai dari tebakan awal $x_{0} = 0$, kita dapat melakukan iterasi
\textit{fixed-point} dengan menggunakan kode Python berikut.

\begin{pythoncode}
# Simple fixed-point iteration
    
import numpy as np # math module also can be used
    
def g(x):
    return .... # lengkapi
    
# Initial guess
x = 0.0
x_true = 0.56714329
    
print()
print("Initial point: x = ", x)
print()
for i in range(1,11):
    xnew = .... # lengkapi
    ε_a = np.abs( (xnew - x)/xnew )*100 # in percent
    ε_t = np.abs( (x_true - xnew)/x_true )*100
    print("%3d %10.6f %10.2f%% %10.2f%%" % (i, xnew, ε_a, ε_t))
    x = xnew    
\end{pythoncode}
\section{Chapra Contoh 6.3}
Pada soal ini, kita akan mencari akar persamaan $f(x) = e^{-x} - x$ dengan
menggunakan metode Newton-Raphson. Diberikan suatu nilai tebakan akar
awal $x_{0}$, metode ini akan memberikan skema iterasi berikut:
\begin{equation}
x_{i+1} = x_{i} - \frac{f(x_{i})}{f'(x_{i})}
\end{equation}

Berikut ini adalah contoh implementasi metode Newton-Raphson untuk
mencari akar dari $f(x)$:
\begin{pythoncode}
import ... # lengkapi

def f(x): # fungsi f(x)
    return ...  # lengkapi
    
def df(x): # turunan pertama dari f(x)
    return ... # lengkapi

x = 0.0 # tebakan akar awal
for i in range(1,6):
    xnew = ... # lengkapi
    fxnew = f(xnew)
    print("%3d %18.10f %18.10e" % (i, xnew, fxnew))
    x = xnew    
\end{pythoncode}

\begin{soal}
Lengkapi program Python untuk metode Newton-Raphson di atas.
Implementasikan program Anda sehingga
dapat melakukan iterasi sampai nilai kesalahan menjadi lebih kecil dari
suatu nilai tertentu yang diberikan. Anda dapat menggunakan loop \txtinline{while}
atau loop \txtinline{for} dengan jumlah iterasi maksimum tertentu.
\end{soal}
\section{Chapra Contoh 6.6}
Salah satu kekurangan dari metode Newton-Raphson adalah perlunya menghitung turunan
pertama dari fungsi yang ingin dicari akarnya. Hal ini biasanya tidak dapat 
Metode \textit{secant} memiliki bentuk iterasi yang mirip dengan metode
Newton-Raphson, namun turunan fungsi diaproksimasi dengan menggunakan:
\begin{equation*}
f'(x_{i}) \approx \frac{f(x_{i-1}) - f(x_{i})}{x_{i-1} - x_{i}}
\end{equation*}
sehingga kita memperoleh bentuk iterasi sebagai berikut:
\begin{equation}
x_{i+1} = x_{i} - \frac{f(x_{i}) (x_{i-1} - x_{i})}{f(x_{i-1}) - f(x_{i})}
\end{equation}
Metode \textit{secant} memerlukan dua titik $x_{-1}$ dan $x_{0}$
sebagai tebakan akar awal. Berbeda
dengan metode \textit{bracketing}, nilai fungsi $f(x_{-1})$ dan $f(x_{0})$
tidak diharuskan memiliki tanda yang berbeda.

\begin{pythoncode}
# ... definisi f dan df sama dengan contoh Newton-Raphson
x0 = 0.0
x1 = 1.0
for i in range(1,6):
    # approximation of derivative of f(x)
    dfx = (f(x0) - f(x1))/(x0 - x1)
    #
    xnew = ... # lengkapi
    fxnew = f(xnew)
    print("%3d %18.10f %18.10e" % (i, xnew, fxnew))
    x0 = x1
    x1 = xnew    
\end{pythoncode}

\begin{soal}
Lengkapi program Python untuk metode \textit{secant} di atas.
Implementasikan program Anda sehingga
dapat melakukan iterasi sampai nilai kesalahan menjadi lebih kecil dari
suatu nilai tertentu yang diberikan. Anda dapat menggunakan loop \txtinline{while}
atau loop \txtinline{for} dengan jumlah iterasi maksimum tertentu.
\end{soal}
\section{Chapra Contoh 6.8}

Kita dapat melakukan modifikasi pada metode \textit{secant} sehingga hanya
memerlukan input satu tebakan akar awal. Aproksimasi turunan fungsi
dihitung dengan cara memberikan perturbasi $\delta$ dari input
\begin{equation*}
f'(x_{i}) \approx \frac{f(x_{i}+\delta) - f(x_{i})}{\delta}
\end{equation*}
sehingga kita memperoleh bentuk iterasi sebagai berikut:
\begin{equation}
x_{i+1} = x_{i} - \frac{\delta f(x_{i})}{f(x_{i}+\delta) - f(x_{i})}
\end{equation}

\begin{pythoncode}
# ... definisi f dan df sama dengan contoh Newton-Raphson
x = 1.0
δ = 0.01
for i in range(1,6):
    # approximation of derivative of f(x)
    dfx = ... # lengkapi
    xnew = ... # lengkapi
    fxnew = f(xnew)
    print("%3d %18.10f %18.10e" % (i, xnew, fxnew))
    x = xnew    
\end{pythoncode}

\begin{soal}
Lengkapi program Python untuk modifikasi metode metode \textit{secant} di atas.
Implementasikan program Anda sehingga
dapat melakukan iterasi sampai nilai kesalahan menjadi lebih kecil dari
suatu nilai tertentu yang diberikan. Anda dapat menggunakan loop \txtinline{while}
atau loop \txtinline{for} dengan jumlah iterasi maksimum tertentu.
\end{soal}
\section{Chapra Contoh 6.10}
Untuk kasus di mana persamaan nonlinear yang memiliki akar
lebih dari satu, metode Newton-Raphson dapat mengalami kesulitan
untuk konvergen. Metode Newton-Raphson perlu untuk dimodifikasi
sebagai berikut:
\begin{equation}
x_{i+1} = x_{i} - m\frac{f(x_{i})}{f'(x_{i})}
\label{eq:newton_raphson_2}
\end{equation}
di mana $m$ adalah multiplisitas dari akar.
Alternatif lain adalah dengan mendefinisikan fungsi:
\begin{equation*}
u(x) = \frac{f(x)}{f'(x)}
\end{equation*}
yang memiliki lokasi akar yang sama dengan fungsi awal $f(x)$.
Dengan definisi tersebut, aplikasi metode Newton-Raphson memberikan skema
iterasi sebagai berikut.
\begin{equation}
x_{i+1} = x_{i} - \frac{f(x_{i}) f'(x_{i})}{[f'(x_{i})]^{2} - f(x_{i})f''(x_{i})}
\label{eq:newton_raphson_3}
\end{equation}

\begin{soal}
Buat implementasi dengan Python untuk mengimplementasikan modifikasi metode
Newton-Raphson untuk akar dengan multiplisitas lebih dari satu
(menggunakan persamaan \eqref{eq:newton_raphson_2} dan \eqref{eq:newton_raphson_3})
dan uji pada persamaan nonlinear $f(x) = x^3 - 5x^2 + 7x - 3$ dengan nilai tebakan akar
awal $x_{0} = 0$. Bandingkan hasil yang Anda dapatkan jika menggunakan metode
Newton-Raphson tanpa modifikasi (persaman \eqref{eq:newton_raphson_1}).
\end{soal}
\section{Chapra Contoh 6.12}
Tinjau suatu sistem persamaan nonlinear berikut:
\begin{align*}
u(x,y) = x^{2} + xy - 10 = 0 \\
v(x,y) = y + 3xy^{2} - 57 = 0
\end{align*}
Dengan menggunakan ekstensi dari metode Newton-Raphson untuk 2 variabel
diperoleh skema iterasi berikut:
\begin{align}
x_{i+1} = x_{i} - \frac{
u_{i}\dfrac{\partial v_{i}}{\partial y} - 
v_{i}\dfrac{\partial u_{i}}{\partial y}
}%
{\dfrac{\partial u_{i}}{\partial x}\dfrac{\partial v_{i}}{\partial y} -
 \dfrac{\partial u_{i}}{\partial y}\dfrac{\partial v_{i}}{\partial x} } \\
y_{i+1} = y_{i} - \frac{
v_{i}\dfrac{\partial u_{i}}{\partial x} - 
u_{i}\dfrac{\partial v_{i}}{\partial x}
}%
{\dfrac{\partial u_{i}}{\partial x}\dfrac{\partial v_{i}}{\partial y} -
 \dfrac{\partial u_{i}}{\partial y}\dfrac{\partial v_{i}}{\partial x} }
\end{align}

Kode berikut ini mengimplementasikan metode Newton-Raphson untuk dua variabel:
\begin{pythoncode}
def u(x,y):
    return x**2 + x*y - 10

def dudx(x,y):
    return 2*x + y

def dudy(x,y):
    return x

def v(x,y):
    return ... # lengkapi

def dvdx(x,y):
    return ... # lengkapi

def dvdy(x,y):
    return ... # lengkapi

# Guess solutions
x = 1.5
y = 3.5

for i in range(1,5):
    # Jacobian matrix elements
    J11 = dudx(x,y)
    J12 = dudy(x,y)
    J21 = dvdx(x,y)
    J22 = dvdy(x,y)
    detJ = J11*J22 - J12*J21
    #
    ui = u(x,y)
    vi = v(x,y)
    # Update x
    xnew = .... # lengkapi
    ynew = .... # lengkapi
    print("x, y = %18.10f %18.10f" % (xnew, ynew))
    # TODO: Check convergence
    x = xnew
    y = ynew
\end{pythoncode}

\begin{soal}
Lengkapi program Python untuk metode Newton-Raphson untuk dua variabel.
Implementasikan program Anda sehingga
dapat melakukan iterasi sampai nilai kesalahan menjadi lebih kecil dari
suatu nilai tertentu yang diberikan. Anda dapat menggunakan loop \txtinline{while}
atau loop \txtinline{for} dengan jumlah iterasi maksimum tertentu.
\end{soal}

Untuk variabel yang lebih dari dua, metode Newton-Raphson dapat dituliskan dengan
menggunakan notasi matriks-vektor.
Persamaan yang diimplementasikan adalah (lihat slide 27 oleh Pak Haris):
\begin{equation}
\mathbf{X}_{i+1} = \mathbf{X}_{i} - \mathbf{J}_{i}^{-1} \mathbf{F}(\mathbf{X}_{i})
\end{equation}
di mana $\mathbf{J}$ adalah matriks Jacobian:
\begin{equation}
\mathbf{J} = \begin{bmatrix}
\dfrac{\partial f_{1}}{\partial x_{1}} & \dfrac{\partial f_{1}}{\partial x_{2}} & \cdots &
\dfrac{\partial f_{1}}{\partial x_{n}} \\[0.4cm]
\dfrac{\partial f_{2}}{\partial x_{1}} & \dfrac{\partial f_{2}}{\partial x_{2}} & \cdots &
\dfrac{\partial f_{2}}{\partial x_{n}} \\[0.4cm]
\vdots & \vdots &  & \vdots \\[0.4cm]
\dfrac{\partial f_{n}}{\partial x_{1}} & \dfrac{\partial f_{n}}{\partial x_{2}} & \cdots &
\dfrac{\partial f_{n}}{\partial x_{n}}
\end{bmatrix}
\label{eq:jacobian}
\end{equation}
Turunan parsial pada persaman \eqref{eq:jacobian} dievaluasi pada titik $\mathbf{X}_{i}$.
Nilai fungsi $\mathbf{F}(\mathbf{X})$ dan $\mathbf{X}$ direpresentasikan sebagai vektor kolom:
\begin{equation}
\mathbf{X} = \begin{bmatrix}
x_{1} \\
x_{2} \\
\vdots \\
x_{n}
\end{bmatrix}
\end{equation}
dan
\begin{equation}
\mathbf{F}(\mathbf{X}) = \begin{bmatrix}
f_{1}(\mathbf{X}) \\
f_{2}(\mathbf{X}) \\
\vdots \\
f_{n}(\mathbf{X})
\end{bmatrix}
\end{equation}

Kode berikut ini adalah alternatif implementasi dari metode Newton-Raphson
dengan menggunakan notasi matriks-vektor.
\begin{pythoncode}
import numpy as np

def f(X):
    x = X[0]
    y = X[1]
    f1 = x**2 + x*y - 10
    f2 = ... # lengkapi 
    return np.array([f1,f2])

def calc_jac(X):
    x = X[0]
    y = X[1]
    dudx = ... # lengkapi
    dudy = ... # lengkapi
    dvdx = ... # lengkapi
    dvdy = ... # lengkapi
    return np.array([
        [dudx, dudy],
        [dvdx, dvdy]
    ])
    
X = np.array([1.5, 3.5]) # initial guess
for i in range(1,6): # change this if needed
    fX = f(X)
    nfX = np.linalg.norm(fX) # calculate norm of f(X)
    print("X = ", X, "nfX = ", nfX)
    if nfX <= 1e-10: # stop the iteration if norm of f(X) become smaller than certain value
        print("Converged")
        break
    J = ... # calculate Jacobian matrix here
    invJ = np.linalg.inv(J) # calculate inverse of Jacobian matrix
    Xnew = X - np.matmul(invJ, fX)  # Update X
    X = np.copy(Xnew)  # replace X with Xnew
\end{pythoncode}

\begin{soal}
Lengkapi program Newton-Raphson di atas dan lakukan modifikasi jika diperlukan.
Perhatikan bahwa pustaka Numpy digunakan untuk melakukan
operasi matriks-vektor. Misalnya fungsi \txtinline{np.linalg.inv} digunakan untuk
menhitung invers dari matriks Jacobian dan \txtinline{np.matmul} untuk melakukan operasi
perkalian matriks $\mathbf{J}^{-1}$ dengan vektor kolom $\mathbf{F}(\mathbf{X})$.
\end{soal}
\section{Metode Bairstow}

\textbf{Chapra Contoh 7.3}
Kode berikut ini adalah implementasi Python (belum lengkap)
dari metode Bairstow pada Chapra Gambar 7.5 dengan sedikit modifikasi.

\begin{pythoncode}
import numpy as np

# a_ is an array containing coefficients of the polynomial, starting
# from the lowest power.
# rr and ss are initial values of r and s (quadratic polynomial used for
# dividing the polynomial defined by a_).
# SMALL is a tolerance parameter
def root_bairstow( a_, rr=1.0, ss=1.0, NiterMax=100, SMALL=1e-10 ):
    
    a = np.copy(a_) # do not modify the input
    Ndeg = len(a) - 1
    
    re = np.zeros(Ndeg)
    im = np.zeros(Ndeg)
    
    b = np.zeros(Ndeg+1)
    c = np.zeros(Ndeg+1)
    
    r = rr
    s = ss
    n = Ndeg
    ier = 0
    
    iteration = 0
    
    while True:
        # Break out from the loop if reduced degree is less than 3
        # or number of iterations exceeds NiterMax    
        if (n < 3) or (iteration >= NiterMax) :
            break
    
        iteration = 0
        while True:
            iteration = iteration + 1
            if iteration >= NiterMax:
                break
            
            b[n]   = a[n]
            b[n-1] = ....
            c[n]   = b[n]
            c[n-1] = ....
            for i in range(n-2,-1,-1):
                b[i] = ....
                c[i] = ....
            # Solve the linear equations for dr and ds
            det = ....
            if abs(det) >= SMALL:
                dr = ....
                ds = ....
                r = r + dr
                s = s + ds
            else:
                # Update with different initial r and s
                r = r + 1.0
                s = s + 1.0
                iteration = 0
            
            # Stop the iteration if s and r did not changed much
            # NOTE: In the original algorithm (Chapra Fig. 7.5) relative
            # error is used, here we directly use the values of dr and ds
            if (abs(dr) <= SMALL) and (abs(ds) <= SMALL):
                break
        
        # Up to this points, we already found r and s of the quadratic
        # polynomial that factors or divide the original polynomial
        # We call quadroot to find the roots of this quadratic polynomial.
        r1, i1, r2, i2  = quadroot(r, s)
        re[n-1] = r1
        im[n-1] = i1
        re[n-2] = r2
        im[n-2] = i2
        
        # Update the original polynomial (using the quotient)
        # The degree of the original polynomial is reduced by 2.
        n = n - 2
        for i in range(0,n+1):
            a[i] = b[i+2]
    
    # Up to this point, n (the degree of the reduced polynomial should be
    # less than 3, i.e. it becomes quadratic or linear polynomial)
    # We solve for the remaining roots.
    if iteration < NiterMax:
        if n == 2:
            r = -a[1]/a[2]
            s = -a[0]/a[2]
            r1, i1, r2, i2 = quadroot(r, s)
            re[n-1] = r1
            im[n-1] = i1
            re[n-2] = r2
            im[n-2] = i2
        else:
            re[n-1] = -a[0]/a[1]
            im[n-1] = 0.0
    else:
        print("WARNING: there are some unknown errors")
        print("WARNING: probably NiterMax is exceeded.)
        ier = 1
    
    # Convert to complex-valued array
    zroots = np.zeros(Ndeg, dtype=np.complex128)
    # `numpy.complex128`: Complex number type composed
    # of 2 64-bit-precision floating-point numbers.
    for i in range(Ndeg):
        zroots[i] = complex(re[i], im[i])
    return zroots
    
    
def quadroot(r, s):
    disc = r**2 + 4.0*s
    if disc > 0:
        # real part of the roots
        r1 = ....
        r2 = ....
        # imaginary part of the roots (set to zero)
        i1 = 0.0
        i2 = 0.0
    else:
        r1 = ....
        r2 = ....
        i1 = ....
        i2 = ....
    #
    return r1, i1, r2, i2    
\end{pythoncode}

\begin{soal}
Lengkapi kode untuk \pyinline{root_bairstow} dan aplikasikan untuk
mencari akar-akar dari polinomial:
\begin{equation*}
f(x) = x^5 - 3.5x^4 + 2.75x^3 + 2.125x^2 - 3.875x + 1.25
\end{equation*}
\end{soal}

\begin{soal}
Gunakan \pyinline{root_bairstow} dan fungsi yang tersedia pada Numpy
(lihat bagian mengenai penggunakan pustaka Python)
untuk menentukan akar-akar dari polinomial berikut:
\begin{itemize}
\item $f(x) = x^3 - x^2 + 2x - 2$
\item $f(x) = 2x^4 + 6x^2 + 8$
\item $f(x) = x^4 - 2x^3 + 6x^2 - 2x + 5$
\item $f(x) = -2 + 6.2x - 4x^2 + 0.7x^3$
\item $f(x) = 9.34 - 21.97x + 16.3x^2 - 3.704x^3$
\item $f(x) = 10x^5 - 2x^3 + 6x^2 - 2x + 5$
\end{itemize}
Untuk Numpy, Anda dapat menggunakan referensi berikut:
\begin{itemize}
\item {\footnotesize\url{https://numpy.org/doc/stable/reference/generated/numpy.roots.html}}
\item {\footnotesize\url{https://numpy.org/doc/stable/reference/routines.polynomials.package.html}}
\end{itemize}
\end{soal}

\section{Menggunakan Pustaka Python}

Pada bagian ini, kita akan menggunakan beberapa pustaka SciPy yang dapat digunakan
untuk mencari akar persamaan nonlinear. Beberapa fungsi tersebut dapat ditemukan
pada modul \txtinline{scipy.optimize}.
\begin{itemize}
\item \txtinline{scipy.optimize.root_scalar}: untuk mencari akar persamaan nonlinear
yang terdiri dari satu variabel.
\item \txtinline{scipy.optimize.fsolve}: untuk mencari akar dari sistem persamaan
nonlinear (lebih dari satu variabel)
\end{itemize}

Pada contoh berikut, kita akan mencari akar dari $f(x) = x - \cos(x)$.
Untuk metode terbuka, kita menggunakan tebakan akar awal $x_0 = 0$
sedangkan untuk 
\begin{pythoncode}
from scipy import optimize
import math
    
def f(x):
    return x - math.cos(x)
    
def fprime(x):
    return 1 + math.sin(x)
    
print("\nUsing Newton-Raphson method")
sol = optimize.root_scalar(f, x0=0.0, fprime=fprime, method='newton')
print(sol)
    
print("\nUsing secant method")
sol = optimize.root_scalar(f, x0=0.0, x1=1.0, method='secant')
print(sol)
    
# Bracketing methods
for method in ["brentq", "brenth", "ridder", "bisect"]:
    print("\nUsing %s method" % (method))
    sol = optimize.root_scalar(f, bracket=[0.0, 1.0], method=method)
    print(sol)

print("\nUsing bisect directly")
xroot, sol = optimize.bisect(f, a=0.0, b=1.0, full_output=True)
print(sol)    
\end{pythoncode}
Modul \txtinline{scipy.optimize} juga menyediakan beberapa fungsi lain seperti
\txtinline{optimize.bisect} yang dapat digunakan untuk mencari akar persamaan
nonlinear, yang berbeda hanyalah \txtinline{interface} atau \txtinline{function signature}-nya.

Pada contoh berikut ini, kita akan mencari akar dari sistem persamaan nonlinear
pada Chapra Contoh 6.12 dengan menggunakan \txtinline{fsolve}:
\begin{pythoncode}
from scipy.optimize import fsolve

def f(x_):
    # x = x_[0] and y = x_[1] 
    x = x_[0]
    y = x_[1]
    u = x**2 + x*y - 10
    v = y + 3*x*y**2 - 57
    return [u, v]
    
root = fsolve(f, [1.0, 3.5]) # using initial guess as in the book
print(root)
\end{pythoncode}

Numpy menyediakan modul khusus untuk merepresentasikan polinomial, yaitu
\txtinline{numpy.polynomial}. Modul ini menyediakan banyak fungsi untuk
melakukan berbagai operasi terkait polinomial.
Pada contoh berikut ini, kita akan mencari akar-akar (real dan kompleks)
dari polinomial:
\begin{equation*}
f(x) = x^5 - 3.5x^4 + 2.75x^3 + 2.125x^2 - 3.875x + 1.25
\end{equation*}

Kode Python:
\begin{pythoncode}
from numpy.polynomial import Polynomial

p = Polynomial([1.25, -3.875, 2.125, 2.75, -3.5, 1.0])
print(p.roots())
\end{pythoncode}

Silakan membaca dokumentasi berikut untuk informasi lebih lanjut.
\begin{itemize}
\item {\scriptsize\url{https://docs.scipy.org/doc/scipy/reference/optimize.html\#root-finding}}
\item {\scriptsize\url{https://docs.scipy.org/doc/scipy/reference/generated/scipy.optimize.fsolve.html}}
\item {\scriptsize\url{https://numpy.org/doc/stable/reference/routines.polynomials.package.html}}
\end{itemize}

\begin{soal}
Gunakan Numpy untuk menentukan akar-akar dari polinomial berikut:
\begin{itemize}
\item $f(x) = x^3 - x^2 + 2x - 2$
\item $f(x) = 2x^4 + 6x^2 + 8$
\item $f(x) = x^4 - 2x^3 + 6x^2 - 2x + 5$
\item $f(x) = -2 + 6.2x - 4x^2 + 0.7x^3$
\item $f(x) = 9.34 - 21.97x + 16.3x^2 - 3.704x^3$
\item $f(x) = x^4 - 2x^3 + 6x^2 - 2x + 5$
\end{itemize}
\end{soal}


\section{Soal Tambahan}

\begin{soal}
Buat subrutin atau fungsi pada Python yang mengimplementasikan metode pencarian akar nonlinear
dari sebuah fungsi satu variabel dengan metode:
\begin{itemize}
\item bisection
\item regula falsi
\item Newton-Raphson
\item secant
\end{itemize}
Buat dokumentasi singkat atau penjelaskan mengenai subrutin yang Anda implementasikan:
\begin{itemize}
\item Input apa saja yang diperlukan
\item Keluaran atau apa saja yang dikembalikan oleh fungsi (\textit{return values})
\item Kondisi terminasi apa yang digunakan
\end{itemize}
Uji subrutin yang Anda buat untuk mencari akar dari salah satu persoalan yang
diberikan pada contoh-contoh sebelumnya.
\end{soal}

\subsection{Chapra Latihan 5.18}

Konsentrasi jenuh dari oksigen yang terlarut dalam air dapat
dihitung dengan menggunakan persamaan:
\begin{equation*}
\mathrm{ln}o_{\mathrm{sf}} = C_{0} + \frac{C_{1}}{T_{a}} +
\frac{C_{2}}{T_{a}^{2}} + \frac{C_{3}}{T_{a}^{3}} +
\frac{C_{4}}{T_{a}^{4}}
\end{equation*}
di mana $o_{\mathrm{sf}}$ adalah konsentrasi oksigen dalam mg/L pada tekanan
1 atm, $T_{a}$ adalah temperatur absolut, $T_{a} = T + 273.15$, $T$ dalam
derajat Celcius, dan parameter:
\begin{align*}
C_{0} & = -139.34411 \\
C_{1} & = 1.575701 \times 10^5 \\
C_{2} & = -6.642308 \times 10^7 \\
C_{3} & = 1.243800 \times 10^{10} \\
C_{4} & = -8.621949 \times 10^{11} \\
\end{align*}
\begin{soal}[Chapra Latihan 5.19]
Menurut prinsip Archimedes, gaya apung sama dengan berat fluida yang dipindahkan
oleh bagian benda yang terendam pada fluida.
Perhatikan gambar di bawah ini.

{\centering
\includegraphics[scale=1.0]{images_priv/Chapra_Fig_P5_19.pdf}
\par}

Tentukan tinggi $h$ yang mewakili bagian bola yang berada di atas air.
Gunakan nilai-nilai berikut: $r=1$ m, $\rho_{s}$ = kerapatan bola =
200 $\mathrm{kg/m}^{3}$ dan $\rho_{w}$ = kerapatan air = 1000 $\mathrm{kg/m}^{3}$.
Volume bagian bola yang berada di dalam air dapat dihitung dari:
\begin{equation}
V = \frac{\pi h^2}{3} (3r - h)
\end{equation}
\end{soal}
\begin{soal}[Chapra Latihan 6.25]
\label{chapra_exe_6.25}
Kesetimbangan massa dari suatu polutan dalam suatu danau
dapat dinyatakan dengan persamaan diferensial:
\begin{equation*}
V\frac{\mathrm{d}c}{\mathrm{d}t} = W - Qc - kV\sqrt{c}
\end{equation*}
di mana $c$ adalah konsentrasi polutan dalam $\mathrm{g/m}^{3}$,
dengan parameter
$V = 1\times10^6\,\mathrm{m}^3$,
$Q = 1\times10^5\,\mathrm{m}^3/\mathrm{tahun}$,
$W = 1\times10^6\,\mathrm{g}/\mathrm{tahun}$, dan
$k = 0.25\,\mathrm{m}^{0.5}/\mathrm{g}^{0.5}/\mathrm{tahun}$.
Hitung konsentrasi polutan dalam keadaan tunak.
\end{soal}
\begin{soal}[Chapra Latihan 8.3]
Dalam proses kimia, uap air ($\mathrm{H}_{\mathrm{2}}\mathrm{O}$) dipanaskan sampai
suhu yang cukup tinggi sehingga sebagian besar dari air akan terdisosiasi membentuk oksigen
($\mathrm{O}_{2}$) dan hidrogen ($\mathrm{H}_{2}$) menurut persamaan reaksi
\begin{equation*}
\mathrm{H}_2\mathrm{O} \rightarrow \mathrm{H}_{2}\,+\,\mathrm{O}_{2}
\end{equation*}
Jika diasumsikan bahwa ini adalah satu-satunya reaksi yang terlibat, fraksi
mol $\ce{H2O}$ yang berdisosiasi, dilambangkan dengan $x$, dapat dihitung dari persamaan
\begin{equation*}
K = \frac{x}{1-x}\sqrt{\frac{2p_t}{2+x}}
\end{equation*}
dimana $K$ = kesetimbangan konstan reaksi dan $p_t$: tekanan total campuran
dalam satuan atm.
Jika diketahui bahwa $p_t = 3.5$ atm dan $K = 0.4$, tentukan nilai $x$ yang memenuhi
persamaan di atas.
\end{soal}
\begin{soal}[Chapra Latihan 8.7]
Persamaan keadan Redlich-Kwong dinyatakan sebagai:
\begin{equation*}
p = \frac{RT}{v - b} - \frac{a}{v(v + b)\sqrt{T}}
\end{equation*}
di mana $R$ = konstanta gas universal = 0.518 kJ/(kg K),
$T$ adalah temperature absolut (K), $p$ adalah tekanan absolut (kPa),
dan $v$ = volume spesifik gas ($\mathrm{m}^3/\mathrm{kg}$).
Parameter $a$ dan $b$ dihitung dengan persamaan:
\begin{align*}
a & = -0.427 \frac{R^{2} T_{c}^{2.5}}{p_{c}} \\
b & = 0.0866 R\frac{T_c}{p_c}
\end{align*}
di mana $p_c$ dan $T_c$ menyatakan tekanan dan temperatur kritis.
Tentukan jumlah methana (yaitu $v$) ($p_c$ = 4600 kPa and $T_c$ = 191 K)
yang dapat disimpan dalam tangki dengan volumne 3 $\mathrm{m}^{3}$
pada temperatur -40$^{\circ}\,\mathrm{C}$ dengan tekanan 65000 MPa.
\end{soal}

\subsection{Chapra Latihan 8.18}
Sebuah balok yang dikenai beban terdistribusi yang meningkat secara linier.
Persamaan untuk kurva elastis diberikan oleh persamaan:
\begin{equation*}
y = \frac{w_0}{120EIL}
\left( -x^5 + 2L^2 x^3 - L^4 x \right)
\end{equation*}
Defleksi maksimum dapat ditentukan dengan cara mencari nilai $x$ yang
memenuhi $\dfrac{\mathrm{d}y}{\mathrm{d}x}$.
Tentukan defleksi maksimum jika diketahui bahwa:
$L = 450\, \mathrm{cm}$,
$E = 50000 \, \mathrm{kN}/\mathrm{cm}^2$,
$I = 30000 \, \mathrm{cm}^4$,
dan $w0 = 2.5$ kN/cm.

\begin{soal}[Chapra Latihan 8.32]
Suatu muatan total $Q$ terdistribusi seragam pada suatu cincin konduktor
dengan radius $a$. Suatu muatan uji $q$ terletak pada jarak $x$
dari titik tengah cincin. Gaya yand bekerja pada muatan uji yang disebabkan
oleh distribusi muatan pada cincin diberikan oleh persamaan:
\begin{equation*}
F = \frac{1}{4\pi\epsilon_{0}}\frac{qQx}{\left( x^2 + a^2 \right)^{3/2}}
\end{equation*}
di mana $\epsilon_{0} = 8.85\times 10^{-12}\,\mathrm{C}^2/(\mathrm{N}\,\mathrm{m}^{2})$.
Tentukan jarak $x$ jika $F = 1$ N, $q = Q = 2\times 10^{-5}$ C, dan $a = 0.9$ m.
\end{soal}
\begin{soal}[Chapra Latihan 8.33]
Impedansi dari rangkaian paralel RLC dinyatakan oleh persamaan
\begin{equation*}
\frac{1}{Z} = \sqrt{\frac{1}{R^2} + \left(
\omega C - \frac{1}{\omega L}
\right)^2}
\end{equation*}
Cari frekuensi angular $\omega$ untuk $Z=75$ ohm,
$R=225$ ohm,
$C = 0.6 \times 10^{-6}$ F
dan $L = 0.5 H$.
\end{soal}

\begin{soal}[Chapra Latihan 8.39]
Kecepatan kearah atas sebuah roket dapat dihitung menggunakan persamaan berikut
\begin{equation*}
v = u \mathrm{ln}\frac{m_{0}}{m_{0} - qt} - gt
\end{equation*}
dimana $v$ = kecepatan kearah atas roket,
$u$ = kecepatan keluar bahan bakar relatif terhadap
roket, $m_0 $ = masa awal roket pada t = 0, $q$ = laju konsumsi bahan bakar,
dan $g$ = percepatan gravitasi. Jika $g = 9.81 \, \mathrm{m/s}^2$,
$u = 2200$ m/s, $m_0 = 160000$ kg,
dan $q = 2680$ kg/s, hitunglah waktu
dimana kecepatan mencapai $v = 1000$ m/s.
\end{soal}
\begin{soal}[Fungsi Bessel bola]
Fungsi Bessel bola (spherical Bessel) $j_{n}(x)$ dapat dituliskan sebagai
berikut:
\begin{equation*}
j_{n}(x) = (-x)^{n}
\left( \frac{1}{x} \frac{\mathrm{d}}{\mathrm{d}x} \right)^n
\frac{\sin(x)}{x}
\end{equation*}
Buatlah plot untuk $j_{3}(x)$ dan carilah semua akar-akarnya pada interval (0,20).
Gunakan salah satu metode untuk mencari akar persamaan nonlinear.
\end{soal}
\begin{soal}[Chapra Latihan 6.23]
Tentukan akar dari sistem persamaan nonlinear berikut
\begin{align*}
(x - 4)^2 + (y - 4)^2 & = 5 \\
x^2 + y^2 & = 16
\end{align*}
Buatlah plot dari fungsi-fungsi tersebut dan gunakan untuk mendapatkan estimasi tebakan
awal untuk metode Newton-Raphson.
\end{soal}


\begin{soal}[Chapra Latihan 6.24]
Tentukan akar dari sistem persamaan nonlinear berikut
\begin{align*}
y = x^2 + 1 \\
y = 2\cos(x)
\end{align*}
Buatlah plot dari fungsi-fungsi tersebut dan gunakan untuk mendapatkan estimasi tebakan
awal untuk metode Newton-Raphson.
\end{soal}

\end{document}

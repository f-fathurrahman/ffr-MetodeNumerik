\begin{soal}[Chapra Latihan 8.39]
Kecepatan kearah atas sebuah roket dapat dihitung menggunakan persamaan berikut
\begin{equation*}
v = u \mathrm{ln}\frac{m_{0}}{m_{0} - qt} - gt
\end{equation*}
dimana $v$ = kecepatan kearah atas roket,
$u$ = kecepatan keluar bahan bakar relatif terhadap
roket, $m_0 $ = masa awal roket pada t = 0, $q$ = laju konsumsi bahan bakar,
dan $g$ = percepatan gravitasi. Jika $g = 9.81 \, \mathrm{m/s}^2$,
$u = 2200$ m/s, $m_0 = 160000$ kg,
dan $q = 2680$ kg/s, hitunglah waktu
dimana kecepatan mencapai $v = 1000$ m/s.
\end{soal}
\documentclass[a4paper,11pt,bahasa]{extarticle} % screen setting
\usepackage[a4paper]{geometry}

%\documentclass[b5paper,11pt,bahasa]{article} % screen setting
%\usepackage[b5paper]{geometry}

%\geometry{verbose,tmargin=1.5cm,bmargin=1.5cm,lmargin=1.5cm,rmargin=1.5cm}

\geometry{verbose,tmargin=2.0cm,bmargin=2.0cm,lmargin=2.0cm,rmargin=2.0cm}

\setlength{\parskip}{\smallskipamount}
\setlength{\parindent}{0pt}

%\usepackage{cmbright}
%\renewcommand{\familydefault}{\sfdefault}

\usepackage{amsmath}
\usepackage{amssymb}

\usepackage[libertine]{newtxmath}

\usepackage[no-math]{fontspec}
\setmainfont{Linux Libertine O}

%\usepackage{fontspec}
%\usepackage{lmodern}

\setmonofont{JuliaMono-Regular}


\usepackage{hyperref}
\usepackage{url}
\usepackage{xcolor}
\usepackage{enumitem}
\usepackage{mhchem}
\usepackage{graphicx}
\usepackage{float}

\usepackage{minted}

\newminted{julia}{breaklines,fontsize=\footnotesize}
\newminted{python}{breaklines,fontsize=\footnotesize}

\newminted{bash}{breaklines,fontsize=\footnotesize}
\newminted{text}{breaklines,fontsize=\footnotesize}

\newcommand{\txtinline}[1]{\mintinline[breaklines,fontsize=\footnotesize]{text}{#1}}
\newcommand{\jlinline}[1]{\mintinline[breaklines,fontsize=\footnotesize]{julia}{#1}}
\newcommand{\pyinline}[1]{\mintinline[breaklines,fontsize=\footnotesize]{python}{#1}}

\newmintedfile[juliafile]{julia}{breaklines,fontsize=\footnotesize}
\newmintedfile[pythonfile]{python}{breaklines,fontsize=\footnotesize}
\newmintedfile[fortranfile]{fortran}{breaklines,fontsize=\footnotesize}

\usepackage{mdframed}
\usepackage{setspace}
\onehalfspacing

\usepackage{babel}
\usepackage{appendix}

\newcommand{\highlighteq}[1]{\colorbox{blue!25}{$\displaystyle#1$}}
\newcommand{\highlight}[1]{\colorbox{red!25}{#1}}

\newcounter{soal}%[section]
\newenvironment{soal}[1][]{\refstepcounter{soal}\par\medskip
   \noindent \textbf{Soal~\thesoal. #1} \sffamily}{\medskip}


\definecolor{mintedbg}{rgb}{0.95,0.95,0.95}
\BeforeBeginEnvironment{minted}{
    \begin{mdframed}[%
        topline=false,bottomline=false,%
        leftline=false,rightline=false]
}
\AfterEndEnvironment{minted}{\end{mdframed}}


\BeforeBeginEnvironment{soal}{
    \begin{mdframed}[%
        topline=true,bottomline=false,%
        leftline=true,rightline=false]
}
\AfterEndEnvironment{soal}{\end{mdframed}}


\begin{document}

\title{%
{\small TF4062 Komputasi Rekayasa Lanjut}\\
Pengenalan Metode Elemen Hingga pada 1d dengan SymPy
}
\author{Fadjar Fathurrahman\\
Teknik Fisika\\
Institut Teknologi Bandung}
\date{}
\maketitle

\section{Model Persamaan Diferensial}

Tinjau masalah konduksi panas tunak (tidak bergantung waktu) dari suatu kawat dengan
panjang $L$ dan penampang melintang konstan. Asumsikan bahwa bagian kiri dikenai fluks
panas konstan $q$ dan bagian kanan dijaga pada suhu tetap, $T(x=L) = T_L$ dan
sepanjang kawat dikelilingi oleh material insulator sehingga tidak ada
panas yang keluar.

Persamaan diferensial yang menjelaskan mengenai konduksi panas ini dapat dituliskan
sebagai berikut.
\begin{equation}
-k \frac{\mathrm{d}^2 T}{\mathrm{d}x^2} = Q
\end{equation}
untuk $0 < x < L$, di mana $k$
adalah konduktivitas termal dari material yang diasumsikan konstan dan
$Q$ adalah sumber panas internal pada kawat.
Syarat batas Neumann atau syarat batas alami (\textit{natural boundary condition})
\begin{equation}
-k\frac{\mathrm{d}T}{\mathrm{d}x} = q \quad \text{at } x = 0
\end{equation}
dan syarat batas Dirichlet atau
 (Dirichlet BC, essential BC)
\begin{equation}
T = T_{L} \quad \text{pada } x = L
\end{equation}

Solusi analitik dari permasalahan ini adalah
\begin{equation}
T(x) = T_{L} + \frac{q}{k}(L - x) + \frac{1}{k} \int_{x}^{L}
\left( \int_{0}^{y} Q(z)\,\mathrm{d}z \right)\,\mathrm{d}y
\label{eq:solusi-analitik1}
\end{equation}
%
Untuk $Q$ konstan, Persamaan \eqref{eq:solusi-analitik1}
tereduksi menjadi
\begin{equation}
T(x) = T_{L} + \frac{q}{k}(L - x) + \frac{Q}{2k}(L^2 - x^2)
\end{equation}

\section{Diskritisasi dan aproksimasi}

Prosedur metode elemen hingga dimulai dari melakukan diskritisasi domain spasial
menjadi elemen-elemen $e_{i}$:
\begin{equation}
e_{i}: \left\{ x_{i} \leq x \leq x_{i+1} \right\},\quad i=1,2,\ldots,n
\end{equation}
Titik-titik $x_{i}$ juga sering disebut sebagai titik nodal atau \textit{nodes}. 
Jika kita memiliki $n$ elemen maka kita memiliki $n+1$ titik nodal.

Solusi dari persamaan konduksi (atau PDE lain) dapat diaproksimasi sebagai
\begin{equation}
T(x) \approx \sum_{i=1}^{n+1} T_{i} N_{i}(x)
\label{eq:sol_pde_approx}
\end{equation}
di mana $T_i$ adalah nilai nodal dari $T(x)$ dan $N_{i}(x)$ adalah fungsi bentuk
(\textit{shape functions}) atau fungsi basis yang terasosiasi pada masing-masing
titik nodal.


\section{Fungsi basis linear}

Fungsi basis atau fungsi bentuk pada metode elemen hingga biasanya dipilih sedemikian rupa
sehingga
memenuhi $N_{i}(x_{i}) = 1$ dan $N_{i}(x_{j})=0$
jika $j \neq i$.
Fungsi basis biasanya dipilih dari fungsi polinomial dengan orde rendah.
Untuk kasus satu elemen atau dua titik nodal fungsi basis dapat didefinisikan sebagai
\begin{equation}
\begin{cases}
N_{1}(x) & = 1 - \dfrac{x}{L} \\[10pt]
N_{2}(x) & = \dfrac{x}{L}
\end{cases}
\label{eq:linear_shape_functions}
\end{equation}
Elemen dengan dua titik nodal yang memiliki bentuk fungsi basis linear
juga dikenal disebagai \textit{elemen linear}.

Untuk kasus elemen linear yang lebih dari satu, atau minimal ada dua elemen
atau tiga titik nodal, kita memiliki fungsi basis sebagai berikut
Untuk titik ujung paling kiri, $i=1$, dengan $i$ adalah indeks basis titik nodal global
\begin{equation}
N_{1}(x) = \begin{cases}
\dfrac{x_{2} - x}{h_1} & x_{1} \leq x \leq x_{2} \\[10pt]
0 & \text{otherwise}
\end{cases}
\end{equation}
%
Untuk titik nodal interior:
\begin{equation}
N_{i}(x) = \begin{cases}
\dfrac{x - x_{i-1}}{h_{i-1}} & x_{i-1} \leq x \leq x_{i} \\[10pt]
\dfrac{x_{i+1} - x}{h_{i}} & x_{i} \leq x \leq x_{i+1},\quad i=2,3,\ldots,n \\[10pt]
0 & \text{otherwise}
\end{cases}
\end{equation}
%
dan untuk titik ujung paling kanan:
\begin{equation}
N_{n+1}(x) = \begin{cases} \dfrac{x - x_{n}}{h_{n}} & x_{n} \leq x \leq x_{n+1} \\[10pt]
0 & \text{otherwise}
\end{cases}
\end{equation}
di mana $h_{i}$ adalah jarak antara titik nodal
$h_{i} = x_{i+1} - x_{i}$ yang juga sama dengan panjang elemen.

Definisi fungsi basis yang merujuk pada titik nodal dengan indeks global
dikenal sebagai representasi global dari fungsi basis. 

Perhatikan bahwa fungsi basis tersebut didefinisikan sebagai fungsi sepotong
atau \textit{piecewise}.
Pada SymPy kita dapat mendefinisikan fungsi basis tersebut dengan menggunakan kelas
\pyinline{Piecewise}.

Contoh pendefinisian elemen, titik nodal, dan fungsi basis.
\begin{pythoncode}
from sympy import *

init_printing(use_unicode=True) # pada terminal
# Change use_unicode to use_latex=True when using Jupyter Notebook

# Spatial variable
x = Symbol("x", real=True)

# Domain length
L = Symbol("L", real=True, positive=True)


# Setup linear element
Nelements = 2 
Nnodes = Nelements + 1

# Setup the grid (manually)
xnodes = [Integer(0), L/2, L]
# I use Integer(0) to make sure that all xnodes are SymPy objects
h = [xnodes[i+1]-xnodes[i] for i in range(Nnodes-1)]

# Some sanity checks
assert(len(xnodes) == Nnodes)
assert(len(h) == Nelements)


# Setup basis functions (global representation)
Nfuncs = []

# First basis
cond1 = (x >= xnodes[0]) & (x <= xnodes[1])
f1 = (xnodes[1] - x)/h[0]
Nfuncs.append( Piecewise( (f1, cond1), (0, True) ) )

# Second basis
cond1 = (x >= xnodes[0]) & (x <= xnodes[1])
f1 = ( x - xnodes[0])/h[0]
#
cond2 = (x >= xnodes[1]) & (x <= xnodes[2])
f2 = (xnodes[2] - x)/h[1]
#
Nfuncs.append( Piecewise( (f1, cond1), (f2, cond2), (0, True) ) )

# Third basis
cond1 = (x >= xnodes[Nnodes-2]) & (x <= xnodes[Nnodes-1])
f1 = (x - xnodes[Nnodes-2])/h[Nnodes-2]
Nfuncs.append( Piecewise( (f1, cond1), (0, True) ) )
\end{pythoncode}

Plot fungsi basis menggunakan Matplotlib.
\begin{pythoncode}
Lnum = 2.0 # why we need this?
import numpy as np
import matplotlib.pyplot as plt
NptsPlot = 200
xplt = np.linspace(0.0, Lnum, 200)
yplt = np.zeros(NptsPlot)
plt.clf()
for ibasis in range(Nnodes):
    for i in range(NptsPlot):
        yplt[i] = Nfuncs[ibasis].subs({x: xplt[i], L: Lnum})
    plt.plot(xplt, yplt, label="Basis-" + str(ibasis+1))
plt.legend()
plt.grid(True)
\end{pythoncode}

{\centering
\includegraphics[scale=0.5]{codes/IMG_basis_lin_01.pdf}
\par}


\section{Integral Residual Terbobot}

When we approximate the solution as in the Equation \eqref{eq:sol_pde_approx},
in general, we cannot obtain the true solution to the differential equation.
So, we will not get an exact equality but
some \textit{residual} associated with the error in the approximation. This residual
can be defined as:
\begin{equation}
R(T,x) \equiv -k \frac{\mathrm{d}^2 T}{\mathrm{d}x^2} - Q
\label{eq:residual_01}
\end{equation}
where $T$ in the above equation is the approximation to the true solution $T^{*}$
for which we have:
\begin{equation}
R(T^{*},x) \equiv 0
\end{equation}
However, for any $T \neq T^{\mathrm{*}}$, we cannot force the residual
to vanish at every
point $x$, no matter how small we make the grid or long the series expansion.
The idea of the \textit{weighted residuals method} is that we can multiply
the residual by a \textit{weighting function} and force the integral of the weighted
expression to vanish:
\begin{equation}
\int_{0}^{L} w(x) R(T,x) \mathrm{d}x = 0
\label{eq:weighted_res}
\end{equation}
where $w(x)$ is the weighting function.
Choosing different weighting functions and replacing each of them in
\eqref{eq:weighted_res}, we can then generate a system
of linear equations in the unknown parameters $T_{i}$ that will determine an
approximation T of the form of the finite series given in Equation \eqref{eq:sol_pde_approx}.
This will satisfy the differential equation in an "average" or "integral" sense.
The type of weighting function chosen depends on the type of weighted
residual technique selected. In the \textit{Galerkin procedure}, the weights are set
equal to the shape functions $N_{i}$, that is
\begin{equation}
w_{i}(x) = N_{i}(x)
\end{equation}
So we have:
\begin{equation}
\int_{0}^{L} N_{i} \left[
-k\frac{\mathrm{d}^2 T}{\mathrm{d}x^2} - Q
\right]\,\mathrm{d}x = 0
\end{equation}

Since the temperature distribution must be a continuous function of $x$,
the simplest way to approximate it would be to use piecewise polynomial
interpolation over each element, in particular, piecewise linear approximation
(by using linear shape functions as in \eqref{eq:linear_shape_functions})
provides the simplest approximation with a continuous function.
Unfortunately, the first derivatives of such functions are not continuous
at the elements' ends and, hence, second derivatives do not exist there;
furthermore, the second derivative of $T$ would vanish inside each element.
However, to require the second-order derivatives to exist everywhere is
too restrictive.
We can weaken this requirement by application of integration by parts to
the second derivative: ($\int\,u\,\mathrm{d}v = uv - \int\,v\,\mathrm{d}u$):
\begin{equation}
\int_{0}^{L} N_{i} \left[ -k\frac{\mathrm{d}^2 T}{\mathrm{d}x^2} \right]\,\mathrm{d}x =
\int_{0}^{L} k \frac{\mathrm{d} N_{i}}{\mathrm{d}x}
\frac{\mathrm{d} T}{\mathrm{d}x} \mathrm{d}x -
\left[ k N_{i} \frac{\mathrm{d}T}{\mathrm{d}x} \right]_{0}^{L}
\label{eq:weak_form_01}
\end{equation}

Substituting the Equation \eqref{eq:sol_pde_approx} to the integral
term at the RHS of equation \eqref{eq:weak_form_01}:
\begin{align}
\int_{0}^{L} k \frac{\mathrm{d} N_{i}}{\mathrm{d}x}
\frac{\mathrm{d}}{\mathrm{d}x}\, \left( \sum_{j=1}^{n+1} T_{j} N_{j} \right) \mathrm{d}x & =
\int_{0}^{L} k \frac{\mathrm{d} N_{i}}{\mathrm{d}x}
\left( \sum_{j=1}^{n+1} \frac{\mathrm{d}N_{j}}{\mathrm{d}x} T_{j} \right)\, \mathrm{d}x \\
& = \int_{0}^{L} k
\left( \sum_{j=1}^{n+1}
\frac{\mathrm{d} N_{i}}{\mathrm{d}x}
\frac{\mathrm{d}N_{j}}{\mathrm{d}x} T_{j} \right)\, \mathrm{d}x
\end{align}

The Equation \eqref{eq:weak_form_01} can be rewritten as
\begin{equation}
\int_{0}^{L} k
\left( \sum_{j=1}^{n+1}
\frac{\mathrm{d} N_{i}}{\mathrm{d}x}
\frac{\mathrm{d}N_{j}}{\mathrm{d}x} T_{j} \right)\, \mathrm{d}x -
\int_{0}^{L} N_{i} Q\, \mathrm{d}x -
\left[ k N_{i} \frac{\mathrm{d}T}{\mathrm{d}x} \right]_{0}^{L} = 0
\label{eq:weak_form_02}
\end{equation}
for $i=1,2,\ldots,n+1$. The quantities $k$, $Q$, $N_{i}$ are known.
The terms containing $T_{i}$ can be isolated to the LHS and the
the known quantities can be moved to the RHS. This is a system
of linear equations with $T_{i}$ as the unknown variables. In matrix form:
\begin{equation}
\mathbf{K} \mathbf{u} = \mathbf{f}
\label{eq:linear_system}
\end{equation}
where $\mathbf{K}$, usually called stiffness matrix,
is a matrix arising from the integral terms,
$\mathbf{f}$, usually called load vector, is
column vector whose elements arise from
the integrals of source term and boundary terms,
and $\mathbf{u}$ is column vector of the the unknown $T_{i}$.
These linear equations
can be solved by standard method such as Gaussian elimination and LU decomposition.

In practice, the integral in Equation \eqref{eq:weak_form_02} can be
calculated analytically or by numerical methods such as Gaussian quadrature.
In the present case, the integral can be calculated analytically.
The integral over all spatial
domain can be divided into integrals over elements:
\begin{equation}
\int_{0}^{L} = \int_{x_{1}=0}^{x_{2}} + \int_{x_{2}}^{x_{3}} + \cdots +
\int_{x_{n}}^{x_{n+1}=L}
\end{equation}

For simplicity, let's consider two elements with equal spacing
$h_{1} = h_{2}$ with nodes
\begin{equation}
x_{1} = 0, x_{2} = L/2, x_{3} = L
\end{equation}
The first integral is done over the first element. In the first element, only
$N_{1}$ and $N_{2}$ contributes to the integrals because $N_{3}$ is zero in
this interval. So the index $i$ and $j$ are limited to 1 and 2 only.
This also applies to other elements.


\end{document}
\input{PREAMBLE}

\begin{document}

\title{%
{\small TF4062 Komputasi Rekayasa Lanjut}\\
Pengenalan Metode Elemen Hingga pada 1d dengan SymPy
}
\author{Fadjar Fathurrahman\\
Teknik Fisika\\
Institut Teknologi Bandung}
\date{}
\maketitle

\section{Model Persamaan Diferensial}

Tinjau masalah konduksi panas tunak (tidak bergantung waktu) dari suatu kawat dengan
panjang $L$ dan penampang melintang konstan. Asumsikan bahwa bagian kiri dikenai fluks
panas konstan $q$ dan bagian kanan dijaga pada suhu tetap, $T(x=L) = T_L$ dan
sepanjang kawat dikelilingi oleh material insulator sehingga tidak ada
panas yang keluar.

Persamaan diferensial yang menjelaskan mengenai konduksi panas ini dapat dituliskan
sebagai berikut.
\begin{equation}
-k \frac{\mathrm{d}^2 T}{\mathrm{d}x^2} = Q
\end{equation}
untuk $0 < x < L$, di mana $k$
adalah konduktivitas termal dari material yang diasumsikan konstan dan
$Q$ adalah sumber panas internal pada kawat.
Syarat batas Neumann atau syarat batas alami (\textit{natural boundary condition})
\begin{equation}
-k\frac{\mathrm{d}T}{\mathrm{d}x} = q \quad \text{at } x = 0
\end{equation}
dan syarat batas Dirichlet atau
 (Dirichlet BC, essential BC)
\begin{equation}
T = T_{L} \quad \text{pada } x = L
\end{equation}

Solusi analitik dari permasalahan ini adalah
\begin{equation}
T(x) = T_{L} + \frac{q}{k}(L - x) + \frac{1}{k} \int_{x}^{L}
\left( \int_{0}^{y} Q(z)\,\mathrm{d}z \right)\,\mathrm{d}y
\label{eq:solusi-analitik1}
\end{equation}
%
Untuk $Q$ konstan, Persamaan \eqref{eq:solusi-analitik1}
tereduksi menjadi
\begin{equation}
T(x) = T_{L} + \frac{q}{k}(L - x) + \frac{Q}{2k}(L^2 - x^2)
\end{equation}


\section{Fungsi basis linear}

Untuk kasus elemen linear kita memiliki fungsi basis sebagai berikut
Untuk titik ujung paling kiri, $i=1$, dengan $i$ adalah indeks basis titik nodal global
\begin{equation}
N_{1}(x) = \begin{cases}
\dfrac{x_{2} - x}{h_1} & x_{1} \leq x \leq x_{2} \\[10pt]
0 & \text{otherwise}
\end{cases}
\end{equation}
%
Untuk titik nodal interior:
\begin{equation}
N_{i}(x) = \begin{cases}
\dfrac{x - x_{i-1}}{h_{i-1}} & x_{i-1} \leq x \leq x_{i} \\[10pt]
\dfrac{x_{i+1} - x}{h_{i}} & x_{i} \leq x \leq x_{i+1},\quad i=2,3,\ldots,n \\[10pt]
0 & \text{otherwise}
\end{cases}
\end{equation}
%
dan untuk titik ujung paling kanan:
\begin{equation}
N_{n+1}(x) = \begin{cases} \dfrac{x - x_{n}}{h_{n}} & x_{n} \leq x \leq x_{n+1} \\[10pt]
0 & \text{otherwise}
\end{cases}
\end{equation}
di mana $h_{i}$ adalah jarak antara titik nodal
$h_{i} = x_{i+1} - x_{i}$ yang juga sama dengan panjang elemen.

Perhatikan bahwa fungsi basis tersebut didefinisikan sebagai fungsi sepotong
atau \textit{piecewise}.
Pada SymPy kita dapat mendefinisikan fungsi basis tersebut dengan menggunakan kelas
\pyinline{Piecewise}.


\end{document}
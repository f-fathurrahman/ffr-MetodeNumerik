\begin{frame}
\frametitle{Persamaan diferensial orde dua}

Suatu persamaan diferensial biasa orde dua dapat diubah menjadi sistem persamaan
diferensial biasa orde 1.

Sebagai contoh, perhatikan persamaan diferensial biasa orde 2 berikut:
\begin{equation*}
m \frac{\mathrm{d}^2 x}{\mathrm{d}t^2} + c \frac{\mathrm{d}x}{\mathrm{d}t} + kx = 0
\end{equation*}
$x$: simpangan dari keadaan setimbang, $m$: massa benda, $c$: koefisien redaman, $k$ konstanta pegas.

{\centering
\includegraphics[scale=0.2]{../../images_priv/Chapra_Fig_P25_16.pdf}
\par}

\end{frame}




\begin{frame}
\frametitle{Persamaan diferensial orde dua}

\begin{columns}

\begin{column}{0.5\textwidth}
Definisikan:
\begin{align*}
y_{1}(t) & = x(t) \\
y_{2}(t) & = x'(t)
\end{align*}
%
sehingga:
%
\begin{align*}
y'_{1}(t) & = x'(t) = y_{2}(t) \\
y'_{2}(t) & = x''(t)
\end{align*}
Persamaan diferensial:
\begin{equation*}
m \frac{\mathrm{d}^2 x}{\mathrm{d}t^2} + c \frac{\mathrm{d}x}{\mathrm{d}t} + kx = 0
\end{equation*}
dituliskan menjadi:
\begin{equation*}
m y'_{2}(t) + c y_{2}(t) + k y_{1}(t) = 0
\end{equation*}

\end{column}


\begin{column}{0.5\textwidth}
Atau:
\begin{equation*}
y'_{2}(t) = -\frac{c y_{2}(t) + k y_{1}(t)}{m}
\end{equation*}
Sehingga diperoleh sistem persamaan diferensial biasa orde-1:
\begin{align*}
y'_{1}(t) & = y_{2}(t) \\
y'_{2}(t) & = -\frac{c y_{2}(t) + k y_{1}(t)}{m}
\end{align*}
\end{column}

\end{columns}

\end{frame}



% -----------------------------
\begin{frame}


{\centering
\includegraphics[scale=0.2]{../../images_priv/Chapra_Fig_P25_16.pdf}
\par}

$x$: simpangan dari keadaan setimbang, $m$: massa benda, $c$: koefisien redaman, $k$ konstanta pegas.

Solusi akan dicari untuk tiga kasus koefisien redaman (dalam Ns/m):
5 (\textit{underdamped}), 40 (\textit{critically damped}), and 200 (\textit{overdamped}).
Nilai parameter lain adalah: $m = 20$ kg dan $k = 20$ N/m

Kecepatan awal adalah 0 dan simpangan awal adalah $x_0$ = 1 m. Persamaan
ingin diselesaikan pada rentang waktu $0 \leq t \leq 15$ s.

\end{frame}



\begin{frame}[fragile]

Contoh kasus vibrasi \textit{underdamped}: $m = 20$ kg, $k = 20$ N/m dan $c = 5$ Ns/m

\begin{pythoncode}
def deriv_underdamped(x, y):
    # Parameters (local)
    m = 20.0   # kg
    k = 20.0   # N/m    
    c = 5.0    # Ns/m underdamped
    Nvec = len(y)
    assert Nvec == 2
    # Output array
    dydx = np.zeros(Nvec)
    dydx[0] = y[1]
    dydx[1] = -(c*y[1] + k*y[0])/m
    #
    return dydx
\end{pythoncode}

\end{frame}



\begin{frame}[fragile]

Contoh kasus vibrasi \textit{overdamped}: $m = 20$ kg, $k = 20$ N/m dan $c = 200$ Ns/m

\begin{pythoncode}
def deriv_overdamped(x, y):
    # Parameters (local)
    m = 20.0   # kg
    k = 20.0   # N/m    
    c = 200.0  # Ns/m overdamped
    Nvec = len(y)
    assert Nvec == 2
    # Output array
    dydx = np.zeros(Nvec)
    dydx[0] = y[1]
    dydx[1] = -(c*y[1] + k*y[0])/m
    #
    return dydx
\end{pythoncode}

\end{frame}



\begin{frame}[fragile]

Contoh kasus vibrasi \textit{critically damped}: $m = 20$ kg, $k = 20$ N/m dan $c = 40$ Ns/m

\begin{pythoncode}
def deriv_critical_damped(x, y):
    # Parameters (local)
    m = 20.0   # kg
    k = 20.0   # N/m    
    c = 40.0   # Ns/m critically damped
    Nvec = len(y)
    assert Nvec == 2
    # Output array
    dydx = np.zeros(Nvec)
    dydx[0] = y[1]
    dydx[1] = -(c*y[1] + k*y[0])/m
    #
    return dydx
\end{pythoncode}

\end{frame}


\begin{frame}[fragile]

\begin{pythoncode}
t0 = 0.0
initial_displ = 1.0
initial_vel = 0.0
x0 = np.array([initial_displ, initial_vel]) # initial conditions
Δt = 0.01 # try playing with this parameter
tf = 15.0
Nstep = int(tf/Δt)
t, x = ode_solve(deriv_underdamped, ode_rk4_1step, t0, x0, Δt, Nstep)
\end{pythoncode}

Tugas: selesaikan untuk kasus lain (crictically damped dan overdamped) serta
buat plot perbandingan tiga kasus tersebut.

\end{frame}
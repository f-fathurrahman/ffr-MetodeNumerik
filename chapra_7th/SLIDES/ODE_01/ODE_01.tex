\documentclass[10pt,aspectratio=169]{beamer}
%\documentclass[fleqn,aspectratio=169]{beamer}

\usepackage{amsmath, amssymb}
\usepackage{fancyvrb, color, graphicx, hyperref, url}

%\setbeamersize{text margin left=5pt, text margin right=5pt}

\setlength{\parskip}{\smallskipamount}
\setlength{\parindent}{0pt}

\usepackage{fontspec}
%\setmonofont{DejaVu Sans Mono}
\setmonofont{JuliaMono-Regular}

\usefonttheme[onlymath]{serif}

\usepackage{minted}
\newminted{python}{breaklines,fontsize=\footnotesize}
\newminted{julia}{breaklines,fontsize=\footnotesize}
\newminted{bash}{breaklines,fontsize=\footnotesize}
\newminted{text}{breaklines,fontsize=\footnotesize}

\newcommand{\txtinline}[1]{\mintinline[breaklines,fontsize=\footnotesize]{text}{#1}}
\newcommand{\pyinline}[1]{\mintinline[breaklines,fontsize=\footnotesize]{python}{#1}}
\newcommand{\jlinline}[1]{\mintinline[breaklines,fontsize=\footnotesize]{julia}{#1}}

\definecolor{mintedbg}{rgb}{0.95,0.95,0.95}
\usepackage{mdframed}

\BeforeBeginEnvironment{minted}{\begin{mdframed}[backgroundcolor=mintedbg,%
  rightline=false,leftline=false,topline=false,bottomline=false]}
\AfterEndEnvironment{minted}{\end{mdframed}}

% https://tex.stackexchange.com/questions/33969/changing-font-size-of-selected-slides-in-beamer

\usepackage{environ}
%
% Custom font for a frame.
%
\newcommand{\customframefont}[1]{
  \setbeamertemplate{itemize/enumerate body begin}{#1}
  \setbeamertemplate{itemize/enumerate subbody begin}{#1}
}

\NewEnviron{framefont}[1]{
  \customframefont{#1} % for itemize/enumerate
  {#1 % For the text outside itemize/enumerate
    \BODY
  }
  \customframefont{\normalsize}
}



\begin{document}

\title{Metode Numerik}
\subtitle{Persamaan Diferensial Biasa}
%\author{Fadjar Fathurrahman}
%\institute{
%Teknik Fisika \\
%Institut Teknologi Bandung
%}
\date{}


\frame{\titlepage}


\begin{frame}
\frametitle{Contoh}

Persamaan gerak Newton:
\begin{equation*}
\frac{\mathrm{d}v}{\mathrm{d}t} = \frac{F}{m}
\end{equation*}

Persamaan kalor Fourier:
\begin{equation*}
q = -k \frac{\mathrm{d}T}{\mathrm{d}x}
\end{equation*}

Hukum difusi Fick:
\begin{equation*}
J = -D \frac{\mathrm{d}c}{\mathrm{d}x}
\end{equation*}

Hukum Faraday:
\begin{equation*}
\Delta V_{L} = L \frac{\mathrm{d}i}{\mathrm{d}t}
\end{equation*}

\end{frame}


% ----------------------------------------------------
\begin{frame}
\frametitle{Bentuk umum permasalahan nilai awal}

Permasalahan nilai awal: \textit{initial value problem} (IVP)

Selesaikan persamaan diferensial
\begin{equation}
\frac{\mathrm{d}y}{\mathrm{d}x} = f(x,y)
\label{eq:ode_gen01}
\end{equation}
diberikan syarat awal: $y(x=x_0) = y_0$.

$y(x)$ adalah variabel dependen (atau fungsi yang ingin dicari)
dan $x$ adalah variabel independen.

Untuk masalah nilai awal, biasanya lebih sering menggunakan $t$ sebagai
variabel independen:
\begin{equation}
\frac{\mathrm{d}y}{\mathrm{d}t} = f(t,y)
\label{eq:ode_gen02}
\end{equation}

\end{frame}

% -----------------------

\begin{frame}
\frametitle{Metode-metode untuk IVP}
  
\begin{itemize}
\item Metode satu-langkah
\item Metode multi-langkah
\item Metode implisit
\end{itemize}

\end{frame}



% -----------------------

\begin{frame}
\frametitle{Metode satu langkah}

Bentuk umum:
\begin{equation*}
y_{i+1} = y_{i} + \phi h
\end{equation*}

$y_{i}$: nilai lama

$y_{i+1}$: nilai baru

$\phi$: estimasi kemiringan (\textit{slope})

\end{frame}


\begin{frame}
\frametitle{Metode Euler}

\begin{columns}

\begin{column}{0.5\textwidth}
  {\centering
  \includegraphics[width=0.9\textwidth]{../../images_priv/Chapra_Fig_25_1.pdf}
  \par}    
\end{column}

\begin{column}{0.5\textwidth}
  Menggunakan turunan pertama pada $x_i$ sebagai estimasi langsung dari kemiringan $\phi$:
  \begin{equation*}
  \phi = f(x_i, y_i)
  \end{equation*}
  sehingga diperoleh:
  \begin{equation*}
  y_{i+1} = y_{i} + f(x_i, y_i) h
  \end{equation*}
\end{column}

\end{columns}

\end{frame}


% ------------------------------------

\begin{frame}
\frametitle{Contoh}
Cari solusi numerik dari IVP
\begin{equation*}
\frac{\mathrm{d}y}{\mathrm{d}x} = -2x^3 + 12x^2 - 20x + 8.5
\end{equation*}
dari $x = 0$ sampai $x = 4$ dengan ukuran langkah $h=0.5$.
Syarat awal: $y(x=0) = 1$.

Dengan membandingkan bentuk umum dari ODE \eqref{eq:ode_gen01}, diperoleh:
\begin{equation*}
f(x,y) = -2x^3 + 12x^2 - 20x + 8.5
\end{equation*}
Perhatikan bahwa pada kasus pada contoh ini, fungsi $f(x,y)$
tidak bergantung dari $y$.

Bandingkan solusi numerik yang diperoleh dengan solusi eksak:
\begin{equation*}
y(x) = -0.5x^4 + 4x^3 - 10x^2 + 8.5x + 1
\end{equation*}

\end{frame}



\begin{frame}[fragile]
\frametitle{Implementasi}

Pertama, kita akan mendefinisikan beberapa fungsi yang akan digunakan nantinya.

Pertama, kita akan mendefinisikan fungsi untuk aplikasi metode Euler satu langkah:
\begin{equation*}
y_{i+1} = y_{i} + f(x_i, y_i) h
\end{equation*}
dengan menggunakan fungsi berikut:
\begin{pythoncode}
def ode_euler_1step(dfunc, xi, yi, h):
    return yi + dfunc(xi,yi)*h
\end{pythoncode}

Fungsi yang dilemparkan \pyinline{dfunc} mengimplementasikan fungsi pada ruas kanan,
yaitu $f(x,y)$. Fungsi ini harus diimplementasikan dalam bentuk sebagai berikut.
\begin{pythoncode}
def dfunc(x,y):
    return ... # implement f(x,y)
\end{pythoncode}

Kita dapat memanggil \pyinline{ode_euler_1step} beberapa kali dari kondisi awal
sampai jumlah langkah yang diperlukan.

\end{frame}


% -------------------------------

\begin{frame}[fragile]
Potongan kode yang diberikan sebelumnya cukup umum. Sekarang kita akan
memberikan kode untuk kasus yang diberikan pada contoh, yaitu:
\begin{equation*}
\frac{\mathrm{d}y}{\mathrm{d}x} = -2x^3 + 12x^2 - 20x + 8.5
\end{equation*}

Fungsi berikut ini mendefinisikan fungsi pada ruas kanan $f(x,y)$
\begin{pythoncode}
def deriv(x, y):
    return -2*x**3 + 12*x**2 - 20*x + 8.5
\end{pythoncode}
Solusi eksak didefinisikan pada fungsi berikut.
\begin{pythoncode}
def exact_sol(x):
    return -0.5*x**4 + 4*x**3 - 10*x**2 + 8.5*x + 1
\end{pythoncode}
\end{frame}


\begin{frame}[fragile]

Pengujian \pyinline{ode_euler_1step}:
\begin{pythoncode}
x0 = 0.0; y0 = 1.0 # Initial condition
x = x0; y = y0; h = 0.5 # Using step of 0.5, starting from x0 and y0
xp1 = x + h # we are searching for solution at x = 0.5, increment by step size
yp1 = ode_euler_1step(deriv, x, y, h)
y_true = exact_sol(xp1)
ε_t = (y_true - yp1)/y_true * 100 # relative error in percent
print("First step : x = %f y_true = %.5f y = %.5f ε_t = %.1f %%" %
  (xp1, y_true, yp1, ε_t))
\end{pythoncode}

Tugas: lanjutkan sampai langkah-langkah selanjutnya.
Anda dapat menggunakan loop.

Contoh keluaran (dua langkah):
\begin{textcode}
First step : x = 0.500000 y_true = 3.21875 y = 5.25000 ε_t = -63.1 %
Second step: x = 1.000000 y_true = 3.00000 y = 5.87500 ε_t = -95.8 %
\end{textcode}

\end{frame}



\begin{frame}[fragile]

Contoh pemanggilan \pyinline{ode_euler_1step} beberapa kali.
\begin{pythoncode}
x = x0; y = y0; h = 0.5
for i in range(0,Nsteps):
    xp1 = x + h
    yp1 = ode_euler_1step(deriv, x, y, h)
    y_true = exact_sol(xp1) # calculate exact solution if available
    # relative error in percent, you can use other
    ε_t = (y_true - yp1)/y_true * 100
    # print the results ...
    # Update x and y for the next step
    x = xp1; y = yp1
\end{pythoncode}

Tugas: simpan hasil kalkulasi, yaitu \pyinline{x} dan
\pyinline{y} ke array dan buat plot.

Anda juga dapat mengimplementasikan ini dalam suatu fungsi.

\end{frame}


\begin{frame}
\frametitle{Metode Heun}

Metode Heun termasuk ke dalam kelompok metode prediktor-korektor. Metode Heun dapat
dituliskan sebagai berikut.
\begin{align}
y^{0}_{i+1} & = y_{i} + f(x_i, y_i) h & \text{   Prediktor}\\
y_{i+1} & = y_{i} + \frac{f(x_i, y_i) + f(x_{i+1}, y^{0}_{i+1})}{2} h & \text{   Korektor}
\end{align}
Karena $y_{i+1}$ berada pada kedua ruas dari persamaan korektor, maka kita harus mengaplikasikan
persamaan tersebut secara iteratif.

\end{frame}


\begin{frame}[fragile]
\frametitle{Metode Heun non-iteratif (\textit{one-shot})}

\begin{align*}
y^{0}_{i+1} & = y_{i} + f(x_i, y_i) h & \text{   Prediktor}\\
y_{i+1} & = y_{i} + \frac{f(x_i, y_i) + f(x_{i+1}, y^{0}_{i+1})}{2} h & \text{   Korektor}
\end{align*}

Jika tidak menggunakan skema iteratif, diperoleh metode Heun non-iteratif:
\begin{pythoncode}
# One-step application of Heun's method for ODE
def ode_heun_1step(dfunc, xi, yi, h):
    y0ip1 = yi + dfunc(xi,yi)*h
    avg = 0.5*( dfunc(xi,yi) + dfunc(xi+h,y0ip1) )*h
    return yi + avg
\end{pythoncode}

\end{frame}



\begin{frame}[fragile]
\frametitle{Metode Heun iteratif}

\begin{align*}
y^{0}_{i+1} & = y_{i} + f(x_i, y_i) h & \text{   Prediktor}\\
y_{i+1} & = y_{i} + \frac{f(x_i, y_i) + f(x_{i+1}, y^{0}_{i+1})}{2} h & \text{   Korektor}
\end{align*}

\begin{pythoncode}
# One-step application of Heun's method for ODE
# Using iterative steps to determine y0ip1
def ode_heun_1step_iterative(dfunc, xi, yi, h, NiterMax=100, Δ=1e-6):
    y0ip1 = yi + dfunc(xi,yi)*h
    y0ip1_old = y0ip1
    for i in range(NiterMax+1):
        avg = 0.5*( dfunc(xi,yi) + dfunc(xi+h,y0ip1) )*h
        y0ip1 = yi + avg
        diff = abs(y0ip1 - y0ip1_old)
        # Uncomment this to see the iteration process
        #print("iter: %2d y0ip1 = %12.7f  diff = %12.7e" % (i+1, y0ip1, diff))
        if diff <= Δ:
            break
        y0ip1_old = y0ip1
    return y0ip1
\end{pythoncode}

\end{frame}


\begin{frame}
\frametitle{Chapra Contoh 25.5}

Gunakan metode Heun untuk mengintegrasikan persamaan diferensial
\begin{equation*}
\frac{\mathrm{d}y}{\mathrm{d}x} = 4e^{0.8x} - 0.5y
\end{equation*}
dari $x=0$ sampai $x=4$ dengan ukuran langkah 1. Syarat awal adalah
$y(x=0) = 2$.
Bandingkan dengan solusi analitik:
\begin{equation*}
y = \frac{4}{1.3} \left( e^{0.8x} - e^{-0.5x} \right) + 2e^{-0.5x}
\end{equation*}

\end{frame}


\begin{frame}[fragile]

\begin{pythoncode}
# .... import library dan/atau definisi fungsi

def deriv(x, y):
    return 4*exp(0.8*x) - 0.5*y
    
def exact_sol(x):
    return 4.0/1.3*( exp(0.8*x) - exp(-0.5*x) ) + 2*exp(-0.5*x)

x0 = 0.0; y0 = 2.0 # Initial cond
xf = 4.0
h = 1.0
Nstep = int(xf/h)
    
x = x0; y = y0
for i in range(0,Nstep):
    xp1 = x + h
    yp1 = ode_heun_1step(deriv, x, y, h) # or use the iterative one
    y_true = exact_sol(xp1)
    ε_t = (y_true - yp1)/y_true * 100
    print("%f %12.7f %12.7f  %5.2f%%" % (xp1, y_true, yp1, abs(ε_t)))
    # For the next step
    x = xp1
    y = yp1    
\end{pythoncode}

\end{frame}



\begin{frame}[fragile]

Berikut ini adalah hasil yang diperoleh jika kita menggunakan metode
non-iteratif:
\begin{textcode}
------------------------------------------
  x         y_true        y_Heun      ε_t
------------------------------------------
0.000000    2.0000000
1.000000    6.1946314    6.7010819   8.18%
2.000000   14.8439219   16.3197819   9.94%
3.000000   33.6771718   37.1992489  10.46%
4.000000   75.3389626   83.3377673  10.62%  
\end{textcode}

Berikut ini adalah hasil dari aplikasi metode Heun iteratif. Error yang dihasilkan lebih
kecil daripada metode Heun non-iteratif.
\begin{textcode}
------------------------------------------
  x         y_true      y_Heun_iter    ε_t
------------------------------------------
0.000000    2.0000000
1.000000    6.1946314    6.3608654   2.68%
2.000000   14.8439219   15.3022364   3.09%
3.000000   33.6771718   34.7432760   3.17%
4.000000   75.3389626   77.7350961   3.18%
\end{textcode}

\end{frame}



\subsection{Metode titik tengah}

Metode titik tengah (midpoint atau improved polygon)
menggunakan metode Euler untuk memprediksi nilai $y$ pada titik tengah
interval:
\begin{equation*}
y_{i+1/2} = y_{i} + f(x_i, y_i) \frac{h}{2}
\end{equation*}
Kemudian nilai ini digunakan untuk menghitung kemiringan pada titik tengah:
\begin{equation*}
y'_{i+1/2} = f(x_{i+1/2}, y_{i+1/2})
\end{equation*}
yang diasumsikan sebagai aproksimasi dari rata-rata kemiringan untuk pada interval.
Kemiringan ini kemudian digunakan untuk mengekstrapolasi linear dari $x_i$ ke $x_{i+1}$:
\begin{equation}
y_{i+1} = y_{i} + f(x_{i+1/2}, y_{i+1/2}) h
\end{equation}

Implementasi:
\begin{pythoncode}
def ode_midpoint_1step(dfunc, xi, yi, h):
    yip12 = yi + deriv(xi,yi)*h/2  # midpoint value
    xip12 = xi + 0.5*h             # midpoint
    yip1 = yi + deriv(xip12,yip12)*h
    return yip1
\end{pythoncode}
\section{Metode Ralston}

\begin{pythoncode}
def ode_ralston_1step(dfunc, xi, yi, h):
    k1 = dfunc(xi, yi)
    k2 = dfunc(xi + 3*h/4, yi + 3*k1*h/4)
    yip1 = yi + (k1/3 + 2*k2/3)*h
    return yip1
\end{pythoncode}

\begin{frame}[fragile]
\frametitle{Metode Runge-Kutta Orde-3}

Skema:
\begin{equation*}
y_{i+1} = y_i + \frac{1}{6}(k_1 + 4k_2 + k_3)h
\end{equation*}
dengan
\begin{align*}
k_1 & = f(x_i, y_i) \\
k_2 & = f(x_i + \frac{1}{2}h, y_i + \frac{1}{2} k_1 h ) \\
k_3 & = f(x_i + h, y_i - k_1 h + 2 k_2 h)
\end{align*}
Perhatikan bahwa, jika fungsi turunan merupakan fungsi dari $x$ saja, maka metode ini menjadi
aturan 1/3 Simpson.

\begin{pythoncode}
def ode_rk3_1step(dfunc, xi, yi, h):
    k1 = dfunc(xi, yi)
    k2 = dfunc(xi + 0.5*h, yi + 0.5*k1*h)
    k3 = dfunc(xi + h, yi - k1*h + 2*k2*h)
    yip1 = yi + (k1 + 4*k2 + k3)*h/6
    return yip1
\end{pythoncode}

\end{frame}


%\begin{soal}
%Gunakan metode \textit{midpoint}, Ralston, Runge-Kutta orde-3, -4, dan -5 untuk
%menyelesaikan persamaan diferensial yang sama pada Chapra Contoh 25.5.
%\end{soal}







\subsection{Metode Runge-Kutta Orde-3}

Skema:
\begin{equation*}
y_{i+1} = y_i + \frac{1}{6}(k_1 + 4k_2 + k_3)h
\end{equation*}
dengan
\begin{align*}
k_1 & = f(x_i, y_i) \\
k_2 & = f(x_i + \frac{1}{2}h, y_i + \frac{1}{2} k_1 h ) \\
k_3 & = f(x_i + h, y_i - k_1 h + 2 k_2 h)
\end{align*}
Perhatikan bahwa, jika fungsi turunan merupakan fungsi dari $x$ saja, maka metode ini menjadi
aturan 1/3 Simpson.


\begin{pythoncode}
def ode_rk3_1step(dfunc, xi, yi, h):
    k1 = dfunc(xi, yi)
    k2 = dfunc(xi + 0.5*h, yi + 0.5*k1*h)
    k3 = dfunc(xi + h, yi - k1*h + 2*k2*h)
    yip1 = yi + (k1 + 4*k2 + k3)*h/6
    return yip1
\end{pythoncode}

\subsection{Metode Runge-Kutta Orde 4}

Skema:
\begin{equation*}
y_{i+1} = y_{i} + \frac{1}{6}(k_1 + 2k_2 + 2k_3 + k_4) h
\end{equation*}
dengan
\begin{align*}
k_1 & = f(x_i, y_i) \\
k_2 & = f\left( x_i + \frac{1}{2}h, y_i + \frac{1}{2} k_1 h \right) \\
k_3 & = f\left( x_i + \frac{1}{2}h, y_i + \frac{1}{2} k_2 h \right) \\
k_4 & = f(x_i + h, y_i + k_3 h)
\end{align*}

\begin{pythoncode}
def ode_rk4_1step(dfunc, xi, yi, h):
    k1 = dfunc(xi, yi)
    k2 = dfunc(xi + 0.5*h, yi + 0.5*k1*h)
    k3 = dfunc(xi + 0.5*h, yi + 0.5*k2*h)
    k4 = dfunc(xi + h, yi + k3*h)
    yip1 = yi + (k1 + 2*k2 + 2*k3 + k4)*h/6
    return yip1
\end{pythoncode}


\subsection{Metode Runge-Kutta Orde 5}

Skema (Butcher 1964):
\begin{equation*}
y_{i+1} = y_i + \frac{1}{90} ( 7 k_1 + 32 k_3 + 12 k_4 + 32 k_5 + 7 k_6) h
\end{equation*}
dengan:
\begin{align*}
k_1 & = f(x_i, y_i) \\
k_2 & = f\left( x_i + \frac{1}{4}h, y_i + \frac{1}{4} k_1 h \right) \\
k_3 & = f\left( x_i + \frac{1}{4}h, y_i + \frac{1}{8} k_1 h + \frac{1}{8} k_2 h \right) \\
k_4 & = f\left( x_i + \frac{1}{2}h, y_i - \frac{1}{2} k_2 h + k_3 h \right) \\
k_5 & = f\left( x_i + \frac{3}{4}h, y_i + \frac{3}{16} k_1 h + \frac{9}{16} k_4 h \right) \\
k_6 & = f\left( x_i + h, y_i - \frac{3}{7} k_1 h + \frac{2}{7} k_2 h +
\frac{12}{7}k_3 h - \frac{12}{7}k_4 h + \frac{8}{7}k_5 h \right)
\end{align*}

\begin{pythoncode}
def ode_rk5_1step(dfunc, xi, yi, h):
    k1 = dfunc(xi, yi)
    k2 = dfunc(xi + h/4, yi + k1*h/4)
    k3 = dfunc(xi + h/4, yi + k1*h/8 + k2*h/8)
    k4 = dfunc(xi + h/2, yi - k2*h/2 + k3*h)
    k5 = dfunc(xi + 3*h/4, yi + 3*k1*h/16 + 9*k4*h/16)
    k6 = dfunc(xi + h, yi - 3*k1*h/7 + 2*k2*h/7 + 12*k3*h/7 - 12*k4*h/7 + 8*k5*h/7)
    yip1 = yi + (7*k1 + 32*k3 + 12*k4 + 32*k5 + 7*k6)*h/90
    return yip1
\end{pythoncode}


\begin{soal}
Gunakan metode \textit{midpoint}, Ralston, Runge-Kutta orde-3, -4, dan -5 untuk
menyelesaikan persamaan diferensial yang sama pada Chapra Contoh 25.5.
\end{soal}






\begin{frame}
\frametitle{Metode Runge-Kutta Orde 5}
\fontsize{9pt}{8.0}\selectfont

Skema (Butcher 1964):
\begin{equation*}
y_{i+1} = y_i + \frac{1}{90} ( 7 k_1 + 32 k_3 + 12 k_4 + 32 k_5 + 7 k_6) h
\end{equation*}
dengan:
\begin{align*}
k_1 & = f(x_i, y_i) \\
k_2 & = f\left( x_i + \frac{1}{4}h, y_i + \frac{1}{4} k_1 h \right) \\
k_3 & = f\left( x_i + \frac{1}{4}h, y_i + \frac{1}{8} k_1 h + \frac{1}{8} k_2 h \right) \\
k_4 & = f\left( x_i + \frac{1}{2}h, y_i - \frac{1}{2} k_2 h + k_3 h \right) \\
k_5 & = f\left( x_i + \frac{3}{4}h, y_i + \frac{3}{16} k_1 h + \frac{9}{16} k_4 h \right) \\
k_6 & = f\left( x_i + h, y_i - \frac{3}{7} k_1 h + \frac{2}{7} k_2 h +
\frac{12}{7}k_3 h - \frac{12}{7}k_4 h + \frac{8}{7}k_5 h \right)
\end{align*}

\end{frame}



\begin{frame}[fragile]
\frametitle{Implementasi Metode Runge-Kutta Orde 5}

\begin{pythoncode}
def ode_rk5_1step(dfunc, xi, yi, h):
    k1 = dfunc(xi, yi)
    k2 = dfunc(xi + h/4, yi + k1*h/4)
    k3 = dfunc(xi + h/4, yi + k1*h/8 + k2*h/8)
    k4 = dfunc(xi + h/2, yi - k2*h/2 + k3*h)
    k5 = dfunc(xi + 3*h/4, yi + 3*k1*h/16 + 9*k4*h/16)
    k6 = dfunc(xi + h, yi - 3*k1*h/7 + 2*k2*h/7 + 12*k3*h/7 - 12*k4*h/7 + 8*k5*h/7)
    yip1 = yi + (7*k1 + 32*k3 + 12*k4 + 32*k5 + 7*k6)*h/90
    return yip1
\end{pythoncode}

\end{frame}


\begin{frame}[fragile]
\frametitle{Contoh}

Gunakan metode \textit{midpoint}, Ralston, Runge-Kutta orde-3, -4, dan -5 untuk
menyelesaikan persamaan diferensial yang sama pada Chapra Contoh 25.5.

\begin{pythoncode}
def deriv(x, y):
    return 4*exp(0.8*x) - 0.5*y
    
def exact_sol(x):
    return 4.0/1.3*( exp(0.8*x) - exp(-0.5*x) ) + 2*exp(-0.5*x)

# .... sama dengan kode sebelumnya
    
x = x0; y = y0
for i in range(0,Nstep):
    xp1 = x + h
    yp1 = ode_midpoint_1step(deriv, x, y, h) # ganti bagian ini
# .... sama dengan kode sebelumnya
#
\end{pythoncode}

\end{frame}


\section{Sistem Persamaan Diferensial Orde-1}

Skema satu langkah yang sudah dipelajari sebelumnya juga dapat digunakan
untuk sistem persamaan diferensial orde 1.

\begin{soal}[Chapra Contoh 25.9]
Cari solusi numerik dari sistem persamaan diferensial berikut:
\begin{align*}
\frac{\mathrm{d}y_1}{\mathrm{d}x} & = -0.5 y_1 \\
\frac{\mathrm{d}y_2}{\mathrm{d}x} & = 4 - 0.3 y_2 - 0.1y_1 \\
\end{align*}
dengan syarat awal pada $x=0$, $y_1 = 4$ dan $y_2 = 6$.
Cari solusi sampai pada $x=2$ dengan ukuran langkah 0.5.
\end{soal}

Berikut ini adalah program yang dapat kita gunakan.
\begin{pythoncode}
import numpy as np

def deriv(x, y):
    Nvec = len(y)
    # Here we make an assertion to make sure that y is a 2-component vector
    # Uncomment this line if the code appears to be slow
    assert Nvec == 2
    # Output array
    dydx = np.zeros(Nvec)
    # remember that in Python the array starts at 0
    # y1 = y[0]
    # y2 = y[1]
    dydx[0] = -0.5*y[0]
    dydx[1] = 4 - 0.3*y[1] - 0.1*y[0]
    # 
    return dydx
    
# One-step application of Euler's method for ODE
def ode_euler_1step(dfunc, xi, yi, h):
    return yi + dfunc(xi,yi)*h
    
def ode_solve(dfunc, do_1step, x0, y0, h, Nstep):
    Nvec = len(y0)
    x = np.zeros(Nstep+1)
    y = np.zeros((Nstep+1,Nvec))
    # Start with initial cond
    x[0] = x0
    y[0,:] = y0[:]
    for i in range(0,Nstep):
        x[i+1] = x[i] + h
        y[i+1,:] = do_1step(dfunc, x[i], y[i,:], h)
    return x, y
    
# initial cond
x0 = 0.0
y0 = np.array([4.0, 6.0])

h = 0.5
Nstep = 4
x, y = ode_solve(deriv, ode_euler_1step, x0, y0, h, Nstep)
print("")
print("---------------------------")
print(" x         y1         y2")
print("---------------------------")
for i in range(len(x)):
    print("%5.1f %10.6f %10.6f" % (x[i], y[i,0], y[i,1]))    
\end{pythoncode}

Contoh keluaran:
\begin{textcode}
---------------------------
   x       y1         y2
---------------------------
  0.0   4.000000   6.000000
  0.5   3.000000   6.900000
  1.0   2.250000   7.715000
  1.5   1.687500   8.445250
  2.0   1.265625   9.094088
\end{textcode}

Beberapa catatan:
\begin{itemize}
\item Tidak ada perubahan pada definisi fungsi \pyinline{ode_euler_1step}
\item Fungsi yang mendefinisikan (sistem) persamaan diferensial sekarang
mengembalikan array satu dimensi atau vektor.
\item Kita telah mendefinisikan fungsi \pyinline{ode_solve} yang dapat dikombinasikan
dengan metode satu langkah yang lain seperti \pyinline{ode_rk4_1step}.
\end{itemize}

\begin{soal}[Chapra Contoh 25.10]
Gunakan metode Runge-Kutta orde-4 untuk sistem persamaan diferensial yang didefinisikan
pada soal sebelumnya (Chapra Contoh 25.9).
\end{soal}

Contoh keluaran:
\begin{textcode}
---------------------------
   x        y1         y2
---------------------------
  0.0   4.000000   6.000000
  0.5   3.115234   6.857670
  1.0   2.426171   7.632106
  1.5   1.889523   8.326886
  2.0   1.471577   8.946865
\end{textcode}



\begin{soal}[Chapra Contoh 25.11]
Cari solusi numerik dari sistem persamaan diferensial berikut:
\begin{align*}
\frac{\mathrm{d}y_1}{\mathrm{d}x} & = y_2 \\
\frac{\mathrm{d}y_2}{\mathrm{d}x} & = -16.1 y_1 \\
\frac{\mathrm{d}y_3}{\mathrm{d}x} & = y_4 \\
\frac{\mathrm{d}y_4}{\mathrm{d}x} & = -16.1 \sin(y_3)
\end{align*}
untuk kasus-kasus syarat awal ($x=0$) berikut
\begin{itemize}
\item Pergeseran kecil: $y_1 = y_3 = 0.1$ radian, $y_2 = y_4 = 0$
\item Pergeseran besar: $y_1 = y_3 = \pi/4$ radian, $y_2 = y_4 = 0$.
\end{itemize}
\end{soal}


Anda dapat melengkapi kode berikut:
\begin{pythoncode}
# .... import dan definisi fungsi

def pendulum_ode(x, y):
    Nvec = len(y)
    # Here we make an assertion to make sure that y is a 4-component vector
    # Uncomment this line if the code appears to be slow
    assert Nvec == 4
    # Output array
    dydx = np.zeros(Nvec)
    # remember that in Python the array starts at 0
    # y1 = y[0]
    # y2 = y[1], etc ...
    dydx[0] = y[1]
    dydx[1] = -16.1*y[0]
    # Nonlinear effect
    dydx[2] = .... # LENGKAPI
    dydx[3] = .... # LENGKAPI
    # 
    return dydx


# initial cond
x0 = 0.0
y0 = np.array([0.1, 0.0, 0.1, 0.0]) # Small displacement

h = 0.01 # try playing with this parameter
xf = 4.0
Nstep = int(xf/h)
x, y = ode_solve(pendulum_ode, ode_rk4_1step, x0, y0, h, Nstep)

plt.clf()
plt.plot(x, y[:,0], label="y1")
plt.plot(x, y[:,1], label="y2")
plt.plot(x, y[:,2], label="y3")
plt.plot(x, y[:,3], label="y4")
plt.title("Small displacement case")
plt.ylim(-4,4) # The same for both small and large displacement
plt.legend()
plt.tight_layout()
plt.grid(True)


# initial cond
x0 = 0.0
y0 = np.array([np.pi/4, 0.0, np.pi/4, 0.0]) # Large displacement

# .... same as before

plt.clf()
# .... same as before
plt.title("Large displacement case")
# .... same as before
\end{pythoncode}

Contoh hasil visualisasi dapat dilihat pada Gambar \ref{fig:chapra_example_25_11}.

\begin{figure}[h]
{\centering
\includegraphics[width=0.45\textwidth]{../../chapra_7th/ch25/IMG_chapra_example_25_11_small.pdf}
\includegraphics[width=0.45\textwidth]{../../chapra_7th/ch25/IMG_chapra_example_25_11_large.pdf}
\par}
\caption{Chapra Contoh 25.11}
\label{fig:chapra_example_25_11}
\end{figure}
\begin{frame}
\frametitle{Persamaan diferensial orde dua}

Suatu persamaan diferensial biasa orde dua dapat diubah menjadi sistem persamaan
diferensial biasa orde 1.

Sebagai contoh, perhatikan persamaan diferensial biasa orde 2 berikut:
\begin{equation*}
m \frac{\mathrm{d}^2 x}{\mathrm{d}t^2} + c \frac{\mathrm{d}x}{\mathrm{d}t} + kx = 0
\end{equation*}
$x$: simpangan dari keadaan setimbang, $m$: massa benda, $c$: koefisien redaman, $k$ konstanta pegas.

{\centering
\includegraphics[scale=0.2]{../../images_priv/Chapra_Fig_P25_16.pdf}
\par}

\end{frame}




\begin{frame}
\frametitle{Persamaan diferensial orde dua}

\begin{columns}

\begin{column}{0.5\textwidth}
Definisikan:
\begin{align*}
y_{1}(t) & = x(t) \\
y_{2}(t) & = x'(t)
\end{align*}
%
sehingga:
%
\begin{align*}
y'_{1}(t) & = x'(t) = y_{2}(t) \\
y'_{2}(t) & = x''(t)
\end{align*}
Persamaan diferensial:
\begin{equation*}
m \frac{\mathrm{d}^2 x}{\mathrm{d}t^2} + c \frac{\mathrm{d}x}{\mathrm{d}t} + kx = 0
\end{equation*}
dituliskan menjadi:
\begin{equation*}
m y'_{2}(t) + c y_{2}(t) + k y_{1}(t) = 0
\end{equation*}

\end{column}


\begin{column}{0.5\textwidth}
Atau:
\begin{equation*}
y'_{2}(t) = -\frac{c y_{2}(t) + k y_{1}(t)}{m}
\end{equation*}
Sehingga diperoleh sistem persamaan diferensial biasa orde-1:
\begin{align*}
y'_{1}(t) & = y_{2}(t) \\
y'_{2}(t) & = -\frac{c y_{2}(t) + k y_{1}(t)}{m}
\end{align*}
\end{column}

\end{columns}

\end{frame}



% -----------------------------
\begin{frame}


{\centering
\includegraphics[scale=0.2]{../../images_priv/Chapra_Fig_P25_16.pdf}
\par}

$x$: simpangan dari keadaan setimbang, $m$: massa benda, $c$: koefisien redaman, $k$ konstanta pegas.

Solusi akan dicari untuk tiga kasus koefisien redaman (dalam Ns/m):
5 (\textit{underdamped}), 40 (\textit{critically damped}), and 200 (\textit{overdamped}).
Nilai parameter lain adalah: $m = 20$ kg dan $k = 20$ N/m

Kecepatan awal adalah 0 dan simpangan awal adalah $x_0$ = 1 m. Persamaan
ingin diselesaikan pada rentang waktu $0 \leq t \leq 15$ s.

\end{frame}



\begin{frame}[fragile]

Contoh kasus vibrasi \textit{underdamped}: $m = 20$ kg, $k = 20$ N/m dan $c = 5$ Ns/m

\begin{pythoncode}
def deriv_underdamped(x, y):
    # Parameters (local)
    m = 20.0   # kg
    k = 20.0   # N/m    
    c = 5.0    # Ns/m underdamped
    Nvec = len(y)
    assert Nvec == 2
    # Output array
    dydx = np.zeros(Nvec)
    dydx[0] = y[1]
    dydx[1] = -(c*y[1] + k*y[0])/m
    #
    return dydx
\end{pythoncode}

\end{frame}



\begin{frame}[fragile]

Contoh kasus vibrasi \textit{overdamped}: $m = 20$ kg, $k = 20$ N/m dan $c = 200$ Ns/m

\begin{pythoncode}
def deriv_overdamped(x, y):
    # Parameters (local)
    m = 20.0   # kg
    k = 20.0   # N/m    
    c = 200.0  # Ns/m overdamped
    Nvec = len(y)
    assert Nvec == 2
    # Output array
    dydx = np.zeros(Nvec)
    dydx[0] = y[1]
    dydx[1] = -(c*y[1] + k*y[0])/m
    #
    return dydx
\end{pythoncode}

\end{frame}



\begin{frame}[fragile]

Contoh kasus vibrasi \textit{critically damped}: $m = 20$ kg, $k = 20$ N/m dan $c = 40$ Ns/m

\begin{pythoncode}
def deriv_critical_damped(x, y):
    # Parameters (local)
    m = 20.0   # kg
    k = 20.0   # N/m    
    c = 40.0   # Ns/m critically damped
    Nvec = len(y)
    assert Nvec == 2
    # Output array
    dydx = np.zeros(Nvec)
    dydx[0] = y[1]
    dydx[1] = -(c*y[1] + k*y[0])/m
    #
    return dydx
\end{pythoncode}

\end{frame}


\begin{frame}[fragile]

\begin{pythoncode}
t0 = 0.0
initial_displ = 1.0
initial_vel = 0.0
x0 = np.array([initial_displ, initial_vel]) # initial conditions
Δt = 0.01 # try playing with this parameter
tf = 15.0
Nstep = int(tf/Δt)
t, x = ode_solve(deriv_underdamped, ode_rk4_1step, t0, x0, Δt, Nstep)
\end{pythoncode}

Tugas: selesaikan untuk kasus lain (crictically damped dan overdamped) serta
buat plot perbandingan tiga kasus tersebut.

\end{frame}
\begin{frame}
\frametitle{Chapra Latihan 25.1}

Selesaikan persamaan diferensial biasa berikut secara numerik
\begin{equation*}
\frac{\mathrm{d} y}{\mathrm{d}t} = yt^2 - 1.1y
\end{equation*}
untuk $t = 0$ sampai $t=2$
dengan syarat awal $y(0) = 1$.

\end{frame}



\begin{frame}
\frametitle{Chapra Latihan 25.2}

Selesaikan persamaan diferensial biasa berikut secara numerik
\begin{equation*}
\frac{\mathrm{d} y}{\mathrm{d}t} = (1 + 4t)\sqrt{y}
\end{equation*}
untuk $t = 0$ sampai $t=1$
dengan syarat awal $y(0) = 1$.

\end{frame}



\begin{frame}
\frametitle{Chapra Latihan 25.4}

Selesaikan persamaan diferensial biasa berikut secara numerik
\begin{equation*}
\frac{\mathrm{d}^2 y}{\mathrm{d}x^2} + 0.6\frac{\mathrm{d}y}{\mathrm{d}x} + 8y = 0
\end{equation*}
untuk $x = 0$ sampai $x=5$
dengan syarat awal $y(0) = 4$ dan $y'(0) = 0$.

\end{frame}


\begin{frame}
\frametitle{Chapra Latihan 25.7}

Selesaikan sistem persamaan diferensial biasa berikut secara numerik
\begin{align*}
\frac{\mathrm{d}y}{\mathrm{d}t} = -2y + 5e^{-t} \\
\frac{\mathrm{d}z}{\mathrm{d}t} = -\frac{yz^2}{2} \\
\end{align*}
untuk $t = 0$ sampai $t=0.4$
dengan syarat awal $y(0) = 2$ dan $z(0) = 4$.

\end{frame}



\begin{frame}
\frametitle{Chapra Latihan 25.18}

Selesaikan persamaan diferensial biasa orde 2 berikut secara numerik
\begin{equation*}
\frac{\mathrm{d}^2 x}{\mathrm{d}t^2} + 5x \frac{\mathrm{d}x}{\mathrm{d}t} +
(x + 7) \sin(\omega t) = 0
\end{equation*}
untuk $t = 0$ sampai $t=15$
dengan syarat awal $x(0) = 6$ dan $x'(0) = 1.5$.
Ambil nilai $\omega = 1$.

\end{frame}


% ----------------------------------
\begin{frame}
\frametitle{Chapra Latihan 25.24}

Selesaikan persamaan diferensial biasa orde 2 berikut secara numerik
\begin{equation*}
\frac{\mathrm{d}^2 y}{\mathrm{d}t^2} + 4y = 0
\end{equation*}
untuk $t = 0$ sampai $t=4$
dengan syarat awal $y(0) = 1$ dan $y'(0) = 0$.
Bandingkan dengan solusi analitik $y = \cos(2t)$.

\end{frame}





\end{document}

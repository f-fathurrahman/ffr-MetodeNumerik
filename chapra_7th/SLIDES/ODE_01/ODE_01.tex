\input{../PREAMBLE_BEAMER}

\begin{document}

\title{Metode Numerik}
\subtitle{Persamaan Diferensial Biasa}
\author{Fadjar Fathurrahman}
\institute{
Teknik Fisika \\
Institut Teknologi Bandung
}
\date{}


\frame{\titlepage}


\begin{frame}
\frametitle{Contoh}

Persamaan gerak Newton:
\begin{equation*}
\frac{\mathrm{d}v}{\mathrm{d}t} = \frac{F}{m}
\end{equation*}

Persamaan kalor Fourier:
\begin{equation*}
q = -k \frac{\mathrm{d}T}{\mathrm{d}x}
\end{equation*}

Hukum difusi Fick:
\begin{equation*}
J = -D \frac{\mathrm{d}c}{\mathrm{d}x}
\end{equation*}

Hukum Faraday:
\begin{equation*}
\Delta V_{L} = L \frac{\mathrm{d}i}{\mathrm{d}t}
\end{equation*}

\end{frame}


% ----------------------------------------------------
\begin{frame}
\frametitle{Bentuk umum permasalahan nilai awal}

Permasalahan nilai awal: \textit{initial value problem} (IVP)

Selesaikan persamaan diferensial
\begin{equation}
\frac{\mathrm{d}y}{\mathrm{d}x} = f(x,y)
\label{eq:ode_gen01}
\end{equation}
diberikan syarat awal: $y(x=x_0) = y_0$.

$y(x)$ adalah variabel dependen (atau fungsi yang ingin dicari)
dan $x$ adalah variabel independen.

Untuk masalah nilai awal, biasanya lebih sering menggunakan $t$ sebagai
variabel independen:
\begin{equation}
\frac{\mathrm{d}y}{\mathrm{d}t} = f(t,y)
\label{eq:ode_gen02}
\end{equation}

\end{frame}

% -----------------------

\begin{frame}
\frametitle{Metode-metode untuk IVP}
  
\begin{itemize}
\item Metode satu-langkah
\item Metode multi-langkah
\item Metode implisit
\end{itemize}

\end{frame}



% -----------------------

\begin{frame}
\frametitle{Metode satu langkah}

Bentuk umum:
\begin{equation*}
y_{i+1} = y_{i} + \phi h
\end{equation*}

$y_{i}$: nilai lama

$y_{i+1}$: nilai baru

$\phi$: estimasi kemiringan (\textit{slope})

\end{frame}


\begin{frame}
\frametitle{Metode Euler}

\begin{columns}

\begin{column}{0.5\textwidth}
  {\centering
  \includegraphics[width=0.9\textwidth]{../../images_priv/Chapra_Fig_25_1.pdf}
  \par}    
\end{column}

\begin{column}{0.5\textwidth}
  Menggunakan turunan pertama pada $x_i$ sebagai estimasi langsung dari kemiringan $\phi$:
  \begin{equation*}
  \phi = f(x_i, y_i)
  \end{equation*}
  sehingga diperoleh:
  \begin{equation*}
  y_{i+1} = y_{i} + f(x_i, y_i) h
  \end{equation*}
\end{column}

\end{columns}

\end{frame}


% ------------------------------------

\begin{frame}
\frametitle{Contoh}
Cari solusi numerik dari IVP
\begin{equation*}
\frac{\mathrm{d}y}{\mathrm{d}x} = -2x^3 + 12x^2 - 20x + 8.5
\end{equation*}
dari $x = 0$ sampai $x = 4$ dengan ukuran langkah $h=0.5$.
Syarat awal: $y(x=0) = 1$.

Dengan membandingkan bentuk umum dari ODE \eqref{eq:ode_gen01}, diperoleh:
\begin{equation*}
f(x,y) = -2x^3 + 12x^2 - 20x + 8.5
\end{equation*}
Perhatikan bahwa pada kasus pada contoh ini, fungsi $f(x,y)$
tidak bergantung dari $y$.

Bandingkan solusi numerik yang diperoleh dengan solusi eksak:
\begin{equation*}
y(x) = -0.5x^4 + 4x^3 - 10x^2 + 8.5x + 1
\end{equation*}

\end{frame}



\begin{frame}[fragile]
\frametitle{Implementasi}

Pertama, kita akan mendefinisikan beberapa fungsi yang akan digunakan nantinya.

Pertama, kita akan mendefinisikan fungsi untuk aplikasi metode Euler satu langkah:
\begin{equation*}
y_{i+1} = y_{i} + f(x_i, y_i) h
\end{equation*}
dengan menggunakan fungsi berikut:
\begin{pythoncode}
def ode_euler_1step(dfunc, xi, yi, h):
    return yi + dfunc(xi,yi)*h
\end{pythoncode}

Fungsi yang dilemparkan \pyinline{dfunc} mengimplementasikan fungsi pada ruas kanan,
yaitu $f(x,y)$. Fungsi ini harus diimplementasikan dalam bentuk sebagai berikut.
\begin{pythoncode}
def dfunc(x,y):
    return ... # implement f(x,y)
\end{pythoncode}

Kita dapat memanggil \pyinline{ode_euler_1step} beberapa kali dari kondisi awal
sampai jumlah langkah yang diperlukan.

\end{frame}


\begin{frame}[fragile]
Potongan kode yang diberikan sebelumnya cukup umum. Sekarang kita akan
memberikan kode untuk kasus yang diberikan pada contoh, yaitu:
\begin{equation*}
\frac{\mathrm{d}y}{\mathrm{d}x} = -2x^3 + 12x^2 - 20x + 8.5
\end{equation*}

Fungsi berikut ini mendefinisikan fungsi pada ruas kanan $f(x,y)$
\begin{pythoncode}
def deriv(x, y):
    return -2*x**3 + 12*x**2 - 20*x + 8.5
\end{pythoncode}
Solusi eksak didefinisikan pada fungsi berikut.
\begin{pythoncode}
def exact_sol(x):
    return -0.5*x**4 + 4*x**3 - 10*x**2 + 8.5*x + 1
\end{pythoncode}
\end{frame}


\begin{frame}[fragile]

Pengujian \pyinline{ode_euler_1step}:
\begin{pythoncode}
x0 = 0.0; y0 = 1.0 # Initial condition
x = x0; y = y0; h = 0.5 # Using step of 0.5, starting from x0 and y0
xp1 = x + h # we are searching for solution at x = 0.5, increment by step size
yp1 = ode_euler_1step(deriv, x, y, h)
y_true = exact_sol(xp1)
ε_t = (y_true - yp1)/y_true * 100 # relative error in percent
print("First step : x = %f y_true = %.5f y = %.5f ε_t = %.1f %%" %
  (xp1, y_true, yp1, ε_t))
\end{pythoncode}

Tugas: lanjutkan sampai langkah-langkah selanjutnya.
Anda dapat menggunakan loop.

Contoh keluaran (dua langkah):
\begin{textcode}
First step : x = 0.500000 y_true = 3.21875 y = 5.25000 ε_t = -63.1 %
Second step: x = 1.000000 y_true = 3.00000 y = 5.87500 ε_t = -95.8 %
\end{textcode}

\end{frame}



\begin{frame}[fragile]

Contoh pemanggilan \pyinline{ode_euler_1step} beberapa kali.
\begin{pythoncode}
x = x0; y = y0; h = 0.5
for i in range(0,Nsteps):
    xp1 = x + h
    yp1 = ode_euler_1step(deriv, x, y, h)
    y_true = exact_sol(xp1) # calculate exact solution if available
    # relative error in percent, you can use other
    ε_t = (y_true - yp1)/y_true * 100
    # print the results ...
    # Update x and y for the next step
    x = xp1; y = yp1
\end{pythoncode}

Tugas: simpan hasil kalkulasi, yaitu \pyinline{x} dan
\pyinline{y} ke array dan buat plot.

Anda juga dapat mengimplementasikan ini dalam suatu fungsi.

\end{frame}


\subsection{Metode Heun}

Metode Heun
\begin{equation*}
y'_{i} = f(x_i, y_i)
\end{equation*}
ekstrapolasi:
\begin{equation*}
y^{0}_{i+1} = y_{i} + f(x_i, y_i) h
\end{equation*}
Untuk metode Heun, $y^{0}_{i+1}$ digunakan sebagai prediksi intermediat.
Persamaan ini dikenal sebagai persamaan prediktor.
Persamaan ini memberikan estimasi $y_{i+1}$ pada $x_{i+1}$:
\begin{equation*}
y'_{i+1} = f(x_{i+1}, y^{0}_{i+1})
\end{equation*}
Dengan menggunakan rata-rata dari nilai kemiringan ini diperoleh:
\begin{equation*}
\overline{y}' = \frac{y'_{i} + y'_{i+1}}{2} = 
\frac{f(x_i, y_i) + f(x_{i+1},y^{0}_{i+1})}{2}
\end{equation*}
Kemiringan rata-rata ini kemudian digunakan untuk ekstrapolasi linear dari $y_{i}$ ke
$y_{i+1}$:
\begin{equation*}
y_{i+1} = y_{i} + \frac{f(x_i, y_i) + f(x_{i+1}, y^{0}_{i+1})}{2} h
\end{equation*}
Persamaan ini dikenal sebagai persamaan korektor.

Metode Heun termasuk ke dalam kelompok metode prediktor-korektor. Metode Heun dapat
dituliskan sebagai berikut.
\begin{align}
y^{0}_{i+1} & = y_{i} + f(x_i, y_i) h & \text{   Prediktor}\\
y_{i+1} & = y_{i} + \frac{f(x_i, y_i) + f(x_{i+1}, y^{0}_{i+1})}{2} h & \text{   Korektor}
\end{align}
Karena $y_{i+1}$ berada pada kedua ruas dari persamaan korektor, maka kita harus mengaplikasikan
persamaan tersebut secara iteratif.


\begin{soal}[Chapra Contoh 25.5]
Gunakan metode Heun untuk mengintegrasikan persamaan diferensial
\begin{equation*}
\frac{\mathrm{d}y}{\mathrm{d}x} = 4e^{0.8x} - 0.5y
\end{equation*}
dari $x=0$ sampai $x=4$ dengan ukuran langkah 1. Syarat awal adalah
$y(x=0) = 2$.
Bandingkan dengan solusi analitik:
\begin{equation*}
y = \frac{4}{1.3} \left( e^{0.8x} - e^{-0.5x} \right) + 2e^{-0.5x}
\end{equation*}
\end{soal}


\begin{pythoncode}
# One-step application of Heun's method for ODE
def ode_heun_1step(dfunc, xi, yi, h):
    y0ip1 = yi + dfunc(xi,yi)*h
    avg = 0.5*( dfunc(xi,yi) + dfunc(xi+h,y0ip1) )*h
    return yi + avg
\end{pythoncode}

\begin{pythoncode}
from math import exp

# .... import dan/atau definisi

def deriv(x, y):
    return 4*exp(0.8*x) - 0.5*y
    
def exact_sol(x):
    return 4.0/1.3*( exp(0.8*x) - exp(-0.5*x) ) + 2*exp(-0.5*x)
    
# Initial cond
x0 = 0.0
y0 = 2.0
xf = 4.0
h = 1.0
Nstep = int(xf/h)
    
x = x0
y = y0
for i in range(0,Nstep):
    xp1 = x + h
    yp1 = ode_heun_1step(deriv, x, y, h)
    y_true = exact_sol(xp1)
    ε_t = (y_true - yp1)/y_true * 100
    print("%f %12.7f %12.7f  %5.2f%%" % (xp1, y_true, yp1, abs(ε_t)))
    # For the next step
    x = xp1
    y = yp1    
\end{pythoncode}


Iteratif:
\begin{pythoncode}
# One-step application of Heun's method for ODE
# Using iterative steps to determine y0ip1
def ode_heun_1step_iterative(dfunc, xi, yi, h, NiterMax=100, Δ=1e-6):
    y0ip1 = yi + dfunc(xi,yi)*h
    y0ip1_old = y0ip1
    for i in range(NiterMax+1):
        avg = 0.5*( dfunc(xi,yi) + dfunc(xi+h,y0ip1) )*h
        y0ip1 = yi + avg
        diff = abs(y0ip1 - y0ip1_old)
        # Uncomment this to see the iteration process
        print("iter: %2d y0ip1 = %12.7f  diff = %12.7e" % (i+1, y0ip1, diff))
        if diff <= Δ:
            break
        y0ip1_old = y0ip1
    return y0ip1
\end{pythoncode}

\section{Metode titik tengah}

Metode titik tengah (midpoint atau improved polygon)
menggunakan metode Euler untuk memprediksi nilai $y$ pada titik tengah
interval:
\begin{equation*}
y_{i+1/2} = y_{i} + f(x_i, y_i) \frac{h}{2}
\end{equation*}
Kemudian nilai ini digunakan untuk menghitung kemiringan pada titik tengah:
\begin{equation*}
y'_{i+1/2} = f(x_{i+1/2}, y_{i+1/2})
\end{equation*}
yang diasumsikan sebagai aproksimasi dari rata-rata kemiringan untuk pada interval.
Kemiringan ini kemudian digunakan untuk mengekstrapolasi linear dari $x_i$ ke $x_{i+1}$:
\begin{equation}
y_{i+1} = y_{i} + f(x_{i+1/2}, y_{i+1/2}) h
\end{equation}

Implementasi:
\begin{pythoncode}
# One-step application of midpoint method for ODE
def ode_midpoint_1step(dfunc, xi, yi, h):
    yip12 = yi + deriv(xi,yi)*h/2  # midpoint value
    xip12 = xi + 0.5*h             # midpoint
    yip1 = yi + deriv(xip12,yip12)*h
    return yip1
\end{pythoncode}
\subsection{Metode Ralston}

Skema:
\begin{equation*}
y_{i+1} = y_i + \left( \frac{1}{3}k_1 + \frac{2}{3}k_2 \right) h
\end{equation*}
dengan
\begin{align*}
k_1 & = f(x_i, y_i) \\
k_2 & = f\left( x_i + \frac{3}{4}h, y_i + \frac{3}{4}k_1 h \right)
\end{align*}

\begin{pythoncode}
def ode_ralston_1step(dfunc, xi, yi, h):
    k1 = dfunc(xi, yi)
    k2 = dfunc(xi + 3*h/4, yi + 3*k1*h/4)
    yip1 = yi + (k1/3 + 2*k2/3)*h
    return yip1
\end{pythoncode}

\begin{frame}[fragile]
\frametitle{Metode Runge-Kutta Orde-3}

Skema:
\begin{equation*}
y_{i+1} = y_i + \frac{1}{6}(k_1 + 4k_2 + k_3)h
\end{equation*}
dengan
\begin{align*}
k_1 & = f(x_i, y_i) \\
k_2 & = f(x_i + \frac{1}{2}h, y_i + \frac{1}{2} k_1 h ) \\
k_3 & = f(x_i + h, y_i - k_1 h + 2 k_2 h)
\end{align*}
Perhatikan bahwa, jika fungsi turunan merupakan fungsi dari $x$ saja, maka metode ini menjadi
aturan 1/3 Simpson.

\begin{pythoncode}
def ode_rk3_1step(dfunc, xi, yi, h):
    k1 = dfunc(xi, yi)
    k2 = dfunc(xi + 0.5*h, yi + 0.5*k1*h)
    k3 = dfunc(xi + h, yi - k1*h + 2*k2*h)
    yip1 = yi + (k1 + 4*k2 + k3)*h/6
    return yip1
\end{pythoncode}

\end{frame}


%\begin{soal}
%Gunakan metode \textit{midpoint}, Ralston, Runge-Kutta orde-3, -4, dan -5 untuk
%menyelesaikan persamaan diferensial yang sama pada Chapra Contoh 25.5.
%\end{soal}







\begin{frame}[fragile]
\frametitle{Metode Runge-Kutta Orde 4}
\fontsize{9pt}{8.0}\selectfont

Skema:
\begin{equation*}
y_{i+1} = y_{i} + \frac{1}{6}(k_1 + 2k_2 + 2k_3 + k_4) h
\end{equation*}
dengan
\begin{align*}
k_1 & = f(x_i, y_i) \\
k_2 & = f\left( x_i + \frac{1}{2}h, y_i + \frac{1}{2} k_1 h \right) \\
k_3 & = f\left( x_i + \frac{1}{2}h, y_i + \frac{1}{2} k_2 h \right) \\
k_4 & = f(x_i + h, y_i + k_3 h)
\end{align*}

\begin{pythoncode}
def ode_rk4_1step(dfunc, xi, yi, h):
    k1 = dfunc(xi, yi)
    k2 = dfunc(xi + 0.5*h, yi + 0.5*k1*h)
    k3 = dfunc(xi + 0.5*h, yi + 0.5*k2*h)
    k4 = dfunc(xi + h, yi + k3*h)
    yip1 = yi + (k1 + 2*k2 + 2*k3 + k4)*h/6
    return yip1
\end{pythoncode}

\end{frame}

\begin{frame}
\frametitle{Metode Runge-Kutta Orde 5}
\fontsize{9pt}{8.0}\selectfont

Skema (Butcher 1964):
\begin{equation*}
y_{i+1} = y_i + \frac{1}{90} ( 7 k_1 + 32 k_3 + 12 k_4 + 32 k_5 + 7 k_6) h
\end{equation*}
dengan:
\begin{align*}
k_1 & = f(x_i, y_i) \\
k_2 & = f\left( x_i + \frac{1}{4}h, y_i + \frac{1}{4} k_1 h \right) \\
k_3 & = f\left( x_i + \frac{1}{4}h, y_i + \frac{1}{8} k_1 h + \frac{1}{8} k_2 h \right) \\
k_4 & = f\left( x_i + \frac{1}{2}h, y_i - \frac{1}{2} k_2 h + k_3 h \right) \\
k_5 & = f\left( x_i + \frac{3}{4}h, y_i + \frac{3}{16} k_1 h + \frac{9}{16} k_4 h \right) \\
k_6 & = f\left( x_i + h, y_i - \frac{3}{7} k_1 h + \frac{2}{7} k_2 h +
\frac{12}{7}k_3 h - \frac{12}{7}k_4 h + \frac{8}{7}k_5 h \right)
\end{align*}

\end{frame}



\begin{frame}[fragile]
\frametitle{Implementasi Metode Runge-Kutta Orde 5}

\begin{pythoncode}
def ode_rk5_1step(dfunc, xi, yi, h):
    k1 = dfunc(xi, yi)
    k2 = dfunc(xi + h/4, yi + k1*h/4)
    k3 = dfunc(xi + h/4, yi + k1*h/8 + k2*h/8)
    k4 = dfunc(xi + h/2, yi - k2*h/2 + k3*h)
    k5 = dfunc(xi + 3*h/4, yi + 3*k1*h/16 + 9*k4*h/16)
    k6 = dfunc(xi + h, yi - 3*k1*h/7 + 2*k2*h/7 + 12*k3*h/7 - 12*k4*h/7 + 8*k5*h/7)
    yip1 = yi + (7*k1 + 32*k3 + 12*k4 + 32*k5 + 7*k6)*h/90
    return yip1
\end{pythoncode}

\end{frame}


\begin{frame}[fragile]
\frametitle{Contoh}

Gunakan metode \textit{midpoint}, Ralston, Runge-Kutta orde-3, -4, dan -5 untuk
menyelesaikan persamaan diferensial yang sama pada Chapra Contoh 25.5.

\begin{pythoncode}
def deriv(x, y):
    return 4*exp(0.8*x) - 0.5*y
    
def exact_sol(x):
    return 4.0/1.3*( exp(0.8*x) - exp(-0.5*x) ) + 2*exp(-0.5*x)

# .... sama dengan kode sebelumnya
    
x = x0; y = y0
for i in range(0,Nstep):
    xp1 = x + h
    yp1 = ode_midpoint_1step(deriv, x, y, h) # ganti bagian ini
# .... sama dengan kode sebelumnya
#
\end{pythoncode}

\end{frame}


\end{document}

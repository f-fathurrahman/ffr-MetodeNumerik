\begin{frame}[fragile]
\frametitle{Metode Runge-Kutta Orde-3}

Skema:
\begin{equation*}
y_{i+1} = y_i + \frac{1}{6}(k_1 + 4k_2 + k_3)h
\end{equation*}
dengan
\begin{align*}
k_1 & = f(x_i, y_i) \\
k_2 & = f(x_i + \frac{1}{2}h, y_i + \frac{1}{2} k_1 h ) \\
k_3 & = f(x_i + h, y_i - k_1 h + 2 k_2 h)
\end{align*}
Perhatikan bahwa, jika fungsi turunan merupakan fungsi dari $x$ saja, maka metode ini menjadi
aturan 1/3 Simpson.

\begin{pythoncode}
def ode_rk3_1step(dfunc, xi, yi, h):
    k1 = dfunc(xi, yi)
    k2 = dfunc(xi + 0.5*h, yi + 0.5*k1*h)
    k3 = dfunc(xi + h, yi - k1*h + 2*k2*h)
    yip1 = yi + (k1 + 4*k2 + k3)*h/6
    return yip1
\end{pythoncode}

\end{frame}


%\begin{soal}
%Gunakan metode \textit{midpoint}, Ralston, Runge-Kutta orde-3, -4, dan -5 untuk
%menyelesaikan persamaan diferensial yang sama pada Chapra Contoh 25.5.
%\end{soal}







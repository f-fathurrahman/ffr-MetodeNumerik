\section{Aturan Simpson}


\begin{frame}
\frametitle{Aturan Simpson}

Aturan Simpson adalah formula integrasi Newton-Cotes yang menggunakan:
\begin{itemize}
\item tiga titik (aturan 1/3 Simpson)
\item empat titik (aturan 3/8 Simpson)
\end{itemize}

{\centering
\includegraphics[width=0.7\textwidth]{images_priv/Chapra_Fig_21_10.pdf}
\par}

\end{frame}




\begin{frame}
\frametitle{Aturan 1/3 Simpson}

Pada aturan Simpson, polinomial yang digunakan adalah orde 2, sehingga memerlukan tidak titik
dasar. Aplikasi aturan Simpson untuk satu selang dapat dituliskan sebagai berikut:
\begin{equation*}
I = \frac{h}{3}( f(x_0) + 4f(x_1) + f(x_2) )
\end{equation*}
di mana $h = (b-a)/2$. Aturan ini dikenal sebagai aturan 1/3 Simpson.
Aturan ini juga dapat dituliskan sebagai:
\begin{equation*}
I = \frac{b-a}{6}( f(x_0) + 4f(x_1) + f(x_2) )
\end{equation*}
Kesalahan pemotongan lokal dapat dituliskan sebagai:
\begin{equation*}
E_{t} = -\frac{(b-a)^5}{2880} f^{(4)}(\xi)
\end{equation*}
dengan $\xi$ berada di dalam interval $[a,b]$.

\end{frame}




\begin{frame}[fragile]
\frametitle{Contoh}

\begin{pythoncode}
def my_func(x):
    return 0.2 + 25*x - 200*x**2 + 675*x**3 - 900*x**4 + 400*x**5

a = 0.0; b = 0.8
h = (b-a)/2
# Base points for Simpson's 1/3 rule
x0 = a
x1 = a + h
x2 = b
# We use 3 points or 2 segments.

I_exact = 1.640533 # from the book

I = .... # LENGKAPI

E_t = (I_exact - I)
ε_t = E_t/I_exact * 100
print("Integral result = %.7f" % I)
print("True integral   = %.6f" % I_exact)
print("True error      = %.7f" % E_t)
print("ε_t             = %.1f%%" % ε_t)
\end{pythoncode}
\end{frame}


\begin{frame}[fragile]
\frametitle{Estimasi error}

\begin{equation*}
E_{t} = -\frac{(b-a)^5}{2880} f^{(4)}(\xi)
\end{equation*}

\begin{equation*}
f^{(4)}(\xi) \approx \dfrac{\int_{a}^{b} f^{(4)}(x) \, \mathrm{d}x}{b - a}
\end{equation*}

\begin{pythoncode}
import sympy
x = sympy.symbols("x")
f = 0.2 + 25*x - 200*x**2 + 675*x**3 - 900*x**4 + 400*x**5
d4f = f.diff(x,4)
avg_d4f_xi = sympy.integrate( d4f, (x,a,b) )/(b - a)
E_a = -1/2880*avg_d4f_xi*(b - a)**5
print("Approx error    = %.7f" % E_a)
\end{pythoncode}

\end{frame}




\begin{frame}[fragile]
\frametitle{Implementasi (dalam fungsi)}

\begin{equation*}
I = \frac{h}{3}( f(x_0) + 4f(x_1) + f(x_2) )
\end{equation*}

\begin{pythoncode}
def integ_simpson13( f, a, b ):
    h = (b - a)/2
    x0 = a
    x1 = a + h
    x2 = b
    I = h/3 * ( f(x0) + 4*f(x1) + f(x2) )
    return I
\end{pythoncode}

\end{frame}



\begin{frame}[fragile]
\frametitle{Contoh keluaran}

\begin{textcode}
Integral result = 1.3674667
True integral   = 1.640533
True error      = 0.2730663
ε_t             = 16.6%
Approx error    = 0.2730667    
\end{textcode}

Bandingkan dengan aturan trapesium.

\end{frame}




\begin{frame}
\frametitle{Aturan 1/3 Simpson Multi-interval}

Aturan 1/3 Simpson dapat digunakan untuk beberapa interval:
\begin{equation*}
\frac{b-a}{3n} \left(
f(x_0) + 4\sum_{i\text{ odd}}^{n-1} f(x_i) + 2 \sum_{j\text{ even}}^{n-2} f(x_j) + f(x_n)
\right)
\end{equation*}
dengan kesalahan pemotongan:
\begin{equation*}
E_{a} = -\frac{(b-a)^5}{180n^4} \overline{f}^{(4)}
\end{equation*}

\end{frame}


\begin{frame}[fragile]
\frametitle{Implementasi}

\begin{pythoncode}
def integ_simpson13_multiple( f, a, b, N ):
    # N: is number of segments
    assert N >= 2
    assert N % 2 == 0 # must be an even number
    
    x0 = a
    xN = b
    h = (b - a)/N
    x = np.linspace(a, b, N+1)
    # Total number of base points is N+1 (an odd number)
    
    ss_odd = 0.0
    for i in range(1,N,2):
        ss_odd = ss_odd + f(x[i])
        
    ss_even = 0.0
    for i in range(2,N-1,2):
        ss_even = ss_even + f(x[i])
    
    I = (b - a)/(3*N) * ( f(x0) + 4*ss_odd + 2*ss_even + f(xN) )
    return I
\end{pythoncode}

\end{frame}




\begin{frame}
\frametitle{Aturan 3/8 Simpson}

Dengan menggunakan 4 titik (polinomial kubik), dapat diturunkan aturan 3/8 Simpson:
\begin{equation*}
I = \frac{3h}{8} ( f(x_0) + 3f(x_1) + 3f(x_2) + f(x_3) )
\end{equation*}
dengan $h = (b-a)/3$.
Dapat juga dituliskan sebagai:
\begin{equation*}
I = \frac{b-a}{8} ( f(x_0) + 3f(x_1) + 3f(x_2) + f(x_3) )
\end{equation*}

Kesalahan pemotongan:
\begin{equation*}
E_{t} = -\frac{(b-a)^5}{6480} f^{(4)}(\xi)
\end{equation*}

\end{frame}



\begin{frame}[fragile]
\frametitle{Implementasi dalam Python}

\begin{pythoncode}
def integ_simpson38( f, a, b ):
    h = (b - a)/3
    x0 = a
    x1 = a + h
    x2 = a + 2*h
    x3 = b
    I = 3*h/8 * ( f(x0) + 3*f(x1) + 3*f(x2) + f(x3) )
    return I
\end{pythoncode}

\end{frame}


\begin{frame}
\frametitle{Simpson 1/3 vs Simpson 3/8}

\begin{itemize}
\item Aturan 3/8 Simpson sedikit lebih akurat karena penyebut pada kesalahan pemotongan
lebih besar dibandingkan aturan 1/3 Simpson.
\item Meskipun demikian, aturan 1/3 Simpson biasanya menjadi pilihan dibandingkan dengan aturan
3/8 karena orde kesalahannya sama meskipun hanya menggunakan tiga titik.
\item Aturan 3/8 Simpson dapat digunakan bersama-sama dengan aturan 1/3 Simpson apabila
jumlah segmen yang digunakan adalah ganjil (atau jumlah titik yang digunakan adalah genap).
\end{itemize}


\end{frame}






\section{Formula Newton-Cotes}

\begin{frame}
\frametitle{Formula Newton-Cotes}
\fontsize{9pt}{8.0}\selectfont
% 7.2 baselineskip

\begin{columns}

\begin{column}{0.5\textwidth}
Formula Newton-Cotes merupakan skema integrasi numerik yang paling umum.
Formula ini menggunakan strategi dengan cara mengganti fungsi atau tabel data
yang ingin dicari nilai integralnya dengan suatu fungsi pendekatan yang
mudah untuk diintegralkan:
\begin{equation*}
I = \int_{a}^{b} f(x) \, \mathrm{d}x \approx \int_{a}^{b} f_{n}(x) \, \mathrm{d}x
\end{equation*}
di mana $f_{n}(x)$ adalah sebuah polinomial:
\begin{equation*}
f_{n}(x) = a_{0} + a_{1}x + \cdots + a_{n-1} x^{n-1} + a_{n} x^{n}
\end{equation*}
di mana $n$ adalah orde dari polinomial.    
\end{column}

\begin{column}{0.5\textwidth}
{\centering
\includegraphics[width=\textwidth]{images_priv/Chapra_Fig_21_1.pdf}
\par}

(a) polinomial orde 1 (b) polinomial orde 2
\end{column}

\end{columns}



\end{frame}
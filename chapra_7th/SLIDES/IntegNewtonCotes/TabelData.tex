\section{Tabel data}


\begin{frame}
\frametitle{Tabel data}

\begin{itemize}
\item Metode-metode yang diberikan sebelumnya dapat digunakan baik untuk fungsi yang diberikan
dalam bentuk analitik, atau tabel data yang memiliki panjang segmen yang sama.
%
\item Jika panjang segmen tidak sama, maka untuk tiap segmen kita dapat menggunakan aturan
trapesium untuk setiap segmen dan menjumlahkannya:
\begin{equation*}
I = h_1 \frac{f(x_0) + f(x_1)}{2} +
    h_2 \frac{f(x_1) + f(x_2)}{2} + \cdots
    h_n \frac{f(x_{n-1}) + f(x_n)}{2}
\end{equation*}
%
\item Aturan Simpson dapat digunakan jika ada beberapa segmen berdekatan yang memiliki panjang
yang sama.
\end{itemize}

\end{frame}


\begin{frame}
\frametitle{Contoh tabel data}

Hitung integral dari fungsi yang diberikan pada data berikut.

{\centering
\includegraphics[width=0.6\textwidth]{images_priv/Chapra_Table_21_3}
\par}

\end{frame}


\begin{frame}[fragile]
\frametitle{Implementasi}

\begin{pythoncode}
def integ_trapz_table( fa, fb, a, b ):
    I = (b - a) * (fa + fb) / 2
    return I

x = [0.0, 0.12, 0.22, 0.32, 0.36, 0.40,
     0.44, 0.54, 0.64, 0.70, 0.80]

fx = [0.200000, 1.309729, 1.305241, 1.743393, 2.074903, 2.456000, 
      2.842985, 3.507297, 3.181929, 2.363000, 0.232000]

I_exact = 1.640533 # from the book

Ndata = len(x)
I = 0.0
for i in range(Ndata-1):
    I = I + integ_trapz_table( fx[i], fx[i+1], x[i], x[i+1] )

E_t = (I_exact - I)
ε_t = E_t/I_exact * 100
print("Integral result = %.6f" % I)
print("True error      = %.6f" % E_t)
print("ε_t             = %.1f%%" % ε_t)
\end{pythoncode}

\end{frame}
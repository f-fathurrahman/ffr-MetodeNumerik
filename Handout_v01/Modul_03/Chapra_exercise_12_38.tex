\begin{soal}[Chapra Latihan 12.38]
Sistem persamaan linear dapat muncul dalam solusi numerik
persamaan diferensial. Sebagai contoh, persamaan diferensial
berikut dapat diturunkan dari kesetimbangan kalor
pada suatu batang tipis yang panjang:
\begin{equation*}
\frac{\mathrm{d}^2 T}{\mathrm{d}x^2} + h'(T_{a} - T) = 0
\end{equation*}
di mana $T$ adalah temperatur (dalam $^{\circ}\mathrm{C}$),
$x$ adalah posisi sepanjang batang (dalam meter), $h'$ adalah
koefisien transfer kalor antara batang dan udara sekitar (dalam
$\mathrm{m}^{-2}$), dan $T_{a}$ adalah temperatur lingkungan
(dalam $^{\circ}\mathrm{C}$). Dengan menggunakan aproksimasi
beda hingga untuk turunan kedua:
\begin{equation}
\frac{\mathrm{d}^2 T}{\mathrm{d}x^2} = \frac{T_{i+1} - 2T_{i} + T_{i-1}}{\Delta x^2}
\label{eq:heatpde1d}
\end{equation}
di mana $T_{i}$ menyatakan temperatur pada titik atau node ke-$i$.
Dengan mensubstitusikan aproksimasi ini ke persamaan diferensial awal,
diperoleh aproksimasi sebagai berikut.
\begin{equation}
-T_{i-1} + (2 + h'\Delta x^2)T_{i} - T_{i+1} = h'\Delta x^2 T_{a}
\label{eq:heatpde1dfd}
\end{equation}
Persamaan ini dapat dituliskan utuk setiap titik interior dari batang dan menghasilkan
sistem persamaan linear dengan matriks tridiagonal.
Titik pertama dan terakhir ditentukan berdasarkan kondisi batas.
(a) Hitung solusi analitik dari persamaan diferensial \eqref{eq:heatpde1d}
untuk suatu batang dengan panjang 10 m, $T_a = 20$, $T(x=0) = 40$, 
$T(x=10) = 200$ dan $h' = 0.02$.
(b) Hitung solusi numerik untuk nilai parameter yang sama pada bagian (a) denga
menggunakan metode beda-hingga berdasarkan persamaan \eqref{eq:heatpde1dfd}
dengan 4 titik internal yang ditunjukkan pada gambar berikut.
($\Delta x = 2$ m)

{\centering
\includegraphics[scale=1.0]{images_priv/Chapra_Fig_12_38.pdf}
\par}

\end{soal}
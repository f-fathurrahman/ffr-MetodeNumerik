\documentclass[a4paper,11pt,bahasa]{extarticle} % screen setting
\usepackage[a4paper]{geometry}

%\documentclass[b5paper,11pt,bahasa]{article} % screen setting
%\usepackage[b5paper]{geometry}

%\geometry{verbose,tmargin=1.5cm,bmargin=1.5cm,lmargin=1.5cm,rmargin=1.5cm}

\geometry{verbose,tmargin=2.0cm,bmargin=2.0cm,lmargin=2.0cm,rmargin=2.0cm}

\setlength{\parskip}{\smallskipamount}
\setlength{\parindent}{0pt}

%\usepackage{cmbright}
%\renewcommand{\familydefault}{\sfdefault}

\usepackage{amsmath}
\usepackage{amssymb}

\usepackage[libertine]{newtxmath}

\usepackage[no-math]{fontspec}
\setmainfont{Linux Libertine O}

%\usepackage{fontspec}
%\usepackage{lmodern}

\setmonofont{JuliaMono-Regular}


\usepackage{hyperref}
\usepackage{url}
\usepackage{xcolor}
\usepackage{enumitem}
\usepackage{mhchem}
\usepackage{graphicx}
\usepackage{float}

\usepackage{minted}

\newminted{julia}{breaklines,fontsize=\footnotesize}
\newminted{python}{breaklines,fontsize=\footnotesize}

\newminted{bash}{breaklines,fontsize=\footnotesize}
\newminted{text}{breaklines,fontsize=\footnotesize}

\newcommand{\txtinline}[1]{\mintinline[breaklines,fontsize=\footnotesize]{text}{#1}}
\newcommand{\jlinline}[1]{\mintinline[breaklines,fontsize=\footnotesize]{julia}{#1}}
\newcommand{\pyinline}[1]{\mintinline[breaklines,fontsize=\footnotesize]{python}{#1}}

\newmintedfile[juliafile]{julia}{breaklines,fontsize=\footnotesize}
\newmintedfile[pythonfile]{python}{breaklines,fontsize=\footnotesize}
\newmintedfile[fortranfile]{fortran}{breaklines,fontsize=\footnotesize}

\usepackage{mdframed}
\usepackage{setspace}
\onehalfspacing

\usepackage{babel}
\usepackage{appendix}

\newcommand{\highlighteq}[1]{\colorbox{blue!25}{$\displaystyle#1$}}
\newcommand{\highlight}[1]{\colorbox{red!25}{#1}}

\newcounter{soal}%[section]
\newenvironment{soal}[1][]{\refstepcounter{soal}\par\medskip
   \noindent \textbf{Soal~\thesoal. #1} \sffamily}{\medskip}


\definecolor{mintedbg}{rgb}{0.95,0.95,0.95}
\BeforeBeginEnvironment{minted}{
    \begin{mdframed}[%
        topline=false,bottomline=false,%
        leftline=false,rightline=false]
}
\AfterEndEnvironment{minted}{\end{mdframed}}


\BeforeBeginEnvironment{soal}{
    \begin{mdframed}[%
        topline=true,bottomline=false,%
        leftline=true,rightline=false]
}
\AfterEndEnvironment{soal}{\end{mdframed}}


% -------------------------
\begin{document}

\title{Sistem Persamaan Linear}
\author{Fadjar Fathurrahman}
\date{}
\maketitle


\section{Eliminasi Gauss}

\begin{soal}
Implementasikan suatu fungsi untuk menghitung solusi persamaan linear:
\begin{equation*}
\mathbf{A}\mathbf{x} = \mathbf{b}
\end{equation*}
di mana $\mathbf{A}$ adalah suatu matriks berukuran $n \times n$, 
$\mathbf{x}$ dan $\mathbf{b}$ adalah vektor kolom.
Anda dapat menggunakan pseudocode yang dijelaskan pada Gambar 9.6
pada buku Chapra dengan mengingat bahwa untuk Python indeks dimulai dari 0.
Lakukan modifikasi jika Anda anggap perlu.
Fungsi Anda harus memiliki opsi atau argumen opsional \pyinline{verbose}.
Jika \pyinline{verbose=True}, maka ketika fungsi yang Anda panggil juga
memberikan informasi mengenai langkah-langkah yang dilakukan, misalnya
menunjukkan matriks dalam proses eliminasi, apakah perlu tukar baris atau tidak,
dan substitusi balik. Uji hasil program Anda dengan menggunakan salah satu contoh
yang ada pada slide.
\end{soal}


\section{Dekomposisi LU}

\begin{soal}
Lakukan hal yang sama dengan soal sebelumnya, namun dengan menggunakan
dekomposisi LU.
Anda dapat menggunakan pseudocode yang ada pada Gambar 10.2 pada Chapra.
Uji fungsi yang Anda implementasikan dan gunakan untuk menyelesaikan
kasus atau contoh sama yang Anda gunakan pada soal pertama.
\end{soal}

\begin{soal}
Jelaskan bagaiman cara mendapatkan invers matriks dengan menggunakan
dekomposisi LU. Gunakan fungsi dekomposisi LU yang Anda implementasikan
untuk mencari invers matriks yang Anda gunakan pada soal sebelumnya.
\end{soal}


\section{Soal tambahan}

Gunakan salah satu dari fungsi atau subrutin Python yang sudah Anda tulis
pada soal-soal sebelumnya untuk menyelesaikan soal-soal berikut.

\begin{soal}
Hitung invers dari matrix:
\begin{equation*}
\mathbf{A} = \begin{bmatrix}
1 & 3 & -9 & 6 & 4 \\
2 & -1 & 6 & 7 & 1 \\
3 & 2 & -3 & 15 & 5 \\
8 & -1 & 1 & 4 & 2 \\
11 & 1 & -2 & 18 & 7
\end{bmatrix}
\end{equation*}
\end{soal}

%\section{Menggunakan Pustaka Python}

Pada bagian ini, kita akan menggunakan beberapa pustaka SciPy yang dapat digunakan
untuk mencari akar persamaan nonlinear. Beberapa fungsi tersebut dapat ditemukan
pada modul \txtinline{scipy.optimize}.
\begin{itemize}
\item \txtinline{scipy.optimize.root_scalar}: untuk mencari akar persamaan nonlinear
yang terdiri dari satu variabel.
\item \txtinline{scipy.optimize.fsolve}: untuk mencari akar dari sistem persamaan
nonlinear (lebih dari satu variabel)
\end{itemize}

Pada contoh berikut, kita akan mencari akar dari $f(x) = x - \cos(x)$.
Untuk metode terbuka, kita menggunakan tebakan akar awal $x_0 = 0$
sedangkan untuk 
\begin{pythoncode}
from scipy import optimize
import math
    
def f(x):
    return x - math.cos(x)
    
def fprime(x):
    return 1 + math.sin(x)
    
print("\nUsing Newton-Raphson method")
sol = optimize.root_scalar(f, x0=0.0, fprime=fprime, method='newton')
print(sol)
    
print("\nUsing secant method")
sol = optimize.root_scalar(f, x0=0.0, x1=1.0, method='secant')
print(sol)
    
# Bracketing methods
for method in ["brentq", "brenth", "ridder", "bisect"]:
    print("\nUsing %s method" % (method))
    sol = optimize.root_scalar(f, bracket=[0.0, 1.0], method=method)
    print(sol)

print("\nUsing bisect directly")
xroot, sol = optimize.bisect(f, a=0.0, b=1.0, full_output=True)
print(sol)    
\end{pythoncode}
Modul \txtinline{scipy.optimize} juga menyediakan beberapa fungsi lain seperti
\txtinline{optimize.bisect} yang dapat digunakan untuk mencari akar persamaan
nonlinear, yang berbeda hanyalah \txtinline{interface} atau \txtinline{function signature}-nya.

Pada contoh berikut ini, kita akan mencari akar dari sistem persamaan nonlinear
pada Chapra Contoh 6.12 dengan menggunakan \txtinline{fsolve}:
\begin{pythoncode}
from scipy.optimize import fsolve

def f(x_):
    # x = x_[0] and y = x_[1] 
    x = x_[0]
    y = x_[1]
    u = x**2 + x*y - 10
    v = y + 3*x*y**2 - 57
    return [u, v]
    
root = fsolve(f, [1.0, 3.5]) # using initial guess as in the book
print(root)
\end{pythoncode}

Numpy menyediakan modul khusus untuk merepresentasikan polinomial, yaitu
\txtinline{numpy.polynomial}. Modul ini menyediakan banyak fungsi untuk
melakukan berbagai operasi terkait polinomial.
Pada contoh berikut ini, kita akan mencari akar-akar (real dan kompleks)
dari polinomial:
\begin{equation*}
f(x) = x^5 - 3.5x^4 + 2.75x^3 + 2.125x^2 - 3.875x + 1.25
\end{equation*}

Kode Python:
\begin{pythoncode}
from numpy.polynomial import Polynomial

p = Polynomial([1.25, -3.875, 2.125, 2.75, -3.5, 1.0])
print(p.roots())
\end{pythoncode}

Silakan membaca dokumentasi berikut untuk informasi lebih lanjut.
\begin{itemize}
\item {\scriptsize\url{https://docs.scipy.org/doc/scipy/reference/optimize.html\#root-finding}}
\item {\scriptsize\url{https://docs.scipy.org/doc/scipy/reference/generated/scipy.optimize.fsolve.html}}
\item {\scriptsize\url{https://numpy.org/doc/stable/reference/routines.polynomials.package.html}}
\end{itemize}

\begin{soal}
Gunakan Numpy untuk menentukan akar-akar dari polinomial berikut:
\begin{itemize}
\item $f(x) = x^3 - x^2 + 2x - 2$
\item $f(x) = 2x^4 + 6x^2 + 8$
\item $f(x) = x^4 - 2x^3 + 6x^2 - 2x + 5$
\item $f(x) = -2 + 6.2x - 4x^2 + 0.7x^3$
\item $f(x) = 9.34 - 21.97x + 16.3x^2 - 3.704x^3$
\item $f(x) = x^4 - 2x^3 + 6x^2 - 2x + 5$
\end{itemize}
\end{soal}


\begin{soal}[Chapra Latihan 12.29]

Perhatikan rangkaian di bawah ini.

{\centering
\includegraphics[scale=1.0]{images_priv/Chapra_Fig_12_29.pdf}
\par}

Turunkan dan cari solusi sistem persamaan linear untuk menghitung arus-arus
pada rangkaian.

\end{soal}

\begin{soal}[Chapra Latihan 12.35]
Tiga massa terhubung oleh tali (anggap bahwa massa tali dapat diabaikan)
dan berada pada suatu bidang miring seperti pada Gambar.

{\centering
\includegraphics[scale=1.0]{images_priv/Chapra_Fig_12_35.pdf}
\par}

Hitung percepatan masing-masing massa dan tegangan masing-masing tali.
\end{soal}

\begin{soal}[Chapra Latihan 12.37]

Tinjau sistem yang terdiri dari 3 massa dan 4 pegas yang terhubung
seperti Gambar berikut.

{\centering
\includegraphics[scale=1.0]{images_priv/Chapra_Fig_12_37}
\par}

Dengan menggunakan Hukum Newton, dapat diturunkan persamaan diferensial
berikut.
\begin{align*}
\ddot{x_{1}} + \frac{k_1 + k_2}{m_1} x_{1} - \frac{k_2}{m_1} = 0 \\
\ddot{x_{2}} - \frac{k_2}{m_2}x_1 + \frac{k_2 + k_3}{m_2}x_2 - \frac{k_3}{m_2}x_3 = 0 \\
\ddot{x_{3}} - \frac{k_3}{m_3}x_2 + \frac{k_3 + k_4}{m_3}x_3 = 0\\
\end{align*}
di mana $k_1 = k_4 = 10\,\mathrm{N/m}$, $k_2 = k_3 = 30\,\mathrm{N/m}$,
dan $m_1 = m_2 = m_3 = 2\,\mathrm{kg}$. Tulis tiga persamaan dalam bentuk matriks:
\begin{equation*}
0 = \text{[vektor akselarasi]} + \text{[matriks }k/m\text{]}\text{vektor perpindahan}
\end{equation*}
pada perpindahan (dalam meter) $x_{1} = 0.05, x_{2} = 0.04, x_{3} = 0.03$.
Cari akselarasi untuk setiap massa.
\end{soal}

\begin{soal}[Chapra Latihan 12.38]
Sistem persamaan linear dapat muncul dalam solusi numerik
persamaan diferensial. Sebagai contoh, persamaan diferensial
berikut dapat diturunkan dari kesetimbangan kalor
pada suatu batang tipis yang panjang:
\begin{equation*}
\frac{\mathrm{d}^2 T}{\mathrm{d}x^2} + h'(T_{a} - T) = 0
\end{equation*}
di mana $T$ adalah temperatur (dalam $^{\circ}\mathrm{C}$),
$x$ adalah posisi sepanjang batang (dalam meter), $h'$ adalah
koefisien transfer kalor antara batang dan udara sekitar (dalam
$\mathrm{m}^{-2}$), dan $T_{a}$ adalah temperatur lingkungan
(dalam $^{\circ}\mathrm{C}$). Dengan menggunakan aproksimasi
beda hingga untuk turunan kedua:
\begin{equation}
\frac{\mathrm{d}^2 T}{\mathrm{d}x^2} = \frac{T_{i+1} - 2T_{i} + T_{i-1}}{\Delta x^2}
\label{eq:heatpde1d}
\end{equation}
di mana $T_{i}$ menyatakan temperatur pada titik atau node ke-$i$.
Dengan mensubstitusikan aproksimasi ini ke persamaan diferensial awal,
diperoleh aproksimasi sebagai berikut.
\begin{equation}
-T_{i-1} + (2 + h'\Delta x^2)T_{i} - T_{i+1} = h'\Delta x^2 T_{a}
\label{eq:heatpde1dfd}
\end{equation}
Persamaan ini dapat dituliskan utuk setiap titik interior dari batang dan menghasilkan
sistem persamaan linear dengan matriks tridiagonal.
Titik pertama dan terakhir ditentukan berdasarkan kondisi batas.
(a) Hitung solusi analitik dari persamaan diferensial \eqref{eq:heatpde1d}
untuk suatu batang dengan panjang 10 m, $T_a = 20$, $T(x=0) = 40$, 
$T(x=10) = 200$ dan $h' = 0.02$.
(b) Hitung solusi numerik untuk nilai parameter yang sama pada bagian (a) denga
menggunakan metode beda-hingga berdasarkan persamaan \eqref{eq:heatpde1dfd}
dengan 4 titik internal yang ditunjukkan pada gambar berikut.
($\Delta x = 2$ m)

{\centering
\includegraphics[scale=1.0]{images_priv/Chapra_Fig_12_38.pdf}
\par}

\end{soal}

\begin{soal}[Chapra Latihan 12.39]
Ditribusi temperatur pada keadaan tunak dari suatu plat yang dipanaskan
dapat dimodelkan dengan menggunakan persamaan Laplace:
\begin{equation*}
\frac{\partial^2 T}{\partial x^2} + \frac{\partial^2 T}{\partial y^2} = 0
\end{equation*}
Jika plat ini direpresentasikan menjadi kumpulan titik-titik yang berjarak
seragam, metode beda hingga dapat diaplikasikan untuk mencari solusi numerik
dari persamaan Laplace. Gunakan aproksimasi beda hingga (seperti pada soal
sebelumnya) untuk menurunkan sistem persamaan linear untuk 4 titik interior
seperti pada gambar berikut.

{\centering
\includegraphics[scale=1.0]{images_priv/Chapra_Fig_12_39.pdf}
\par}

Asumsikan $\Delta x = \Delta y$. Tulis juga persamaan linear yang diperoleh
jika ada 9 dan 16 titik interior. 
Implementasikan program untuk menyelesaikan sistem persamaan linear
ini dengan menggunakan metode Gauss-Seidel. Program Anda harus dapat menerima input
$N$ di mana $N = N$ adalah jumlah titik interior yang digunakan. Asumsikan juga bahwa $N$
merupakan bilangan kuadrat, artinya jumlah titik diskritisasi pada arah $x$ dan $y$
bernilai sama.
(Lihat Chapra Contoh 29.1 untuk informasi lebih lanjut, jika diperlukan)
\end{soal}



\begin{soal}
Apa keuntungan dan kekurangan dari metode Gauss-Seidel (dan metode iteratif lainnya)
dibandingkan dengan metode eliminasi Gaussian?
\end{soal}

\end{document}

\subsection{Metode Euler}

Pada metode Euler, turunan pertama sebagai estimasi kemiringan pada
$x_i$:
\begin{equation*}
\phi = f(x_i, y_i)
\end{equation*}
sehingga diperoleh skema berikut.
\begin{equation}
y_{i+1} = y_{i} + f(x_i, y_i) h
\label{eq:euler_1step}
\end{equation}

Pada contoh berikut ini kita akan mendemonstrasikan penggunaan metode Euler.

\begin{soal}[Chapra Contoh 25.1]
Gunakan metode Euler untuk menyelesaikan persamaan diferensial berikut:
\begin{equation*}
\frac{\mathrm{d}y}{\mathrm{d}x} = -2x^3 + 12x^2 - 20x + 8.5
\end{equation*}
dari $x=0$ sampai $x=4$ dengan ukuran langkah 0.5.
Syarat awal adalah $y(x=0) = 1$.
Bandingkan dengan solusi eksak:
\begin{equation*}
y(x) = -0.5x^4 + 4x^3 - 10x^2 + 8.5x + 1
\end{equation*}
\end{soal}

Pada kode berikut ini kita akan melakukan dua langkah dari metode Euler.
\begin{pythoncode}
# initial cond
x0 = 0.0
y0 = 1.0

# Using step of 0.5, starting from x0 and y0
x = x0
y = y0
h = 0.5
xp1 = x + h # we are searching for solution at x = 0.5
ϕ = -2*x**3 + 12*x**2 - 20*x + 8.5
yp1 = y + ϕ*h
y_true = -0.5*xp1**4 + 4*xp1**3 - 10*xp1**2 + 8.5*xp1 + 1
ε_t = (y_true - yp1)/y_true * 100
print("First step : x = %f y_true = %.5f y = %.5f ε_t = %.1f %%" % (xp1, y_true, yp1, ε_t))

# Second step
x = xp1 # x from the previous step
y = yp1 # y from the previous step
h = 0.5
xp1 = x + h # we are searching for solution at x = 1.0
ϕ = -2*x**3 + 12*x**2 - 20*x + 8.5
yp1 = y + ϕ*h
y_true = -0.5*xp1**4 + 4*xp1**3 - 10*xp1**2 + 8.5*xp1 + 1
ε_t = (y_true - yp1)/y_true * 100
print("Second step: x = %f y_true = %.5f y = %.5f ε_t = %.1f %%" % (xp1, y_true, yp1, ε_t))
\end{pythoncode}

Contoh keluaran:
\begin{textcode}
First step : x = 0.500000 y_true = 3.21875 y = 5.25000 ε_t = -63.1 %
Second step: x = 1.000000 y_true = 3.00000 y = 5.87500 ε_t = -95.8 %
\end{textcode}

Anda dapat melanjutkan kode di atas sampai $x=4$.


Pada kode yang baru saja kita gunakan, perhitungan $\phi$ dilakukan secara
manual dengan memberikan ekspresinya secara eksplisit. Hal yang sama juga
kita lakukan untuk perhitungan solusi eksak \pyinline{y_true}.
Jika kita memiliki
persamaan diferensial yang lain kita perlu mengubah bagian ini. Jika kita melakukan
banyak perhitungan dengan menggunakan ekspresi yang sama, maka kita juga harus melakukan
perubahannya pada seluruh program kita.
Hal ini tentu saja akan menyulitkan kita nanti ketika kita ingin menyelesaikan
persamaan diferensial yang berbeda.
Oleh karena itu, kita akan mengimplementasikan skema metode Euler yang diberikan pada
Persamaan \eqref{eq:euler_1step} pada suatu fungsi.
\begin{pythoncode}
# One-step aplication of Euler's method for ODE
def ode_euler_1step(dfunc, xi, yi, h):
    return yi + dfunc(xi,yi)*h
\end{pythoncode}
Fungsi ini memiliki argumen sebagai berikut.
\begin{itemize}
\item \pyinline{dfunc}: expresi dari $\mathrm{d}y/\mathrm{d}x$, yaitu ruas kanan
dari Persamaan \eqref{eq:dydx_umum}. Tipe dari \pyinline{dfunc} adalah fungsi.
\pyinline{dfunc} harus didefinisikan sebagai fungsi yang menerima dua argumen.
\item \pyinline{xi} dan \pyinline{yi}: titik $x_{i}$ dan $y_{i}$ pada
Persamaan \eqref{eq:euler_1step}
\item \pyinline{h}: ukuran langkah.
\end{itemize}
Fungsi ini mengembalikan nilai $y_{i+1}$.

Sebagai contoh, kita akan memodifikasi program sebelumnya dengan menggunakan
\pyinline{ode_euler_1step}.
Untuk ini kita akan mendefinisikan dua fungsi: untuk perhitungan
$\mathrm{d}y/\mathrm{d}x$ dan solusi eksak.
\begin{pythoncode}
# the left hand side of dy/dx=... (in general depends on x and y)
# In the present case it only depends on x
def deriv(x, y):
    return -2*x**3 + 12*x**2 - 20*x + 8.5

def exact_sol(x):
    return -0.5*x**4 + 4*x**3 - 10*x**2 + 8.5*x + 1
\end{pythoncode}

Pada program sebelumnya, kita mengganti bagian berikut:
\begin{pythoncode}
ϕ = -2*x**3 + 12*x**2 - 20*x + 8.5
yp1 = y + ϕ*h
y_true = -0.5*xp1**4 + 4*xp1**3 - 10*xp1**2 + 8.5*xp1 + 1
\end{pythoncode}
menjadi:
\begin{pythoncode}
yp1 = ode_euler_1step(deriv, x, y, h)
y_true = exact_sol(xp1)
\end{pythoncode}
Keluaran yang diperoleh seharusnya sama dengan cara manual (tanpa fungsi).


Akan lebih baik lagi jika kita menggunakan loop, seperti pada potongan
program berikut.
\begin{pythoncode}
print("   x     y_true   y_Euler       ε_t")
print("-----------------------------------")

print("%5.1f  %8.5f  %8.5f" % (x0, y0, y0)) # Initial cond

# Start from initial condition
x = x0; y = y0
h = 0.5 # step size
for i in range(0,8):
    xp1 = x + h
    yp1 = ode_euler_1step(deriv, x, y, h)
    y_true = exact_sol(xp1)
    ε_t = (y_true - yp1)/y_true * 100 # in percent
    print("%5.1f  %8.5f  %8.5f  %8.1f%%" % (xp1, y_true, yp1, ε_t))
    # Update x and y for the next step
    x = xp1
    y = yp1
\end{pythoncode}

Contoh keluaran:
\begin{textcode}
   x     y_true   y_Euler       ε_t 
------------------------------------
  0.0   1.00000   1.00000
  0.5   3.21875   5.25000     -63.1%
  1.0   3.00000   5.87500     -95.8%
  1.5   2.21875   5.12500    -131.0%
  2.0   2.00000   4.50000    -125.0%
  2.5   2.71875   4.75000     -74.7%
  3.0   4.00000   5.87500     -46.9%
  3.5   4.71875   7.12500     -51.0%
  4.0   3.00000   7.00000    -133.3%
\end{textcode}


%\subsection*{Analisis kesalahan metode Euler}

Kesalahan pada metode Euler terdiri dari dua kontribusi:
\begin{itemize}
\item Kesalahan pemotongan (truncation)
\item Kesalahan pembulatan (round-off)
\end{itemize}

Kesalahan pemotongan terdiri dari dua bagian:
\begin{itemize}
\item kesalahan pemotongan lokal yang berasal dari aplikasi metode pada satu
ukuran langkah.
\item kesalahan pemotongan perambatan (propagated truncation error) yang diakibatkan
aproksimasi pada langkah-langkah sebelumnya.
\end{itemize}
Penjumlahan dari kedua jenis kesalahan tersebut adalah kesalahan pemotongan global.

Misalkan, persamaan diferensial yang akan diintegralkan adalah:
\begin{equation*}
y' = \frac{\mathrm{d}y}{\mathrm{d}x} = f(x,y)
\end{equation*}
Jika solusi, yaitu fungsi yang mendeskripsikan $y$, memiliki turunan yang kontinu, maka
deret Taylor dapat digunakan untuk merepresentasikan fungsi di sekitar nilai awal
$(x_i, y_i)$:
\begin{equation*}
y_{i+1} = y_{i} + y'_{i} h + \frac{y''_{i}}{2!}h^2 + \cdots + \frac{y^{(n)}_{i}}{n!} h^n + R_n
\end{equation*}
dengan $h = x_{i+1} - x_{i}$ dan $R_n$ adalah suku sisa yang didefinisikan sebagai:
\begin{equation*}
R_n = \frac{y^{n+1}(\xi)}{(n+1)!} h^{n+1}
\end{equation*}
di mana $\xi$ berada di dalam interval $[x_i, x_{i+1}]$.
Alternatif penulisan:
\begin{equation*}
y_{i+1} = y_{i} + f(x_i, y_i) h + \frac{f'(x_i, y_i)}{2!}h^2 + \cdots +
\frac{f^{(n-1)}(x_i, y_i)}{n!} h^n + \mathcal{O}(h^{n+1})
\end{equation*}
di mana $\mathcal{O}(h^{n+1})$ menyatakan kontribusi kesalahan pemotongan lokal
yang sebanding dengan ukuran langkah pangkat $(n+1)$.

Dengan membandingkan persamaan ini dengan skema metode Euler, diperoleh
kesalahan pemotongan pada metode Euler sebagai berikut:
\begin{equation*}
E_{t} = \frac{f'(x_i, y_i)}{2!} h^2 + \cdots + \mathcal{O}(h^{n+1})
\end{equation*}
untuk nilai $h$ yang cukup kecil, dapat digunakan hanya satu suku saja:
\begin{equation*}
E_{a} = \frac{f'(x_i, y_i)}{2!} h^2
\end{equation*}
atau
\begin{equation*}
E_a = \mathcal{O}(h^2)
\end{equation*}
di mana $E_a$ adalah aproksimasi kesalahan pemotongan lokal.

Dengan loop, perhitungan error pemotongan
\begin{pythoncode}
# .... definisi deriv
# .... definisi ode_euler_1step
# .... definisi exact_sol

# For local truncation errors
from math import factorial
def trunc_err_second(x,y,h):
    return (-6*x**2 + 24*x - 20)*h**2/factorial(2)

def trunc_err_third(x,y,h):
    return (-12*x + 24)*h**3/factorial(3)

def trunc_err_fourth(x,y,h):
    return -12*h**4/factorial(4)

# initial cond
x0 = 0.0
y0 = 1.0

print("   x     y_true   y_Euler       ε_t   ε_t local")
print("-----------------------------------------------")

print("%5.1f  %8.5f  %8.5f" % (x0, y0, y0)) # Initial cond

x = x0
y = y0
h = 0.5
for i in range(0,8):
    xp1 = x + h
    yp1 = ode_euler_1step(deriv, x, y, h)
    y_true = exact_sol(xp1)
    ε_t = (y_true - yp1)/y_true * 100
    ε_t_local = (trunc_err_second(x,y,h) + trunc_err_third(x,y,h) + \
        trunc_err_fourth(x,y,h))/y_true * 100
    print("%5.1f  %8.5f  %8.5f  %8.1f%%  %8.1f%%" % (xp1, y_true, yp1, ε_t, ε_t_local))
    # Update x and y for the next step
    x = xp1
    y = yp1
\end{pythoncode}

Contoh keluaran:
\begin{textcode}
   x     y_true   y_Euler       ε_t   ε_t local
-----------------------------------------------
  0.0   1.00000   1.00000
  0.5   3.21875   5.25000     -63.1%     -63.1%
  1.0   3.00000   5.87500     -95.8%     -28.1%
  1.5   2.21875   5.12500    -131.0%      -1.4%
  2.0   2.00000   4.50000    -125.0%      20.3%
  2.5   2.71875   4.75000     -74.7%      17.2%
  3.0   4.00000   5.87500     -46.9%       3.9%
  3.5   4.71875   7.12500     -51.0%     -11.3%
  4.0   3.00000   7.00000    -133.3%     -53.1%
\end{textcode}





\subsection*{Subrutin untuk metode satu langkah}
Kita juga dapat menggunakan fungsi atau subrutin berikut sebagai implementasi
metode Euler.
\begin{pythoncode}
def ode_euler(dfunc, x0, y0, h, Nstep):
    x = np.zeros(Nstep+1)
    y = np.zeros(Nstep+1)
    # Start with initial cond
    x[0] = x0
    y[0] = y0
    for i in range(0,Nstep):
        x[i+1] = x[i] + h
        y[i+1] = ode_euler_1step(dfunc, x[i], y[i], h)
    return x, y
\end{pythoncode}
Fungsi ini memiliki argumen sebagai berikut yang sama dengan \pyinline{ode_euler_1step}
dengan satu tambahan argumen \pyinline{Nstep}, yaitu jumlah langkah yang dilakukan.
Perbedaan lain adalah fungsi ini mengembalikan array solusi \pyinline{x} dan \pyinline{y}
yang dapat diproses lebih lanjut, misalnya divisualisasikan dalam bentuk plot.

Perhatikan bahwa kode ini dapat digeneralisasi untuk metode satu langkah lain yang akan
diberikan dalam modul ini.

Sebagai contoh penggunakan \pyinline{ode_euler}, kita akan mengerjakan Chapra Contoh 25.3,
di mana dilakukan perbandingan hasil metode Euler untuk ukuran langkah yang berbeda, yaitu
$h=0.5$ dan $h=0.25$.
\begin{pythoncode}
# .... Definisi dan/atau import fungsi-fungsi yang diperlukan

x0 = 0.0; y0 = 1.0 # initial cond
xf = 4.0 # last or final x

# Using h=0.5
h = 0.5
Nstep = int(xf/h)
xs1, ys1 = ode_euler(deriv, x0, y0, h, Nstep)

# Using h=0.25
h = 0.25
Nstep = int(xf/h)
xs2, ys2 = ode_euler(deriv, x0, y0, h, Nstep)

# True solution
x_true = np.linspace(0.0, 4.0, 200)
y_true = exact_sol(x_true)

# Plot
plt.clf()
plt.plot(xs1, ys1, label="h=0.5", marker="o")
plt.plot(xs2, ys2, label="h=0.5", marker="o")
plt.plot(x_true, y_true, label="true") # Do not show the markers
plt.legend()
plt.tight_layout()
\end{pythoncode}
Hasil dari program ini dapat dilihat pada Gambar \ref{fig:example_25_3}.


\begin{figure}[h]
{\centering
\includegraphics[scale=0.7]{../../chapra_7th/ch25/IMG_chapra_example_25_3.pdf}
\par}
\caption{Perbandingan hasil metode Euler dengan ukuran langkah yang berbeda}
\label{fig:example_25_3}
\end{figure}


\begin{soal}[Chapra Contoh 25.4]
Bandingkan solusi yang diperoleh dari model linear dengan persamaan diferensial:
\begin{equation*}
\frac{\mathrm{d}v}{\mathrm{d}t} = g - \frac{c}{m}v
\end{equation*}
dan model nonlinear yang memperhitungkan gaya gesek akibat angin:
\begin{equation*}
\frac{\mathrm{d}v}{\mathrm{d}t} = g - \frac{c}{m}\left[
v + a\left(\frac{v}{v_{\mathrm{max}}}\right)^b
\right]
\end{equation*}
dengan parameter $g = 9.81$, $c=12.5$, $m=68.1$, and kondisi awal $v=0$ pada $t=0$
(satuan dalam SI).
Parameter tambahan untuk model nonlinear adalah sebagai berikut: $a=8.3$, $b=2.2$,
$v_{\mathrm{max}}=46$.
Gunakan metode Euler dengan ukuran langkah 0.1 untuk menyelesaikan persamaan
diferensial secara numerik.
\end{soal}


\begin{pythoncode}
# .... import library yang diperlukan
# .... definisi fungsi-fungsi yang diperlukan

def model_linear(t, v):
    g = 9.81; c = 12.5; m = 68.1
    return g - c*v/m

def model_nonlinear(t, v):
    g = 9.81; c = 12.5; m = 68.1
    a = 8.3; b = 2.2; vmax = 46
    return g - c/m*( v + a*(v/vmax)**b )
    
t0 = 0.0; v0 = 0.0 # initial cond
tf = 15.0
h = 0.1
Nstep = int(tf/h)

t_linear, v_linear = ode_euler(....) # Lengkapi
t_nonlinear, v_nonlinear = ode_euler(....) # Lengkapi
    
# Plot
plt.clf()
plt.plot(t_linear, v_linear, label="model linear")
plt.plot(t_nonlinear, v_nonlinear, label="model nonlinear")
plt.xlabel("t (s)")
plt.ylabel("y (m)")
plt.legend()
plt.grid(True)
plt.tight_layout()
\end{pythoncode}

Hasil dapat dilihat pada Gambar \ref{fig:example_25_4}.

\begin{figure}[h]
{\centering
\includegraphics[scale=0.7]{../../chapra_7th/ch25/IMG_chapra_example_25_4.pdf}
\par}
\caption{Perbandingan hasil metode Euler untuk model linear dan nonlinear
dengan ukuran langkah $h=0.1$}
\label{fig:example_25_4}
\end{figure}

\section{Persoalan Nilai Batas}

Sebagai contoh persoalan nilai batas kita akan meninjau masalah distribusi
panas pada suatu batang panjang. Kedua ujung dari batang terinsulasi dan
dijaga pada satu suhu tertentu. Bagian lain dari batang tidak terinsulasi
sehingga memungkikan terjadinya pertukaran energi.
Distribusi temperatur pada batang dapat dijelaskan dengan persamaan diferensial
orde 2 berikut.
\begin{equation}
\frac{\mathrm{d}^2 T}{\mathrm{d}x^2} + h'(T_a - T) = 0
\label{eq:chapra_eq_27_1}
\end{equation}
dengan syarat batas: $T(x=0) = T_1$ dan $T(x=L) = T_2$.


\subsection{Metode \textit{shooting}}

Pada metode \textit{shooting}, permasalahan nilai batas diubah menjadi permasalahan
nilai awal. Solusi kemudian dicari dengan menggunakan metode \textit{trial-and-error}
pada.

\begin{soal}[Chapra Contoh 27.1]
Untuk $L=10$, $T_a = 20$, $T_1 = 40$, $T_2 = 200$, dan $h' = 0.01$, solusi
dari Persamaan \eqref{eq:chapra_eq_27_1} adalah.
\begin{equation*}
T(x) = 73.4523 e^{0.1x} - 53.4523 e^{-0.1x} + 20
\end{equation*}
Gunakan metode \textit{shooting} untuk menyelesaikan persamaan ini secara numerik
\end{soal}


Kita mengubah persamaan diferensial orde-2 menjadi sistem persamaan diferensial
orde-1 sebagai berikut:
\begin{align*}
\frac{\mathrm{d}T}{\mathrm{d}x} & = z \\
\frac{\mathrm{d}z}{\mathrm{d}x} & = h'(T - T_a)
\end{align*}
Berikut ini adalah implementasi dari persamaan di atas:
\begin{pythoncode}
# T == y[0]
# dT/dx == y[1]
# d2T/dx2 == dydx[1]
def deriv(x, y):
    #
    Nvec = len(y)
    assert Nvec == 2
    dydx = np.zeros(Nvec)
    # Parameters
    h = 0.01
    T_a = 20.0
    #
    dydx[0] = y[1]
    dydx[1] = h*(y[0] - T_a)
    return dydx
\end{pythoncode}

Kita juga mendefinisikan beberapa variabel yang akan digunakan nanti.
\begin{pythoncode}
x0 = 0.0 # Initial cond
y0 = np.zeros(2) # y0[0] should be equal to T(x=) and y0[1] will be set later
xf = 10.0  # end interval
Tf = 200.0 # Boundary condition, T(10) = 200
\end{pythoncode}


Nilai $T(x=0)$ sudah diberikan. Akan tetapi untuk menyelesaikan persamaan nilai
awal, kita juga memerlukan nilai $\frac{\mathrm{d}T}{\mathrm{d}x} = z$ pada $x=0$.
Misalkan kita menebaknya dengan nilai $z(0) = 10$.
Dengan menggunakan metode Runge-Kutta orde-4 dengan ukuran langkah 2 kita dapat memperoleh
nilai $T(x=L)$. Potongan kode berikut ini dapat digunakan:
\begin{pythoncode}
y0[0] = 40.0 # from the boundary condition, T(0) = 40
z0_1 = 10.0 # Guess for z = dT/dx == y[1], save to it variable for later use
y0[1] = z0_1

h = 2.0 # Step size
Nstep = int( (xf-x0)/h )
x, y = ode_solve(deriv, ode_rk4_1step, x0, y0, h, Nstep)
# At the end of the interval
Tf_1 = y[-1,0]
print("First guess: T(10) = y[-1,0] = ", Tf_1)
\end{pythoncode}
Keluaran:
\begin{textcode}
First guess: T(10) = y[-1,0] =  168.37965867134406
\end{textcode}
Diperleh $T(x=L)$ sekitar 168.3797. Ini masih berbeda dengan nilai yang kita perlukan
yaitu 200.

Berikut ini adalah plot dari solusi yang diperoleh (belum memenuhi syarat batas).

{\centering
\includegraphics[scale=0.7]{../../chapra_7th/ch27/IMG_example_27_1_v2_1st.pdf}
\par}

Karena masih berbeda, kita akan menggunakan tebakan lain, misalnya $z=10$:
\begin{pythoncode}
# Integrate again, now with new guess for z(0) = y0[1]
z0_2 = 20.0
y0[1] = z0_2 # Guess for z = dT/dx == y[1]
x, y = ode_solve(deriv, ode_rk4_1step, x0, y0, h, Nstep)
# At the end of the interval
Tf_2 = y[-1,0]
print("Second guess: T(10) = y[-1,0] = ", Tf_2)
\end{pythoncode}
Diperoleh:
\begin{textcode}
Second guess: T(10) = y[-1,0] =  285.89795359537567
\end{textcode}
Hasil yang diperoleh masih belum sama dengan yang diperlukan, yaitu 200.

Berikut ini adalah plot dari solusi yang diperoleh (belum memenuhi syarat batas).

{\centering
\includegraphics[scale=0.7]{../../chapra_7th/ch27/IMG_example_27_1_v2_2nd.pdf}
\par}

Karena persamaan diferensial \eqref{eq:chapra_eq_27_1} adalah linear, maka
dua solusi tersebut saling terkait.
Kita dapat menggunakan interpolasi linear untuk mendapatkan nilai dari $z(0)$.
Berikut adalah data yang sudah diperoleh.

{\centering
\begin{tabular}{|c|c|}
\hline
$z(0) = 10$ & $T(10) = 168.3797$ \\
$z(0) = 20$ & $T(10) = 285.8980$ \\
$z(0) = ??$ & $T(10) = 200$ \\
\hline
\end{tabular}
\par}

Kode Python:
\begin{pythoncode}
# Using linear interp to guess what value of z(0) which gives T(10) = 200
z0_new = z0_1 + (z0_2 - z0_1)/(Tf_2 - Tf_1) * (Tf - Tf_1)
print("z0_new = ", z0_new)
\end{pythoncode}
Diperoleh:
\begin{textcode}
z0_new =  12.690673937117328
\end{textcode}

Nilai ini dapat digunakan sebagai syarat awal.
\begin{pythoncode}
# Now solve the IVP using z0_new
y0[1] = z0_new # Guess for z = dT/dx == y[1]
x, y = ode_solve(deriv, ode_rk4_1step, x0, y0, h, Nstep)
# At the end of the interval
Tf_3 = y[-1,0]
print("Third guess: T(10) = y[-1,0] = ", Tf_3)
\end{pythoncode}
Akhirnya kita mendpatkan nilai yang diinginkan:
\begin{textcode}
Third guess: T(10) = y[-1,0] =  200.0
\end{textcode}

Solusi yang diperoleh dapat diplot pada gambar berikut (sudah memenuhi syarat batas).

{\centering
\includegraphics[scale=0.7]{../../chapra_7th/ch27/IMG_example_27_1_v2_3rd.pdf}
\par}


Berikut ini perbandingannya dengan solusi eksak:
\begin{pythoncode}
def exact_sol(x):
    #return 73.4532*np.exp(0.1*x) - 53.4523*np.exp(-0.1*x) + 20.0 # from book
    # Using SymPy
    return 20*((1 - np.exp(2))*np.exp(x/10) + (1 - 9*np.e)*np.exp(x/5) + \
            np.e*(9 - np.e))*np.exp(-x/10)/(1 - np.exp(2))

T_exact = exact_sol(x)
T_num = y[:,0]
error = np.abs(T_exact - T_num)
for i in range(len(x)):
    print("%18.10f %18.10f %18.10f %15.10e" % (x[i], T_num[i], T_exact[i], error[i]))
\end{pythoncode}

Hasil keluaran ($x$, numerik, eksak, error):
\begin{textcode}
    0.0000000000      40.0000000000      40.0000000000 0.0000000000e+00
    2.0000000000      65.9518901934      65.9517913981 9.8795269608e-05
    4.0000000000      93.7479650466      93.7477895327 1.7551381092e-04
    6.0000000000     124.5037505132     124.5035454074 2.0510586421e-04
    8.0000000000     159.4535539523     159.4533954960 1.5845629972e-04
   10.0000000000     200.0000000000     200.0000000000 0.0000000000e+00
\end{textcode}


Untuk persamaan diferensial nonlinear, interpolasi linear biasanya
tidak cukup untuk mendapatkan solusi yang memenuhi nilai batas.
Kita perlu menggunakan metode pencarian akar seperti metode bagi
dua untuk mendapatkan solusi.

\begin{soal}
Dengan menggunakan parameter numerik yang sama dengan soal sebelumnya, namun
persamaan diferensial yang digunakan adalah:
\begin{equation*}
\frac{\mathrm{d}^2 T}{\mathrm{d}x^2} + h''(T_a - T)^4 = 0
\end{equation*}
dengan $h'' = 5\times 10^{-8}$
\end{soal}


\begin{pythoncode}
# .... import dan definisi fungsi yang digunakan

# T == y[0]
# dT/dx == y[1]
# d2T/dx2 == dydx[1]
def deriv(x, y):
    Nvec = len(y)
    assert Nvec == 2
    dydx = np.zeros(Nvec)
    h = 5e-8
    T_a = 20.0
    dydx[0] = y[1]
    dydx[1] = h*(y[0] - T_a)**4
    return dydx

def obj_func(z0_guess):
    # Initial cond
    x0 = 0.0
    y0 = np.zeros(2)
    y0[0] = 40.0 # from the boundary condition, T(0) = 40
    y0[1] = z0_guess
    xf = 10.0  # end interval
    Tf = 200.0 # Boundary condition, T(10) = 200
    #
    h = 2.0 # Step size
    Nstep = int( (xf-x0)/h )
    x, y = ode_solve(deriv, ode_rk4_1step, x0, y0, h, Nstep)
    # At the end of the interval
    Tf_guess = y[-1,0]
    return Tf_guess - Tf


# For testing values of z0_1 and z0_2 which brackets obj_func
z0_1 = 5.0
Tf_1 = obj_func(z0_1)
z0_2 = 11.0
Tf_2 = obj_func(z0_2)
print("Tf_1 = ", Tf_1)
print("Tf_2 = ", Tf_2)

z0 = root_bisection(obj_func, z0_1, z0_2, TOL=1.0e-9)

# Now solve the ODE with the obtained z0
x0 = 0.0
y0 = np.zeros(2)
y0[0] = 40.0 # from the boundary condition, T(0) = 40
y0[1] = z0
xf = 10.0  # end interval
#
h = 2.0 # Step size
Nstep = int( (xf-x0)/h )
x, y = ode_solve(deriv, ode_rk4_1step, x0, y0, h, Nstep)
print("Tf = ", y[-1,0])  # CHECK: Should give a value close to 200.0

# Now plot the solution
plt.clf()
plt.plot(x, y[:,0], marker="o", label="Temperature")
# ....
\end{pythoncode}


Contoh hasil keluaran:
\begin{textcode}
Tf_1 =  -101.16453273552762
Tf_2 =  98.71355087438758
             Iter      Estimated          f(x)
             ----      ---------          ----

bisection:     1       8.0000000000     4.37036e+01
bisection:     2       9.5000000000     6.70737e+00
......  # output removed
bisection:    36       9.3398893460     3.14060e-10

bisection is converged in 36 iterations
Tf =  200.00000000031406
\end{textcode}


Berikut ini adalah plot solusi yang diperoleh.

{\centering
\includegraphics[scale=0.7]{../../chapra_7th/ch27/IMG_example_27_2.pdf}
\par}

Anda dapat menggunakan jumlah titik yang lebih banyak (ukuran langkah yang lebih
kecil untuk mendapatkan solusi yang lebih akurat).

Implementasi \pyinline{root_bisection} (jika diperlukan)
\begin{pythoncode}
from math import ceil, log10

def root_bisection(f, x1, x2, TOL=1.0e-9, NiterMax=None ):
    f1 = f(x1)
    if abs(f1) <= TOL:
        return x1, 0.0
    f2 = f(x2)
    if abs(f2) <= TOL:
        return x2, 0.0
    if f1*f2 > 0.0:
        raise RuntimeError("Root is not bracketed")

    # No NiterMax is provided
    # We calculate the default value here.
    if NiterMax == None:
        NiterMax = int(ceil( log10(abs(x2-x1)/TOL) )/ log10(2.0) ) + 10
        # extra 10 iterations

    # For the purpose of calculating relative error
    x3 = 0.0
    x3_old = 0.0

    print(13*" "+"Iter      Estimated          f(x)")
    print(13*" "+"----      ---------          ----")
    print("")

    for i in range(1,NiterMax+1):

        x3_old = x3
        x3 = 0.5*(x1 + x2)
        f3 = f(x3)

        print("bisection: %5d %18.10f %15.5e" % (i, x3, abs(f3)))

        if abs(f3) <= TOL:
            print("")
            print("bisection is converged in %d iterations" % i)
            # return the result
            return x3

        if f2*f3 < 0.0:
            # sign of f2 and f3 is different
            # root is in [x2,x3]
            # change the interval bound of x1 to x3
            x1 = x3
            f1 = f3
        else:
            # sign of f1 and f3 is different
            # root is in [x1,x3]
            # change the interval bound of x2 to x3
            x2 = x3
            f2 = f3

    print("No root is found")
    return None
\end{pythoncode}


\subsection{Metode Beda Hingga}

Alternatif lain yang dapat digunakan untuk menyelesaikan persoalan nilai
batas yang melibatkan persamaan diferensial adalah metode beda hingga.
Pada metode ini, operasi turunan diganti dengan aproksimasi
beda hingga yang melibatkan titik-titik diskrit.
Misalnya:
\begin{equation*}
\frac{\mathrm{d}^2 T}{\mathrm{d}x^2} = \frac{T_{i+1} - 2T_{i} + T_{i-1}}{\Delta x^2}
\end{equation*} 
yang dapat disubstitusikan kedalam persamaan 27.1 (pada buku Chapra) menjadi:
\begin{equation*}
\frac{T_{i+1} - 2T_{i} + T_{i-1}}{\Delta x^2} - h'(T_i - T_a) = 0
\end{equation*}
yang dapat disusun ulang menjadi sistem persamaan linear
\begin{equation*}
-T_{i-1} + (2 + h'\Delta x^2)T_i - T_{i+1} = h' \Delta x^2 T_a
\end{equation*}
Persamaan ini diaplikasikan pada setiap titik interior. Titik pertama dan
titik terakhir diperoleh nilainya dari syarat batas.


\begin{soal}
Gunakan metode beda hingga untuk menyelesaikan persoalan nilai
batas pada Chapra Contoh 27.1
\end{soal}

\begin{pythoncode}
# Finite-difference method for linear BVP

# d2T/dx2 + h'(T_a - T) = 0
# Boundary condition:
#   T(0) = 40
#   T(10) = 200

import numpy as np

h = 0.01
T_a = 20.0

x0 = 0.0
T0 = 40.0 # Boundary condition, T(0) = 40
xf = 10.0
Tf = 200.0 # Boundary condition, T(10) = 200
Δx = 2.0 # segment length (or step size in shooting method)
Nstep = int( (xf-x0)/Δx )
Npoints = Nstep + 1

T = np.zeros(Npoints)
T[0] = T0 # first point
T[-1] = Tf # last point 

# Finite-difference operator of second derivative matrix.
# In general we should use sparse matrix. However, because
# the size is rather small, we use full (dense) matrix.
# Please refer to the left-hand-side of Eq. 27.3 for the matrix elements.
Npointsm2 = Npoints-2 # Number of interior points
d2dx2 =  np.zeros((Npointsm2,Npointsm2))
for i in range(Npointsm2):
    d2dx2[i,i] = 2 + h*Δx**2
    if i != 0:
        d2dx2[i-1,i] = -1.0
    if i != (Npointsm2-1):
        d2dx2[i+1,i] = -1.0
# Display the matrix
print("FD representation of second-derivative operator:")
print(d2dx2)

# The vector represented by the right hand side of Eq. 27.3
f = np.zeros(Npointsm2)
for i in range(1,Npointsm2-1):
    f[i] = h*Δx**2*T_a
# From the left BC
f[0] = h*Δx**2*T_a + T0
# From the right BC 
f[-1] = h*Δx**2*T_a + Tf
# Display
print("f = ", f)

# Solve the linear equations
T[1:Npoints-1] = np.linalg.solve(d2dx2,f)

def exact_sol(x):
    return 20*((1 - np.exp(2))*np.exp(x/10) + (1 - 9*np.e)*np.exp(x/5) + \
            np.e*(9 - np.e))*np.exp(-x/10)/(1 - np.exp(2))

x = np.zeros(Npoints)
for i in range(Npoints):
    x[i] = x0 + i*Δx
    T_exact = exact_sol(x[i])
    error = abs(T[i] - T_exact)
    print("%18.10f %18.10f %18.10f %18.10e" % (x[i], T[i], T_exact, error))

plt.clf()
plt.plot(x, T, marker="o", label="Temperature")
plt.xlabel("x")
plt.ylabel("T")
plt.legend()
\end{pythoncode}

Contoh keluaran
\begin{textcode}
FD representation of second-derivative operator:
[[ 2.04 -1.    0.    0.  ]
 [-1.    2.04 -1.    0.  ]
 [ 0.   -1.    2.04 -1.  ]
 [ 0.    0.   -1.    2.04]]
f =  [ 40.8   0.8   0.8 200.8]
      0.0000000000      40.0000000000      40.0000000000   0.0000000000e+00
      2.0000000000      65.9698343668      65.9517913981   1.8042968650e-02
      4.0000000000      93.7784621082      93.7477895327   3.0672575481e-02
      6.0000000000     124.5382283340     124.5035454074   3.4682926640e-02
      8.0000000000     159.4795236931     159.4533954960   2.6128197126e-02
     10.0000000000     200.0000000000     200.0000000000   0.0000000000e+00
\end{textcode}




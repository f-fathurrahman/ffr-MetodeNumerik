\section{Metode Bairstow}

\textbf{Chapra Contoh 7.3}
Kode berikut ini adalah implementasi Python (belum lengkap)
dari metode Bairstow pada Chapra Gambar 7.5 dengan sedikit modifikasi.

\begin{pythoncode}
import numpy as np

# a_ is an array containing coefficients of the polynomial, starting
# from the lowest power.
# rr and ss are initial values of r and s (quadratic polynomial used for
# dividing the polynomial defined by a_).
# SMALL is a tolerance parameter
def root_bairstow( a_, rr=1.0, ss=1.0, NiterMax=100, SMALL=1e-10 ):
    
    a = np.copy(a_) # do not modify the input
    Ndeg = len(a) - 1
    
    re = np.zeros(Ndeg)
    im = np.zeros(Ndeg)
    
    b = np.zeros(Ndeg+1)
    c = np.zeros(Ndeg+1)
    
    r = rr
    s = ss
    n = Ndeg
    ier = 0
    
    iteration = 0
    
    while True:
        # Break out from the loop if reduced degree is less than 3
        # or number of iterations exceeds NiterMax    
        if (n < 3) or (iteration >= NiterMax) :
            break
    
        iteration = 0
        while True:
            iteration = iteration + 1
            if iteration >= NiterMax:
                break
            
            b[n]   = a[n]
            b[n-1] = ....
            c[n]   = b[n]
            c[n-1] = ....
            for i in range(n-2,-1,-1):
                b[i] = ....
                c[i] = ....
            # Solve the linear equations for dr and ds
            det = ....
            if abs(det) >= SMALL:
                dr = ....
                ds = ....
                r = r + dr
                s = s + ds
            else:
                # Update with different initial r and s
                r = r + 1.0
                s = s + 1.0
                iteration = 0
            
            # Stop the iteration if s and r did not changed much
            # NOTE: In the original algorithm (Chapra Fig. 7.5) relative
            # error is used, here we directly use the values of dr and ds
            if (abs(dr) <= SMALL) and (abs(ds) <= SMALL):
                break
        
        # Up to this points, we already found r and s of the quadratic
        # polynomial that factors or divide the original polynomial
        # We call quadroot to find the roots of this quadratic polynomial.
        r1, i1, r2, i2  = quadroot(r, s)
        re[n-1] = r1
        im[n-1] = i1
        re[n-2] = r2
        im[n-2] = i2
        
        # Update the original polynomial (using the quotient)
        # The degree of the original polynomial is reduced by 2.
        n = n - 2
        for i in range(0,n+1):
            a[i] = b[i+2]
    
    # Up to this point, n (the degree of the reduced polynomial should be
    # less than 3, i.e. it becomes quadratic or linear polynomial)
    # We solve for the remaining roots.
    if iteration < NiterMax:
        if n == 2:
            r = -a[1]/a[2]
            s = -a[0]/a[2]
            r1, i1, r2, i2 = quadroot(r, s)
            re[n-1] = r1
            im[n-1] = i1
            re[n-2] = r2
            im[n-2] = i2
        else:
            re[n-1] = -a[0]/a[1]
            im[n-1] = 0.0
    else:
        print("WARNING: there are some unknown errors")
        print("WARNING: probably NiterMax is exceeded.)
        ier = 1
    
    # Convert to complex-valued array
    zroots = np.zeros(Ndeg, dtype=np.complex128)
    # `numpy.complex128`: Complex number type composed
    # of 2 64-bit-precision floating-point numbers.
    for i in range(Ndeg):
        zroots[i] = complex(re[i], im[i])
    return zroots
    
    
def quadroot(r, s):
    disc = r**2 + 4.0*s
    if disc > 0:
        # real part of the roots
        r1 = ....
        r2 = ....
        # imaginary part of the roots (set to zero)
        i1 = 0.0
        i2 = 0.0
    else:
        r1 = ....
        r2 = ....
        i1 = ....
        i2 = ....
    #
    return r1, i1, r2, i2    
\end{pythoncode}

\begin{soal}
Lengkapi kode untuk \pyinline{root_bairstow} dan aplikasikan untuk
mencari akar-akar dari polinomial:
\begin{equation*}
f(x) = x^5 - 3.5x^4 + 2.75x^3 + 2.125x^2 - 3.875x + 1.25
\end{equation*}
\end{soal}

\begin{soal}
Gunakan \pyinline{root_bairstow} dan fungsi yang tersedia pada Numpy
(lihat bagian mengenai penggunakan pustaka Python)
untuk menentukan akar-akar dari polinomial berikut:
\begin{itemize}
\item $f(x) = x^3 - x^2 + 2x - 2$
\item $f(x) = 2x^4 + 6x^2 + 8$
\item $f(x) = x^4 - 2x^3 + 6x^2 - 2x + 5$
\item $f(x) = -2 + 6.2x - 4x^2 + 0.7x^3$
\item $f(x) = 9.34 - 21.97x + 16.3x^2 - 3.704x^3$
\item $f(x) = 10x^5 - 2x^3 + 6x^2 - 2x + 5$
\end{itemize}
\end{soal}
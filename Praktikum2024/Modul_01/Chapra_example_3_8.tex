\subsection{Galat pembulatan (rounding error)}

\textbf{Chapra Contoh 3.8}

Akar-akar suatu polinomial kuadrat:
\begin{equation*}
ax^2 + bx + c = 0
\end{equation*}
diberikan oleh formula berikut
\begin{equation}
x_{1,2} = \frac{-b \pm \sqrt{b^2 - 4ac}}{2a}
\label{eq:quadeq1}
\end{equation}
Untuk kasus di mana $b^2 \gg 4ac$ perbedaan antara pembilang dapat menjadi sangat kecil.
Pada kasus tersebut, kita dapat menggunakan \textit{double precision} untuk mengurangi
kesalahan pembulatan. Selain itu, kita juga dapat menggunakan formula:
\begin{equation}
x_{1,2} = \frac{-2c}{b \pm \sqrt{b^2 - 4ac}}
\label{eq:quadeq2}
\end{equation}

Mengikuti contoh yang diberikan pada buku, kita akan menggunakan
$a = 1$, $b = 3000.001$, dan $c = 3$. Akar-akar eksaknya adalah
$x_{1} = -0.001$ dan $x_2 = -3000$.

\begin{soal}
Buat program Python dengan menggunakan \textit{single precision} dan
\textit{double precision} untuk melihat perbedaan hasil yang diberikan 
dari Persamaan \eqref{eq:quadeq1} dan Persamaan \eqref{eq:quadeq2}.
Bandingkan akar-akar yang Anda peroleh dengan akar-akar eksak.
Untuk masing-masing akar, formula mana yang memberikan hasil yang paling dekat dengan
hasil eksak?
\end{soal}

Program berikut ini adalah dalam \textit{single precision}
yang dapat Anda lengkapi. Anda juga dapat menggunakan program yang Anda tulis sendiri
dari awal atau modifikasi dari program ini.
\begin{pythoncode}
import numpy as np

def calc_quad_root_v1(a, b, c):
    D = np.float32(b**2) - np.float32(4)*a*c
    x1 = (-b + np.sqrt(D))/(np.float32(2)*a)
    x2 = # ... lengkapi
    return x1, x2
  
def calc_quad_root_v2(a, b, c):
    D = # ... lengkapi
    x1 = # ... lengkapi
    x2 = # ... lengkapi
    return x1, x2
  
a = np.float32(1.0)
b = np.float32(3000.001)
c = np.float32(3.0)
  
x1_true = np.float32(-0.001)
x2_true = np.float32(-3000.0)
  
x1, x2 = calc_quad_root_v1(a, b, c)
print("Using 1st formula: approx roots: ", x1, " ", x2)
print(type(x1), type(x2)) # pastikan x1 dan x2 merupakan np.float32
  
x1, x2 = calc_quad_root_v2(a, b, c)
print(type(x1), type(x2))
print("Using 2nd formula: approx roots: ", x1, " ", x2)  
print("True roots: ", x1_true, " ", x2_true)
\end{pythoncode}


\begin{soal}
Program berikut ini, kita akan menggunakan CAS atau \textit{computer algebra system}
untuk memastikan bahwa formula \eqref{eq:quadeq1} dan \eqref{eq:quadeq2} memberikan
hasil yang identik. Lengkapi kode berikut ini dan cek apakah hasil yang diberikan
dari kedua formula tersebut adalah sama.
Selain itu coba turunkan formula \eqref{eq:quadeq2} dari \eqref{eq:quadeq1}.
\end{soal}

\begin{pythoncode}
from sympy import *

def calc_quad_root_v1(a, b, c):
    D = b**2 - 4*a*c
    x1 = (-b + sqrt(D))/(2*a)
    x2 = (-b - sqrt(D))/(2*a)
    return x1, x2
  
def calc_quad_root_v2(a, b, c):
    D = # lengkapi ...
    x1 = # lengkapi ...
    x2 = # lengkapi ... 
    return x1, x2
  
a = Rational(1)
b = Rational(3000001, 1000)
c = Rational(3)
  
x1_true = -Rational(1, 1000)
x2_true = -3000
  
x1, x2 = calc_quad_root_v1(a, b, c)
print("Using 1st formula: appprox roots: ", x1, " ", x2)

x1, x2 = calc_quad_root_v2(a, b, c)
print("Using 2nd formula: appprox roots: ", x1, " ", x2)

print("True roots: ", x1_true, " ", x2_true)
\end{pythoncode}
Perhatikan bahwa kode di atas juga mencetak tipe dari variabel \txtinline{x1} dan
\txtinline{x2} adalah bilangan integer atau rasional.
Pada SymPy, tipe untuk integer dan rasional adalah:
\begin{textcode}
<class 'sympy.core.numbers.Integer'> <class 'sympy.core.numbers.Rational'>
\end{textcode}

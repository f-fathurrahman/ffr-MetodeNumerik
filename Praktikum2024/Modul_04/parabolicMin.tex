\subsection{Interpolasi parabola}

Polinomial orde dua atau parabola seringkali dapat digunakan sebagai pendekatan
dari $f(x)$ di sekitar minimum, misalnya untuk mengaproksimasi fungsi potensial Morse
antara dua atom.
Pada metode interpolasi parabola, dipilih tiga titik $x_0, x_1, x_2$
yang kemudian diinterpolasi menjadi
fungsi polinomial orde dua. Nilai optimum dari $f(x)$ kemudian diaproksimasi sebagai
optimum dari parabola yang dihasilkan. Nilai optimum diperoleh dari
\begin{equation}
x_3 = \frac{f(x_0)(x_{1}^2 - x_{2}^2) + f(x_1)(x_{2}^2 - x_{0}^2) + f(x_2)(x_{0}^2 - x_{1}^2)}%
{2f(x_0)(x_{1} - x_{2}) + 2f(x_1)(x_{2} - x_{0}) + 2f(x_2)(x_{0} - x_{1})}
\end{equation}
Untuk iterasi selanjutnya dapat digunakan strategi yang sama dengan pemilihan titik pada
metode rasio emas. Alternatif lain yang akan kita gunakan di sini adalah
penggunaan titik secara sekuensial (lebih mudah untuk diimplementasikan):
$x_0 \leftarrow x_1$, $x_1 \leftarrow x_2$, dan $x_2 \leftarrow x_3$.

\begin{soal}
Program Python berikut ini mengimplementasikan metode interpolasi parabola untuk
mencari maksimum dari $f(x) = 2\sin(x) - \frac{x^2}{10}$. Lengkapi kode yang tersebut.
\end{soal}

Catatan: kode berikut ini belum dibuat dalam bentuk fungsi seperti pada kasus
\pyinline{optim_golden_ratio}, Anda dapat mengubahnya jika diperlukan.
Kriteria konvergensi yang digunakan adalah ketika nilai optimum sudah tidak berubah
berdasarkan suatu nilai tertentu.

\begin{pythoncode}
import numpy as np

def my_func(x):
    return 2*np.sin(x) - x**2/10

# f0 = f(x0), f1 = f(x1), f2 = f(x2)
def calc_parabolic_x3(x0, f0, x1, f1, x2, f2):
    num = ....  # lengkapi
    denum = .... # lengkapi
    return num/denum

# Initial guess
x0 = 0.0; f0 = my_func(x0)
x1 = 1.0; f1 = my_func(x1)
x2 = 4.0; f2 = my_func(x2)

x3 = calc_parabolic_x3(x0, f0, x1, f1, x2, f2)
f3 = my_func(x3)
print("x3 = %18.10f f3 = %18.10f" % (x3, f3))

TOL = 1e-10
NiterMax = 100

for iiter in range(1,NiterMax+1):
    xopt_old = x3
    fopt_old = f3

    # Sequentially choose the next points
    x0 = x1; f0 = f1
    x1 = x2; f1 = f2
    x2 = x3; f2 = f3

    x3 = calc_parabolic_x3(x0, f0, x1, f1, x2, f2)
    f3 = my_func(x3)
    print("x3 = %18.10f f3 = %18.10f" % (x3, f3))

    if abs(fopt_old - f3) < TOL:
        print("Converged")
        break
\end{pythoncode}

Contoh keluaran:
\begin{textcode}
x3 =       1.5055348740 f3 =       1.7690789285
x3 =       1.4902527509 f3 =       1.7714309125
x3 =       1.3908075360 f3 =       1.7742568388
x3 =       1.4275400017 f3 =       1.7757256530
x3 =       1.4275037854 f3 =       1.7757256506
x3 =       1.4275518296 f3 =       1.7757256531
x3 =       1.4275526174 f3 =       1.7757256531
Converged
\end{textcode}
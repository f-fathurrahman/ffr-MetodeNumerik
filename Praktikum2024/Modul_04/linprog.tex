\section{Pemrograman Linear}

Pada bagian ini, kita akan menggunakan pustaka SciPy untuk menyelesaikan permasalahan
pemrograman linear, di mana fungsi objektif dan kendala yang terlibat berbentuk
persamaan dan pertidaksamaan linear. Fungsi yang akan digunakan adalah \pyinline{linprog}
yang terdefinisi pada modul \pyinline{scipy.optimize}.
Deklarasi API (\textit{application programming interface}) dari fungsi ini adalah sebagai berikut.
%
\begin{pythoncode}
scipy.optimize.linprog(
    c, A_ub=None, b_ub=None, A_eq=None, b_eq=None, bounds=None,
    method='highs', callback=None, options=None, x0=None, integrality=None)
\end{pythoncode}
%
Fungsi ini menyelesaikan permasalahan
pemrograman linear yang memiliki bentuk standard sebagai berikut.
\begin{align*}
& \min_{x} \mathbf{c}^{\mathrm{T}} \mathbf{x} \\
\text{dengan kendala:}\,\, & \mathbf{A}_{\mathrm{ub}} \leq \mathbf{b}_{\mathrm{ub}} \\
& \mathbf{A}_{\mathrm{eq}} = \mathbf{b}_{\mathrm{eq}} \\
& \mathbf{l} \leq \mathbf{x} \leq \mathbf{u}
\end{align*}
dengan:
\begin{itemize}
\item $\mathbf{x}$: (output, array 1d) variabel yang akan diubah-ubah
atau dicari nilainya, atau variabel keputusan (\textit{decision variables}).
\item $\mathbf{c}$: (argumen \pyinline{c}, array 1d) koefisien fungsi objektif (linear).
\item $\mathbf{A}_{\mathrm{ub}}$: (argumen \pyinline{A_ub}, array 2d) matriks kendala batas atas
(\pyinline{ub}: \textit{upper bound})
\item $\mathbf{b}_{\mathrm{ub}}$: (argumen \pyinline{b_ub}, array 1d) vektor kendala batas atas
\item $\mathbf{A}_{\mathrm{eq}}$: (argumen \pyinline{A_eq}, array 2d) matriks kendala persamaan
(\pyinline{eq}: \textit{equality})
\item $\mathbf{b}_{\mathrm{eq}}$: (argumen \pyinline{b_eq}, array 1d) vektor kendala persamaan
\item $\mathbf{l}$ dan $\mathbf{u}$: (argumen \pyinline{l} dan \pyinline{b},
array 1d) kendala rentang nilai untuk
variabel keputusan.
\end{itemize}

Untuk selengkapnya silakan membaca dokumentasi terkait
\footnote{\url{https://docs.scipy.org/doc/scipy/reference/generated/scipy.optimize.linprog.html}}.

Sebagai contoh, kita ingin mencari nilai minimum dari
\begin{equation*}
f(x_0, x_1) = -x_0 + 4x_1
\end{equation*}
dan $x_0$ serta $x_1$ yang membuat $f(x_0, x_1)$ minimum dengan kendala sebagai
berikut:
\begin{align*}
-3x_0 + x_1 & \leq 6 \\
-x_0 - 2x_1 & \geq -4 \\
x_1 \geq -3
\end{align*}
Bentuk di atas belum sesuai dengan bentuk standard yang diasumsikan oleh
\pyinline{linprog}. Oleh karena itu kita perlu mengubah terlebih dahulu bentuk kendala
standard. Kendala kedua dikalikan dulu dengan -1 sehingga kendala menjadi
sebagai berikut.
\begin{align*}
-3x_0 + x_1 & \leq 6 \\
x_0 + 2x_1 & \leq 4 \\
x_1 \geq -3
\end{align*}
Pertidaksamaan terakhir memberikan batas untuk bawah untuk $x_1$.
Kita tidak diberikan bentuk kendala persamaan sehingga \pyinline{A_eq=None} dan
\pyinline{b_eq=None}, yang sudah merupakan nilai \textit{default} dari \pyinline{linprog}.

Kode berikut ini akan mencari solusi dari permasalahan di atas.
\begin{pythoncode}
from scipy.optimize import linprog
c = [-1, 4] # perhatikan bentuk fungsi objektif
# Matriks pertidaksamaan A_ub,
# perhatikan kendala pertidaksamaan untuk variabel keputusan dalam bentuk standard.
A = [
    [-3, 1],
    [1, 2]
]
# Vektor pertidaksamaan A_ub,
b = [6, 4]
x0_bounds = (None, None) # tidak ada batasan untuk x0
x1_bounds = (-3, None) # batas untuk x1, hanya batas bawah yang diberikan

res = linprog(c, A_ub=A, b_ub=b, bounds=[x0_bounds, x1_bounds])

print("Min value: ", res.fun)
print("Decision variables: ", res.x)
\end{pythoncode}

Contoh keluaran:
\begin{textcode}
Min value:  -22.0
Decision variables:  [10. -3.]
\end{textcode}


\begin{soal}
Selesaikan masalah pemrograman linear yang diberikan pada Chapra Contoh 15.2
dengan menggunakan \pyinline{linprog}.
Pada contoh ini kita diminta untuk mencari \textbf{nilai maksimum} dari:
\begin{equation*}
Z = 150x_1 + 175x_2
\end{equation*}
dengan kendala
\begin{align*}
7x_1 + 11x_2 & \leq 77 \\
10x_1 + 8x_2 & \leq 80
\end{align*}
serta $0 \leq x_1 \leq 9$, $0 \leq x_1 \leq 9$ dan bandingkan hasil yang Anda
peroleh dengan metode grafis dan simplex seperti pada buku.
\end{soal}

Anda dapat melengkapi kode berikut.
\begin{pythoncode}
from scipy.optimize import linprog

c = [-150, -175]
# menggunakan tanda berlawan (negatif) karena kita ingin mencari nilai maksimum
# sedangkan bentuk standard linprog hanya menerima masalah minimisasi.

# Bentuk kendala pertidaksamaan sudah dalam bentuk standard yang diasumsikan
# oleh linprog.
A = [
    [7, 11],
    [10, 8]
]
b = [..., ...] # lengkapi

x0_bounds = (0, 9)
x1_bounds = (0, 6)

res = linprog(c, A_ub=A, b_ub=b, bounds=[x0_bounds, x1_bounds])

# We search for max, the result is negative the minimum we found from linprog
print("Max value: ", -res.fun)
print("Decision variables: ", res.x)
\end{pythoncode}


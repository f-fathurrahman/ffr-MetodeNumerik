\documentclass[a4paper,11pt,bahasa]{extarticle} % screen setting
\usepackage[a4paper]{geometry}

%\documentclass[b5paper,11pt,bahasa]{article} % screen setting
%\usepackage[b5paper]{geometry}

%\geometry{verbose,tmargin=1.5cm,bmargin=1.5cm,lmargin=1.5cm,rmargin=1.5cm}

\geometry{verbose,tmargin=2.0cm,bmargin=2.0cm,lmargin=2.0cm,rmargin=2.0cm}

\setlength{\parskip}{\smallskipamount}
\setlength{\parindent}{0pt}

%\usepackage{cmbright}
%\renewcommand{\familydefault}{\sfdefault}

\usepackage{amsmath}
\usepackage{amssymb}

\usepackage[libertine]{newtxmath}

\usepackage[no-math]{fontspec}
\setmainfont{Linux Libertine O}

%\usepackage{fontspec}
%\usepackage{lmodern}

\setmonofont{JuliaMono-Regular}


\usepackage{hyperref}
\usepackage{url}
\usepackage{xcolor}
\usepackage{enumitem}
\usepackage{mhchem}
\usepackage{graphicx}
\usepackage{float}

\usepackage{minted}

\newminted{julia}{breaklines,fontsize=\footnotesize}
\newminted{python}{breaklines,fontsize=\footnotesize}

\newminted{bash}{breaklines,fontsize=\footnotesize}
\newminted{text}{breaklines,fontsize=\footnotesize}

\newcommand{\txtinline}[1]{\mintinline[breaklines,fontsize=\footnotesize]{text}{#1}}
\newcommand{\jlinline}[1]{\mintinline[breaklines,fontsize=\footnotesize]{julia}{#1}}
\newcommand{\pyinline}[1]{\mintinline[breaklines,fontsize=\footnotesize]{python}{#1}}

\newmintedfile[juliafile]{julia}{breaklines,fontsize=\footnotesize}
\newmintedfile[pythonfile]{python}{breaklines,fontsize=\footnotesize}
\newmintedfile[fortranfile]{fortran}{breaklines,fontsize=\footnotesize}

\usepackage{mdframed}
\usepackage{setspace}
\onehalfspacing

\usepackage{babel}
\usepackage{appendix}

\newcommand{\highlighteq}[1]{\colorbox{blue!25}{$\displaystyle#1$}}
\newcommand{\highlight}[1]{\colorbox{red!25}{#1}}

\newcounter{soal}%[section]
\newenvironment{soal}[1][]{\refstepcounter{soal}\par\medskip
   \noindent \textbf{Soal~\thesoal. #1} \sffamily}{\medskip}


\definecolor{mintedbg}{rgb}{0.95,0.95,0.95}
\BeforeBeginEnvironment{minted}{
    \begin{mdframed}[%
        topline=false,bottomline=false,%
        leftline=false,rightline=false]
}
\AfterEndEnvironment{minted}{\end{mdframed}}


\BeforeBeginEnvironment{soal}{
    \begin{mdframed}[%
        topline=true,bottomline=false,%
        leftline=true,rightline=false]
}
\AfterEndEnvironment{soal}{\end{mdframed}}


\begin{document}

\title{%
{\small TF2202 Komputasi Rekayasa}\\
Persamaan Diferensial Parsial
}
\author{Tim Praktikum Komputasi Rekayasa 2024\\
Teknik Fisika\\
Institut Teknologi Bandung}
\date{}
\maketitle

Persamaan difusi pada 1d dapat dituliskan sebagai:
\begin{equation*}
\frac{\partial u}{\partial t} =\alpha\frac{\partial^{2}u}{\partial x^{2}}+f(x,t)
\end{equation*}

Aproksimasi turunan pertama terhadap waktu
\begin{equation*}
\frac{\partial u}{\partial t}\approx\frac{u_{i}^{n+1}-u_{i}^{n}}{\Delta t}    
\end{equation*}

Aproksimasi turunan kedua terhadap $x$
\begin{equation*}
\frac{\partial^{2}u}{\partial x^{2}}=\frac{u_{i+1}^{n}-2u_{i}^{n}+u_{i-1}^{n}}{\Delta x^{2}}
\end{equation*}

Diperoleh
\begin{equation*}
\frac{u_{i}^{n+1}-u_{i}^{n}}{\Delta t}=\alpha\frac{u_{i+1}^{n}-2u_{i}^{n}+u_{i-1}^{n}}{\Delta x^{2}}+f_{i}^{n}
\end{equation*}

Skema eksplisit (Forward Euler):
\begin{equation*}
u_{i}^{n+1}-u_{i}^{n}=F\left(u_{i+1}^{n}-2u_{i}^{n}+u_{i-1}^{n}\right)+\Delta t\ f_{i}^{n}
\end{equation*}


menjadi
\begin{equation*}
u_{i}^{n+1} = u_{i}^{n} + F\left(u_{i+1}^{n} - 2u_{i}^{n} + u_{i-1}^{n}\right)+\Delta t\ f_{i}^{n}
\end{equation*}

\begin{equation*}
F = \alpha\frac{\Delta t}{\Delta x^{2}}
\end{equation*}

Syarat kestabilan: $F\leq\dfrac{1}{2}$.

Contoh, menggunakan solusi buatan:
\begin{equation*}
u(x,t) = 5tx(L-x)    
\end{equation*}

\begin{equation*}
\frac{\partial u}{\partial t} = 5x(L-x)    
\end{equation*}

\begin{equation*}
\frac{\partial^{2}u}{\partial x^{2}} = -10t
\end{equation*}

\begin{equation*}
f(x,t) = \frac{\partial u}{\partial t} - \alpha\frac{\partial^{2}u}{\partial x^{2}}=5tx(L-x)+10t
\end{equation*}



Aproksimasi turunan pertama terhadap waktu menggunakan beda-hingga mundur
\textit{backward finite difference}:
\begin{equation*}
\frac{\partial u}{\partial t}\approx\frac{u_{i}^{n}-u_{i}^{n-1}}{\Delta t}
\end{equation*}

Aproksimasi turunan kedua terhadap $x$
\begin{equation*}
\frac{\partial^{2}u}{\partial x^{2}}=\frac{u_{i+1}^{n}-2u_{i}^{n}+u_{i-1}^{n}}{\Delta x^{2}}
\end{equation*}

Diperoleh
\begin{equation*}
\frac{u_{i}^{n}-u_{i}^{n-1}}{\Delta t}=\alpha\frac{u_{i+1}^{n}-2u_{i}^{n}+u_{i-1}^{n}}{\Delta x^{2}}+f_{i}^{n}
\end{equation*}

$N_{x}=3$, kita memiliki 4 titik:~$u_{0}^{n},u_{1}^{n},u_{2}^{n},u_{3}^{n}$
(dua titik ujung diketahui nilainya). Yang ingin dicari adalah $u_{1}^{n},u_{2}^{n}$.

Untuk $i=1$:
\begin{equation*}
\frac{u_{1}^{n}-u_{1}^{n-1}}{\Delta t}=\alpha\frac{u_{2}^{n}-2u_{1}^{n}+u_{0}^{n}}{\Delta x^{2}}+f_{1}^{n}
\end{equation*}

Untuk $i=2:$
\begin{equation*}
\frac{u_{2}^{n}-u_{2}^{n-1}}{\Delta t}=\alpha\frac{u_{3}^{n}-2u_{2}^{n}+u_{1}^{n}}{\Delta x^{2}}+f_{2}^{n}
\end{equation*}


Rearrange: gunakan nilai batas: $u_{0}^{n}=0$ dan $u_{3}^{n}=0$
\begin{equation*}
u_{1}^{n}-u_{1}^{n-1}=\alpha\Delta t\frac{u_{2}^{n}-2u_{1}^{n}}{\Delta x^{2}}+\Delta tf_{1}^{n}
\end{equation*}

\begin{equation*}
u_{1}^{n}-\alpha\Delta t\frac{u_{2}^{n}-2u_{1}^{n}}{\Delta x^{2}}=u_{1}^{n-1}+\Delta tf_{1}^{n}
\end{equation*}

\begin{equation*}
u_{1}^{n}-F\left(u_{2}^{n}-2u_{1}^{n}\right)=u_{1}^{n-1}+\Delta tf_{1}^{n}
\end{equation*}

\begin{equation*}
u_{1}^{n} + 2Fu_{1}^{n} - Fu_{2}^{n} = u_{1}^{n-1} + \Delta t f_{1}^{n}
\end{equation*}

\begin{equation*}
(1 + 2F)u_{1}^{n} - Fu_{2}^{n} = u_{1}^{n-1} + \Delta t f_{1}^{n}
\end{equation*}


Dapat dituliskan juga untuk $i=2$
\begin{equation*}
-Fu_{1}^{n}+(1+2F)u_{2}^{n}=u_{2}^{n-1}+\Delta tf_{2}^{n}
\end{equation*}


Diperoleh sistem persamaan linear: $2\times2$
\begin{align*}
(1+2F)u_{1}^{n} - Fu_{2}^{n} & = u_{1}^{n-1} + \Delta t f_{1}^{n} \\
-Fu_{1}^{n} + (1+2F)u_{2}^{n} & = u_{2}^{n-1} + \Delta t f_{2}^{n}
\end{align*}

Dalam bentuk matriks:
\[
\left(\begin{array}{cc}
1+2F & -F\\
-F & 1+2F
\end{array}\right) = \left(\begin{array}{c}
u_{1}^{n}\\
u_{2}^{n}
\end{array}\right)=\left(\begin{array}{c}
u_{1}^{n-1}+\Delta tf_{1}^{n}\\
u_{2}^{n-1}+\Delta tf_{2}^{n}
\end{array}\right)
\]
Diketahui $u_{i}^{0}$ kita dapat memperoleh $u_{i}^{1}$ dengan cara
menyelesaikan sistem persamaan persamaan linear. 

Untuk kasus yang lebih yang lebih umum:
\[
-Fu_{i-1}^{n}+(1+2F)u_{i}^{n}-Fu_{i+1}^{n}=u_{i}^{n-1}+\Delta tf_{i}^{n}
\]

Jika ingin memasukkan juga titik-titik ujung, misalnya untuk $N_{x}=3$
\[
\left(\begin{array}{cccc}
1 & 0 & 0 & 0\\
0 & 1+2F & -F & 0\\
0 & -F & 1+2F & 0\\
0 & 0 & 0 & 1
\end{array}\right)\left(\begin{array}{c}
u_{0}^{n}\\
u_{1}^{n}\\
u_{2}^{n}\\
u_{3}^{n}
\end{array}\right)=\left(\begin{array}{c}
u_{0}^{n-1}+\Delta tf_{0}^{n}\\
u_{1}^{n-1}+\Delta tf_{1}^{n}\\
u_{2}^{n-1}+\Delta tf_{2}^{n}\\
u_{3}^{n-1}+\Delta tf_{3}^{n}
\end{array}\right)
\]

Untuk kasus $N_{x}=4$
\[
\left(\begin{array}{ccc}
1+2F & -F & 0\\
-F & 1+2F & -F\\
0 & -F & 1+2F
\end{array}\right)
\]







\end{document}

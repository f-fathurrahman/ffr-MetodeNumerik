\documentclass[12pt]{article}
\usepackage[a4paper]{geometry}
\geometry{verbose,tmargin=2cm,bmargin=2cm,lmargin=2cm,rmargin=2cm}
\usepackage{amsmath}
    
\setlength{\parindent}{0cm}
  
\usepackage{float}
\usepackage{graphicx}

\usepackage{hyperref}
\usepackage{url}
\usepackage{xcolor}

\usepackage{minted}
\newminted{scilab}{breaklines}

\definecolor{mintedbg}{rgb}{0.95,0.95,0.95}
\usepackage{mdframed}

\BeforeBeginEnvironment{minted}{\begin{mdframed}[backgroundcolor=mintedbg]}
\AfterEndEnvironment{minted}{\end{mdframed}}

\begin{document}

\title{Tugas Besar TF2202}
\date{}
\maketitle

\begin{scilabcode}
function [a,b] = func1(c,d)
  // Test comment
endfunction
\end{scilabcode}

Selesaikan soal-soal berikut ini dengan menggunakan Scilab.

\section{Soal 1: Perbandingan akurasi beberapa metode}

Carilah solusi numerik dari persamaan diferensial dengan syarat awal:
\begin{align}
y''(t) + y(t) = 0 \\
y(0) = 0, \hspace{1cm} y'(0) = 1
\end{align}

Bandingkan solusi yang diperoleh dengan solusi analitik:
\begin{equation}
y(t) = \sin(t)
\end{equation}

Gunakan menggunakan metode-metode berikut ini untuk mencari solusi numeriknya.
\begin{itemize}
\item Euler
\item Euler dengan prediktor-korektor (Runge-Kutta orde-2)
\item Runge-Kutta orde-4
\end{itemize}

Dari solusi numerik yang didapatkan, buatlah (1) plot antara $y$ dan $y'$
dan (2) plot antara $t$ dan $y$.

\textbf{SOLUSI}

Dengan menggunakan notasi berikut:
\begin{equation}
y_1 \equiv y, \hspace{1cm} y_2 \equiv y'
\end{equation}

%\begin{mdframed}[backgroundcolor=mintedbg]
%\inputminted[breaklines]{scilab}{soal_01.sce}
%\end{mdframed}


\begin{figure}[H]
\centering
\includegraphics[width=0.3\textwidth]{images/soal_01_ode_euler_y1_y2.pdf}
\includegraphics[width=0.3\textwidth]{images/soal_01_ode_euler_PC_y1_y2.pdf}
\includegraphics[width=0.3\textwidth]{images/soal_01_ode_RK4_y1_y2.pdf}
\par
\end{figure}

\begin{figure}[H]
\centering
\includegraphics[width=0.3\textwidth]{images/soal_01_ode_euler_t_y1.pdf}
\includegraphics[width=0.3\textwidth]{images/soal_01_ode_euler_PC_t_y1.pdf}
\includegraphics[width=0.3\textwidth]{images/soal_01_ode_RK4_t_y1.pdf}
\par
\end{figure}


\section{Soal 2: Gerakan pendulum}

Gerakan pendulum
\input{solusi_03}
\section{Soal 4: Persamaan Schrodinger melalui eigenvalue}

(a) harmonic oscillator
\section{Soal 5: Persamaan Poisson 2D}
\input{solusi_06}
\section{Soal 7: Persamaan gelombang}

\end{document}
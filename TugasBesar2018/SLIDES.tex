\documentclass[bahasa,10pt]{beamer}

\setlength{\parskip}{\smallskipamount}
\setlength{\parindent}{0pt}

\setbeamersize{text margin left=5pt, text margin right=5pt}

\usepackage{amsmath}
\usepackage{amssymb}
\usepackage{braket}

\usepackage{minted}
\newminted{scilab}{breaklines,fontsize=\footnotesize,texcomments=true}

\definecolor{mintedbg}{rgb}{0.95,0.95,0.95}
\usepackage{mdframed}

\BeforeBeginEnvironment{minted}{\begin{mdframed}[backgroundcolor=mintedbg]}
\AfterEndEnvironment{minted}{\end{mdframed}}

\setcounter{secnumdepth}{3}
\setcounter{tocdepth}{3}

\makeatletter
%%%%%%%%%%%%%%%%%%%%%%%%%%%%%% Textclass specific LaTeX commands.
 % this default might be overridden by plain title style
 \newcommand\makebeamertitle{\frame{\maketitle}}%
 % (ERT) argument for the TOC
 \AtBeginDocument{%
   \let\origtableofcontents=\tableofcontents
   \def\tableofcontents{\@ifnextchar[{\origtableofcontents}{\gobbletableofcontents}}
   \def\gobbletableofcontents#1{\origtableofcontents}
 }

\makeatother

\usepackage{babel}

\begin{document}


\title{Tugas Besar TF2202}
\author{}
\institute{}
%\institute{
%Program Studi Teknik Fisika \\
%Divisi Komputasi Pusat Penelitian Nanosains dan Nanoteknologi \\
%Institut Teknologi Bandung
%}
\date{}

\frame{\titlepage}

\begin{frame}
\frametitle{Format Laporan}

Bagian-bagian yang harus ada dalam laporan:
\begin{itemize}
\item Deskripsi masalah
\item Penurunan model numerik (diskrit) dan/atau metode yang digunakan
\item Implementasi solusi dalam Scilab
\item Solusi numerik yang diperoleh beserta visualisasinya. Gunakan beberapa
parameter numerik (misalnya jumlah titik yang digunakan untuk diskritisasi)
untuk menghitung solusi numerik dan berikan komentar atas hasil yang Anda
dapatkan.
\item Referensi yang digunakan
\item Informasi lain yang Anda anggap perlu.
\end{itemize}

\end{frame}




\begin{frame}[fragile]
\frametitle{Pengumpulan}

\begin{itemize}
\item Laporan ditulis dalam format PDF.
\item File-file pendukung, seperti kode Scilab juga disertakan.
\item Kumpulkan laporan dan file-file pendukung dalam satu folder dengan
nama \texttt{TF2202\_NIM1\_NIM2}.
\item Kompresi folder dalam file \texttt{*.zip}: {\color{blue}\texttt{TF2202\_NIM1\_NIM2.zip}}
\item Pengumpulan secara online melalui \url{kuliah2013.tf.itb.ac.id}.
\end{itemize}

\end{frame}


\begin{frame}[fragile]
\frametitle{Kode Scilab}

\begin{itemize}
\item Kode Scilab ditulis dengan rapi, perhatikan indentasi.
\item Berikan komentar mengenai kode yang Anda buat: deskripsi input dan output,
arti dari variabel-variabel yang digunakan, dsb.
\end{itemize}

\begin{scilabcode}
function [t,y] = solve_ode(func,ti,tf,N)
// Komentar ...
// t: 
// y:
// func:
// ... etc

  // cek argumen

  // Algoritma Euler
  for k = 1:N
    y(k+1,:) = y(k,:) + h*func(t(k),y(k,:))
  end

endfunction
\end{scilabcode}


\end{frame}



\end{document}


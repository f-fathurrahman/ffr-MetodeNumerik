%\documentclass[a4paper,11pt]{article} % print setting
\documentclass[a4paper,12pt]{article} % screen setting

\usepackage[a4paper]{geometry}
\geometry{verbose,tmargin=1.5cm,bmargin=1.5cm,lmargin=1.5cm,rmargin=1.5cm}

\setlength{\parskip}{\smallskipamount}
\setlength{\parindent}{0pt}

\usepackage{cmbright}
\renewcommand{\familydefault}{\sfdefault}

\usepackage{fontspec}
\setmonofont{FreeMono}

%\usepackage{sfmath}
%\usepackage{mathptmx}
%\usepackage{mathpazo}

\usepackage{hyperref}
\usepackage{url}
\usepackage{xcolor}

\usepackage{amsmath}
\usepackage{amssymb}

\usepackage{graphicx}
\usepackage{float}

\usepackage{minted}
\newminted{julia}{breaklines,fontsize=\small}
\newminted{bash}{breaklines,fontsize=\small}
\newminted{text}{breaklines,fontsize=\small}

\newcommand{\txtinline}[1]{\mintinline{text}{#1}}
\newcommand{\jlinline}[1]{\mintinline{julia}{#1}}

\newmintedfile[juliafile]{julia}{breaklines,fontsize=\small}

\definecolor{mintedbg}{rgb}{0.90,0.90,0.90}
\usepackage{mdframed}

\BeforeBeginEnvironment{minted}{\begin{mdframed}[backgroundcolor=mintedbg]}
\AfterEndEnvironment{minted}{\end{mdframed}}

\usepackage{setspace}

\onehalfspacing

\usepackage{appendix}


\begin{document}


\title{Diffusion Equation with Finite Element Method \\
TF4062}
\author{Iwan Prasetyo \\
Fadjar Fathurrahman}
\date{}
\maketitle


\section{Steady heat equation}

Consider steady heat equation:
\begin{equation}
-k \frac{\mathrm{d}^2 T}{\mathrm{d}x^2} = Q
\end{equation}
for $0 < x < L$ and boundary conditions:
\begin{equation}
-k\frac{\mathrm{d}T}{\mathrm{d}x} = q \quad \text{at } x = 0
\end{equation}
and
\begin{equation}
T = T_{L} \quad \text{at } x = L
\end{equation}

The analytical solution to this problem can be expressed as \cite{Heinrich2017}:
\begin{equation}
T(x) = T_{L} + \frac{q}{k}(L - x) + \frac{1}{k} \int_{x}^{L}
\left( \int_{0}^{y} Q(z)\,\mathrm{d}z \right)\,\mathrm{d}y
\end{equation}
%
For constant $Q$, this equation reduces to
\begin{equation}
T(x) = T_{L} + \frac{q}{k}(L - x) + \frac{Q}{2k}(L^2 - x^2)
\end{equation}

\subsection{Discretization and approximation}

The FEM procedure start by discretizing the spatial domain into elements $e_{i}$
where:
\begin{equation}
e_{i}: \left\{ x_{i} \leq x \leq x_{i+1} \right\},\quad i=1,2,\ldots,n
\end{equation}
The points $x_{i}$ are also called \textit{nodes}. If we have $n$ elements then we
have $n+1$ nodes (or nodal points).
The solution to the PDE can be approximated as:
\begin{equation}
T(x) \approx \sum_{i}^{n+1} T_{i} N_{i}(x)
\end{equation}
where $T_i$ are (unknown) nodal values of $T(x)$ and $N_{i}(x)$ are \textit{shape functions}
(also called \textit{trial functions} or \textit{basis functions}) associated with each node.


\subsection{Linear shape functions}
The shape functions are defined such that $N_{i}(x_{i}) = 1$ and $N_{i}(x_{j})=0$
if $j \neq i$. In FEM, they are usually chosen to be low order, piecewise polynomials.
As an example, we will consider the case of one element, $n=1$, or two nodes. The shape functions
can be defined as:
\begin{align}
N_{1}(x) & = 1 - \frac{x}{L} \\
N_{2}(x) & = \frac{x}{L}
\end{align}
An illustration of this case is shows in Figure \ref{fig:Simpson_2_2}.

\begin{figure}[H]
{\centering
\includegraphics[scale=1.0]{images/Simpson_Fig_2_2.pdf}
\par}
\caption{The case of one element with linear shape functions. (from \cite{Simpson2017})}
\label{fig:Simpson_2_2}
\end{figure}

For general case of linear elements we have the following shape functions. For left
boundary points:
\begin{equation}
N_{1}(x) = \begin{cases}
\dfrac{x_{2} - x}{h_1} & x_{1} \leq x \leq x_{2} \\[10pt]
0                     & \text{otherwise}
\end{cases}
\end{equation}
%
For interior points:
%
\begin{equation}
N_{i}(x) = \begin{cases}
\dfrac{x - x}{h_{i-1}} & x_{i-1} \leq x \leq x_{i} \\[10pt]
\dfrac{x_{i+1} - x}{h_{i}} & x_{i} \leq x \leq x_{i+1},\quad i=2,3,\ldots,n \\[10pt]
0                     & \text{otherwise}
\end{cases}
\end{equation}
%
And for right boundary points:
%
\begin{equation}
N_{n+1}(x) = \begin{cases}
\dfrac{x - x_{n}}{h_{n}} & x_{n} \leq x \leq x_{n+1} \\[10pt]
0                     & \text{otherwise}
\end{cases}
\end{equation}
%
where $h_{i}$ are the spacings between nodal points.
%
\begin{equation}
h_{i} = x_{i+1} - x_{i}
\end{equation}

\subsection{Weighted residuals}

When we approximate the solution as ??ref??, in general, we cannot obtain the true
solution to the differential equation. So, we will not get an exact equality but
some \textit{residual} associated with the error in the approximation. This residual
can be defined as:
\begin{equation}
R(T,x) \equiv -k \frac{\mathrm{d}^2 T}{\mathrm{d}x^2} - Q
\end{equation}
where $T$ in the above equation is the approximation to the true solution $T^{*}$
for which we have:
\begin{equation}
R(T^{*},x) \equiv 0
\end{equation}
However, for any $T \neq T^{\mathrm{*}}$, we cannot force the residual
to vanish at every
point $x$, no matter how small we make the grid or long the series expansion.
The idea of the \textit{weighted residuals method} is that we can multiply
the residual by a \textit{weighting function} and force the integral of the weighted
expression to vanish:
\begin{equation}
\int_{0}^{L} w(x) R(T,x) \mathrm{d}x = 0
\end{equation}
where $w(x)$ is the weighting function.
Choosing different weighting functions and replacing each of them in
??ref??, we can then generate a system
of linear equations in the unknown parameters $T_{i}$ that will determine an
approximation T of the form of the finite series given in Equation ??ref??.
This will satisfy the differential equation in an "average" or "integral" sense.
The type of weighting function chosen depends on the type of weighted
residual technique selected. In the \textit{Galerkin procedure}, the weights are set
equal to the shape functions $N_{i}$, that is
\begin{equation}
w_{i}(x) = N_{i}(x)
\end{equation}
So we have:
\begin{equation}
\int_{0}^{L} N_{i} \left[
-k\frac{\mathrm{d}^2 T}{\mathrm{d}x^2} - Q
\right]\,\mathrm{d}x = 0
\end{equation}

Since the temperature distribution must be a continuous function of $x$,
the simplest way to approximate it would be to use piecewise polynomial
interpolation over each element, in particular, piecewise linear approxi-
mation (by using linear shape functions as in ??ref??)
provides the simplest approximation with a continuous function.
Unfortunately, the first derivatives of such functions are not continuous
at the elements' ends and, hence, second derivatives do not exist there;
furthermore, the second derivative of $T$ would vanish inside each element.
However, to require the second-order derivatives to exist everywhere is
too restrictive.
We can weaken this requirement by application of integration by parts to
the second derivative:
\begin{equation}
\int_{0}^{L} N_{i} \left[ -k\frac{\mathrm{d}^2 T}{\mathrm{d}x^2} \right]\,\mathrm{d}x =
\int_{0}^{L} k \frac{\mathrm{d} N_{i}}{\mathrm{d}x}
\frac{\mathrm{d} T}{\mathrm{d}x} \mathrm{d}x -
\left[ k N_{i} \frac{\mathrm{d}T}{\mathrm{d}x} \right]_{0}^{L}
\end{equation}

Substituting the Equation ??ref?? to the integral term at the RHS:
\begin{align}
\int_{0}^{L} k \frac{\mathrm{d} N_{i}}{\mathrm{d}x}
\frac{\mathrm{d}}{\mathrm{d}x}\, \left( \sum_{j=1}^{n+1} T_{j} N_{j} \right) \mathrm{d}x & =
\int_{0}^{L} k \frac{\mathrm{d} N_{i}}{\mathrm{d}x}
\left( \sum_{j=1}^{n+1} \frac{\mathrm{d}N_{j}}{\mathrm{d}x} T_{j} \right)\, \mathrm{d}x \\
& = \int_{0}^{L} k
\left( \sum_{j=1}^{n+1}
\frac{\mathrm{d} N_{i}}{\mathrm{d}x}
\frac{\mathrm{d}N_{j}}{\mathrm{d}x} T_{j} \right)\, \mathrm{d}x
\end{align}

The Equation ??ref?? can be rewritten as
\begin{equation}
\int_{0}^{L} k
\left( \sum_{j=1}^{n+1}
\frac{\mathrm{d} N_{i}}{\mathrm{d}x}
\frac{\mathrm{d}N_{j}}{\mathrm{d}x} T_{j} \right)\, \mathrm{d}x -
\int_{0}^{L} N_{i} Q\, \mathrm{d}x -
\left[ k N_{i} \frac{\mathrm{d}T}{\mathrm{d}x} \right]_{0}^{L} = 0
\end{equation}
for $i=1,2,\ldots,n+1$. The quantities $k$, $Q$, $N_{i}$ are known.
The terms containing $T_{i}$ can be isolated to the LHS and the
the known quantities can be moved to the RHS. This is a system
of linear equations with $T_{i}$ as the unknown variables. In matrix form:
\begin{equation}
\mathbf{K} \mathbf{u} = \mathbf{f}
\end{equation}
where $\mathbf{K}$ is a matrix arising from the integral terms, $\mathbf{f}$ are
column vector whose elements arise from
the integrals of source term and boundary terms,
and $\mathbf{u}$ is column vector of the the unknown $T_{i}$.
These linear equations
can be solved by standard method such as Gaussian elimination and LU decomposition.

In practice, the integral in Equation ??ref?? can be calculated analytically or
by numerical methods such as Gaussian quadrature.
In the present case, the integral can be calculated analytically.
The integral over all spatial
domain can be divided into integrals over elements:
\begin{equation}
\int_{0}^{L} = \int_{x_{1}=0}^{x_{2}} + \int_{x_{2}}^{x_{3}} + \cdots +
\int_{x_{n}}^{x_{n+1}=L}
\end{equation}

For simplicity, let's consider two elements with equal spacing
$h_{1} = h_{2}$ with nodes
\begin{equation}
x_{1} = 0, x_{2} = L/2, x_{3} = L
\end{equation}
The first integral is done over the first element. In the first element, only
$N_{1}$ and $N_{2}$ contributes to the integrals because $N_{3}$ is zero in
this interval. So the index $i$ and $j$ are limited to 1 and 2 only.
This also applies to other elements.

These integrals can be written as a matrix:
\begin{equation}
\mathbf{K}^{e_{1}} = \int_{0}^{L/2} k \begin{bmatrix}
\dfrac{\mathrm{d}N_{1}}{\mathrm{d}x} \dfrac{\mathrm{d}N_{1}}{\mathrm{d}x} &
\dfrac{\mathrm{d}N_{1}}{\mathrm{d}x} \dfrac{\mathrm{d}N_{2}}{\mathrm{d}x} \\[10pt]
\dfrac{\mathrm{d}N_{2}}{\mathrm{d}x} \dfrac{\mathrm{d}N_{1}}{\mathrm{d}x} &
\dfrac{\mathrm{d}N_{2}}{\mathrm{d}x} \dfrac{\mathrm{d}N_{2}}{\mathrm{d}x}
\end{bmatrix}\,\mathrm{d}x
\end{equation}

The needed integrals are:
\begin{align*}
\int_{0}^{L/2} \frac{\mathrm{d}N_{1}}{\mathrm{d}x} \dfrac{\mathrm{d}N_{1}}{\mathrm{d}x}\, \mathrm{d}x & =
\int_{0}^{L/2} \left(-\frac{1}{h_{1}}\right) \left( -\frac{1}{h_{1}} \right)\, \mathrm{d}x \\
& = \int_{0}^{L/2} \frac{1}{(L/2)^2}\, \mathrm{d}x \\
& = \left[ \frac{4}{L^2}x \right]_{0}^{L/2} \\
& = \frac{2}{L}
\end{align*}
and
\begin{align*}
\int_{0}^{L/2} \frac{\mathrm{d}N_{1}}{\mathrm{d}x} \dfrac{\mathrm{d}N_{2}}{\mathrm{d}x}\, \mathrm{d}x & =
\int_{0}^{L/2} \left(-\frac{1}{h_{1}}\right) \left( \frac{1}{h_{1}} \right)\, \mathrm{d}x \\
& = -\frac{2}{L}
\end{align*}
So the matrix elements can be written as
\begin{equation}
\mathbf{K}^{e_{1}} = \frac{k}{L/2} \begin{bmatrix}
1  & -1 \\
-1 & 1
\end{bmatrix}
\end{equation}
%
For more general case where $h_{1} \neq h_{2}$, we obtain:
\begin{equation}
\mathbf{K}^{e_{1}} = \frac{k}{h_{1}} \begin{bmatrix}
1  & -1 \\
-1 & 1
\end{bmatrix}
\end{equation}

Source term for the first element:
\begin{equation}
\mathbf{f}^{e_{1},s} = \begin{bmatrix}
f^{e_{1},s}_{1} \\
f^{e_{1},s}_{2} \\
\end{bmatrix}
\end{equation}

\begin{equation}
f^{e_{1},s}_{1} = Q \int_{0}^{L/2} N_{1}(x)\,\mathrm{d}x =
Q \int_{0}^{L/2} \left( 1 - \dfrac{2x}{L} \right) \,\mathrm{d}x =
Q \left[ x - \frac{1}{L} x^{2} \right]_{0}^{L/2} =
Q \frac{L}{4} = Q \frac{h_1}{2}
\end{equation}

\begin{equation}
f^{e_{1},s}_{2} = Q \int_{0}^{L/2} N_{2}(x)\,\mathrm{d}x =
Q \int_{0}^{L/2} \left( \dfrac{2x}{L} \right) \,\mathrm{d}x =
Q \left[ \frac{1}{L}x^{2} \right]_{0}^{L/2} =
Q \frac{L}{4} = Q \frac{h_1}{2}
\end{equation}

Boundary terms for the first element
\begin{equation}
\mathbf{f}^{e_{1},b} = \begin{bmatrix}
f^{e_{1},b}_{1} \\
f^{e_{1},b}_{2} \\
\end{bmatrix}
\end{equation}

By using boundary condition at $x=0$ and using the fact that $N_{1}(x=0)=1$
and $N_{1}(x=L/2)=0$:
\begin{equation}
f^{e_{1},b}_{1} = \left[ N_{1}(x) \left( k\frac{\mathrm{d}T}{\mathrm{d}x} \right) \right]_{0}^{L/2} =
N_{1}(L/2) \left. \left( k\frac{\mathrm{d}T}{\mathrm{d}x} \right)\right|_{x=L/2} -
N_{1}(0) \left. \left( k\frac{\mathrm{d}T}{\mathrm{d}x} \right)\right|_{x=0} = q
\end{equation}
Meanwhile using using the fact that $N_{2}(x=0)=0$
and $N_{2}(x=L/2)=1$:
\begin{equation}
f^{e_{1},b}_{2} = \left[ N_{2}(x) \left( k\frac{\mathrm{d}T}{\mathrm{d}x} \right) \right]_{0}^{L/2} =
N_{2}(L/2) \left. \left( k\frac{\mathrm{d}T}{\mathrm{d}x} \right)\right|_{x=L/2} -
N_{2}(0) \left. \left( k\frac{\mathrm{d}T}{\mathrm{d}x} \right)\right|_{x=0} =
\left. \left( k\frac{\mathrm{d}T}{\mathrm{d}x} \right)\right|_{x=L/2}
\end{equation}

Local stiffness matrix

The matrix $\mathbf{K}^{e_{2}}$ associated with the second element also has similar
form. The matrix $\mathbf{K}^{e_{2}}$

\begin{figure}[H]
{\centering
\includegraphics[width=\textwidth]{images/Heinrich_Fig_3_7.pdf}
\par}
\end{figure}


\section{Time-dependent problem}

Consider the unsteady heat PDE in one spatial dimension $x$:
\begin{equation}
\frac{\partial T}{\partial t} = k \frac{\partial^2 T}{\partial t^2} + Q
\end{equation}
where $k$ is thermal diffusivity constant and $Q$ is a constant heat source.
%
The initial conditions is
\begin{equation}
T(x,t=0) = 0,\qquad \forall \, x \in [0,L]
\end{equation}
and the boundary conditions are
\begin{equation}
T(x=0,t) = 0 \qquad \text{and} \qquad T(x=L,t) = 0
\end{equation}


\section{Test}

Test cite: \cite{Heinrich2017}

Test code:
\begin{juliacode}
α + β
\end{juliacode}

\bibliographystyle{unsrt}
\bibliography{BIBLIO}

\end{document}

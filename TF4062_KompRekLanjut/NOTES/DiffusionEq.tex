\documentclass[a4paper,12pt]{article}

\usepackage[a4paper]{geometry}
\geometry{verbose,tmargin=1.5cm,bmargin=1.5cm,lmargin=1.5cm,rmargin=1.5cm}

\setlength{\parskip}{\smallskipamount}
\setlength{\parindent}{0pt}

\usepackage{fontspec}
\setmonofont{FreeMono}

\usepackage{hyperref}
\usepackage{url}
\usepackage{xcolor}

\usepackage{amsmath}
\usepackage{amssymb}

\usepackage{graphicx}
\usepackage{float}

\usepackage{minted}
\newminted{julia}{breaklines,fontsize=\small}
\newminted{bash}{breaklines,fontsize=\small}
\newminted{text}{breaklines,fontsize=\small}

\newcommand{\txtinline}[1]{\mintinline{text}{#1}}
\newcommand{\jlinline}[1]{\mintinline{julia}{#1}}

\newmintedfile[juliafile]{julia}{breaklines,fontsize=\small}

\definecolor{mintedbg}{rgb}{0.90,0.90,0.90}
\usepackage{mdframed}

\BeforeBeginEnvironment{minted}{\begin{mdframed}[backgroundcolor=mintedbg]}
\AfterEndEnvironment{minted}{\end{mdframed}}


\usepackage{appendix}


\begin{document}

\title{Diffusion Equation with Finite Difference Method \\
TF4062}
\author{Iwan Prasetyo \\
Fadjar Fathurrahman}
\date{}
\maketitle

\section{Initial-boundary value problem for 1d diffusion}

1d diffusion equation:
\begin{equation}
\frac{\partial u}{\partial t} = \alpha \frac{\partial^2 u}{\partial t^2} + f(x,t)
\end{equation}

Initial condition:
\begin{equation}
u(x,0) = I(x), \qquad x \in [0,L]
\end{equation}

Boundary condition:
\begin{equation}
u(0,t) = 0, \qquad u(L,t) = 0, \qquad t > 0
\end{equation}

Spatial grid:
\begin{equation}
x_{i} = (i-1)\Delta x, \qquad i = 1, \ldots, N_{x}
\end{equation}
Temporal grid:
\begin{equation}
t_{n} = (n-1)\Delta t, \qquad n = 1, \ldots, N_{t}
\end{equation}


\section{Forward Euler scheme}

\begin{equation}
\frac{u_{i}^{n+1} - u_{i}^{n}}{\Delta t} = \alpha \frac{u_{i+1}^{n} -2u_{i}^{n} + u_{i-1}^{n}}{\Delta x^2} + f_{i}^{n}
\label{eq:expl_forward_01}
\end{equation}

By using the following definition of \textit{mesh Fourier number}:
\begin{equation}
F = \alpha \frac{\Delta t}{\Delta x^2}
\end{equation}
we can rearrange the equation \eqref{eq:expl_forward_01} to:
\begin{equation}
u_{i}^{n+1} = u_{i} + F\left( u_{i+1}^{n} - 2u_{i}^{n} + u_{i+1}^{n} \right) +
f_{i}^{n}\Delta t
\label{eq:expl_forward}
\end{equation}
The equation \eqref{eq:expl_forward} can be used 


\section{Backward Euler scheme}

\begin{equation}
\frac{u_{i}^{n} - u_{i}^{n-1}}{\Delta t} = \alpha
\frac{u^{n}_{i+1} - 2u^{n}_{i} + u^{n}_{i-1}}{\Delta x^2} + f_{i}^{n}
\end{equation}

\begin{equation}
-Fu_{i-1}^{n} + (1 + 2F)u_{i}^{n} - F u_{i+1}^{n} = u^{n-1}_{i-1} + f^{n}_{i}
\end{equation}

\begin{equation}
\mathbf{A}\mathbf{u} = \mathbf{b}
\end{equation}


\begin{equation}
\begin{bmatrix}
A_{1,1} & A_{1,2} & 0       & \cdots  & \cdots    & \cdots  & \cdots    & \cdots        & \cdots \\
A_{1,2} & A_{2,2} & A_{2,3} & \cdots  & \cdots    & \cdots  &           &               & \vdots \\
0       & A_{3,2} & A_{3,3} & A_{3,4} & \cdots    & \cdots  &           &               & \vdots \\
\vdots  & \ddots  &         & \ddots  &           &  0      &           &               & \vdots \\
\vdots  &         &         & \ddots  & \ddots    & \ddots  & \ddots    &               & \vdots \\
\vdots  &         &         & 0       & A_{i,j-1} & A_{i,j} & A_{i,j+1} & \ddots        & \vdots \\
\vdots  &         &         &         & \ddots    & \ddots  & \ddots    & \ddots        & 0 \\
\vdots  &         &         &         &           & \ddots  & \ddots    & \ddots        & A_{N_x-1,N_x} \\
0       & \cdots  & \cdots  & \cdots  & \cdots    & \cdots  & \cdots    & A_{N_x,N_x-1} & A_{N_x,N_x} \\
\end{bmatrix}
\end{equation}

\begin{align*}
A_{i,i-1} & = -F \\
A_{i,i}   & = 1 + 2F \\
A_{i,i+1} & = -F
\end{align*}

\begin{align*}
A_{1,1} = 1 \\
A_{1,2} = 0 \\
A_{N_x-1,N_x-1} = 0 \\
A_{N_x,N_x} 1
\end{align*}

\begin{equation}
\mathbf{b} = \begin{bmatrix}
b_{1} \\
b_{2} \\
\vdots \\
b_{i} \\
\vdots \\
b_{N_x}
\end{bmatrix}
\end{equation}

Right-hand side $b_{1} = 0$ and $b_{N_x} = 0$
\begin{equation}
b_{i} = u^{n-1}_{i} + f_{i}^{n-1} \Delta t, \qquad i = 2, \ldots, N_{x}-1
\end{equation}



\section{Crank-Nicolson (CN) method}

In the Crank-Nicolson method we require the PDE to be satisfied at the spatial
mesh point $x_{i}$ but midway between the points in the time mesh ($t_{n+\frac{1}{2}})$:
\begin{equation}
\frac{\partial}{\partial t} u_{i}^{n+\frac{1}{2}} =
\alpha\frac{\partial^2}{\partial x^2} u_{i}^{n+\frac{1}{2}} + 
f_{i}^{n+\frac{1}{2}}
\end{equation}
%
Using centered difference in space and time:
%
\begin{equation}
\frac{u_{i}^{n+1} - u_{i}^{n}}{\Delta t} = 
\frac{1}{\Delta x^2} \left(
u_{i+1}^{n+\frac{1}{2}} - 2u_{i}^{n+\frac{1}{2}} + u_{i-1}^{n+\frac{1}{2}}
\right) + f_{i}^{n+\frac{1}{2}}
\end{equation}
%
Because $u_{i}^{n+\frac{1}{2}}$ is not the quantity that we want to calculate,
we must approximate it.
We can approximate it by an average between the value at $t_n$ and $t_{n+1}$:
\begin{equation}
u_{i}^{n+\frac{1}{2}} \approx \frac{1}{2}( u_{i}^{n} + u_{i}^{n+1} )
\end{equation}
We also can use the same approximation for $f_{i}^{n+\frac{1}{2}}$:
\begin{equation}
f_{i}^{n+\frac{1}{2}} \approx \frac{1}{2}( f_{i}^{n} + f_{i}^{n+1} )
\end{equation}
%
Substituting these approximations we obtain:
%
\begin{equation}
u_{i}^{n+1} - \frac{1}{2} F \left( u_{i-1}^{n+1} - 2u^{n+1}_{i} + u^{n+1}_{i+1} \right) =
u_{i}^{n} + \frac{1}{2} F \left( u_{i-1}^{n} - 2u^{n}_{i} + u^{n}_{i+1} \right) +
\frac{1}{2} f_{i}^{n+1} + \frac{1}{2} f_{i}^{n}
\label{eq:CN_v1}
\end{equation}
%
We notice that the equation \eqref{eq:CN_v1} has similar structure as the one we obtained
for backward Euler method:
%
\begin{equation}
\mathbf{A}\mathbf{u} = \mathbf{b}
\end{equation}
%
The element of the matrix $\mathbf{A}$ are:
%
\begin{align*}
A_{i,i-1} & = -\frac{1}{2}F \\
A_{i,i}   & = 1 + F \\
A_{i,i+1} & = -\frac{1}{2}F
\end{align*}
%
for internal points $i = 2,\ldots,N_{x}-1$. For boundary points we have:
%
\begin{align*}
A_{1,1} & = 1 \\
A_{1,2} & = 0 \\
A_{N_{x},N_{x}-1} & = 0 \\
A_{N_{x},N_{x}} & = 1 \\
\end{align*}
%
For the right-hand side vector $\mathbf{b}$ we have $b_{1} = 0$ and $b_{N_x} = 0$ and
\begin{equation}
b_{i} = u^{n}_{i} + \frac{1}{2}F \left( u^{n}_{i+1} -2u^{n}_{i} + u^{n}_{i-1} \right) +
\frac{1}{2} \left( f_{i}^{n+1} + f_{i}^{n} \right) \Delta t
\end{equation}
for internal points $i = 2, \ldots, N_{x}-1$.



%\appendix

%\appendixpage

%\section{Conjugate gradient for system of linear equations}


\bibliographystyle{unsrt}
\bibliography{BIBLIO}


\end{document}
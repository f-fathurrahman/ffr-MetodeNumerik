\section{Sistem Linear Tridiagonal}

Algoritma Thomas
\begin{equation}
\begin{bmatrix}
b_{1} & c_{1} & 0 & \cdots & \cdots & 0 & 0 \\
a_{2} & b_{2} & c_{2} & 0 & \cdots & 0 & 0 \\
\cdots & \cdots & \cdots & \cdots & \cdots & \cdots & \cdots \\
\cdots & \cdots & \cdots & \cdots & \cdots & \cdots & \cdots \\
0 & 0 & 0 & 0 & a_{n-1} & b_{n-1} & c_{n-1} \\
0 & 0 & 0 & 0 & \cdots & a_{n} & b_{n} \\
\end{bmatrix}
\begin{bmatrix}
X_{1} \\ X_{2} \\ \vdots \\ \vdots \\ X_{n-1} \\ X_{n}
\end{bmatrix} =
\begin{bmatrix}
f_{1} \\ f_{2} \\ \vdots \\ \vdots \\ f_{n-1} \\ f_{n}
\end{bmatrix}
\end{equation}
dengan menggunakan \textit{recurrence relation}:
\begin{align}
X_{k} & = \gamma_{k} - \frac{c_{k}}{\beta_{k}} X_{k+1}, \qquad k = 1,\ldots,(n-1) \\
X_{n} & = \gamma_{n}
\end{align}
Dapat diperoleh:
\begin{align}
\beta_{1} = b_1 \\
\beta_{k} = b_{k} - \frac{c_{k-1}}{\beta_{k-1}}a_{k},\qquad k=2,\ldots,n
\end{align}
dan
\begin{align}
\gamma_{1} = \frac{f_{1}}{\beta_{1}} = \frac{f_{1}}{b_{1}} \\
\gamma_{k} = \frac{f_{k} - a_{k}\gamma_{k-1}}{\beta_{k}}
\end{align}

Persamaan pertama:
\begin{equation}
b_{1}X_{1} + c_{1}X_{2} = f_{1}
\end{equation}
Substitusi: $X_{1} = \gamma_{1} - \dfrac{c_{1}}{\beta_{1}} X_{2}$
\begin{equation}
b_{1}(\gamma_{1} - \dfrac{c_{1}}{\beta_{1}} X_{2}) + c_{1}X_{2} = f_{1}
\end{equation}
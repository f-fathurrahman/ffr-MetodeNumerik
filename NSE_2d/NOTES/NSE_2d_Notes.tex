\documentclass[a4paper,11pt,bahasa]{article} % screen setting
\usepackage[a4paper]{geometry}

%\documentclass[b5paper,11pt,bahasa]{article} % screen setting
%\usepackage[b5paper]{geometry}

\geometry{verbose,tmargin=1.5cm,bmargin=1.5cm,lmargin=1.5cm,rmargin=1.5cm}

\setlength{\parskip}{\smallskipamount}
\setlength{\parindent}{0pt}

%\usepackage{cmbright}
%\renewcommand{\familydefault}{\sfdefault}

\usepackage[libertine]{newtxmath}
\usepackage[no-math]{fontspec}

\setmainfont{Linux Libertine O}

\setmonofont{JuliaMono-Regular}


\usepackage{hyperref}
\usepackage{url}
\usepackage{xcolor}

\usepackage{amsmath}
\usepackage{amssymb}

\usepackage{graphicx}
\usepackage{float}

\usepackage{minted}

\newminted{julia}{breaklines,fontsize=\scriptsize}
\newminted{python}{breaklines,fontsize=\scriptsize}

\newminted{bash}{breaklines,fontsize=\scriptsize}
\newminted{text}{breaklines,fontsize=\scriptsize}

\newcommand{\txtinline}[1]{\mintinline[breaklines,fontsize=\scriptsize]{text}{#1}}
\newcommand{\jlinline}[1]{\mintinline[breaklines,fontsize=\scriptsize]{julia}{#1}}
\newcommand{\pyinline}[1]{\mintinline[breaklines,fontsize=\scriptsize]{python}{#1}}

\newmintedfile[juliafile]{julia}{breaklines,fontsize=\scriptsize}
\newmintedfile[pythonfile]{python}{breaklines,fontsize=\scriptsize}
% f-o-otnotesize

\usepackage{mdframed}
\usepackage{setspace}
\onehalfspacing

\usepackage{appendix}

\newcommand{\highlighteq}[1]{\colorbox{blue!25}{$\displaystyle#1$}}
\newcommand{\highlight}[1]{\colorbox{red!25}{#1}}

\newcounter{soal}%[section]
\newenvironment{soal}[1][]{\refstepcounter{soal}\par\medskip
   \noindent \textbf{Soal~\thesoal. #1} \sffamily}{\medskip}


\definecolor{mintedbg}{rgb}{0.95,0.95,0.95}
\BeforeBeginEnvironment{minted}{
    \begin{mdframed}[backgroundcolor=mintedbg,%
        topline=false,bottomline=false,%
        leftline=false,rightline=false]
}
\AfterEndEnvironment{minted}{\end{mdframed}}


\BeforeBeginEnvironment{soal}{
    \begin{mdframed}[%
        topline=true,bottomline=true,%
        leftline=true,rightline=true]
}
\AfterEndEnvironment{soal}{\end{mdframed}}

% -------------------------
\begin{document}

\title{%
{\small TF2202 Komputasi Rekayasa}\\
Solusi Numerik Persamaan Navier-Stokes 2d
}
\author{Tim Praktikum Komputasi Rekayasa 2021\\
Teknik Fisika\\
Institut Teknologi Bandung}
\date{}
\maketitle

Aliran fluida 2d, vektor kecepatan
$q = ( u(x,y), v(x,y) ) \in \mathbb{R}^2$ dan tekanan $p(x,y) \in \mathbb{R}$.

Konvervasi massa:
\begin{equation}
\mathrm{div}(q) = 0
\end{equation}
atau:
\begin{equation}
\frac{\partial u}{\partial x} + \frac{\partial v}{\partial y} = 0
\end{equation}

Konvervasi momentum
\begin{equation}
\frac{\partial q}{\partial t} + \mathrm{div}(q \otimes q) =
-\mathcal{G}p + \frac{1}{\mathrm{Re}}\nabla q
\end{equation}
atau dalam bentuk eksplisit:
\begin{align}
\frac{\partial u}{\partial t} + \frac{\partial u^2}{\partial x} +
\frac{\partial uv}{\partial y} = -\frac{\partial p}{\partial x} +
\frac{1}{\mathrm{Re}}\left( \frac{\partial^2 u}{\partial x^2} +
\frac{\partial^2 u}{\partial y^2}
\right) \\
\frac{\partial v}{\partial t} + \frac{\partial uv}{\partial x} +
\frac{\partial v^2}{\partial y} = -\frac{\partial p}{\partial y} +
\frac{1}{\mathrm{Re}}\left( \frac{\partial^2 v}{\partial x^2} +
\frac{\partial^2 v}{\partial y^2}
\right)
\end{align}
dengan bilangan Reynolds
\begin{equation}
\mathrm{Re} = \frac{V_{0} L}{\nu}
\end{equation}

Menggunakan metode \textit{fractional-step}:
\begin{itemize}
\item Predictor step, menyelesaikan persamaan momentum yang ditulis dalam bentuk
\begin{equation}
\frac{\partial q}{\partial t} = -\mathcal{G}p + \mathcal{H} + \frac{1}{\mathrm{Re}}\nabla q
\end{equation}
\item s
\end{itemize}

\section{Sistem Linear Tridiagonal}

Algoritma Thomas
\begin{equation}
\begin{bmatrix}
b_{1} & c_{1} & 0 & \cdots & \cdots & 0 & 0 \\
a_{2} & b_{2} & c_{2} & 0 & \cdots & 0 & 0 \\
\cdots & \cdots & \cdots & \cdots & \cdots & \cdots & \cdots \\
\cdots & \cdots & \cdots & \cdots & \cdots & \cdots & \cdots \\
0 & 0 & 0 & 0 & a_{n-1} & b_{n-1} & c_{n-1} \\
0 & 0 & 0 & 0 & \cdots & a_{n} & b_{n} \\
\end{bmatrix}
\begin{bmatrix}
X_{1} \\ X_{2} \\ \vdots \\ \vdots \\ X_{n-1} \\ X_{n}
\end{bmatrix} =
\begin{bmatrix}
f_{1} \\ f_{2} \\ \vdots \\ \vdots \\ f_{n-1} \\ f_{n}
\end{bmatrix}
\end{equation}
dengan menggunakan \textit{recurrence relation}:
\begin{align}
X_{k} & = \gamma_{k} - \frac{c_{k}}{\beta_{k}} X_{k+1}, \qquad k = 1,\ldots,(n-1) \\
X_{n} & = \gamma_{n}
\end{align}
Dapat diperoleh:
\begin{align}
\beta_{1} = b_1 \\
\beta_{k} = b_{k} - \frac{c_{k-1}}{\beta_{k-1}}a_{k},\qquad k=2,\ldots,n
\end{align}
dan
\begin{align}
\gamma_{1} = \frac{f_{1}}{\beta_{1}} = \frac{f_{1}}{b_{1}} \\
\gamma_{k} = \frac{f_{k} - a_{k}\gamma_{k-1}}{\beta_{k}}
\end{align}

Persamaan pertama:
\begin{equation}
b_{1}X_{1} + c_{1}X_{2} = f_{1}
\end{equation}
Substitusi: $X_{1} = \gamma_{1} - \dfrac{c_{1}}{\beta_{1}} X_{2}$
\begin{equation}
b_{1}(\gamma_{1} - \dfrac{c_{1}}{\beta_{1}} X_{2}) + c_{1}X_{2} = f_{1}
\end{equation}

\end{document}

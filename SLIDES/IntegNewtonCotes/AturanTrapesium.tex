\section{Aturan trapesium}

\begin{frame}
\frametitle{Aturan trapesium}

Aturan trapesium adalah formula pertama Newton-Cotes di mana polinomial yang digunakan
adalah orde 1, dan hasil dari integrasinya adalah sebagai berikut:
\begin{equation*}
I = (b - a) \frac{f(a) + f(b)}{2}
\end{equation*}
dengan estimasi error pemotongan lokal untuk aturan trapesium adalah sebagai berikut:
\begin{equation*}
E_{t} = -\frac{1}{12} f''(\xi) (b - a)^3
\end{equation*}
di mana $\xi$ berada di dalam interval $[a,b]$. Nilai $f''(\xi)$ dapat diganti dengan nilai
rata-rata $f''(x)$ pada interval $[a,b]$ sebagai aproksimasi.

\end{frame}

\begin{frame}
\frametitle{Ilustrasi aturan trapesium}

\begin{equation*}
I = (b - a) \frac{f(a) + f(b)}{2}
\end{equation*}
    

\begin{columns}

\begin{column}{0.5\textwidth}
{\centering
\includegraphics[width=\textwidth]{images_priv/Chapra_Fig_21_4.pdf}
\par}    
\end{column}

\begin{column}{0.5\textwidth}
{\centering
\includegraphics[width=\textwidth]{images_priv/Chapra_Fig_21_5.pdf}
\par}    
\end{column}

\end{columns}

\end{frame}


\begin{frame}
\frametitle{Contoh}

Gunakan aturan trapesium untuk aproksimasi integral berikut.
\begin{equation*}
f(x) = 0.2 + 25x - 200x^2 + 675x^3 - 900x^4 + 400x^5
\end{equation*}
dari $a = 0$ sampai $b = 0.8$. Bandingkan hasilnya dengan hasil analitik,
yaitu 1.640533. Hitung juga estimasi error dari aproksimasi yang digunakan.

\end{frame}


\begin{frame}[fragile]
\frametitle{Kode Python}

\begin{pythoncode}
def my_func(x):
    return 0.2 + 25*x - 200*x**2 + 675*x**3 - 900*x**4 + 400*x**5
        
a = 0.0
b = 0.8
f_a = my_func(a)
f_b = my_func(b)
        
I_exact = 1.640533 # from the book
# Alternatively, you can use SymPy to do the integration
        
# Use trapezoid rule here
I = .... # Lengkapi
        
E_t = I_exact - I
ε_t = E_t/I_exact * 100 # in percent
print("Integral result = %.6f" % I)
print("True integral   = %.6f" % I_exact)
print("True error      = %.6f" % E_t)
print("ε_t             = %.1f%%" % ε_t)
\end{pythoncode}

\end{frame}


\begin{frame}[fragile]
\frametitle{Kode Python}

\begin{equation*}
E_{t} = -\frac{1}{12} f''(\xi) (b - a)^3
\end{equation*}

Dengan mengganti $f''(\xi)$ dengan nilai rata-rata:
\begin{pythoncode}
import sympy
x = sympy.symbols("x")
f = 0.2 + 25*x - 200*x**2 + 675*x**3 - 900*x**4 + 400*x**5
d2f = f.diff(x,2) # calculate 2nd derivative
avg_d2f_xi = sympy.integrate( d2f, (x,a,b) )/(b - a)
E_a = -1/12*avg_d2f_xi*(b - a)**3
print("Approx error    = %.6f" % E_a)
\end{pythoncode}

\end{frame}


\begin{frame}[fragile]
\frametitle{Contoh keluaran}

\begin{columns}

\begin{column}{0.5\textwidth}
Contoh hasil keluaran:
\begin{textcode}
Integral result = 0.172800
True integral   = 1.640533
True error      = 1.467733
ε_t             = 89.5%
Approx error    = 2.560000
\end{textcode}    
\end{column}

\begin{column}{0.5\textwidth}
{\centering
\includegraphics[width=\textwidth]{images_priv/Chapra_Fig_21_6.pdf}
\par}
\end{column}

\end{columns}

Error yang dihasilkan masih cukup besar.

\end{frame}


\begin{frame} 
\frametitle{Aturan trapesium multi-interval}
\fontsize{9pt}{8.0}\selectfont

Aturan trapesium (dan aturan integrasi Newton-Cotes yang lain)
juga dapat digunakan untuk beberapa interval. Misalkan interval
awal $[a,b]$ dibagi menjadi $n$ segmen dengan panjang yang sama sehingga terdapat
total $n+1$ titik: $(x_0, x_1, \ldots, x_n)$. Panjang dari segmen atau subinterval
adalah
\begin{equation*}
h = \frac{b - a}{n}
\end{equation*}
dengan $x_0 = a$ dan $x_n = b$. Integral dari $a$ ke $b$ dapat diurai menjadi
integral pada masing-masing segmen:
\begin{equation*}
I = \int_{a}^{b} = \int_{x_0}^{x_1} + \int_{x_1}^{x_2} + \cdots + \int_{x_{n-1}}^{x_n}
\end{equation*}
Dengan menggunakan aturan trapesium untuk masing-masing selang, diperoleh:
\begin{equation*}
I = \frac{h}{2}\left[
    f(x_{0}) + 2\sum_{i=1}^{n-1} f(x_{i}) + f(x_{n})
\right]
\end{equation*}

\end{frame}


\begin{frame}
\frametitle{Aturan trapesium multi-interval}
\fontsize{9pt}{8.0}\selectfont

Dalam bentuk yang lain dapat dituliskan menjadi:
\begin{equation*}
I = (b - a) \frac{f(x_{0}) + 2\sum_{i=1}^{n-1} + f(x_{n})}{2n}
\end{equation*}

Estimasi error adalah sebagai berikut.
\begin{equation*}
E_{t} = -\frac{(b - a)^3}{12n^3} \sum_{i=1}^{n} f''(\xi_{i})
\end{equation*}
atau dengan menggunakan aproksimasi nilai rata-rata:
\begin{equation*}
\overline{f}'' \approx \frac{\sum_{i=1}^{n} f''(\xi_{i})}{n}
\end{equation*}
diperoleh:
\begin{equation*}
E_{a} = -\frac{(b - a)^3}{12n^2} \overline{f}''
\end{equation*}

\end{frame}


\begin{frame}[fragile]
\frametitle{Kode Python}

\begin{equation*}
I = \frac{h}{2}\left[
    f(x_{0}) + 2\sum_{i=1}^{n-1} f(x_{i}) + f(x_{n})
\right]
\end{equation*}

\begin{pythoncode}
# N is number of segments
# Npoints is N + 1
# f is function
def integ_trapz_multiple( f, a, b, N ):
    x0 = a
    xN = b
    h = (b - a)/N
    ss = 0.0
    for i in range(1,N):
        xi = x0 + i*h
        ss = ss + f(xi)
    I = .... # LENGKAPI
    return I
\end{pythoncode}

\end{frame}


\begin{frame}[fragile]
\frametitle{Kode Python}

Contoh pemanggilan fungsi:
\begin{pythoncode}
Nsegments = 2
a = 0.0
b = 0.8
I_exact = 1.640533

I = integ_trapz_multiple( my_func, a, b, Nsegments )

print("Nsegments = ", Nsegments)
print("Integral result = %.6f" % I)
E_t = I_exact - I
print("True integral   = %.6f" % I_exact)
print("True error      = %.6f" % E_t)
ε_t = E_t/I_exact * 100
print("ε_t             = %.1f%%" % ε_t)
\end{pythoncode}
\end{frame}


\begin{frame}[fragile]
\frametitle{Kode Python}

Perhitungan estimasi error:

\begin{equation*}
E_{a} = -\frac{(b - a)^3}{12n^2} \overline{f}''
\end{equation*}

\begin{pythoncode}
import sympy
x = sympy.symbols("x")
f = 0.2 + 25*x - 200*x**2 + 675*x**3 - 900*x**4 + 400*x**5
d2f = f.diff(x,2)
avg_d2f_xi = sympy.integrate( d2f, (x,a,b) )/(b - a)
E_a = -1/12*avg_d2f_xi*(b - a)**3/Nsegments**2
print("Approx error    = %.6f" % E_a)
\end{pythoncode}

\end{frame}


\begin{frame}[fragile]
\frametitle{Contoh keluaran}

\begin{columns}

\begin{column}{0.5\textwidth}
\begin{textcode}
Nsegments =  2
Integral result = 1.068800
True integral   = 1.640533
True error      = 0.571733
ε_t             = 34.9%
Approx error    = 0.640000
\end{textcode}

\begin{textcode}
Nsegments =  3
Integral result = 1.369574
True integral   = 1.640533
True error      = 0.270959
ε_t             = 16.5%
Approx error    = 0.284444
\end{textcode}    
\end{column}

\begin{column}{0.5\textwidth}

\begin{textcode}
Nsegments =  4
Integral result = 1.484800
True integral   = 1.640533
True error      = 0.155733
ε_t             = 9.5%
Approx error    = 0.160000
\end{textcode}

\begin{textcode}
Nsegments =  10
Integral result = 1.615043
True integral   = 1.640533
True error      = 0.025490
ε_t             = 1.6%
Approx error    = 0.025600    
\end{textcode}
\end{column}

\end{columns}


\end{frame}
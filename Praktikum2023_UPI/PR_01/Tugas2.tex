\documentclass[a4paper,11pt,bahasa]{extarticle} % screen setting
\usepackage[a4paper]{geometry}

%\documentclass[b5paper,11pt,bahasa]{article} % screen setting
%\usepackage[b5paper]{geometry}

%\geometry{verbose,tmargin=1.5cm,bmargin=1.5cm,lmargin=1.5cm,rmargin=1.5cm}

\geometry{verbose,tmargin=2.0cm,bmargin=2.0cm,lmargin=2.0cm,rmargin=2.0cm}

\setlength{\parskip}{\smallskipamount}
\setlength{\parindent}{0pt}

%\usepackage{cmbright}
%\renewcommand{\familydefault}{\sfdefault}

\usepackage{amsmath}
\usepackage{amssymb}

\usepackage[libertine]{newtxmath}

\usepackage[no-math]{fontspec}
\setmainfont{Linux Libertine O}

%\usepackage{fontspec}
%\usepackage{lmodern}

\setmonofont{JuliaMono-Regular}


\usepackage{hyperref}
\usepackage{url}
\usepackage{xcolor}
\usepackage{enumitem}
\usepackage{mhchem}
\usepackage{graphicx}
\usepackage{float}

\usepackage{minted}

\newminted{julia}{breaklines,fontsize=\footnotesize}
\newminted{python}{breaklines,fontsize=\footnotesize}

\newminted{bash}{breaklines,fontsize=\footnotesize}
\newminted{text}{breaklines,fontsize=\footnotesize}

\newcommand{\txtinline}[1]{\mintinline[breaklines,fontsize=\footnotesize]{text}{#1}}
\newcommand{\jlinline}[1]{\mintinline[breaklines,fontsize=\footnotesize]{julia}{#1}}
\newcommand{\pyinline}[1]{\mintinline[breaklines,fontsize=\footnotesize]{python}{#1}}

\newmintedfile[juliafile]{julia}{breaklines,fontsize=\footnotesize}
\newmintedfile[pythonfile]{python}{breaklines,fontsize=\footnotesize}
\newmintedfile[fortranfile]{fortran}{breaklines,fontsize=\footnotesize}

\usepackage{mdframed}
\usepackage{setspace}
\onehalfspacing

\usepackage{babel}
\usepackage{appendix}

\newcommand{\highlighteq}[1]{\colorbox{blue!25}{$\displaystyle#1$}}
\newcommand{\highlight}[1]{\colorbox{red!25}{#1}}

\newcounter{soal}%[section]
\newenvironment{soal}[1][]{\refstepcounter{soal}\par\medskip
   \noindent \textbf{Soal~\thesoal. #1} \sffamily}{\medskip}


\definecolor{mintedbg}{rgb}{0.95,0.95,0.95}
\BeforeBeginEnvironment{minted}{
    \begin{mdframed}[%
        topline=false,bottomline=false,%
        leftline=false,rightline=false]
}
\AfterEndEnvironment{minted}{\end{mdframed}}


\BeforeBeginEnvironment{soal}{
    \begin{mdframed}[%
        topline=true,bottomline=false,%
        leftline=true,rightline=false]
}
\AfterEndEnvironment{soal}{\end{mdframed}}


% -------------------------
\begin{document}

\title{Tugas 2}
\author{ER344 Komputasi dan Analisis Numerik}
\date{}
\maketitle

Kerjakan dengan menggunakan Python (dalam Jupyter Notebook).
Lengkapi setiap file \txtinline{ipynb}
yang Anda gunakan dengan nama dan NIM.
File yang tidak ada identitas tidak akan dinilai.


Jawaban dan file-file terkait dikumpulkan dalam satu folder dan
dikompres dengan ekstensi .zip dengan
format nama \txtinline{KelompokXX.zip} , file dengan ekstensi lain tidak
akan diterima.


\begin{soal}
Jelaskan mengenai kesalahan pemotongan (\textit{truncation error})
dan kesalahan pembulatan (\textit{round-off error}) pada perhitungan numerik
dan bagaimana cara mengatasi atau mengurangi masing-masing jenis-jenis
kesalahan tersebut.
\end{soal}


\begin{soal}
Bandingkan dua persamaan berikut:
\begin{align*}
x_{1,2} & = \frac{-b \pm \sqrt{b^2 - 4ac}}{2a} \\
x_{1,2} & = \frac{-2c}{b \pm \sqrt{b^2 - 4ac}}
\end{align*}
untuk mencari akar-akar pada persamaan kuadrat:
\begin{equation*}
x^2 - 6000.001x + 10
\end{equation*}
Lakukan dalam \textit{single precision} dan \textit{double precision}.
\end{soal}


% Chapra Latihan 4.4
\begin{soal}
Ekspansi deret Maclaurin untuk $\tan^{-1}(x)$ untuk $|x| \leq 1$ diberikan sebagai
\begin{equation*}
\tan^{-1}(x) = \sum_{n=0}^{\infty} \frac{(-1)^{n}}{2n + 1} x^{2n + 1}
\end{equation*}
Buatlah program Python untuk menghitung aproksimasi dari $\tan^{-1}(\pi/6)$ menggunakan deret
Maclaurin tersebut. Hitung juga kesalahan untuk tiap penambahan suku baru dengan referensi
nilai yang diberikan oleh fungsi \textit{built-in} pada modul \txtinline{math} atau
\txtinline{numpy} atau pada Microsoft Excel.
\end{soal}

\begin{soal}[Fungsi Bessel bola]
Fungsi Bessel bola (\textit{spherical Bessel}) $j_{n}(x)$ dapat dituliskan sebagai
berikut:
\begin{equation}
j_{n}(x) = (-x)^{n}
\left( \frac{1}{x} \frac{\mathrm{d}}{\mathrm{d}x} \right)^n
\frac{\sin(x)}{x}
\label{eq:bessel_sferis}
\end{equation}
Buatlah plot untuk $j_{2}(x)$ dan carilah semua akar-akarnya pada interval (0,20).
Gunakan salah satu metode untuk mencari akar persamaan nonlinear. Jelaskan metode
yang Anda gunakan untuk mendapatkan seluruh akar tersebut.
(Catatan: Anda boleh menggunakan bentuk eksplisit dari $j_2(x)$ dari referensi
tanpa menggunakan Persamaan \ref{eq:bessel_sferis})
\end{soal}


\begin{soal}
Tentukan akar dari sistem persamaan nonlinear berikut
\begin{align*}
(x - 4)^2 + (y - 4)^2 & = 5 \\
x^2 + y^2 & = 16
\end{align*}
Buatlah plot dari fungsi-fungsi tersebut dan gunakan untuk mendapatkan estimasi tebakan
awal untuk metode Newton-Raphson.
\end{soal}






\end{document}

\input{PREAMBLE}

% -------------------------
\begin{document}

\title{Tugas 2}
\author{ER344 Komputasi dan Analisis Numerik}
\date{}
\maketitle

Kerjakan dengan menggunakan Python (dalam Jupyter Notebook).
Lengkapi setiap file \txtinline{ipynb}
yang Anda gunakan dengan nama dan NIM.
File yang tidak ada identitas tidak akan dinilai.


Jawaban dan file-file terkait dikumpulkan dalam satu folder dan
dikompres dengan ekstensi .zip dengan
format nama \txtinline{KelompokXX.zip} , file dengan ekstensi lain tidak
akan diterima.


\begin{soal}
Jelaskan mengenai kesalahan pemotongan (\textit{truncation error})
dan kesalahan pembulatan (\textit{round-off error}) pada perhitungan numerik
dan bagaimana cara mengatasi atau mengurangi masing-masing jenis-jenis
kesalahan tersebut.
\end{soal}


\begin{soal}
Bandingkan dua persamaan berikut:
\begin{align*}
x_{1,2} & = \frac{-b \pm \sqrt{b^2 - 4ac}}{2a} \\
x_{1,2} & = \frac{-2c}{b \pm \sqrt{b^2 - 4ac}}
\end{align*}
untuk mencari akar-akar pada persamaan kuadrat:
\begin{equation*}
x^2 - 6000.001x + 10
\end{equation*}
Lakukan dalam \textit{single precision} dan \textit{double precision}.
\end{soal}


% Chapra Latihan 4.4
\begin{soal}
Ekspansi deret Maclaurin untuk $\tan^{-1}(x)$ untuk $|x| \leq 1$ diberikan sebagai
\begin{equation*}
\tan^{-1}(x) = \sum_{n=0}^{\infty} \frac{(-1)^{n}}{2n + 1} x^{2n + 1}
\end{equation*}
Buatlah program Python untuk menghitung aproksimasi dari $\tan^{-1}(\pi/6)$ menggunakan deret
Maclaurin tersebut. Hitung juga kesalahan untuk tiap penambahan suku baru dengan referensi
nilai yang diberikan oleh fungsi \textit{built-in} pada modul \txtinline{math} atau
\txtinline{numpy} atau pada Microsoft Excel.
\end{soal}

\begin{soal}[Fungsi Bessel bola]
Fungsi Bessel bola (\textit{spherical Bessel}) $j_{n}(x)$ dapat dituliskan sebagai
berikut:
\begin{equation}
j_{n}(x) = (-x)^{n}
\left( \frac{1}{x} \frac{\mathrm{d}}{\mathrm{d}x} \right)^n
\frac{\sin(x)}{x}
\label{eq:bessel_sferis}
\end{equation}
Buatlah plot untuk $j_{2}(x)$ dan carilah semua akar-akarnya pada interval (0,20).
Gunakan salah satu metode untuk mencari akar persamaan nonlinear. Jelaskan metode
yang Anda gunakan untuk mendapatkan seluruh akar tersebut.
(Catatan: Anda boleh menggunakan bentuk eksplisit dari $j_2(x)$ dari referensi
tanpa menggunakan Persamaan \ref{eq:bessel_sferis})
\end{soal}


\begin{soal}
Tentukan akar dari sistem persamaan nonlinear berikut
\begin{align*}
(x - 4)^2 + (y - 4)^2 & = 5 \\
x^2 + y^2 & = 16
\end{align*}
Buatlah plot dari fungsi-fungsi tersebut dan gunakan untuk mendapatkan estimasi tebakan
awal untuk metode Newton-Raphson.
\end{soal}






\end{document}

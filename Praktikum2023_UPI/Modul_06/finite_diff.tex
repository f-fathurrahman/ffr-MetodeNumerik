\subsection{Metode Beda Hingga}

\begin{soal}
Gunakan metode beda hingga untuk menyelesaikan persoalan nilai
batas pada Chapra Contoh 27.1
\end{soal}

\begin{pythoncode}
# Finite-difference method for linear BVP

# d2T/dx2 + h'(T_a - T) = 0
# Boundary condition:
#   T(0) = 40
#   T(10) = 200

import numpy as np

h = 0.01
T_a = 20.0

x0 = 0.0
T0 = 40.0 # Boundary condition, T(0) = 40
xf = 10.0
Tf = 200.0 # Boundary condition, T(10) = 200
Δx = 2.0 # segment length (or step size in shooting method)
Nstep = int( (xf-x0)/Δx )
Npoints = Nstep + 1

T = np.zeros(Npoints)
T[0] = T0
T[-1] = Tf

# Finite-difference operator of second derivative matrix
# In general we should use sparse matrix. However, because
# The size is rather small, we use full (dense) matrix
# Please refer to the left-hand-side of Eq. 27.3 for the matrix elements.
Npointsm2 = Npoints-2 # Number of interior points
d2dx2 =  np.zeros((Npointsm2,Npointsm2))
for i in range(Npointsm2):
    d2dx2[i,i] = 2 + h*Δx**2
    if i != 0:
        d2dx2[i-1,i] = -1.0
    if i != (Npointsm2-1):
        d2dx2[i+1,i] = -1.0
# Display the matrix
print("FD representation of second-derivative operator:")
print(d2dx2)

# The vector represented by the right hand side of Eq. 27.3
f = np.zeros(Npointsm2)
for i in range(1,Npointsm2-1):
    f[i] = h*Δx**2*T_a
# From the left BC
f[0] = h*Δx**2*T_a + T0
# From the right BC 
f[-1] = h*Δx**2*T_a + Tf
# Display
print("f = ", f)

# Solve the linear equations
T[1:Npoints-1] = np.linalg.solve(d2dx2,f)

def exact_sol(x):
    return 20*((1 - np.exp(2))*np.exp(x/10) + (1 - 9*np.e)*np.exp(x/5) + \
            np.e*(9 - np.e))*np.exp(-x/10)/(1 - np.exp(2))

x = np.zeros(Npoints)
for i in range(Npoints):
    x[i] = x0 + i*Δx
    T_exact = exact_sol(x[i])
    error = abs(T[i] - T_exact)
    print("%18.10f %18.10f %18.10f %18.10e" % (x[i], T[i], T_exact, error))

plt.clf()
plt.plot(x, T, marker="o", label="Temperature")
plt.xlabel("x")
plt.ylabel("T")
plt.legend()
\end{pythoncode}

Contoh keluaran
\begin{textcode}
FD representation of second-derivative operator:
[[ 2.04 -1.    0.    0.  ]
 [-1.    2.04 -1.    0.  ]
 [ 0.   -1.    2.04 -1.  ]
 [ 0.    0.   -1.    2.04]]
f =  [ 40.8   0.8   0.8 200.8]
      0.0000000000      40.0000000000      40.0000000000   0.0000000000e+00
      2.0000000000      65.9698343668      65.9517913981   1.8042968650e-02
      4.0000000000      93.7784621082      93.7477895327   3.0672575481e-02
      6.0000000000     124.5382283340     124.5035454074   3.4682926640e-02
      8.0000000000     159.4795236931     159.4533954960   2.6128197126e-02
     10.0000000000     200.0000000000     200.0000000000   0.0000000000e+00
\end{textcode}



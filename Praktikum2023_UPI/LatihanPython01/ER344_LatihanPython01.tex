\documentclass[a4paper,11pt,bahasa]{extarticle} % screen setting
\usepackage[a4paper]{geometry}

%\documentclass[b5paper,11pt,bahasa]{article} % screen setting
%\usepackage[b5paper]{geometry}

%\geometry{verbose,tmargin=1.5cm,bmargin=1.5cm,lmargin=1.5cm,rmargin=1.5cm}

\geometry{verbose,tmargin=2.0cm,bmargin=2.0cm,lmargin=2.0cm,rmargin=2.0cm}

\setlength{\parskip}{\smallskipamount}
\setlength{\parindent}{0pt}

%\usepackage{cmbright}
%\renewcommand{\familydefault}{\sfdefault}

\usepackage{amsmath}
\usepackage{amssymb}

\usepackage[libertine]{newtxmath}

\usepackage[no-math]{fontspec}
\setmainfont{Linux Libertine O}

%\usepackage{fontspec}
%\usepackage{lmodern}

\setmonofont{JuliaMono-Regular}


\usepackage{hyperref}
\usepackage{url}
\usepackage{xcolor}
\usepackage{enumitem}
\usepackage{mhchem}
\usepackage{graphicx}
\usepackage{float}

\usepackage{minted}

\newminted{julia}{breaklines,fontsize=\footnotesize}
\newminted{python}{breaklines,fontsize=\footnotesize}

\newminted{bash}{breaklines,fontsize=\footnotesize}
\newminted{text}{breaklines,fontsize=\footnotesize}

\newcommand{\txtinline}[1]{\mintinline[breaklines,fontsize=\footnotesize]{text}{#1}}
\newcommand{\jlinline}[1]{\mintinline[breaklines,fontsize=\footnotesize]{julia}{#1}}
\newcommand{\pyinline}[1]{\mintinline[breaklines,fontsize=\footnotesize]{python}{#1}}

\newmintedfile[juliafile]{julia}{breaklines,fontsize=\footnotesize}
\newmintedfile[pythonfile]{python}{breaklines,fontsize=\footnotesize}
\newmintedfile[fortranfile]{fortran}{breaklines,fontsize=\footnotesize}

\usepackage{mdframed}
\usepackage{setspace}
\onehalfspacing

\usepackage{babel}
\usepackage{appendix}

\newcommand{\highlighteq}[1]{\colorbox{blue!25}{$\displaystyle#1$}}
\newcommand{\highlight}[1]{\colorbox{red!25}{#1}}

\newcounter{soal}%[section]
\newenvironment{soal}[1][]{\refstepcounter{soal}\par\medskip
   \noindent \textbf{Soal~\thesoal. #1} \sffamily}{\medskip}


\definecolor{mintedbg}{rgb}{0.95,0.95,0.95}
\BeforeBeginEnvironment{minted}{
    \begin{mdframed}[%
        topline=false,bottomline=false,%
        leftline=false,rightline=false]
}
\AfterEndEnvironment{minted}{\end{mdframed}}


\BeforeBeginEnvironment{soal}{
    \begin{mdframed}[%
        topline=true,bottomline=false,%
        leftline=true,rightline=false]
}
\AfterEndEnvironment{soal}{\end{mdframed}}


% -------------------------
\begin{document}

\title{%
{\small ER344 Komputasi dan Analisis Numerik}\\
Latihan Python
}
\author{Mariya Al Qibtiya Nasution}
\date{}
\maketitle

\section{Soal 1}
Buat program Python sederhana untuk menentukan akar-akar dari polinomial
kuadrat:
\begin{equation*}
ax^2 + bx + c = 0
\end{equation*}
Input dari koefisien-koefisien $a$, $b$, dan $c$ diperoleh dari pengguna.
Pada bagian awal jalannya program, nama-nama dari anggota kelompok harus
ditampilkan.


\section{Soal 2}
Buatlah suatu plot (menggunakan Python dan Matplotlib)
dari fungsi sinus dan cosinus (dalam satu plot).
Anda dapat menetapkan sendiri parameter-parameter terkait seperti
periode, rentang nilai pada sumbu-$x$ dan sumbu-$y$, warna garis yang
digunakan, serta parameter lain yang mungkin diperlukan.
Lengkapi plot Anda dengan label para sumbu-$x$ dan sumbu-$y$, legenda,
dan judul yang menampilkan nama-nama dari anggota kelompok.
Simpan plot yang dihasilkan dalam format JPG, PNG dan PDF.

Anda dapat menggunakan referensi yang banyak ditemukan pada web, misalnya:

{\footnotesize
\url{https://matplotlib.org/stable/gallery/lines_bars_and_markers/simple_plot.html}}

\section{Soal 3}
Buat plot permukaan (\textit{surface}) dan kontur (\textit{contour})
dari fungsi dua variabel berikut:
\begin{equation*}
f(x,y) = A_{1} \exp\left[ -\frac{(x-1)^2 + (y-3)^2}{\sigma_{1}^{2}} \right]
A_{2} \exp\left[ -\frac{(x+1)^2 + (y+2)^2}{\sigma_{2}^{2}} \right]
\end{equation*}
Anda dapat memilih nilai dari parameter-parameter $A_{1}$, $\sigma_{1}$,
$A_{2}$, dan $\sigma_{2}$.
Lengkapi plot-plot Anda dengan keterangan nama dari anggota kelompok.

Beberapa referensi yang mungkin dapat Anda gunakan:

{\footnotesize
  \url{https://matplotlib.org/stable/gallery/mplot3d/surface3d.html}
}

{\footnotesize
  \url{https://matplotlib.org/stable/gallery/images_contours_and_fields/contour_demo.html}
}

\section{Soal 4}
Tinjau salah satu dari representasi fungsi periodik yang diberikan dalah deret Fourier, misalnya
yang ditampilkan pada laman:
{\footnotesize
\url{https://en.wikipedia.org/wiki/Fourier_series#Table_of_common_Fourier_series}}
Seperti yang telah dipelajari dalam Matematika Teknik II, deret Fourier melibatkan
suku ekspansi yang tak hingga, akan tetapi hal ini tidak dapat diperoleh pada perhitungan
numerik. Oleh karena itu, kita biasanya membatasi penjumlahan deret ini hanya sampai beberapa
suku saja.
Buat plot dari fungsi tersebut dengan menggunakan Matplotlib untuk beberapa
suku ekspansi, dan bandingkan hasilnya, misalnya perbandingan hasil plot menggunakan satu suku
dengan dua suku, tiga suku dan seterusnya. Semakin banyak suku yang Anda masukkan
dalam perhitungan, maka plot yang Anda tampilkan akan semakin mendekati fungsi periodik atau sinyal
yang diinginkan.


\end{document}

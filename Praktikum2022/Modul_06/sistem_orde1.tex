\section{Sistem Persamaan Diferensial Orde-1}

Skema satu langkah yang sudah dipelajari sebelumnya juga dapat digunakan
untuk sistem persamaan diferensial orde 1.

\begin{soal}[Chapra Contoh 25.9]
Cari solusi numerik dari sistem persamaan diferensial berikut:
\begin{align*}
\frac{\mathrm{d}y_1}{\mathrm{d}x} & = -0.5 y_1 \\
\frac{\mathrm{d}y_2}{\mathrm{d}x} & = 4 - 0.3 y_2 - 0.1y_1
\end{align*}
dengan syarat awal pada $x=0$, $y_1 = 4$ dan $y_2 = 6$.
Cari solusi sampai pada $x=2$ dengan ukuran langkah 0.5.
\end{soal}

Berikut ini adalah program yang dapat kita gunakan.
\begin{pythoncode}
import numpy as np

def deriv(x, y):
    Nvec = len(y)
    # Here we make an assertion to make sure that y is a 2-component vector
    # Uncomment this line if the code appears to be slow
    assert Nvec == 2
    # Output array
    dydx = np.zeros(Nvec)
    # remember that in Python the array starts at 0
    # y1 = y[0]
    # y2 = y[1]
    dydx[0] = -0.5*y[0]
    dydx[1] = 4 - 0.3*y[1] - 0.1*y[0]
    # 
    return dydx
    
# One-step application of Euler's method for ODE
def ode_euler_1step(dfunc, xi, yi, h):
    return yi + dfunc(xi,yi)*h
    
def ode_solve(dfunc, do_1step, x0, y0, h, Nstep):
    Nvec = len(y0)
    x = np.zeros(Nstep+1)
    y = np.zeros((Nstep+1,Nvec))
    # Start with initial cond
    x[0] = x0
    y[0,:] = y0[:]
    for i in range(0,Nstep):
        x[i+1] = x[i] + h
        y[i+1,:] = do_1step(dfunc, x[i], y[i,:], h)
    return x, y
    
# initial cond
x0 = 0.0
y0 = np.array([4.0, 6.0])

h = 0.5
Nstep = 4
x, y = ode_solve(deriv, ode_euler_1step, x0, y0, h, Nstep)
print("")
print("---------------------------")
print(" x         y1         y2")
print("---------------------------")
for i in range(len(x)):
    print("%5.1f %10.6f %10.6f" % (x[i], y[i,0], y[i,1]))    
\end{pythoncode}

Contoh keluaran:
\begin{textcode}
---------------------------
   x       y1         y2
---------------------------
  0.0   4.000000   6.000000
  0.5   3.000000   6.900000
  1.0   2.250000   7.715000
  1.5   1.687500   8.445250
  2.0   1.265625   9.094088
\end{textcode}

Beberapa catatan:
\begin{itemize}
\item Tidak ada perubahan pada definisi fungsi \pyinline{ode_euler_1step}
\item Fungsi yang mendefinisikan (sistem) persamaan diferensial sekarang
mengembalikan array satu dimensi atau vektor.
\item Kita telah mendefinisikan fungsi \pyinline{ode_solve} yang dapat dikombinasikan
dengan metode satu langkah yang lain seperti \pyinline{ode_rk4_1step}.
\end{itemize}

\begin{soal}[Chapra Contoh 25.10]
Gunakan metode Runge-Kutta orde-4 untuk sistem persamaan diferensial yang didefinisikan
pada soal sebelumnya (Chapra Contoh 25.9).
\end{soal}

Contoh keluaran:
\begin{textcode}
---------------------------
   x        y1         y2
---------------------------
  0.0   4.000000   6.000000
  0.5   3.115234   6.857670
  1.0   2.426171   7.632106
  1.5   1.889523   8.326886
  2.0   1.471577   8.946865
\end{textcode}



\begin{soal}[Chapra Contoh 25.11]
Cari solusi numerik dari sistem persamaan diferensial berikut:
\begin{align*}
\frac{\mathrm{d}y_1}{\mathrm{d}x} & = y_2 \\
\frac{\mathrm{d}y_2}{\mathrm{d}x} & = -16.1 y_1 \\
\frac{\mathrm{d}y_3}{\mathrm{d}x} & = y_4 \\
\frac{\mathrm{d}y_4}{\mathrm{d}x} & = -16.1 \sin(y_3)
\end{align*}
untuk kasus-kasus syarat awal ($x=0$) berikut
\begin{itemize}
\item Pergeseran kecil: $y_1 = y_3 = 0.1$ radian, $y_2 = y_4 = 0$
\item Pergeseran besar: $y_1 = y_3 = \pi/4$ radian, $y_2 = y_4 = 0$.
\end{itemize}
\end{soal}


Anda dapat melengkapi kode berikut:
\begin{pythoncode}
# .... import dan definisi fungsi

def pendulum_ode(x, y):
    Nvec = len(y)
    # Here we make an assertion to make sure that y is a 4-component vector
    # Uncomment this line if the code appears to be slow
    assert Nvec == 4
    # Output array
    dydx = np.zeros(Nvec)
    # remember that in Python the array starts at 0
    # y1 = y[0]
    # y2 = y[1], etc ...
    dydx[0] = y[1]
    dydx[1] = -16.1*y[0]
    # Nonlinear effect
    dydx[2] = .... # LENGKAPI
    dydx[3] = .... # LENGKAPI
    # 
    return dydx


# initial cond
x0 = 0.0
y0 = np.array([0.1, 0.0, 0.1, 0.0]) # Small displacement

h = 0.01 # try playing with this parameter
xf = 4.0
Nstep = int(xf/h)
x, y = ode_solve(pendulum_ode, ode_rk4_1step, x0, y0, h, Nstep)

plt.clf()
plt.plot(x, y[:,0], label="y1")
plt.plot(x, y[:,1], label="y2")
plt.plot(x, y[:,2], label="y3")
plt.plot(x, y[:,3], label="y4")
plt.title("Small displacement case")
plt.ylim(-4,4) # The same for both small and large displacement
plt.legend()
plt.tight_layout()
plt.grid(True)


# initial cond
x0 = 0.0
y0 = np.array([np.pi/4, 0.0, np.pi/4, 0.0]) # Large displacement

# .... same as before

plt.clf()
# .... same as before
plt.title("Large displacement case")
# .... same as before
\end{pythoncode}

Contoh hasil visualisasi dapat dilihat pada Gambar \ref{fig:chapra_example_25_11}.

\begin{figure}[h]
{\centering
\includegraphics[width=0.45\textwidth]{../../chapra_7th/ch25/IMG_chapra_example_25_11_small.pdf}
\includegraphics[width=0.45\textwidth]{../../chapra_7th/ch25/IMG_chapra_example_25_11_large.pdf}
\par}
\caption{Chapra Contoh 25.11}
\label{fig:chapra_example_25_11}
\end{figure}
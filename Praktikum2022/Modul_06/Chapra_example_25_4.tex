\begin{soal}[Chapra Contoh 25.4]
Bandingkan solusi yang diperoleh dari model linear dengan persamaan diferensial:
\begin{equation*}
\frac{\mathrm{d}v}{\mathrm{d}t} = g - \frac{c}{m}v
\end{equation*}
dan model nonlinear yang memperhitungkan gaya gesek akibat angin:
\begin{equation*}
\frac{\mathrm{d}v}{\mathrm{d}t} = g - \frac{c}{m}\left[
v + a\left(\frac{v}{v_{\mathrm{max}}}\right)^b
\right]
\end{equation*}
dengan parameter $g = 9.81$, $c=12.5$, $m=68.1$, and kondisi awal $v=0$ pada $t=0$
(satuan dalam SI).
Parameter tambahan untuk model nonlinear adalah sebagai berikut: $a=8.3$, $b=2.2$,
$v_{\mathrm{max}}=46$.
Gunakan metode Euler dengan ukuran langkah 0.1 untuk menyelesaikan persamaan
diferensial secara numerik.
\end{soal}


\begin{pythoncode}
# .... import library yang diperlukan
# .... definisi fungsi-fungsi yang diperlukan

def model_linear(t, v):
    g = 9.81; c = 12.5; m = 68.1
    return g - c*v/m

def model_nonlinear(t, v):
    g = 9.81; c = 12.5; m = 68.1
    a = 8.3; b = 2.2; vmax = 46
    return g - c/m*( v + a*(v/vmax)**b )
    
t0 = 0.0; v0 = 0.0 # initial cond
tf = 15.0
h = 0.1
Nstep = int(tf/h)

t_linear, v_linear = ode_euler(....) # Lengkapi
t_nonlinear, v_nonlinear = ode_euler(....) # Lengkapi
    
# Plot
plt.clf()
plt.plot(t_linear, v_linear, label="model linear")
plt.plot(t_nonlinear, v_nonlinear, label="model nonlinear")
plt.xlabel("t (s)")
plt.ylabel("y (m)")
plt.legend()
plt.grid(True)
plt.tight_layout()
\end{pythoncode}

Hasil dapat dilihat pada Gambar \ref{fig:example_25_4}.

\begin{figure}[h]
{\centering
\includegraphics[scale=0.7]{../../chapra_7th/ch25/IMG_chapra_example_25_4.pdf}
\par}
\caption{Perbandingan hasil metode Euler untuk model linear dan nonlinear
dengan ukuran langkah $h=0.1$}
\label{fig:example_25_4}
\end{figure}
% ----------------------------------
Menggunakan deret Taylor orde-tinggi

Misalnya, menggunakan orde dua:
\begin{equation*}
y_{i+1} = y_{i} + f(x_i, y_i) + \frac{f'(x_i, y_i)}{2!}h^2
\end{equation*}
dengan kesalahan pemotongan lokal:
\begin{equation*}
E_{a} = \frac{f''(x_i, y_i)}{6} h^3
\end{equation*}
Dengan menggunakan aturan rantai untuk turunan:
\begin{equation*}
f'(x_i, y_i) = \frac{\partial f(x,y)}{\partial x} +
\frac{\partial f(x,y)}{\partial y}\frac{\mathrm{d}y}{\mathrm{d}x}
\end{equation*}
Metode Runge-Kutta

\begin{equation*}
y_{i+1} = y_{i} + \phi(x_i, y_i, h) h
\end{equation*}

$\phi(x_i, y_i, h)$: fungsi inkremen

\begin{equation*}
\phi = a_1 k_1 + a_2 k_2 + \cdots + a_n k_n
\end{equation*}

di mana $a$ adalah konstanta dan $k$ adalah:
\begin{align*}
k_1 & = f(x_i, y_i) \\
k_2 & = f(x_i + p_1 h, y_i + q_{11} k_1 h ) \\
k_3 & = f(x_i + p_2 h, y_i + q_{21} k_1 h + q_{22} k_2 h) \\
\cdots & \cdots \\
k_n & = f(x_i + p_{n-1}h, y_i + q_{n-1,1} k_1 h + q_{n-1,2} k_2 h + \cdots + q_{n-1,n-1} k_{n-1} h)
\end{align*}
di mana $p$ dan $q$ adalah konstanta.
Perhatikan bahwa $k$ memiliki hubungan rekurensi (perulangan).

Runge-Kutta orde 2

\begin{equation*}
y_{i+1} = y_i + (a_1 k_1 + a_2 k_2) h
\end{equation*}
dengan
\begin{align*}
k_1 = f(x_i, y_i) \\
k_2 = f(x_i + p_1 h, y_i + q_{11} k_1 h)
\end{align*}
Nilai-nilai dari $a_1$, $a_2$, $p_1$, dan $q_{11}$ dapat diperoleh dari persamaan
berikut.
\begin{align*}
a_1 + a_2  & = 1 \\
a_2 p_1    & = \frac{1}{2} \\
a_2 q_{11} & = \frac{1}{2}
\end{align*}
Karena ada tiga persamaan dan empat variabel yang tidak diketahui, kita harus mengasumsikan satu
nilai dari variabel tersebut untuk mendapatkan tiga variabel yang lain.
Misalkan nilai $a_2$ telah dipilih, maka variabel-variabel yang lain
dapat ditentukan sebagai berikut.
\begin{align*}
a_1 = 1 - a_2
p_1 = q_{11} = \frac{1}{2a_2}
\end{align*}
Dengan kata lain, ada tak hingga versi dari metode Runge-Kutta orde-2.

Ada tiga versi yang populer:

Metode Heun dengan korektor tunggal, $a_2 = 1/2$
$a_1 = 1/2$ dan $p_1 = q_{11} = 1$. Dengan parameter tersebut, diperoleh skema
sebagai berikut:
\begin{equation*}
y_{i+1} = y_{i} + \left(
\frac{1}{2}k_1 + \frac{1}{2}k_2
\right)h
\end{equation*}
dengan
\begin{align*}
k_1 = f(x_i, y_i) \\
k_2 = f(x_i + h, y_i + k_1 h)
\end{align*}
Perhatikan bahwa $k_1$ adalah kemiringan pada awal interval dan $k_2$ adalah
kemiringan pada akhir interval. Oleh karena itu, metode Runge-Kutta orde-2 ini
tidak lain adalah metode Heun tanpa iterasi.


Metode titik tengah, $a_2 = 1$, $a_1 = 0$, $p_1 = q_{11} = 1/2$, diperoleh
skema sebagai berikut.
\begin{equation*}
y_{i+1} = y_i + k_2 h
\end{equation*}
dengan
\begin{align*}
k_1 = f(x_i, y_i) \\
k_2 = f\left( x_i + \frac{1}{2}h, y_i + \frac{1}{2} k_1 h \right)
\end{align*}
yang merupakan metode titik tengah.

Metode Ralston, dikembangkan oleh Ralston (1962) dan Ralston dan Rabinowitz (1978),
yang memilih parameter-parameter sehingga batas minimum untuk kesalahan pemotongan,
dengan parameter $a_2 = 2/3$, $a_1 = 1/3$, dan $p_1 = q_{11} = 3/4$, yang
memberikan skema sebagai berikut:
\begin{equation*}
y_{i+1} = y_i + \left( \frac{1}{3}k_1 + \frac{2}{3}k_2 \right) h
\end{equation*}
dengan
\begin{align*}
k_1 = f(x_i, y_i) \\
k_2 = f\left( x_i + \frac{3}{4}h, y_i + \frac{3}{4}k_1 h \right)
\end{align*}


Metode Runge-Kutta orde-3

Skema:
\begin{equation*}
y_{i+1} = y_i + \frac{1}{6}(k_1 + 4k_2 + k_3)h
\end{equation*}
dengan
\begin{align*}
k_1 = f(x_i, y_i) \\
k_2 = f(x_i + \frac{1}{2}h, y_i + \frac{1}{2} k_1 h ) \\
k_3 = f(x_i + h, y_i - k_1 h + 2 k_2 h)
\end{align*}
Perhatikan bahwa, jika fungsi turunan merupakan fungsi dari $x$ saja, maka metode ini menjadi
aturan 1/3 Simpson.

Metode Runge-Kutta orde-4

Skema:
\begin{equation*}
y_{i+1} = y_{i} + \frac{1}{6}(k_1 + 2k_2 + 2k_3 + k_4) h
\end{equation*}
dengan
\begin{align*}
k_1 = f(x_i, y_i) \\
k_2 = f\left( x_i + \frac{1}{2}h, y_i + \frac{1}{2} k_1 h \right) \\
k_3 = f\left( x_i + \frac{1}{2}h, y_i + \frac{1}{2} k_2 h \right) \\
k_4 = f(x_i + h, y_i + k_3 h)
\end{align*}

Metode Runge-Kutta orde-5

Oleh Butcher (1964)

Skema:
\begin{equation*}
y_{i+1} = y_i + \frac{1}{90} ( 7 k_1 + 32 k_3 + 12 k_4 + 32 k_5 + 7 k_6) h
\end{equation*}
dengan:
\begin{align*}
k_1 & = f(x_i, y_i) \\
k_2 & = f\left( x_i + \frac{1}{4}h, y_i + \frac{1}{4} k_1 h \right) \\
k_3 & = f\left( x_i + \frac{1}{4}h, y_i + \frac{1}{8} k_1 h + \frac{1}{8} k_2 h \right) \\
k_4 & = f\left( x_i + \frac{1}{2}h, y_i - \frac{1}{2} k_2 h + k_3 h \right) \\
k_5 & = f\left( x_i + \frac{3}{4}h, y_i + \frac{3}{16} k_1 h + \frac{9}{16} k_4 h \right) \\
k_6 & = f\left( x_i + h, y_i - \frac{3}{7} k_1 h + \frac{2}{7} k_2 h +
\frac{12}{7}k_3 h - \frac{12}{7}k_4 h + \frac{8}{7}k_5 h \right)
\end{align*}


Persamaan diferensial orde dua

Contoh:
\begin{equation*}
m \frac{\mathrm{d}^2 x}{\mathrm{d}t^2} + c \frac{\mathrm{d}x}{\mathrm{d}t} + kx = 0
\end{equation*}
$c$: koefisien redaman, $k$ konstant pegas.

Definisikan $y_{1}(t) = x(t)$, $y_{2}(t) = x'(t)$, sehingga:
$y'_{1}(t) = x'(t) = y_{2}(t)$, dan
$y'_{2}(t) = x''(t)$:
\begin{align*}
m \frac{\mathrm{d}^2 x}{\mathrm{d}t^2} + c \frac{\mathrm{d}x}{\mathrm{d}t} + kx & = 0 \\
m y'_{2}(t) + c y_{2}(t) + k y_{1}(t) & = 0 \\
y'_{2}(t) & = -\frac{c y_{2}(t) + k y_{1}(t)}{m}
\end{align*}

\begin{align*}
y'_{1}(t) & = y_{2}(t) \\
y'_{2}(t) & = -\frac{c y_{2}(t) + k y_{1}(t)}{m}
\end{align*}
%m \frac{\mathrm{d}^2 x}{\mathrm{d}t^2} + c \frac{\mathrm{d}x}{\mathrm{d}t} + kx = 0

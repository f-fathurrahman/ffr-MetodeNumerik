\input{../PREAMBLE}

\begin{document}

\title{%
{\small TF2202 Komputasi Rekayasa}\\
Integral dan Diferensiasi Numerik
}
\author{Tim Praktikum Komputasi Rekayasa 2022\\
Teknik Fisika\\
Institut Teknologi Bandung}
\date{}
\maketitle


\section{Aturan trapesium}

\begin{soal}[Chapra Contoh 21.1]
Gunakan aturan trapesium untuk aproksimasi integral berikut.
\begin{equation*}
f(x) = 0.2 + 25x - 200x^2 + 675x^3 - 900x^4 + 400x^5
\end{equation*}
dari $a = 0$ sampai $b = 0.8$. Bandingkan hasilnya dengan hasil analitik,
yaitu 1.640533. Hitung juga estimasi error dari aproksimasi yang digunakan.
\end{soal}

Anda dapat melengkapi kode berikut ini untuk Chapra Contoh 21.1
\begin{pythoncode}
def my_func(x):
    return 0.2 + 25*x - 200*x**2 + 675*x**3 - 900*x**4 + 400*x**5

a = 0.0
b = 0.8
f_a = my_func(a)
f_b = my_func(b)

I_exact = 1.640533 # from the book
# Alternatively, you can use SymPy to do the integration

# Use trapezoid rule here
I = .... # Lengkapi

E_t = I_exact - I
ε_t = E_t/I_exact * 100 # in percent
print("Integral result = %.6f" % I)
print("True integral   = %.6f" % I_exact)
print("True error      = %.6f" % E_t)
print("ε_t             = %.1f%%" % ε_t)

import sympy
x = sympy.symbols("x")
f = 0.2 + 25*x - 200*x**2 + 675*x**3 - 900*x**4 + 400*x**5
d2f = f.diff(x,2) # calculate 2nd derivative
avg_d2f_xi = sympy.integrate( d2f, (x,a,b) )/(b - a)
E_a = -1/12*avg_d2f_xi*(b - a)**3 # Persamaan 21.6
print("Approx error    = %.6f" % E_a)
\end{pythoncode}

Aturan trapesium (dan aturan integrasi Newton-Cotes lainnya) dapat digunakan
untuk lebih dari satu interval.

\begin{soal}
Gunakan aturan trapesium multi-interval dengan jumlah segmen 2 untuk menghitung
aproksimasi integral pada Soal sebelumnya (Chapra Contoh 21.1).
Coba juga ganti jumlah segmen yang digunakan dan bandingkan hasilnya
dengan hasil analitik.
\end{soal}

Anda dapat melengkapi kode berikut ini. Lihat juga persaman yang digunakan pada
buku Chapra.
\begin{pythoncode}
def my_func(x):
    return 0.2 + 25*x - 200*x**2 + 675*x**3 - 900*x**4 + 400*x**5

# N is number of segments
# Npoints is N + 1
# f is function
def integ_trapz_multiple( f, a, b, N ):
    x0 = a
    xN = b
    h = (b - a)/N
    ss = 0.0
    for i in range(1,N):
        xi = x0 + i*h
        ss = ss + f(xi)
    I = .... # lengkapi
    return I

Nsegments = 2
a = 0.0
b = 0.8
I_exact = 1.640533 # from the book

I = integ_trapz_multiple( my_func, a, b, Nsegments )

print("Nsegments = ", Nsegments)
print("Integral result = %.6f" % I)
E_t = I_exact - I
print("True integral   = %.6f" % I_exact)
print("True error      = %.6f" % E_t)
ε_t = E_t/I_exact * 100
print("ε_t             = %.1f%%" % ε_t)

import sympy
x = sympy.symbols("x")
f = 0.2 + 25*x - 200*x**2 + 675*x**3 - 900*x**4 + 400*x**5
d2f = f.diff(x,2)
#sympy.pprint(d2f)
avg_d2f_xi = sympy.integrate( d2f, (x,a,b) )/(b - a)
#print("avg_d2f_xi = ", avg_d2f_xi)
E_a = -1/12*avg_d2f_xi*(b - a)**3/Nsegments**2
print("Approx error    = %.6f" % E_a)
\end{pythoncode}

\begin{soal}[Chapra Contoh 21.3]
Hitung integral berikut ini dengan menggunakan aturan trapesium:
\begin{equation*}
d = \frac{gm}{c} \int_{0}^{10} (1 - e^{-(c/m)t}) \, \mathrm{d}t
\end{equation*}
Bandingkan hasil yang diperoleh dengan hasil analitik. Lihat buku Chapra untuk nilai-nilai
numerik yang diperlukan atau lihat pada kode Python di bawah.
\end{soal}

Anda dapat melengkapi kode berikut ini.
\begin{pythoncode}
from math import exp

def my_func(t):
    g = 9.8
    m = 68.1
    c = 12.5
    return g*m/c * (1 - exp(-(c/m)*t) )
    
# ... definisi integ_trapz_multiple
# ... LENGKAPI

t = 10.0
a = 0.0
b = t
    
d_exact = 289.43515 # dari buku
    
print("------------------------------------------------------")
print("   N          h          d          E_t         ε_t")
print("------------------------------------------------------")
for Nsegments in [10, 20, 50, 100, 200, 500, 1000, 2000, 5000, 5000, 10000]:
    h = (b - a)/Nsegments
    d = integ_trapz_multiple( my_func, a, b, Nsegments )
    E_t = d_exact - d
    ε_t = E_t/d_exact * 100
    print("%5d  %10.4f  %10.4f   %10.4e %10.2e%%" % (Nsegments, h, d, E_t, ε_t))


# Calculate "Exact" result using SymPy
import sympy
g = 9.8
m = 68.1
c = 12.5
t = sympy.symbols("t")
f = g*m/c * (1 - sympy.exp(-(c/m)*t) )
d_sympy = sympy.integrate( f, (t,0,10))
    
print()
print("SymPy result:")
print("d_sympy = ", d_sympy)    
\end{pythoncode}

Apakah hasil yang Anda peroleh sama dengan yang diberikan pada Chapra?
Jika ada perbedaan, apakah yang menyebabkan perbedaan tersebut?

\begin{soal}[Chapra Latihan 21.1-21.7]
Untuk integral-integral berikut ini
\begin{align}
& \int_{0}^{\pi/2} (6 + 3\cos(x))\ \mathrm{d}x \\
& \int_{0}^{3} (1 - e^{-2x})\ \mathrm{d}x \\
& \int_{-2}^{4} (1 - x - 4x^3 + 2x^5)\ \mathrm{d}x \\
& \int_{1}^{2} (x - 2/x)^2\ \mathrm{d}x \\
& \int_{-3}^{5} (4x - 3)^3 \mathrm{d}x \\
& \int_{0}^{3} x^2 e^x \mathrm{d}x \\
& \int_{0}^{1} 14^{2x}\ \mathrm{d}x
\end{align}
dengan menggunakan metode:
\begin{enumerate}[label=(\alph*)]
\item analitik
\item aturan trapesium
\item aturan 1/3 Simpson
\item aturan 3/8 Simpson
\item aturan Boole
\end{enumerate}
Variasikan parameter numerik yang terkait, seperti jumlah titik yang dievaluasi,
untuk setiap metode yang digunakan.
Bandingkan hasil numerik yang diberikan dengan hasil analitik.
Anda dapat menggunakan SymPy untuk menghitung integral secara analitik.
\end{soal}

\begin{soal}[Chapra 21.10 dan 21.11]
    Hitung integral dari data berikut ini dengan menggunakan aturan trapesium dan aturan
    Simpson.
    
    Data Chapra 21.10
    
    {\centering
    \begin{tabular}{|c|cccccc|}
    \hline
    $x$    & 0 & 0.1 & 0.2 & 0.3 & 0.4 & 0.5 \\
    $f(x)$ & 1 & 8   & 4   & 3.5 & 5   & 1 \\
    \hline
    \end{tabular}
    \par}
    
    Data Chapra 21.11
    
    {\centering
    \begin{tabular}{|c|ccccccc|}
    \hline
    $x$    & -2 & 0 & 2 & 4 & 6 & 8 & 10 \\
    $f(x)$ & 35 & 5 & -10 & 2 & 5 & 3 & 20 \\
    \hline
    \end{tabular}
    \par}
    \end{soal}

\begin{soal}[Chapra Latihan 22.1, 22.2, dan 22.3]
Gunakan metode Romberg untuk menghitung integral berikut.
\begin{align}
& \int_{0}^{3} x e^{2x}\ \mathrm{d}x \\
& \int_{1}^{2} \left( x + \frac{1}{x} \right)^2 \ \mathrm{d}x \\
& \int_{0}^{2} \frac{e^{x}\sin(x)}{1 + x^2}\ \mathrm{d}x
\end{align}
Variasikan parameter numerik yang Anda gunakan, misalnya jumlah maksimal selang
yang digunakan.
Bandingkan hasil yang Anda dapatkan dengan hasil analitik atau dari SymPy.
\end{soal}


\begin{soal}[Chapra Latihan 22.8]
Gunakan formula Gauss-Legendre dengan dua sampai 6 titik untuk menghitung integral
\begin{equation}
\int_{-3}^{3} \frac{1}{1 + x^2}\ \mathrm{d}x
\end{equation}
Bandingkan hasil yang Anda dapatkan dengan hasil analitik atau dari SymPy.
\end{soal}

\begin{soal}[Chapra Latihan 22.9]
Gunakan metode integrasi numerik untuk menghitung integral:
\begin{align}
& \int_{2}^{\infty} \frac{1}{x(x + 2)}\ \mathrm{dx} \\
& \int_{0}^{\infty} e^{-x} \sin^{2} x\ \mathrm{dx}
\end{align}
Bandingkan hasil yang Anda peroleh dengan hasil analitik atau dari SymPy.
Gunakan metode standard (trapesium, Simpson, Romberg, dll) secara
langsung dan dengan melakukan penggantian variabel. Anda juga dapat menggunakan
kombinasi metode tersebut jika integral yang Anda lakukan dibagi menjadi dua
integral (Lihat Contoh 22.6 pada Chapra). Plot juga integran yang terlibat.
\end{soal}


\begin{soal}[Chapra Latihan 23.8]
Hitung turunan pertama dari fungsi-fungsi berikut dengan menggunakan metode beda hingga
tengah $\mathcal{O}(h^4)$:
\begin{itemize}
\item $y = x^3 + 4x - 15$ pada $x=0$ dengan $h=0.25$
\item $y = x^2 \cos(x)$ pada $x=0.4$ dengan $h=0.1$
\item $y = \tan(x/3)$ pada $x=3$ dengan $h=0.5$
\item $y = \sin(0.5\sqrt{x})/x$ pada $x=1$ dengan $h=0.2$
\item $y = e^{x} + x$ pada $x=2$ dengan $h=0.2$
\end{itemize}
Bandingkan hasil yang Anda peroleh dengan hasil analitik atau SymPy.
\end{soal}


\begin{soal}[Chapra Latihan 23.9]
Diketahui data jarak dan waktu yang ditempuh pada suatu roket.

{\centering
\begin{tabular}{|c|cccccc|}
\hline
t (s)  & 0 & 25 & 50 & 75 & 100 & 125 \\
y (km) & 0 & 32 & 58 & 78 &  92 & 100 \\
\hline
\end{tabular}
\par}

Gunakan metode numerik untuk untuk mengestimasi kecepatan dan percepatan roket
pada masing-masing waktu pada tabel (jika memungkinkan dengan metode yang Anda
gunakan).
\end{soal}


%\input{KodePython}

\end{document}

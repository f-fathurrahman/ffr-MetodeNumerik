\subsection{Chapra Contoh 4.2}

Deret Taylor dapat dituliskan sebagai berikut.
\begin{equation}
f(x_{i+1}) = f(x_{i}) + f'(x_{i}) h + \frac{f''(x_{i})}{2!}h^2 + 
\frac{f^{(3)}(x_{i})}{3!}h^3 + \cdots +
\frac{f^{(n)}(x_{i})}{n!}h^n + R_{n}
\end{equation}
dengan $h = x_{i+1} - x_{i}$ dan $R_{n}$ adalah suku sisa (\textit{remainder}).

Kita akan menggunakat deret Taylor dari $n=0$ sampai $n=6$ untuk mengaproksimasi
$f(x) = \cos(x)$ pada $x_{i+1} = \pi/3$ dengan nilai $f(x)$ dari turunan-turunannya
pada $x_{i} = \pi/4$, atau $h = x_{i+1} - x_{i} = \pi/12$.

Program berikut ini menggunakan kombinasi perhitungan simbolik dan numerik
dengan SymPy.
\begin{pythoncode}
from sympy import * # be very careful when using this!

init_printing(use_unicode=True)
# if you are using Jupyter Lab or Notebook, use the following line instead:
#init_printing(use_latex=True)

x = symbols("x")

f = cos(x)

xi = pi/4
xip1 = pi/3
h = xip1 - xi

# zeroth order
f_approx = diff(f, x, 0).subs({x: xi}) # or simply call cos(xi)

for n in range(1,7): # from 1 to 6
    new_term = diff(f, x, n) * h**n / factorial(n)
    f_approx = f_approx + new_term
    pprint(f_approx)
    print(N(f_approx.subs({x: xi}))) # use N to force numerical expression

f_true = N(f.subs({x: xip1}))
print("f_true = ", f_true)
print(type(f_true))
\end{pythoncode}


\begin{soal}
Lakukan modifikasi pada program di atas sehingga dapat menampilkan error atau
perbedaan antara nilai aproksimasi dan nilai benar. Program di atas juga menampilkan
deret Taylor yang digunakan secara simbolik. Anda dapat menonaktifkan baris
kode yang sesuai dengan cara menghapusnya atau menjadikannya komentar.
\end{soal}

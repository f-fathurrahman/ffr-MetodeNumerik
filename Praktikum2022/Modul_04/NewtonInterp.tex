\section{Polinomial Interpolasi Newton}

\begin{soal}
Implementasikan fungsi atau subroutin dalam Python untuk implementasi algoritma
pada Gambar 18.7 untuk implementasi polinomial Newton.
Uji hasil yang Anda dapatkan dengan menggunakan data-data yang diberikan pada contoh
18.2 (polinomial kuadrat) dan 18.3 (polinomial kubik) pada Chapra.
Lengkapi jawaban Anda dengan membuat plot seperti pada Gambar 18.4 dan 18.6 pada Chapra.
\end{soal}

\begin{soal}
Gunakan data pada soal Chapra 18.5 untuk mengevaluasi nilai $f(x) = \mathrm{ln}(x)$
pada $x = 2$ dengan menggunakan polinomial kubik. Coba variasikan titik-titik
yang digunakan (\textit{base points}) dan perhatikan nilai estimasi
kesalahan yang diberikan oleh
fungsi/subrutin yang sudah Anda buat pada soal sebelumnya.
\end{soal}

\begin{pythoncode}
import numpy as np
def newton_interp(x, y, xi):
    assert(len(x) == len(y))
    N = len(x) - 1  # length of array is (N + 1)
    yint = np.zeros(N+1)
    ea = np.zeros(N)
    # finite divided difference table
    fdd = np.zeros((N+1,N+1))
    for i in range(0,N+1):
        fdd[i,0] = y[i]
    for j in range(1,N+1):
        for i in range(0,N-j+1):
            fdd[i,j] = ( fdd[i+1,j-1] - fdd[i,j-1] ) / ( x[i+j] - x[i] )
    xterm = 1.0
    yint[0] = fdd[0,0]  
    for order in range(1,N+1):
        xterm = xterm * ( xi - x[order-1] )
        yint2 = yint[order-1] + fdd[0,order]*xterm
        ea[order-1] = yint2 - yint[order-1]
        yint[order] = yint2 
      
    return yint, ea  
\end{pythoncode}
\documentclass[a4paper,11pt,bahasa]{article} % screen setting
\usepackage[a4paper]{geometry}

%\documentclass[b5paper,11pt,bahasa]{article} % screen setting
%\usepackage[b5paper]{geometry}

\geometry{verbose,tmargin=1.5cm,bmargin=1.5cm,lmargin=1.5cm,rmargin=1.5cm}

\setlength{\parskip}{\smallskipamount}
\setlength{\parindent}{0pt}

%\usepackage{cmbright}
%\renewcommand{\familydefault}{\sfdefault}

\usepackage[libertine]{newtxmath}
\usepackage[no-math]{fontspec}

\setmainfont{Linux Libertine O}

\setmonofont{JuliaMono-Regular}


\usepackage{hyperref}
\usepackage{url}
\usepackage{xcolor}

\usepackage{amsmath}
\usepackage{amssymb}

\usepackage{graphicx}
\usepackage{float}

\usepackage{minted}

\newminted{julia}{breaklines,fontsize=\scriptsize}
\newminted{python}{breaklines,fontsize=\scriptsize}

\newminted{bash}{breaklines,fontsize=\scriptsize}
\newminted{text}{breaklines,fontsize=\scriptsize}

\newcommand{\txtinline}[1]{\mintinline[breaklines,fontsize=\scriptsize]{text}{#1}}
\newcommand{\jlinline}[1]{\mintinline[breaklines,fontsize=\scriptsize]{julia}{#1}}
\newcommand{\pyinline}[1]{\mintinline[breaklines,fontsize=\scriptsize]{python}{#1}}

\newmintedfile[juliafile]{julia}{breaklines,fontsize=\scriptsize}
\newmintedfile[pythonfile]{python}{breaklines,fontsize=\scriptsize}
\newmintedfile[fortranfile]{fortran}{breaklines,fontsize=\scriptsize}
% f-o-otnotesize

\usepackage{mdframed}
\usepackage{setspace}
\onehalfspacing

\usepackage{enumitem}

\usepackage{mhchem}
\usepackage{appendix}

\newcommand{\highlighteq}[1]{\colorbox{blue!25}{$\displaystyle#1$}}
\newcommand{\highlight}[1]{\colorbox{red!25}{#1}}

\newcounter{soal}%[section]
\newenvironment{soal}[1][]{\refstepcounter{soal}\par\medskip
   \noindent \textbf{Soal~\thesoal. #1} \sffamily}{\medskip}


\definecolor{mintedbg}{rgb}{0.95,0.95,0.95}
\BeforeBeginEnvironment{minted}{
    \begin{mdframed}[backgroundcolor=mintedbg,%
        topline=false,bottomline=false,%
        leftline=false,rightline=false]
}
\AfterEndEnvironment{minted}{\end{mdframed}}


\BeforeBeginEnvironment{soal}{
    \begin{mdframed}[%
        topline=true,bottomline=true,%
        leftline=true,rightline=true]
}
\AfterEndEnvironment{soal}{\end{mdframed}}

% -------------------------
\begin{document}

\title{%
{\small TF2202 Komputasi Rekayasa}\\
Interpolasi dan Pencocokan Kurva
}
\author{Tim Praktikum Komputasi Rekayasa 2021\\
Teknik Fisika\\
Institut Teknologi Bandung}
\date{}
\maketitle

\section{Polinomial Interpolasi Newton}

\begin{soal}
Implementasikan fungsi atau subroutin dalam Python untuk implementasi algoritma
pada Gambar 18.7 untuk implementasi polinomial Newton.
Uji hasil yang Anda dapatkan dengan menggunakan data-data yang diberikan pada contoh
18.2 (polinomial kuadrat) dan 18.3 (polinomial kubik) pada Chapra.
Lengkapi jawaban Anda dengan membuat plot seperti pada Gambar 18.4 dan 18.6 pada Chapra.
\end{soal}

\begin{soal}
Gunakan data pada soal Chapra 18.5 untuk mengevaluasi nilai $f(x) = \mathrm{ln}(x)$
pada $x = 2$ dengan menggunakan polinomial kubik. Coba variasikan titik-titik
yang digunakan (\textit{base points}) dan perhatikan nilai estimasi
kesalahan yang diberikan oleh
fungsi/subrutin yang sudah Anda buat pada soal sebelumnya.
\end{soal}

\section{Polinomial Interpolasi Lagrange}
\begin{soal}
Implementasikan fungsi atau subroutin dalam Python untuk implementasi algoritma
pada Gambar 18.11 untuk implementasi polinomial Lagrange.
Uji hasil yang Anda dapatkan dengan menggunakan data-data yang diberikan pada contoh
18.2 (polinomial kuadrat) dan 18.3 (polinomial kubik) pada Chapra.
\end{soal}

\begin{soal}
Kerjakan Contoh 18.7 pada Chapra dengan menggunakan fungsi/subrutin interpolasi Lagrange
yang sudah Anda buat pada soal sebelumnya sehingga Anda dapat mereproduksi Gambar 18.12
pada Chapra. Pada Contoh 18.17 Anda diminta untuk mengestimasi kecepatan penerjun
pada $t=10$ s, yang berada di antara dua titik data terakhir.
Untuk polinom orde-1, gunakan dua titik data terakhir, untuk orde-2 gunakan tiga titik
data terakhir, dan seterusnya sampai orde-4.
\end{soal}

\section{Interpolasi Spline Kubik}
\begin{soal}
Implementasikan fungsi atau subroutin dalam Python untuk implementasi algoritma
pada Gambar 18.18 untuk implementasi spline kubik (natural).
Uji hasil yang Anda dapatkan dengan menggunakan data-data yang diberikan pada contoh
18.10.
\end{soal}


\section{Regresi Linear}
\begin{soal}
Implementasikan fungsi atau subroutin dalam Python untuk implementasi algoritma
pada Gambar 17.6 untuk implementasi regresi linear. Uji dengan menggunakan data
pada Contoh 17.1.
\end{soal}

\section{Regresi Linear (Notasi Matriks-Vektor)}

Tinjau model linear berikut:
\begin{equation}
y = f(x; w_{0}, w_{1}) = w_{0} + w_{1}x
\end{equation}
di mana $w_{1}$ (atau kemiringan atau \textit{slope}) dan $w_{0}$
(titik potong sumbu $y$ atau \textit{intercept}) adalah parameter model.
Misalnya, diberikan suatu nilai atau input $x_{n}$, kita dapat menghitung output dari model
sebagai:
\begin{equation}
y_{n} = f(x_{n}; w_{0}, w_{1}) = w_{0} + w_{1}x_{n}
\label{eq:linmodel1}
\end{equation}
Definisikan:
\begin{equation}
\mathbf{x}_{n} = \begin{bmatrix}
1 \\
x_{n}
\end{bmatrix}
,\,\,\,%
\mathbf{w} = \begin{bmatrix}
w_{0} \\
w_{1}
\end{bmatrix}
\end{equation}
Dengan menggunakan notasi ini, model linear pada Persamaan \eqref{eq:linmodel1} dapat ditulis
menjadi:
\begin{equation}
y_{n} = f(x_{n}; w_0, w_1) = \mathbf{w}^{\mathsf{T}} \mathbf{x}_{n}
\label{eq:linmodel2}
\end{equation}
Dalam bentuk matriks-vektor, dapat dituliskan sebagai berikut:
\begin{equation}
\mathbf{y} = \mathbf{X}\mathbf{w}
\end{equation}
dengan $\mathbf{X}$ adalah matriks input:
\begin{equation}
\mathbf{X} = \begin{bmatrix}
\mathbf{x}^{\mathsf{T}}_{1} \\
\mathbf{x}^{\mathsf{T}}_{2} \\
\vdots \\
\mathbf{x}^{\mathsf{T}}_{N}
\end{bmatrix} =
\begin{bmatrix}
1 & x_{1} \\
1 & x_{2} \\
\vdots & \vdots \\
1 & x_{N} \\
\end{bmatrix}
\end{equation}
dan $\mathbf{y}$ adalah vektor kolom:
\begin{equation}
\mathbf{y} =
\begin{bmatrix}
y_{1} \\
y_{2} \\
\vdots \\
y_{N} \\
\end{bmatrix}
\end{equation}

Diberikan himpunan pasangan data
$\left\{(x_{1},y_{1}), (x_{2}, y_{2}), \ldots, (x_{n}, y_{n})\right\}$
kita ingin
mencari parameter $\mathbf{w}$ yang meminimumkan rata-rata kesalahan kuadrat
$\mathcal{L}$ yang didefinisikan sebagai:
\begin{equation}
\mathcal{L} \equiv \frac{1}{N} \sum_{n=1}^{N} \left( y_{n} - \mathbf{w}^{\mathsf{T}}
\mathbf{x}_{n} \right)^2
\end{equation}
Dapat ditunjukkan bahwa parameter yang membuat $\mathcal{L}$ menjadi mininum adalah
\begin{equation}
\mathbf{w} = \left(\mathbf{X}^{\mathsf{T}}\mathbf{X} \right)^{-1} \mathbf{X}^{\mathsf{T}} \mathbf{y}
\label{eq:w_vektor}
\end{equation}

\begin{soal}
Buat fungsi/subrutin dalam Python untuk membuat matriks $\mathbf{X}$ dengan argumen
input $\mathbf{x}$ dan $\mathbf{y}$ dan menghitung $\mathbf{w}$ berdasarkan
persamaan \eqref{eq:w_vektor}.
Uji fungsi yang sudah Anda buat dengan menggunakan data pada Contoh 17.1 dan bandingkan
parameter \textit{slope} dan \textit{intercept} yang Anda dapatkan pada soal ini dan
soal sebelumnya. Anda dapat menggunakan \pyinline{np.linalg.inv} untuk menghitung
invers matriks atau \pyinline{np.linalg.linsolve} jika Anda mengubah permasalahan
ini menjadi sistem persamaan linear.
\end{soal}

Kita juga dapat menggunakan Persamaan \eqref{eq:w_vektor} untuk model linear
\footnote{Model linear didefinisikan sebagai model yang parameternya linear
(berpangkat satu)} polinom lebih tinggi dari satu. Misalkan, pada model
polinom kuadrat:
\begin{equation}
y = w_{0} + w_{1} x + w_{2} x^2
\end{equation}
matriks $\mathbf{X}$ menjadi:
\begin{equation}
\mathbf{X} = \begin{bmatrix}
1 & x_{1} & x_{1}^2 \\
1 & x_{2} & x_{2}^2 \\
\vdots & \vdots & \vdots \\
1 & x_{N} & x_{N}^2
\end{bmatrix}
\end{equation}
dan vektor $\mathbf{w}$ menjadi:
\begin{equation}
\mathbf{w} = 
\begin{bmatrix}
w_{0} \\
w_{1} \\
w_{2}
\end{bmatrix}
\end{equation}

Formula yang sama juga dapat digunakan untuk model multilinear dengan
dua variabel independen $x^{(1)}$ dan $x^{(2)}$ (tanda $^{(1)}$ dan $^{(2)}$ bukan
menyatakan pangkat, namun variabel yang berbeda)
\footnote{Silakan gunakan variabel lain seperti $t$, $x$, $y$, $z$}:
\begin{equation}
y = w_{0} + w_{1} x^{(1)} + w_{2} x^{(2)}
\end{equation}
di mana sekarang matrix $\mathbf{X}$ menjadi:
\begin{equation}
\mathbf{X} = \begin{bmatrix}
1 & x^{(1)}_{1} & x^{(2)}_{1} \\
1 & x^{(1)}_{2} & x^{(2)}_{2} \\
\vdots & \vdots & \vdots \\
1 & x^{(1)}_{N} & x^{(2)}_{N}
\end{bmatrix}
\end{equation}

\begin{soal}
Kembangkan fungsi/subrutin Python yang sudah Anda buat sehingga dapat menerima argumen
opsional \pyinline{m}, di mana \pyinline{m >= 1} adalah orde polinomial yang ingin digunakan.
Uji dengan menggunakan data pada Contoh 17.5 di buku Chapra.
\end{soal}

\begin{soal}
Modifikasi fungsi/subrutin Python sudah Anda buat untuk regresi linear sehingga
dapat digunakan untuk regresi multilinear dan uji fungsi yang Anda buat
dengan menggunakan data pada Contoh 17.6 di buku Chapra.
\end{soal}



\section{Soal Tambahan}

\begin{soal}[Chapra Latihan 18.5]

Diberikan data berikut:

{\centering
\begin{tabular}{|c|cccccc|}
\hline
$x$    & 1.6 & 2 & 2.5 & 3.2 & 4 & 4.5 \\
$f(x)$ & 2   & 8 & 14  & 15  & 8 & 2 \\
\hline
\end{tabular}
\par}

\begin{enumerate}[label=(\alph*)]
\item Hitung $f(2.8)$ dengan menggunakan polinomial interpolasi Newton dengan orde 1 sampai 3.
\item Gunakan Pers. (18.18) pada Chapra untuk mengestimasi kesalahan untuk setiap prediksi.
\end{enumerate}
\end{soal}

\begin{soal}[Chapra Latihan 18.11]
Gunakan interpolasi invers dengan menggunakan polinomial interpolasi kubik dan
metode bagi dua (\textit{bisection}) untuk menentukan nilai $x$ yang memenuhi
$f(x) = 0.23$ untuk data dalam tabel berikut.

{\centering
\begin{tabular}{|c|cccccc|}
\hline
x & 2   &   3   & 4    & 5 & 6 & 7 \\
y & 0.5 & 0.333 & 0.25 & 0.2 & 0.1667 & 1.1429 \\
\hline
\end{tabular}
\par}
\end{soal}

\begin{soal}{Chapra Latihan 18.26}
Fungsi Runge dapat dinyatakan sebagai berikut:
\begin{equation*}
f(x) = \frac{1}{1 + 25x^2}
\end{equation*}
\begin{enumerate}[label=(\alph*)]
\item Buat plot dari fungsi tersebut interval dari $x=-1$ sampai $x=1$.
\item Buat polinomial interpolasi Lagrange orde 4 dengan menggunakan nilai
fungsi yang disampel secara seragam: $x = -1, -0.5, 0, 0.5, 1$. Buat juga
plot dari polinomial tersebut. Gunakan untuk menghitung nilai $f(0.8)$.
\item Ulangi bagian sebelumnya dengan menggunakan polinom orde 5 sampai 10.
\end{enumerate}
\end{soal}

\begin{soal}[Chapra Latihan 18.27]
Fungsi \textit{humps} dapat ditulis sebagai berikut.
\begin{equation*}
f(x) = \frac{1}{(x - 0.3)^2 + 0.01} + \frac{1}{(x - 0.9)^2 + 0.04} - 6
\end{equation*}
Hitung nilai fungsi ini pada titik-titik dalam interval x = 0 sampai x = 1
dengan jarak antar titik 0.1. Gunakan interpolasi spline kubik pada data
yang Anda hasilkan dan buat plot dari fungsi \textit{humps} dengan
hasil interpolan spline yang Anda dapatkan.
\end{soal}


\begin{soal}[Chapra Latihan 17.4]
Gunakan regresi kuadrat terkecil untuk mencocokkan garis lurus
ke data berikut.

{\centering
\begin{tabular}{|c|ccccccccccc|}
\hline
$x$ & 6 & 7 & 11 & 15 & 17 & 21 & 23 & 29 & 29 & 37 & 39 \\
$x$ & 29 & 21 & 29 & 14 & 21 & 15 & 7 & 7 & 13 & 0 & 3 \\
\hline
\end{tabular}
\par}

Plot data dan garis lurus (persamaan linear) yang Anda dapatkan
dalam satu plot. Hitung juga koefisien korelasi dan determinasi.
\end{soal}




\begin{soal}[Chapra Latihan 17.6]
Gunakan regresi kuadrat terkecil untuk mencocokkan polinomial
orde-1 dan orde-2

{\centering
\begin{tabular}{|c|ccccccccc|}
\hline
$x$ & 1 & 2 & 3 & 4 & 5 & 6 & 7 & 8 & 9 \\
$x$ & 1 & 1.5 & 2 & 3 & 4 & 5 & 8 & 10 & 13 \\
\hline
\end{tabular}
\par}
\end{soal}

\begin{soal}[Chapra Latihan 17.8]

Lakukan pencocokan model pangkat:
\begin{equation*}
y = a x^{b}
\end{equation*}
pada data berikut.

{\centering
\begin{tabular}{|c|cccc cccc cc|}
\hline
$x$ & 2.5 & 3.5 & 5   & 6   & 7.5 & 10  & 12.5 & 15  & 17.5 & 20 \\
$y$ & 13  & 11  & 8.5 & 8.2 & 7   & 6.2 & 5.2  & 4.8 & 4.6  & 4.3 \\
\hline
\end{tabular}
\par}

Gunakan persamaan yang diperoleh utuk memprediksi keluaran model jika
$x = 9$.
Jelaskan transformasi apa yang harus Anda buat untuk mengubah permasalahan
ini menjadi permasalahan regresi linear.
Plot data dan kurva yang dihasilkan pada koordinat $x-y$ dan koordinat di mana
model menjadi linear (Lihat Gambar 17.9).
\end{soal}

\begin{soal}[Chapra Latihan 17.9]

Lakukan pencocokan model eksponensial:
\begin{equation*}
y = \alpha_{1} e^{\beta_{1}x}
\end{equation*}
pada data berikut.

{\centering
\begin{tabular}{|c|ccccccccc|}
\hline
$x$ & 0.4 & 0.8 & 1.2 & 1.6 & 2 & 2.3 \\
$y$ & 800 & 975 & 1500 & 1950 & 2900 & 3600 \\
\hline
\end{tabular}
\par}
Jelaskan transformasi apa yang harus Anda buat untuk mengubah permasalahan
ini menjadi permasalahan regresi linear.
Plot data dan kurva yang dihasilkan pada koordinat $x-y$ dan koordinat di mana
model menjadi linear (Lihat Gambar 17.9).
\end{soal}

\begin{soal}
Titik simpul Chebyshev jenis pertama didefinisikan pada selang $[-1,1]$ dan
dapat dituliskan sebagai berikut:
\begin{equation}
t_{k} = -\cos\left( \frac{k\pi}{n} \right), \,\,\,\, k = 0, \ldots, n
\end{equation}
Gunakan titik simpul Chebyshev sebagai ganti dari titik-titik $x$ yang memiliki jarak seragam
untuk melakukan interpolasi dengan polinom orde 10 dan orde 20 untuk fungsi Runge
pada interval $[-1,1]$.
Bandingkan hasilnya jika Anda menggunakan titik-titik $x$ yang memiliki jarak
seragam.
\end{soal}


\section{Kode Fortran}

Pada bagian ini akan diberikan kode Fortran dari beberapa algoritma yang
ada pada buku Chapra mengenai interpolasi.

Beberapa catatan mengenai Fortran:
\begin{itemize}
\item Kode Fortran disusun berdasarkan blok dengan kata kunci dan nama tertentu.
Yang biasa kita gunakan adalah kata kunci \txtinline{PROGRAM} dan \txtinline{SUBROUTINE}.
Akhir tiap blok ditandai dengan kata kunci \txtinline{END}.
\item Fortran dapat menebak tipe data yang kita gunakan berdasarkan nama variabel
yang digunakan. Kita memaksa agar hal ini tidak terjadi dengan menggunakan pernyataan
\txtinline{IMPLICIT NONE}, artinya seluruh variabel yang digunakan harus dideklarasikan
terlebih dahulu beserta tipenya. Beberapa korespondensi tipe variabel:
  \begin{itemize}
  \item \txtinline{REAL(8)} dengan \txtinline{float64} pada Python
  \item \txtinline{INTEGER} dengan \txtinline{int}
  \item \txtinline{LOGICAL} dengan \txtinline{bool}
  \end{itemize}
\item Array pada Fortran dapat dimulai dari indeks bilangan bulat apa saja, namun secara default
dimulai dari 1. Contoh:
  \begin{itemize}
  \item \txtinline{a(N)}: array \txtinline{a} memiliki indeks \txtinline{1}
  sampai \txtinline{N}. Jumlah elemen array adalah \txtinline{N}.
  \item \txtinline{a(0:N)}: array \txtinline{a} memiliki indeks \txtinline{0}
  sampai \txtinline{N}. Jumlah elemen array adalah \txtinline{N+1}.
  \end{itemize}
\end{itemize}

Anda dapat menggunakan laman berikut jika pada komputer Anda tidak tersedia
\textit{compiler} untuk Fortran:
{\footnotesize \url{https://www.onlinegdb.com/online_fortran_compiler}}

\subsection{Interpolasi Newton}
\fortranfile{../../chapra_7th/ch18/newton_interp.f90}

\subsection{Contoh 18.2}
\fortranfile{../../chapra_7th/ch18/example_18_2.f90}

\subsection{Contoh 18.3}
\fortranfile{../../chapra_7th/ch18/example_18_3.f90}

\subsection{Interpolasi Lagrange}
\fortranfile{../../chapra_7th/ch18/lagrange_interp.f90}


\subsection{Interpolasi Spline Kubik Natural}
\fortranfile{../../chapra_7th/ch18/natural_cubic_spline.f90}


\end{document}


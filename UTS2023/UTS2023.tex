\documentclass[a4paper,11pt,bahasa]{extarticle} % screen setting
\usepackage[a4paper]{geometry}

%\documentclass[b5paper,11pt,bahasa]{article} % screen setting
%\usepackage[b5paper]{geometry}

%\geometry{verbose,tmargin=1.5cm,bmargin=1.5cm,lmargin=1.5cm,rmargin=1.5cm}

\geometry{verbose,tmargin=2.0cm,bmargin=2.0cm,lmargin=2.0cm,rmargin=2.0cm}

\setlength{\parskip}{\smallskipamount}
\setlength{\parindent}{0pt}

%\usepackage{cmbright}
%\renewcommand{\familydefault}{\sfdefault}

\usepackage{amsmath}
\usepackage{amssymb}

\usepackage[libertine]{newtxmath}

\usepackage[no-math]{fontspec}
\setmainfont{Linux Libertine O}

%\usepackage{fontspec}
%\usepackage{lmodern}

\setmonofont{JuliaMono-Regular}


\usepackage{hyperref}
\usepackage{url}
\usepackage{xcolor}
\usepackage{enumitem}
\usepackage{mhchem}
\usepackage{graphicx}
\usepackage{float}

\usepackage{minted}

\newminted{julia}{breaklines,fontsize=\footnotesize}
\newminted{python}{breaklines,fontsize=\footnotesize}

\newminted{bash}{breaklines,fontsize=\footnotesize}
\newminted{text}{breaklines,fontsize=\footnotesize}

\newcommand{\txtinline}[1]{\mintinline[breaklines,fontsize=\footnotesize]{text}{#1}}
\newcommand{\jlinline}[1]{\mintinline[breaklines,fontsize=\footnotesize]{julia}{#1}}
\newcommand{\pyinline}[1]{\mintinline[breaklines,fontsize=\footnotesize]{python}{#1}}

\newmintedfile[juliafile]{julia}{breaklines,fontsize=\footnotesize}
\newmintedfile[pythonfile]{python}{breaklines,fontsize=\footnotesize}
\newmintedfile[fortranfile]{fortran}{breaklines,fontsize=\footnotesize}

\usepackage{mdframed}
\usepackage{setspace}
\onehalfspacing

\usepackage{babel}
\usepackage{appendix}

\newcommand{\highlighteq}[1]{\colorbox{blue!25}{$\displaystyle#1$}}
\newcommand{\highlight}[1]{\colorbox{red!25}{#1}}

\newcounter{soal}%[section]
\newenvironment{soal}[1][]{\refstepcounter{soal}\par\medskip
   \noindent \textbf{Soal~\thesoal. #1} \sffamily}{\medskip}


\definecolor{mintedbg}{rgb}{0.95,0.95,0.95}
\BeforeBeginEnvironment{minted}{
    \begin{mdframed}[%
        topline=false,bottomline=false,%
        leftline=false,rightline=false]
}
\AfterEndEnvironment{minted}{\end{mdframed}}


\BeforeBeginEnvironment{soal}{
    \begin{mdframed}[%
        topline=true,bottomline=false,%
        leftline=true,rightline=false]
}
\AfterEndEnvironment{soal}{\end{mdframed}}


% -------------------------
\begin{document}

\title{Ujian Tengah Semester}
\author{TF2202 Komputasi Rekayasa}
\date{9 Maret 2023}
\maketitle

Kerjakan dengan menggunakan Python (dalam Jupyter Notebook)
atau Microsoft Excel atau aplikasi \textit{spreadsheet},
silakan pilih salah satu
jika tidak ada instruksi khusus.
Lengkapi setiap file \txtinline{ipynb} dan \textit{spreadsheet}
yang Anda gunakan dengan nama dan NIM.
File yang tidak ada identitas tidak akan dinilai.


Jawaban dan file-file terkait dikumpulkan dalam satu folder dan
dikompres dengan ekstensi .zip dengan
format nama \txtinline{NIM_UTS_TF2202.zip} , file dengan ekstensi lain tidak
akan diterima.


\begin{soal}
Jelaskan mengenai kesalahan pemotongan (\textit{truncation error})
dan kesalahan pembulatan (\textit{round-off error}) pada perhitungan numerik
dan bagaimana cara mengatasi atau mengurangi masing-masing jenis-jenis
kesalahan tersebut.
\end{soal}


\begin{soal}
Bandingkan dua persamaan berikut:
\begin{align*}
x_{1,2} & = \frac{-b \pm \sqrt{b^2 - 4ac}}{2a} \\
x_{1,2} & = \frac{-2c}{b \pm \sqrt{b^2 - 4ac}}
\end{align*}
untuk mencari akar-akar pada persamaan kuadrat:
\begin{equation*}
x^2 - 6000.001x + 10
\end{equation*}
Lakukan dalam \textit{single precision} dan \textit{double precision}.
\end{soal}


% Chapra Latihan 4.4
\begin{soal}
Ekspansi deret Maclaurin untuk $\tan^{-1}(x)$ untuk $|x| \leq 1$ diberikan sebagai
\begin{equation*}
\tan^{-1}(x) = \sum_{n=0}^{\infty} \frac{(-1)^{n}}{2n + 1} x^{2n + 1}
\end{equation*}
Buatlah program Python untuk menghitung aproksimasi dari $\tan^{-1}(\pi/6)$ menggunakan deret
Maclaurin tersebut. Hitung juga kesalahan untuk tiap penambahan suku baru dengan referensi
nilai yang diberikan oleh fungsi \textit{built-in} pada modul \txtinline{math} atau
\txtinline{numpy} atau pada Microsoft Excel.
\end{soal}

\begin{soal}[Fungsi Bessel bola]
Fungsi Bessel bola (\textit{spherical Bessel}) $j_{n}(x)$ dapat dituliskan sebagai
berikut:
\begin{equation}
j_{n}(x) = (-x)^{n}
\left( \frac{1}{x} \frac{\mathrm{d}}{\mathrm{d}x} \right)^n
\frac{\sin(x)}{x}
\label{eq:bessel_sferis}
\end{equation}
Buatlah plot untuk $j_{2}(x)$ dan carilah semua akar-akarnya pada interval (0,20).
Gunakan salah satu metode untuk mencari akar persamaan nonlinear. Jelaskan metode
yang Anda gunakan untuk mendapatkan seluruh akar tersebut.
(Catatan: Anda boleh menggunakan bentuk eksplisit dari $j_2(x)$ dari referensi
tanpa menggunakan Persamaan \ref{eq:bessel_sferis})
\end{soal}


\begin{soal}
Tentukan akar dari sistem persamaan nonlinear berikut
\begin{align*}
(x - 4)^2 + (y - 4)^2 & = 5 \\
x^2 + y^2 & = 16
\end{align*}
Buatlah plot dari fungsi-fungsi tersebut dan gunakan untuk mendapatkan estimasi tebakan
awal untuk metode Newton-Raphson.
\end{soal}




\begin{soal}
Sistem pegas-beban ideal memiliki berbagai aplikasi kerekayasaan.
Gambar di bawah ini menujukkan contohnya.

{\centering
\includegraphics[scale=1.0]{images/IMG_soal_pegas.png}
\par}

Setelah semua beban dilepaskan dan jatuh karena gravitasi,
terjadi perpindahan $x_1$, $x_2$, $x_3$. Dengan menggunakan metode eliminasi Gauss,
tentukan ketiga nilai perpindahan tersebut jika diketahui $m_1=10$ kg, $m_2=3.5$ kg,
dan $m_3=2$ kg.
Buatlah dulu skema gaya yang terjadi pada setiap beban, kemudian tentukan
tiga persamaan linier yang merupakan fungsi dari $x_1$, $x_2$, $x_3$,
dan tuliskan dalam file pdf/jpg (tulisan tangan difoto/scan) atau diketik.
Tulis atau implementasikan sendiri fungsi tersebut (tidak boleh menggunakan
\pyinline{linsolve} pada Python atau fungsi bawaan seperti MMULT atau MINVERSE
dari Microsoft Excel).
\end{soal}


\begin{soal}
Apakah penggunaan dekomposisi LU untuk menyelesaikan sistem persamaan linear memiliki keuntungan
dibandingkan dengan metode eliminasi Gauss? Jelaskan jawaban Anda (misalnya dengan menjelaskan
situasi atau masalah tertentu di mana metode dekomposisi LU memiliki keuntungan dibandingkan
dengan eliminasi Gauss).
\end{soal}


\begin{soal}
Cari $\mathbf{x}$ yang memenuhi persamaan linear
\begin{equation}
\mathbf{A} \mathbf{x} = \mathbf{b}
\label{eq:linear_eq}
\end{equation}
dengan
\begin{equation*}
\mathbf{A} = \begin{bmatrix}
888445 & 887112\\
887112 & 885781
\end{bmatrix}
\end{equation*}
dan
\begin{equation*}
\mathbf{b} = \begin{bmatrix}
10\\
0
\end{bmatrix}
\end{equation*}
menggunakan fungsi \pyinline{linsolve} pada modul \pyinline{numpy.linalg}
atau fungsi MINVERSE pada Microsoft Excel.
\footnote{Misalnya: \url{https://www.excel-easy.com/examples/system-of-linear-equations.html}}
Lakukan pengecekan hasil yang Anda
dapatkan dengan cara mengecek apakah persamaan \ref{eq:linear_eq} terpenuhi atau tidak
(ruas kiri sama dengan ruas kanan).
Bandingkan hasil yang Anda peroleh dengan cara manual (menggunakan kalkulator atau
perhitungan analitik) atau dengan bantuan SymPy (lihat kode yang diberikan).
Apakah hasil yang Anda peroleh sama? Jika tidak apa yang
menyebabkan perbedaan tersebut? Apa yang harus Anda lakukan agar hasilnya sama (jika
hasil yang Anda peroleh berbeda) ?
\end{soal}

Anda dapat menggunakan kode berikut.
\begin{pythoncode}
from sympy import Matrix, pprint, init_printing

init_printing()

A = Matrix([
    [888445, 887112],
    [887112, 885781]
])

b = Matrix([[10], [0]])

x = A.solve(b)
print("x = ");
pprint(x)
\end{pythoncode}
atau (untuk mendapatkan bentuk eksplisit dari $\mathbf{x}$ pada kasus
ukuran matriks $\mathbf{A}$ adalah $2\times2$)
\begin{pythoncode}
from sympy import *

# Elemen-elemen matriks dan vektor, indeks mulai dari 0.
A00, A01, A10, A11 = symbols("A00 A01 A10 A11")
b0, b1 = symbols("b0 b1")

A = Matrix([
    [A00, A01],
    [A10, A11]
])

b = Matrix([[b0], [b1]])

x = A.solve(b)
print("x = ");
print(x)
\end{pythoncode}


\begin{soal}
Merujuk pada jenis kesalahan yang ditanyakan pada soal No. 1, apakah jenis kesalahan
yang ada pada metode eliminasi Gauss untuk penyelesaian sistem persamaan linear?
Bagaimana dengan metode iteratif, misalnya metode Gauss-Seidel? Jelaskan jawaban Anda.
(apakah kedua jenis kesalahan ada atau hanya salah satunya saja).
\end{soal}



\begin{soal}
Cari nilai maksimum dari
\begin{equation*}
f(x,y) = 4x + 2y + x^2 - 2x^4 + 2xy - 3y^2
\end{equation*}
dengan menggunakan metode \textit{steepest descent} atau \textit{steepest ascent}.
Buat plot kontur
dari $f(x,y)$ dan tunjukkan titik yang dilalui ketika proses optimisasi.
Gunakan (0,0) sebagai titik awal.
\end{soal}


\end{document}
